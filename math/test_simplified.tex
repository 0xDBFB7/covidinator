\documentclass[]{article}

\usepackage{pdfpages}

\begin{document}


Existing worked example from a paper:

$$ e_x(z,t) = \frac{1}{\sqrt{2 \pi}} \int_{-\infty}^{+\infty}{F(\omega) e^{- j (\omega/c_0)n(\omega)z}\ e^{j\omega t} d\omega} $$

$$W = \frac{1}{\eta_0} \int_{-\infty}^{+\infty}{(n_r(\omega))\ |F(\omega)|^2}\ d\omega$$\\

Where $n(\omega)$ is a complex function, $n_r()$ is the real part of the same, everything else is a real constant. An existing paper sets up the functional with a lagrange multiplier\\

$$\xi = -e_x(z,t) + \lambda W$$\\

Takes the functional ("variational") gradient with respect to $F(\omega)$.

$$ \nabla \xi = -\frac{1}{\sqrt{2\pi}} \left(e^{- j (\omega/c_0)n(\omega)z}\right)^\star \  e^{-j\omega T} + \lambda...$$

Where * is the complex conjugate.\\

The $\nabla H() = H()^\star$ thing seems to come from Franks, Signal Theory, 1969, page 140 onwards; I can't seem to make heads or tails of his formulation... \\

Neither Mathematica's VariationalD, Matlab's Fundiff, nor Maple's functionalDerivative seem to yield any meaningful result for grad xi, but I probably set things up wrong.\\

They then set

$$\nabla \xi = 0$$

and solve for $F(\omega)$ normally. (then fourier transform back to $F(t)$)\\

\pagebreak

What I'd like to do is modify this to maximize the amplitude x(t) of oscillation of a harmonic oscillator. Using the Greens function solution, \\

where q is a charge, oscillation amplitude is


listed therein (eq. 15, for the underdamped case).\\

Anyhow, then modify the lagrange multiplier,

$$\xi = -x(t) + \lambda W$$\\

And now,

$$\nabla \xi = 0$$

That seems to be the hard part.\\

Is there any chance you could suggest a route to attack this gradient?

\end{document}
