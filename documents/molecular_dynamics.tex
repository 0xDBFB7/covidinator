
\paragraph{Molecular dynamics simulation}
\
%\begin{multicols}{1}
%%%%%%%%%%%%%%%%%%%%%%%%%%%%%%%%%%%%%%%%%%%%%%%%

It was originally expected that a large part of this work would be done in-silico, as we did not anticipate having access to suitable RF test equipment. 

Some time was spent attempting to set up a molecular dynamics toolchain capable of simulating an entire virus. 

Having an approximate simulation of the this technique would be useful for a number of reasons: 

We would have a better idea of the transferability to SARS, without wasting the time of experts with BSL-3/4 labs.

The impulse could be subjected to the same optimization as the RF feedback loop. A number of parameters (such as phase, polarization)



Coarse-graining also greatly increases the allowable timestep.

We also did not understand the *confined* part in the *confined* acoustic resonance.

Simulating the aggregate bose condensate wavefunction is well beyond us.


Both finite-element, molecular-dynamics, and stiffness-eigenvalue normal mode techniques have been used to great effect for this purpose. 


%\end{multicols}





\PRLsep{{\itshape Virus structural data }}

One extreme can be seen in 

Because CryoEM is sensitive to the 



