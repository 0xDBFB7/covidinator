\documentclass[fleqn,10pt]{paper}

\usepackage[left=2cm,right=2cm,
top=0.5cm,
bottom=1.5cm,%
headheight=11pt,%
letterpaper]{geometry}



\usepackage[left=2cm,right=2cm,
			top=1.25cm,
			bottom=2.25cm,%
			headheight=11pt,%
			letterpaper]{geometry}
			
\frenchspacing			

\nonstopmode




\usepackage{lmodern}
\usepackage[T1]{fontenc}
\usepackage[utf8]{inputenc}



\usepackage{noweb}

\usepackage{multicol}
\usepackage{fancyhdr}
\usepackage{blindtext,graphicx}
\usepackage[absolute]{textpos}
%\usepackage[parfill]{parskip}
\usepackage{parskip}
\setlength{\parskip}{\baselineskip}

\usepackage[colorlinks=true,citecolor=brown]{hyperref}
\usepackage{gensymb}
\usepackage{csquotes}
\usepackage{amsmath}
\usepackage{fontawesome}
\usepackage{orcidlink}
\usepackage{standalone}
\usepackage{pdfpages}
\usepackage{subfiles}
\usepackage{svg}
\usepackage{sidecap}
\usepackage{float}
\usepackage{amssymb}
\usepackage{textcomp}
\usepackage{lettrine}
%\usepackage[T1]{fontenc}

\usepackage{soul}


%\usepackage{draftwatermark}
%\SetWatermarkText{DRAFT}
%\SetWatermarkScale{0.25}

\usepackage{booktabs,caption}
\usepackage[flushleft]{threeparttable}

%\usepackage{biblatex}
\usepackage[backend=bibtex8, sorting=none, style=chem-angew]{biblatex}

\let\cite\footfullcite

%\let\cite\footcite

\addbibresource{processed.bib}
%biblatex has a zoterordfxml
% might avoid the need for python bibtex_collections.py



\usepackage{etoolbox}
\AtBeginEnvironment{quote}{\small}




\usepackage{pifont}
\newcommand{\cmark}{\ding{51}}%
\newcommand{\xmark}{\ding{55}}%


\newcommand{\citationneeded}[1][]{\textsuperscript{[\color{blue}{\it \bf{citation needed}#1}]}}
\newcommand{\dubiousdiscuss}[1][]{\textsuperscript{\color{blue} [{\it \bf{dubious-discuss}}]} }

\newcommand{\light}[1]{\textcolor{gray}{#1}}

%
%
\usepackage{titlesec}
%
%% custom section


\titleformat{\section}
{\normalfont\LARGE\bfseries}{\thesection}{1em}{}
%\titleformat{\section}
%{\normalfont\LARGE\bfseries\PRLsep}
%{{{{\itshape \thesection\hskip 9pt\textpipe\hskip 9pt}}}}{0pt}{}
%
%% custom section
%\titleformat{\subsection}
%{\normalfont\Large\bfseries\PRLsep}
%{{{{\itshape \thesection\hskip 9pt\textpipe\hskip 9pt}}}}{0pt}{}
%
%
%


\newcommand{\Wsqm}{$\text{ W/m}^2$}

\newcommand{\ghfile}[1]{\href{https://github.com/0xDBFB7/covidinator/tree/master/#1}{\faGithub/\url{#1} }}

%\newcommand{\supercite}[1]{}
%\newcommand{\supercollect}[1]{}


\newlength{\PRLlen}
\newcommand*\PRLsep[1]{{\itshape \Large\settowidth{\PRLlen}{#1}\advance\PRLlen by -\textwidth\divide\PRLlen by -2\noindent\makebox[\the\PRLlen]{\resizebox{\the\PRLlen}{1pt}{$\blacktriangleleft$}}\raisebox{-.5ex}{#1}\makebox[\the\PRLlen]{\resizebox{\the\PRLlen}{1pt}{$\blacktriangleright$}}\bigskip}}


\renewcommand{\thefootnote}{\textcolor{gray}{\arabic{footnote}}}


\usepackage{graphicx}
\graphicspath{ {../media/} 
				{../firmware/eppenwolf/runs/sic_susceptor/} 
			}

\usepackage{tcolorbox}
\newtcolorbox{protocol}{colback=yellow!5!white,colframe=yellow!75!black}
\newtcolorbox{equipment}{colback=orange!5!white,colframe=orange!75!black}
\newtcolorbox{autem}{colback=red!5!white,colframe=red!75!black}
\newtcolorbox{toolchain}{colback=blue!5!white,colframe=blue!40!black!40}
\newtcolorbox{sidenote}{colback=cyan!5!white,colframe=blue!40!black!40}
%https://tex.stackexchange.com/questions/66154/how-to-construct-a-coloured-box-with-rounded-corners

%\usepackage[sfdefault,light]{roboto}

\setlength{\TPHorizModule}{1cm}
\setlength{\TPVertModule}{1cm}





%%%%********************************************************************
% fancy quotes
\definecolor{quotemark}{gray}{0.7}
\makeatletter
\def\fquote{%
	\@ifnextchar[{\fquote@i}{\fquote@i[]}%]
}%
\def\fquote@i[#1]{%
	\def\tempa{#1}%
	\@ifnextchar[{\fquote@ii}{\fquote@ii[]}%]
}%
\def\fquote@ii[#1]{%
	\def\tempb{#1}%
	\@ifnextchar[{\fquote@iii}{\fquote@iii[]}%]
}%
\def\fquote@iii[#1]{%
	\def\tempc{#1}%
	\vspace{1em}%
	\noindent%
	\begin{list}{}{%
			\setlength{\leftmargin}{0.1\textwidth}%
			\setlength{\rightmargin}{0.1\textwidth}%
		}%
		\item[]%
		\begin{picture}(0,0)%
		\put(-15,-5){\makebox(0,0){\scalebox{3}{\textcolor{quotemark}{``}}}}%
		\end{picture}%
		\begingroup\itshape}%
	%%%%********************************************************************
	\def\endfquote{%
		\endgroup\par%
		\makebox[0pt][l]{%
			\hspace{0.8\textwidth}%
			\begin{picture}(0,0)(0,0)%
			\put(15,15){\makebox(0,0){%
					\scalebox{3}{\color{quotemark}''}}}%
			\end{picture}}%
		\ifx\tempa\empty%
		\else%
		\ifx\tempc\empty%
		\hfill\rule{100pt}{0.5pt}\\\mbox{}\hfill\tempa,\ \emph{\tempb}%
		\else%
		\hfill\rule{100pt}{0.5pt}\\\mbox{}\hfill\tempa,\ \emph{\tempb},\ \tempc%
		\fi\fi\par%
		\vspace{0.5em}%
	\end{list}%
}%
\makeatother







%%%%********************************************************************
%title link to doi
\newbibmacro{string+doiurlisbn}[1]{%
	\iffieldundef{doi}{%
		\iffieldundef{url}{%
			\iffieldundef{isbn}{%
				\iffieldundef{issn}{%
					#1%
				}{%
					\href{http://books.google.com/books?vid=ISSN\thefield{issn}}{#1}%
				}%
			}{%
				\href{http://books.google.com/books?vid=ISBN\thefield{isbn}}{#1}%
			}%
		}{%
			\href{\thefield{url}}{#1}%
		}%
	}{%
		\href{https://doi.org/\thefield{doi}}{#1}%
	}%
}

\DeclareFieldFormat{journaltitle}{\usebibmacro{string+doiurlisbn}{\mkbibemph{#1}}}


\title{{\it Maxwell's Silver Hammer:}\\ On pulsed microwave acoustic resonant viral inactivation}


\begin{document}

\maketitle

%\footnote{The}

This is a condensed version of a (horribly unfinished) paper at \href{https://www.github.com/0xDBFB7/covidinator/documents/paper.pdf}{\faGithub/0xDBFB7/covidinator}.\footnote{Daniel Correia [TODO affiliation] \orcidlink{0000-0002-9353-0216}}


\subfile{pulse_1.tex}



Both the microwave and microbiology components of this field\footnote{field field?} appear to be unusually sensitive to experimental design and exceptionally prone to artifacts that cause false positive results: \cite{Microwave1982} $\neq$ \cite{Resonances1987}, \cite{Effects1985a} $\neq$ \cite{Cytogenetic1986}, \cite{Comprehensive2018}. Though the cited articles have taken care to avoid these effects, it is entirely possible that none of what is discussed below actually exists.  

This project was rushed and slow, terribly managed, and not nearly as rigorous as the field demands. This may well end up being a waste of time.

%The literature is littered with irreproducible claims of non-thermal effects; however, this appears to be a singular anomaly.

{\Large \textbf{The mechanism}}


\cite{Microwave2009} \textrightarrow \ (\cite{focusing2014} $\parallel$ \cite{Efficient2015} $\parallel$ \cite{Resonant2017})\footnote{A great many papers are involved.} appear to establish that the net charge, mass distribution, and envelope stiffness of a few species of viruses support a weak resonance mode in the microwave spectrum, with a peak at roughly 8 GHz for Inf. A.

They appear to demonstrate that this allows a comparatively inconsequential electromagnetic field magnitude to produce mechanical stresses sufficient to crack the lipid envelope, destroying the virus.

\cite{Efficient2015} specifically provides a reasonable spherical-cow model for how this could occur that agrees well with experiment.\footnote{See also \ghfile{documents/biology.ipynb}}

However, this work would seem to be more useful if extended with the following assertion: the virus' Q factor is only about 2, meaning that the steady-state amplitude and stresses should be reached in some nanoseconds.

Barring protein mechanical fatigue \cite{Mechanical2013} or some other mechanism that could prolong the inactivation duration, it appears to be possible to reduce the required power density to a minuscule fraction of all safety limits by pulsing the field at nanosecond scales. This incidentally allows scanning of a focal point, a significant area can be treated with a relatively small transmit power.

\cite{Efficient2015} test with various strains of Influenza A, we are testing with Coliphage T4; structural similarities with NCoV mean this is may be transferable. 

These authors above deserve all the credit for this finding; nothing in this work is even remotely novel.

{\Large \textbf{Is it safe?}}

The consensus is that conditions in our cells are highly unfavorable for any such resonance modes; damping from biological solvents is too strong \cite{Vibrational2002}. Harm occurs only through thermal mechanisms; therefore an almost arbitrarily powerful pulse can be safely applied for infinitesimal periods.

Quality in-vitro data supports this assertion. \cite{Cytogenetic2006} expose cells to pulse power in precisely the regime required in this work; 8.2 GHz, 8 ns duration, 50 khz repetition rate, a whopping pulse power density of 250,000\Wsqm\footnote{computed from average power / duty cycle} (2500x the time-averaged power density safety limit), average power density 100\Wsqm (the safety limit), for 2 hours, finding no change in any of the measured quantities. \cite{DNA2004} replicate this finding, finding genotoxicity purely through the expected thermal mode. \footnote{These papers were cherry-picked from a sea of positive results. See the full paper for the rationale.}

The literature reviews of \cite{ICNIRP2020} and \cite{C95} are exhaustive; however, there is not a great deal of in-vivo evidence in this frequency range\cite{New2019}\cite{Comprehensive2018}.

We are unsure what specific feature of the virion accounts for this discrepancy. The lipid bilayer of the virion differs in composition from that of the host cell, imparting a different net charge.

\clearpage 
{\Large \textbf{How do we wield this sword?}}

If the above is correct, any region with a power density in the 7 to 9 GHz band greater than 300\Wsqm \ for 100 ns will be sterilized. (TODO field strength)

This is not a particularly demanding requirement, and lends itself to a wide variety of applications, many of which could be implemented almost immediately. 

First, depending on the type of tissue, this field strength can apparently be safely produced a minimum of 3 cm deep (suitable for targeted sterilization of lungs or other organs with bronchoscopic techniques), up to 200 cm (eradicating the virus from the body) \cite{Physical1982}. We are working to narrow these bounds now.

On an individual level, the free-space-combined output of, say, a few dozen SiGe:C or GaAs transistors (\$0.3 apiece [BFP620]), very similarly to our reference design, should suffice to locally protect the major vulnerabilities like mouth, nose, eyes, hair and any surfaces that the hands contact. This arguably wouldn't provide many advantages over simple masks, however. 

Unlike UV, the X-band is minimally attenuated by air. Common areas could be illuminated by magnetron systems.

On the other extreme, installations with megawatt-scale klystrons could sterilize square-kilometer areas at a time. 



{\Large \textbf{Tasks}}

We can see the following 

\begin{itemize}
	\item To prevent duplication of work, are any other teams working on this now? It seems a rather obvious technique. A thorough review of arxiv etc should be performed.
	\item Double-check all figures and assumptions.
	\item Send off spectrometer (or, ideally, a more precise piece of equipment) to a BSL with a sample of NCoV to obtain the spectra and thresholds.
	\item get the gosh-darned AustinMan voxel team to reply to my gosh-darned emails
	\item Determine the ideal application to focus effort on
	\item ???????
	\item Check with the FCC/Industry Canada if we can get a temporary exemption for these intentional radiators?
	
\end{itemize}

%\centering {\color{red} \faHeart}

\clearpage




\printbibliography[heading=none, title={}]





\end{document}

