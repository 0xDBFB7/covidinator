%!TeX root = summary
\documentclass[paper.tex]{subfiles}
\begin{document}

\subfile{title}


\begin{figure}[H]
\captionsetup{singlelinecheck = false, justification=justified}
	\centering
	
	\subfloat[]{
		\includegraphics*[width=0.5\textwidth]{pulse_exposure_setup.JPG}
		
	}
	\subfloat[]{
		\includegraphics*[width=0.5\textwidth]{eppenwolf_output_cap.jpg}
	}
    \hfill
	\subfloat[]{
		\includegraphics*[width=0.5\textwidth]{bronch_9GHz_500W_2_transparent.png}
	}
	\subfloat[]{
		\includegraphics*[width=0.5\textwidth]{CPWG_sim_pretty_2}
	}
    \hfill

	\subfloat[]{
		\includegraphics[width=0.3\textwidth]{x_gal.jpg}
	}
	\subfloat[]{
		\includegraphics[width=0.5\textwidth]{fluro_1}
	}
%	\caption{Simulation of a (largely ineffective) bronchoscopic method.}

	\caption*{\small{\\
	(a) \textit{Chengxiang} nanosecond pulse exposure jig. Aluminum disc belongs to the scan-conversion digitizer, not used in this study.\\
	(b) Pulsed X-Band microwave spectrometer used.\\
	(c) Simulation of a (largely ineffective) bronchoscopic method.\\
	(d) FDTD simulation geometry of the 0.2 mm coplanar waveguide exposure cell.\\
	(e)	The very pretty opalescent blue culture caused by E. coli B uptake of X-Gal.\\
	(f) Fluorescence reader used.\\
}}
%		\caption*{\small{}}
%				\caption*{\small{FDTD simulation geometry of the 0.2 mm coplanar waveguide exposure cell. }}
%						\caption*{\small{The very pretty opalescent blue culture caused by E. coli B and X-gal.}}
%						\caption*{\small{Fluorescence reader used.}} 
%	\caption{Brillouin precursor propagation through tissue (reproduced from section 6).\\
%		(a) Gaussian monopulse with 16 ps FWHM.\\
%		(b) Optimized 10 GHz sawtooth wave. Note the slight high-frequency ripple.\\ This has a frequency of approximately 1.5 GHz.\\
%		(c) Prototypical precursor 10 GHz sine burst.\\
%		(d) Continuous 10 GHz sine tone.}
\end{figure}
\clearpage

%	\caption{FDTD simulation geometry of the 0.3 mm coplanar waveguide exposure cell. Red object is the fluid channel. Visualized with Paraview.
%	\ghfile{covidinator/electronics/simple_fdtd/run/microfluidic_coplanar.py}}
%

\subfile{experiment_1_data}

\clearpage
%
%\begin{figure}[H]
%	\makebox[\textwidth][c]{
%		\includegraphics[width=\textwidth*2]{chunk_4_line_2_1.png}
%	}%
%	\caption{Chunk 4, it.is Founda}
%\end{figure}


%\clearpage




%\section{Terse? I can be terse. Once, in flight school, I was laconic.}

\begin{multicols}{1}

%\begin{refsection}
\lettrine{W}e were inspired to undertake this project by a set of publications by a certain Sun group, (primarily Liu et al \cite{Microwave2009} and Yang et al \cite{Efficient2015}, simultaneously demonstrated by Hung et al \cite{focusing2014}), partially replicated by \cite{Optical2020} and discussed by \cite{Theoretical2020}.

%We have some minor reservations regarding experimental technique in the original papers that can only be addressed by more extensive replication studies, and luckily are due to be answered by other groups\cite{Generating}, so we try not to defer greatly to their observations. 

These studies are all founded on the idea that some large-scale structural difference between viruses and all host cellular structures might usable as leverage to selectively inactivate the virus by some physical - rather than biological - means. For example, such an inactivation "window" between host cell and virion damage has been found in certain wavelengths of CW Far UV\cite{Germicidal2017} - but also apparently in certain modulated UV pulses\cite{Use1987a}\cite{Can1993}, and also sporadically reported for simple thermal treatment\cite{Summary}; however, the modes exhibiting differential inactivation that have been found so far do not appear to be of particular use in the clinic.

Some possible minor shortcomings of the original experiments are due to be addressed by other groups\cite{Generating}, so we try not to defer greatly to those observations.

A few well-established non-resonant non-thermal effects caused by extreme electric fields have already been put to practical use. Electroporation\cite{Electroporation1988}, now a common laboratory procedure, occurs when an enormous electric field causes ions in solution to diffuse across a cell membrane, capacitively charging it. This leads to a change in lipid conformation, reversibly rendering the membrane permeable (among other, more subtle mechanisms; the minutia of poration are extraordinarily complex and beyond the author's understanding\cite{Theoretical2007}). 

Therapies based on irreversible electroporation\cite{Nonthermal2013}\cite{Lipid2017} have recently been used in the clinic\cite{Irreversible2013} for tumor ablation. There is much discussion on electroporation as an alternative to viral vectors, but we have not yet found any discussion of clinical treatment of viral infection with these conventional electroporation therapies.% In vitro there is evidence to suggest that sub-cellular effects take place under truly tremendous fields. 

If the duration of exposure is sufficiently short that thermal and membrane breakdown effects can be neglected, tissue appears to withstand peak electric fields of great magnitude - above 1 nanosecond megavolt/meter, even in in-vivo studies - without severe acute effects (\cite{Repeated2020}\cite{Review2011}\cite{DNA2004} and below).

The specific concept of driving a virus as a harmonic oscillator suggested by \cite{Efficient2015} has been previously considered only briefly \cite{MECHANICAL1968}\cite{Comment2004}\cite{Vibrational2009}\cite{Maximum2010}\cite{Effects1951}, perhaps first because in almost all cases structures in tissue are so over-damped by the surrounding solvent\cite{Vibrational2002}\cite{Biological2002}\cite{Biophysics2000}\cite{Viscous2000} as to have no non-zero resonant frequency at all \cite{dielectric1996}\cite{gabriel1996compilation}\wikinote{not in citation}\wikinote{discuss}; and second, because the charge distributions found in biology are too small for energies significantly above the thermal background to be absorbed via applied fields. 

On the other hand, the possibility of a slightly underdamped virion does not appear to be entirely unexpected; classical viscous Stokes drag predicts such behavior\cite{nature1986} (see supplemental), and large proteins have been found with similar properties\cite{Microwave1994}; and while MHz transverse resonances in microtubules are ruled out\cite{Viscous2000}, the same arguments suggest that very weak GHz longitudinal modes may exist.

However, whether the small difference in driven amplitude between, for example, a $Q\approx0.5$ host structure normal mode and a $Q\lessapprox6$ virion mode could possibly be of any clinical value is not clear to us.

%
%\printbibliography[heading=subbibliography]
%\end{refsection}	

The idea of driven oscillation in biological systems appears to have originated from work by Frohlich\cite{Longrange1968}\cite{Evidence1983}\cite{Biological1980}\cite{Coherence1983} on the possibility of large-scale coherent QM Bose-Einstein condensates, and then many more subtle lines of inquiry\cite{Mechanisms1992}\cite{mechanisms1981}; but, to our knowledge, no such large-scale non-classical effects have ever been substantiated; specifically, suggestions of non-classical damping due to hydration layer effects have not been observed. In fact, evidence to date\cite{Exposure2009}\cite{ICNIRP2020}\cite{C95} appears to agree that, not only are oscillatory effects implausible, there are no relevant, substantiated, non-diffusion-like non-thermal mechanisms. 

Because the effect seen by\cite{Efficient2015} is so striking compared to previous work in the non-thermal literature, we have tried to invent ways of explaining the observed results without invoking a non-thermal effect. Sufficient orthogonal measurements were made, so this is quite difficult. The most plausible timeline was:
\begin{itemize}
	\item The use of microwave absorption spectroscopy techniques somewhat similar to those which have previously shown resonance-like artifacts\cite{Resonances1987}. In particular, an effective charge of $q=10^7 e^-$ was found via this method, which seems quite large for a $10^9$ atom structure. Genome and protein charge can hardly explain $10^5 e^-$ (see below), which would be reduced further by ionic screening. It appears that the potential energy of such a sphere of charge would exceed the observed metabolic burden to build a virus by a significant factor \footnote{
		$$ U = \frac{3}{5}  \frac{1}{4 \pi \epsilon_0} \frac{Q^2}{R} \approx 2.3
		\times10^{-7} \text{ J} \approx 6 \times 10^{11}\  \text{ATP-equivalent} $$
		%U = (3/5) * 1/(4*pi*electricConstant) *((1e7 electricCharge)^2 / 60 nanometers) = 2.30708e-7 J
		
		%50 kJ per mole, 6.022*10^23 = 9.9634673e-13 * 50 kj = 5e-8 J
		
		Mahmoudabadi\cite{Energetic2017} provide a total per-virion energetic cost of $10^8 $ ATP-equivalent , so this appears to be 3 orders of magnitude too large.
		
		Of course, much of the potential energy may have already existed in the host cell, so this is possibly a flawed comparison. If most of the charge is contributed by osmotic, ion, this potential energy may not be considered in the above. Autem: is it possible for the charged core to compensate for the potential energy of the charged shell?}. Ion polarization effects are generally not indicated at these timescales\cite{ICNIRP2020}, but faster electron polarization might occur to some small degree. 
	
	
	
	
	\item Infrared thermography only determines the surface temperature. Based on heating patterns seen in other experiments\cite{Effects1950}, the observed surface temperature rise does not seem to conclusively rule out a very localized thermal effect. 
	\item Therefore, without free-space dosimetry or simulation of power density, the frequency-selective inactivation could possibly be explained by the frequency response of the power amplifier, antenna gain, and absorption of the cuvette.
\end{itemize}

\footnote{It is interesting to consider that a field with a strong positive publication bias\cite{Comprehensive2018}, in the presence of stringent peer review, could unintentionally tend towards a literature populated with only the most abstruse and difficult to explain false-positive results.}

However, if this is a thermal artifact, the fact that the spectrometry and inactivation data shown align perfectly is very difficult to explain; and so this chain of events seems unlikely.

Permeabilization at high field strengths is not necessarily completely implausible. \cite{182015} observe a similar non-ionic photoporation effect.

An interesting effect, superficially similar to the leakage found by Yang et al, has been observed on lipid liposomes similar to viral envelopes\cite{MicrowaveStimulated1985}\cite{Influence}\cite{Correlation1994}\cite{contribution2019}. These results are of excellent quality, with good thermal controls. \footnote{One founding researcher has been implicated in a research scandal; however, the Livermore report on the topic does not appear to be public, so we cannot evaluate the basis for these claims.} This effect appears to depend on dissolved oxygen\cite{Microwaves1987}\cite{Microwaves1988}, somewhat contrary to a mechanical effect. The phase transition temperature of the lipids used (the point at which pores reseal) is also a relevant quantity. 

One such report has been found to be a very strange artifact caused by the Teflon dish used for microwave trials\cite{Effect1994a}, highlighting the need for exactly identical sham trials.


With\footnote{The dipole spring constant can be extracted from the resonant frequency and therefore from the speed of sound and $m^*$, the reduced mass ($k = \omega_{res}^2 m^2$). } relatively sane choices of solvent charge screening and resonant frequency, the energy absorbed by the whole virus per cycle at 5 MV/m might vary from $10^{-5}$ to 100 times the thermal energy $k_b T$.

Compare to $~5 k_b T$ per protein for 1 capsid segment binding energy \cite{Energies2012} \cite{Weak2002} and 6 to 30 $k_b T$ to form a single pore in a lipid bilayer \cite{Atomistic2014}. 
%17 or 78 kJ/mol / (boltzmannConstant * 310 K) / avogadroConstant = 6.5 to 30.3 kT

With a relaxation time on the order of 500 ps, it is somewhat difficult to imagine that this energy will be sufficiently concentrated in the capsid to be destructive. After the relaxation time, only the mechanical effects that were induced can remain, unless some progressive chemical or fatigue-like degradation occurs.


Brackley \cite{Electrostatic2020} describe various analytic representations of the charge screening about the virion with the Poisson-Boltzmann equation; by translating the obtained charge into a steady-state Poisson simulation (ions will not migrate in nanosecond pulse timescales), a relatively representative simulation may be obtained\footnote{implementation attempted at \ghfile{/math/mathematica/analytic_force.nb}}.

\end{multicols}
\begin{autem}
	autem, discuss\\
	At the risk of seeming a fool, what is the most useful metric for the charge quantity being discussed? We are interested in the stress produced by a certain arbitrary distribution of charge. Assuming a spherically symmetric virus composed of two charged shells, there is no intrinsic dipole moment. Polarizability or induced dipole moment are not obviously relevant; stress could be produced in an extremely stiff structure with no electron mobility. The virion has little net charge. Multipole expansions are not clearly relevant.
\end{autem}
\begin{multicols}{1}
From a force perspective, the $10^3$ genome charge easily provides the nanonewton force scale required\cite{Bacteriophage2004} to break capsids or envelopes in 5 MV/m fields; but accurate estimation of the capsid stress will likely require application of the Maxwell stress tensor \cite{Electrostatic2013}\cite{Deformation1991}.

On the other hand, such a sub-angstrom movement can only induce approximately a 0.05\% deflection of the capsid, whereas a general value for the rupture of host cells is about 3\% deflection\cite{Electromechanical1989}. The stiffness of the genome could be used to explain this. 

Accurately modelling this sort of multi-scale inactivation mechanism involving the collective motion of some $10^9$ atoms, possibly simultaneously with fatigue from atomic dislocations\cite{Physical2010} and reactive oxidation effects, poses a significant computational and theoretical hassle, which has far exceeded our capabilities.

All-atom simulations of whole capsids are now possible, but existing pipelines tend to lack genome or nucleocapsid\cite{Mesoscale2020}\cite{Physical2017} and are extremely computationally demanding.

Coarse-graining is one route out; but, as with all abstractions in simulations, the problem must be set up with great care and experience (polarizable water; proper heat-bath damping coefficients and choice of integrator)\cite{Membrane2016}\cite{Determining2014}, and with an understanding of possible failure modes; we did not trust our abilities to produce correct results. An illustrative GROMACS simulation of a liposome, trajectories fourier transformed with \cite{TRAVIS2011}, was used to determine some parameters\footnote{implemented at \ghfile{/biology/simulation/GROMACS/BUMPy_bilayer/}, see supplemental}.

Parameterizing coarse bead-spring models based on all-atom simulations of small capsid segments or proteins \cite{Elucidating2009} is another technique. In this case, DNA/RNA can be represented by homogeneous beads restrained by Lennard-Jones forces \cite{Communication2013}. Mechanically reasonable coarse-grained models have recently been published for SARS-NCoV-2\cite{Multiscale2020}. Finite-element analysis in biology is also somewhat common \cite{Finite2007}; coupled finite-element and Poisson-Boltzmann\cite{Electrostatic2020a} seems particularly effective.


An approximate charge distribution of the proteins of the virion can be obtained by multiplying primary structure charges (in our case, extracted \cite{CIDER2017}, or (\cite{complete2003})) by observed protein contribution to total MW\cite{Quantitative1981}, old 2D-PAGE gels \cite{Influenza1978}\cite{Structurea} (which also provides an estimate of the variation in the charge, important for accurate spectroscopy, see supplemental) - or perhaps directly via extraction from whole-virion CryoEM maps, enabled by the charge-sensitive nature of the technique \cite{Identification2018}\cite{Electrostatic2020}. This has not yet been implemented in any software tools and seems quite tricky. 

The genome charge is almost entirely neutralized by the nucleocapsid. 

In our cursory testing, Bacteriophage T4 did not appear to be lysed by 10 microsecond pulses between 6 and 12 GHz at about $10^2$ to $10^3$ V/m, nor, exploring non-resonant inactivation, did nanosecond pulses up to a few MV/m appear to have any effect. Note that neither of these regimes overlap with the results of the previous studies.

Non-resonant capsid and envelope breakage with 120 pulses, 500 ns each, at 3 MV/m, has been previously observed\cite{Inactivation1990} but the RNA damage suggests that this is an electrochemical artifact \cite{Formation1996} (and \cite{Microwave1987}). Long pulses at $10$ MV/m electroporate enveloped viruses similarly to host cells\cite{AC2017} but do not necessarily affect T4\cite{Manipulation2013}. 

In terms of host safety, Pakhomov \cite{Comparative} find no non-thermal effects on functioning frog heart pacemaker cells at 0.9 MV/m (although the field inside the tissue was not measured and could have been lower). Excellent recent \textit{in vivo} data by de Seze following continuous exposure to nanosecond pulse trains at an external field up to 3 MV/m \cite{Repeated2020} notes little acute effects, but a massive rate of long-term carcinogenicity. 

In-vitro, there appears to be a consistent set of effects of nanosecond pulses, or nsPEF, distinct from membrane effects, that occur on small intracellular organelles (see reviews \cite{Penetration2016c} \cite{Effects2016} ), \cite{Bioeffects}, 

In almost all recently updated standards, the precautionary 100 kV/m instantaneous field limits established by ANSI have been abolished due to a lack of evidence of harm; an excellent review in this regime is \footnote{\cite{treatyelectromagnetic}, as cited by \textit{C95.1-2019, B.4.3 Rationale for pulsed RF field limits}.\footnote{That same review also contains the following, which highlights the incredible difficulty of obtaining reliable results in this field: "Dr. de Seze also described an experiment in which two sham groups were mistakenly run and a significant difference was found. Dr. Klauenberg noted another series of experiments he reviewed where experimental treatment groups did not vary while the sham groups compared to each other did resulting in a significant difference that was otherwise meaningless."}}. Field limits are now generally set only by integrated pulse energy.

Previous field limits were set by the microwave hearing effect, a primarily thermoelastic mechanical \cite{MICROWAVEINDUCED1975}. Thermoelastic effects can cause damage, but primarily when \cite{Thermoacoustic2017}\cite{MECHANICAL1968}. \cite{Radiation1996}

There is a body of work on coherent control of viral inactivation using picosecond laser exposure \cite{Maximum2010}; targeting small proteins seems to require very large fields\cite{Picosecond2016b}, but breaking viral capsids by exciting long-wavelength vibrations using Raman scattering has been reported experimentally \cite{Inactivation2007}\cite{Prospects2012}\cite{Studies2014}, however these results have failed replication\cite{No2011}. The author has not investigated this mechanism in any detail. Molecular dynamics of ISRU has been considered \cite{Maximum2010}, (\cite{Vibrational2009} but without considering heat-bath damping). Interestingly, if the well-established 1 TW power density\cite{Targeted2002} for mammalian cell damage by femtosecond laser is scaled linearly with time to the nanosecond regime, a power density similar to that of \cite{Repeated2020} is found ($1\ \text{TW/m}^2 / (100\ \text{ps} / 0.3\ \text{ps}) = 3\ \text{GW/m}^2$). 

Optical centrifuges use a rotating polarization to induce a dipole moment and spin up small molecules \cite{Forced2000}\cite{Spinning2000}. Electrorotation\cite{RotatingFieldInduced1982}\cite{Electrorotation1988}\cite{Electrorotation1997}\cite{Dielectric1988} of viruses has been reported\cite{Analysis2004}\cite{New1999}\cite{comprehensive2001}, but as yet only in a non-destructive low-speed sub-synchronous regime for sensing, using optical means to detect rotation - using the virion as a kind of electrostatic induction motor. High-frequency dielectrophoresis and electrostatic trapping of viruses has also been reported\cite{Accumulation2006}, without suggestions of electrostatic damage, implying that conventional irreversible electroporation of viruses should require longer than 10 microseconds.

A rotational drag coefficient can extracted from MD simulation\cite{Evaluating2017}. We simulated a 30 nm lipid liposome using the MARTINI coarse-grained \cite{MARTINI2007}\cite{BUMPy2018}\cite{Tsjerk2020} lipid force fields with GROMACS\cite{GROMACS2005}' torque measurement feature, kindly implemented by Kutzner \cite{Keep2011}. This suggests a maximum rotation frequency on the order of $10^3$, but this value requires so many assumptions that it's hardly worth mentioning. From \cite{Spinning2008}, the upper bound of a homogenous spherical virus with 10 MPa overall tensile strength appears to be about $10^9$ Hz. \footnote{\ghfile{documents/optical_centrifuge.ipynb}} 



Almost all biological spectroscopy techniques have produced resonance-like artifacts at some point in the literature, and we suggest that even the most striking and competently-obtained absorption data should not necessarily be trusted implicitly. Trustworthiness probably depends most significantly on the quality of error analysis; many have made good use of otherwise unreliable techniques. However, particularly in a field with a high positive publication bias, it seems most productive to use the most advanced techniques, those that are foolproof by their very nature. Based only on our experience of the literature, in order of decreasing reliability for resonance measurement:

\begin{itemize}
\item Background-free photothermal optical \cite{Microwave1993a}\cite{Broadband1988}
\item Time-domain spectroscopy (many advantages cited by \cite{Time2003}, including instant, drift-free capture of all frequencies and limited instrumentation artifacts), \cite{Dielectric2004}\cite{Microwave1994}
\item Raman or other non-photothermal optical (\cite{optical1983} $\neq$ [\cite{Resonances1987} \cite{Dielectric1989}]) and also \footnote{see Taylor \cite{mechanisms1981} citing unpublished, personal communication}, but we have no understanding of how such artifacts could form
\item Frequency-domain CW VNA absorption or reflection (\cite{Microwave1982} $\neq$ [\cite{Resonances1987} \cite{Dielectric1989}]) \cite{Substitution1982} \cite{Millimeter1980}.
\end{itemize}

Although not representative of careful, well-calibrated VNA testing because of the crudity of our homebrew spectrometer, we were able to produce many candidate resonances in our data with perfectly justifiable modifications to post-processing, which raises interesting (but not novel\cite{Reanalysis2020}\cite{statistical2020}) questions regarding software development in such conditions. \footnote{One solution might be to keep all data blinded while developing the analysis software, and use synthetic data to calibrate. One objection might be that the data often inform the required calibration; we might retort that if the technique is so sensitive that the calibration procedure cannot be known beforehand, a different technique should be used.}


Positive artifacts are very common in this field, and extreme care in experimental design appears to be required to produce any meaningful result. Excellent best-practices guidelines and meta-analyses are available in \cite{Biological2016} \cite{Comprehensive2018} \cite{Funding2019}, \cite{chou1996radio} and \cite{Effects2016}.

Thermal effects appear to be most common. For instance, Manikowska\cite{Effects1985a} showed a "highly significant (P > 0.001)" result, and state that "...under our experimental conditions, effects of heat stress and local heating are unlikely". This result was promptly refuted by \cite{Cytogenetic1986}, who suggest a thermal artifact. Most critically, the localized hotspots created by microwave exposure are often almost impossible to measure. Ideally, experimental design should guarantee that even slight localized temperature rise above 3 C cannot occur \cite{Sharp1983}\cite{Effects1951}\cite{DNA2004}, or use thermally-matched shams\cite{Basic1983}.

FDTD simulation of exposure geometry and sample heating is now usually possible with free, open-source software and should be performed. A PyTorch-based FDTD library from \cite{Highly2019} was used in our study with some modification, but \cite{CUDAbased2019} and \cite{openEMS} are also excellent. 

Next appear to be electrochemical\cite{Comparative2003} effects; as noted above, metal ions from electrodes can wreak havoc. Even more obscure differences between positive and sham, such as very strange polymer surface effects observed by\cite{Effect1994a}, must be ruled out by experimental design.

If the effect exists, efficient use of the lower damping would seem to require more than one cycle of oscillation. It appears to be impossible at any input power to effectively drive a 10 GHz resonator in the lungs from outside the torso, partly because of the layer of conductive muscle that shields the chest cavity (see supplemental). Open-ended waveguides\cite{OpenEnded1982}\cite{Analysis1989}\cite{142018} or water bolus allow efficient coupling to the body. There is precedent for bronchoscopic \cite{Flexible2019}\cite{Antenna2018} or more invasive methods\cite{Implantable1980}\cite{Implantable1982}\cite{Electromagnetic1983}. 

Dispersion and Brillouin precursors do offer a means to produce a significant improvement in loss through tissue, and the required electrical waveforms appear to be synthesizeable by variable-impedance or stub transmission-line pulse shaping\cite{Coaxial1985}\cite{Arbitrarya}. 

A sawtooth might be the optimal waveform (see below). However, obtaining 10 GHz tones deep in the body is still very hard, and may require very complicated dispersive phased arrays\cite{Microwave1982a}.

%Early albanese with mm used vircator diodes. 

E. coli phage T4r was selected for biosafety reasons, and because of the large amount of experience\cite{History1995} and structural data available. T4 does not have a nucleocapsid; the charge distribution is likely much stronger than other RNA-like viruses Phi 6\footnote{Naming organisms with greek letters makes literature review somewhat more difficult, because many search engines do not support Unicode.}. Some more useful surrogates may include Pseudomonas Phi 6\cite{Selection2017}, the lipid-bearing PM2, PRD1 or PR4\cite{Lipidcontaining1979} if the outer protein capsid is dissolved\cite{Bacteriophage2002}, \cite{Dissociation1993}; or synthetic lipid liposomes. Certain phages also allow efficient concentration without ultracentrifuge, either by PEG precipitation\cite{Rapid1970} and settling, or via VIRADEL\cite{GROWTH}\cite{Chemical1982}\cite{highly1988}\cite{AdsorptionElution1989}, by altering the zeta potential of a syringe filter. 

Custom single-lipid liposome surrogates appear to be relatively easy to produce and fill with a DNA tracer by sonication and dehydration\cite{Encapsulation1982}\cite{OPTIMIZATION2017}\cite{Liposome2014}. We encountered one problem in removing unencapsulated DNA, which typically requires ultracentrifugation as precipitation of liposomes is difficult. However, \cite{novela} appear to demonstrate that ethanol extraction is satisfactory - we have not tested this. This likely does not lead to DNA precipitation because of the lack of salt. 

When phage is rapidly immersed in solution of differing osmotic strength, the active system that maintains osmotic equilibrium is overwhelmed, and the resulting pressure ruptures the capsid when it exceeds about 100 atm \cite{Osmotic2003}. By $\sigma = pr / 2t $ t = 5.7 nm\cite{Head1988} and r=60 nm, capsid strength should be about 53 MPa, less than the 100 to 300 MPa cited by \cite{Bacteriophage2004}.
%((100 atm * 60 nanometers) / (2*(5.7 nanometers))) to MPa

\subsection{Experiment methods}

The original paper by Yang et al used plaque assays and RT-PCR to measure envelope damage. Quantification of viral titer is a thoroughly solved problem, which did not justify the amount of time we spent developing the below procedures.

Although not sensitive to changes in infectivity, we used an amplification-free fluorescent dsDNA detection method, exactly as suggested by \cite{Quantification2020}, using a custom photon-counting reader and a 1:1 mix of viral solution with 1/4000 Biotium GelGreen (a safe Acridine Orange (N-alkylacridinium) based flurophore). 

This was a very convenient (stable over long periods of time, requires no host culture) and sufficiently sensitive (titers down to $10^8$ of T4 (170 kbp dsDNA)) assay both for capsid damage and, by melting the capsid with an autoclave, for general virus titer estimation, far faster and less laborious\footnote{"Fast turnaround time has always been important to me. Mistakes are unavoidable, so I always wanted an apparatus that would allow mistakes to be corrected as rapidly as possible." \cite{manipulation1998}} than high-throughput plaque assays evaluated\cite{Simple2018}\cite{Streamlining2018}. More useful enveloped viral surrogates like Phi6 are RNA and a lower sensitivity can be expected.\cite{Selected} 

This assay works because the flurophore drastically changes fluorescence intensity and lifetime when in complex with DNA\cite{SYBR2012}\cite{Characterization2010}, and the large size of the GelGreen molecule prevents diffusion through intact membranes.

While their bandwidth is narrow, very cheap 0.5" laser line thin-film filters still work great for both excitation and emission for fluorescence techniques. Blue LEDs [Cree XLamp XP-E2 Blue Starboard] are then sufficient for excitation (though high CRI white LEDs emit more 480 nm blue, green leakage is too high). Dichroic mirrors are unnecessary for this application. Use a 1 mm plastic fiber optic at right angles to decrease excitation light scattering. Beware material autofluorescence - custom Lexan cuvettes unexpectedly overwhelmed the DNA signal (final set: [emission: Tiffen Orange 16 gel\cite{lide2004crc} in series with Thorlabs FL05532, 532nm, 10 nm FWHM] [Edmund 28432, 486 nm]). A surplus Hammamatsu R4220 with HC123 current-limiting base at maximum sensitivity was used. A low-voltage silicon photomultiplier like ON Semi's C-Series SiPMs will probably be sufficient in most cases.

PMT dark counts and ambient light were removed by synchronous photon counting\cite{Measurement1966} - add when light is on, subtract when light is off. The fluorescence excitation light was 50\% square-wave modulated at 100 khz.  This cannot be done with luminescence. Further filtering could be done by fluorescence lifetime discrimination. 

Many luminescent assays require high-performance DSLR CCDs capable of several minute exposures. \cite{Image2012} use ImageJ\cite{NIH2012} to stack frames from standard CMOS video cameras. In our testing, this was ineffective with expensive, high-quality industrial CMOS cameras, but extremely effective with webcams and ELP-brand cameras. Performance of such a stacked video approached that of the photomultiplier. 

A number of other capsid damage assays were evaluated. However, these initial tests were marred by the 1 uL sample volume in our initial cuvette attempts, our use of a domestic fridge to store phage, which unexpectedly froze regularly overnight, apparently reducing phage titer, along with our accidental use of T4r+ (not an "improved" T4r, as we believed, but rather a mutation away from rapid-lysis, meaning that the lysis behavior is far slower (see \cite{Spontaneous1946}). In our testing, lysis of T4r+ only occurred on plaque plates [where culture growth and log stage infectivity is maintained for many days]). 

%Nephelometry (right-angle light scatter detection) and turbidimetry (transmission) using CdS and LtoF sensors; despite being visible to the naked eye "clean" 2 hour E. coli cultures 1.5 mL. Turbidimetry\cite{Fast2019} Impedimietric In addition, removing the necessitating 

E. coli produces the enzyme $\beta$-galactosidase to break lactose into glucose for later metabolism. A luciferase-based luminescent assay, Promega Beta-Glo, was evaluated, as described in\cite{rapid2014}. One component of the beta-glo contains a proprietary detergent designed to lyse cells\cite{BETAGLO2003} and expose the enzyme, counterproductive for phage use. Perhaps the detergent is not effective on bacteria, or the paper substrate that was used in the previous work was sufficient to bind the detergent. The detergent could perhaps be removed with BioBeads if this is desired.
 
While undoubtedly essential and very effective for its intended purpose (accurately quantifying lac operon activity), because of ambiguities in the rise and decay of the luminescence signal, short shelf life after the two components are mixed, oxygen depletion in small cuvettes, concerns regarding the detergent, and cost, Beta-Glo compared unfavorably to direct fluorescence or X-Gal for this specific application.

Positive control samples were put in a commercial pressure cooker to autoclave for either 6 or 20 minutes in an Eppendorf tube: the pressure was released after the cycle to prevent dilution. The flurophore was added after autoclaving.

We hoped to shape the nanosecond pulse into a high-voltage 10 GHz tone to drive the virus. This requires an edge faster than ~100 ps. It appears to be difficult, but not impossible\cite{Fundamental1998}to make spark gaps produce such fast edges without high-pressure\cite{kHz1995} \cite{Design2007d}\cite{Picosecond1993} or building the gap into a Marx\cite{Simple1991} pulse-forming line. 

Avalanche transistors appear to be the easiest way to produce high power sub-100 picosecond pulses. We stole an avalanche Marx design verbatim from Li (\cite{Development2016b} \cite{Design2018c}), which allows large pulses from safer lower voltage supplies (albeit requiring somewhat costly capacitors). 2N5551 transistors used to be common \cite{Avalanche1991}\cite{high1994}\cite{High1998}; our testing with Raj's FMMT417

In both of our final experimental setups, non-grounded, single-sided coplanar waveguides were used as exposure chambers, inspired by the CPWG heater of \cite{Microwave2007}. \cite{Nanosecond2006} describe a very similar arrangement. 

A programmable syringe withdrew and pumped 50 uL of fluid at a constant rate through one of the coplanar channels, via a 4 cm long silicone capillary (McMaster-Carr \#51845K65) immersed in a 1.5 mL centrifuge tube using a 26 Ga Luer-lock (\#6710A76). Forcing phage through the capillary is similar to lysis with a French pressure cell press, but no such effects were obviously seen.

 FDTD simulations showed a significant decrease in field. \cite{Microchamber2011} and \cite{Characterization2012} provide a good review of pulse exposure cells and a design which is much better than ours and should be used instead. \cite{Electromagnetic1989} \footnote{The utility wcalc was used extensively}.






%
%For in-vitro research, it seems to me that perhaps this field is running into the issues Feynman's rat-races \cite{Cargo}. There are a large number of very tricky variables. 


The fluor-meter printed a constant stream of results, all within the random error between capture windows $\sigma=0.5$ kcounts. Only after most of the data was taken was it realized how dumb this was.


Data rounded to the nearest thousand photon counts per 10 seconds. %thoroughly characterized. 

If a fast laser is available, Auston switches appear to be equally effective.


Repeatability between test conditions was not terrific. The source of the \ntilde 5000 count variation between some equivalent samples, and between "blank" and "sham", is not known.

Background ('blank', in notes) readings measure the amount of DNA present in the phage culture tube before any treatment. 'sham' indicates a blank sample that was injected through the capillary identically to the treatment samples, except with either the VCO or Marx pulser disabled.

Ishii and Yanagida \cite{two1977}, (Qin \cite{Structure2010}), find at a pH of 10.6, about 0.3x inactivation after a few minutes of incubation at 37 C, and almost complete loss after a few hours. The capsid is weakened greatly by the base, relying on the "clamp proteins" hoc and soc. It was thought that, if the phage capsid was weakened, it might be easier to destroy electrically. A sodium carbonate-bicarbonate buffer was used to raise the pH. Run 9 and Weakened show these results. 

Note the difference in base inactivation timeline between Run 9 and Weakened. In one case, base inactivation is found exactly as predicted; little inactivation is found in the other. We cannot explain this result; it is possible that the buffer degraded somehow between experiments.

In no case did any field-exposed sample\wikinote{verify in notes} approach the 30 kcounts of the autoclave; therefore, we do not consider this a positive result.

Stock culture had titer $3\times 10^9$ T4r.

The electrodes were only 1 oz copper thick (35 micrometers), but the engraved channel is significantly deeper. 
\cite{Nanosecond2006} and simulations seem to suggest that this is not a serious problem, but it is possible that the field in much of the channel was much lower than predicted.

Results aggregated over the course of several months.

In all cases, a thermal effect was made implausible; the total energy in the pulse train precluded heating above the capsid melting temperature\cite{Effects1951}.

Only tubes 4 onwards are shown from Run 8 data because the capillary purging protocol was changed. In the final iteration, the autosampler was disconnected. The capillary was flushed with water; then the leuer lock was disconnected, air withdrawn into the syringe, and then purged through the capillary and fluidic channel.

In an early iteration of the experiment, the firmware and data analysis scripts randomly generated blinded sham and autoclave assignments and automatically treated the samples accordingly. It also printed blinded templates for pipetting. This was done partly in the hopes that massively oversampling the microwave absorption spectrometer across the range of physical tolerances would highlight any resonances.\cite{first2000} 

However, as spectrometry was abandoned, this non-deterministic property caused considerable issues while troubleshooting the fluorescence detection system, and \st{expedience} \footnote{laziness} prompted us to disable this system at run 9, defying best practices on blinded trials.

A thin layer of silicone conformal coating was applied to the 0.2 mm waveguide to avoid the copper ion artifact, but this could not be done on the 0.03 mm waveguide.


For the Eppenwolf, the lower bound is the lowest voltage measured on the diode detector through the sweep (at 12 GHz), (raw unconverted voltage across frequency can be seen at\ghfile{/firmware/eppenwolf/runs/phage_experiment_12/screenshot_192.168.0.38_2020-11-26_13:38:24.png} - it is not recorded which detector this was seen at) multiplied by the FDTD results for the coplanar waveguide voltage to channel field scaling factor. The upper bound is the maximum voltage seen over the frequency sweep multiplied by the same FDTD factor.





%For the pulser, the lower bound is 

\begin{tcolorbox}
	
	The following is probably self-evident to all good scientists; unfortunately, we only learned it during this project, so it seems worthwhile to make explicit.\\
	
	As noted later, this field reached a reasonable level of quality many decades ago, and many of the most troubling issues had already been encountered and addressed satisfactorily in published experimental designs\cite{Biological1984}. 
	
	In some cases, these are thoroughly characterized and described; field intensities are mapped, thermal profiles taken; even the dielectric properties of the media must be considered. But it seems that some workers continue to develop their own test arrangements; yet the studies are sometimes not so different as to require such customization. \\
	
	This demands that the reader perform analysis anew to verify details of how the study was conducted, rather than simply evaluate its results; a waste of time on all counts.\\
	
	By using yet another exposure cell, introducing a new set of unknown artifacts, we hamper this field; and so we encourage anyone who may wish to persue similar results to not use this arrangement, but instead one of the existing, well-characterized exposure systems mentioned below or in the literature.
	
	In this case, we were limited by budget - primarily the power amplifiers we have access to.
	
	%\blfootnote{It occurs to me that perhaps a set of standardized test cells and procedures would be helpful; along with a requirement that any papers on the topic either use such a cell, or, if their study requires some exceptional treatment, to follow some established step-by-step procedure to add a new cell to the standard.}
\end{tcolorbox}




We are particularly excited by the prospect of using picosecond time-domain measurements not only to perform steady state dielectric measurements or monitor the input pulse, as is now common practice, but to resolve the internal effects of a single high-voltage pulse on the virus in real time. Scan-converter based transient digitizers, such as the Tektronix SCD-5000 series, appear to be the lowest-cost method of obtaining the >0.5 THz sampling bandwidths required, when the tube bandwidth is deconvolved.

%When high voltage pulses are available, miniaturized \ntilde 2 mm internal diameter continuously pumped scan converter tubes like Fecher\cite{Production1949} micro-oscillographs using in-vacuum CMOS image sensors for direct-electron-detection appear to be an easy way to digitize 20+ GHz transients. The main bandwidth limitation appears to be the length of the Y deflection plate and the sensitivity of the recording medium.

%\begin{figure}[H]
%	%	\makebox[\textwidth][c]{
%	\centering
%	\subfloat[Direct electron beam detection]{
%		\includegraphics[width=0.3\textwidth]{direct_electron_beam}
%	}
%	\subfloat[Damage to the sensor, believed to be due to. ]{
%		\includegraphics[width=0.3\textwidth]{e_beam_damage_2}
%	}
%	\subfloat[]{
%		\includegraphics[width=0.3\textwidth]{e_beam_damage}
%	}
%	\caption{Second.}
%	\hfill
%	
%\end{figure}

Before the HMC732 was used, a misguided effort was spent in trying to develop an inexpensive wideband tunable oscillator for spectrometry. This will be chronicled in the GitHub repository.

\end{multicols}

\section{Acknowledgements}

Funded in part by the author's company, SafeSump Inc., and Ideal Consulting Ltd., and the author's parents. The author was at YorkU for the duration of this project.

Special thanks to my parents, Doris and Brian, without whom this work would have been accomplished by someone else, and whose extraordinary support made this paper possible.

Thanks to a group with Chengxiang Li for the Marx generator design we use in this paper. It is an excellent, very high performance design! Professor Shahramian's video blog informed this work, as did Dr. Dainis' on microbiology. Thanks to the legend of dispersion himself, Professor Oughstun at Vermont, Prof. Sarris at U of Toronto, both for excellent information on optimal dispersive pulses; Professor Voinigescu at UofT for a tip on experiment engineering; Sandra and colleagues at York library for spending an hour looking for the Gray paper; Prof. Griffiths and Dr. Brovko at UGuelph and Dr. Anany at the GRDC for useful tips on the Beta-Glo assay; Professor Z-G Wang at Caltech for a note on viral electrostatics; Dr. James Hawthorne for the Autoleveller utility; Yuriy at Indiana; Barfuss for reviewing an early draft; Greenland Engineering for patience; and lots of friends - a few at York and many beyond. 

This paper would have been absolutely impossible without the support of thousands of open-source software packages, ranging from monolithic state-supported tools like Paraview to poorly-compensated but equally invaluable scripts like zonca/qucs\cite{Modeling2010}. A special thanks to the dedicated authors that develop and maintain these tools, and especially to those who choose not to put them behind a paywall.

Hamming was wrong; despite the name above the abstract, this paper is of course entirely the result of circumstance, luck, and awesome friends. 

Special thanks to Alexandra Elbakyan and Aaron Swartz.

\rule{\linewidth}{0.2pt}



\subfile{brillouin_pulse_math}

\end{document}