%!TeX root = summary
\documentclass[paper.tex]{subfiles}
\begin{document}



\section{Terse? I can be terse. Once, in flight school, I was laconic.}

There might be some large-scale structural difference between viruses and all host cellular structures that might used as leverage to inactivate the virus by some physical means. Such an inactivation "window" between host cell and virion damage has been found in certain far wavelengths of UV\cite{Germicidal2017} - but also \cite{Use1987a}\cite{Can1993}; in fact 

More specifically, the idea of driving the virus as a harmonic oscillator has been considered very little \cite{Comment2004}, perhaps first because in almost all cases structures in tissue are so over-damped by the surrounding solvent\cite{Vibrational2002}\cite{Biological2002}\cite{Biophysics2000}\cite{Viscous2000} as to have no non-zero resonant frequency at all \cite{dielectric1996}\cite{gabriel1996compilation}\wikinote{not in citation}\wikinote{discuss}; and second, because the charge distribution on biological cells is too. On the other hand, the possibility of a slightly underdamped virion does not appear to be entirely unexpected; classical viscous Stokes drag predicts such behaviour\cite{nature1986}, and large proteins have been found with similar properties\cite{Microwave1994}; and while MHz transverse resonances in microtubules are ruled out\cite{Viscous2000}, the same arguments suggest that GHz longitudinal modes may exist. 

Whether the small difference in driven amplitude between, for example, a $Q\approx0.5$ host structure normal mode and a $Q\approx6$ virion mode can be of any clinical value is unknown to us.

Originally these were borne from work by Frolich on the formation of large-scale bose-like QM condensates, then to many more subtle lines of inquiry; but, to our knowledge, no non-classical effects have ever been substantiated; in fact, consensus appears to be that, not only are oscillatory effects very unlikely, no relevant non-diffusion-like non-thermal effects have ever been substantiated. 


Pulses beyond 1 ns in length \cite{Effect2008} 



In our cursory testing, Bacteriophage T4 did not appear to be lysed by microsecond pulses between 6 and 12 GHz at about 70 V/m, nor did nanosecond pulses up to 1 MV/m appear to have any effect. 

Capsid and envelope breakage at 100 pulses, 500 ns at 3 MV/m, has been previously observed\cite{Inactivation1990} but the RNA damage suggests that this is an electrochemical artifact \cite{Formation1996} (and \cite{Microwave1987}). Long pulses electroporate enveloped viruses similarly to host cells\cite{AC2017} but do not necessarily affect T4\cite{Manipulation2013}. 

Pakhomov \cite{Comparative} find no non-thermal effects on functioning frog heart pacemaker cells at 0.9 MV/m (although the field inside the tissue was not measured and could have been lower). 

Recent data by de Seze following continuous exposure to nanosecond fields at an external field of 3 MV/m \cite{Repeated2020} notes no acute effects, but 

Almost all biological spectroscopy techniques have produced resonance-like artifacts at some point in the literature, and even the most striking and competently-obtained absorption data should not necessarily be trusted implicitly. In order of highest to lowest apparent reliability for resonance measurement:

Background-free photothermal optical \cite{Microwave1993a}\cite{Broadband1988} >\\ time-domain spectroscopy\cite{Time2003}\cite{Dielectric2004}\cite{Microwave1994} >\\ Raman or other non-photothermal optical (\cite{optical1983} $\neq$ [\cite{Resonances1987} \cite{Dielectric1989}]) and also \footnote{see Taylor \cite{mechanisms1981} citing unpublished, personal communication} (but we have no understanding of how such artifacts could form) >\\ VNA CW absorption or reflection ( \cite{Microwave1982} $\neq$ [\cite{Resonances1987} \cite{Dielectric1989}]) \footnote{Anywhere we liked}; but trustworthiness probably depends most significantly on the quality of error analysis; many have made good use of otherwise unreliable techniques \cite{Substitution1982} \cite{Millimeter1980}.

Positive artifacts are very common in this field, and extreme care in experimental design appears to be required to produce any meaningful result. Excellent best-practices guidelines and meta-analyses are available in \cite{Biological2016} \cite{Comprehensive2018} \cite{Funding2019}, \cite{chou1996radio} and \cite{Effects2016}.

Thermal effects appear to be most common. Most critically, the localized hotspots created by microwave exposure are often almost impossible to measure. Ideally, experimental design should guarantee that even slight temperature rise, more than about 3 C, cannot occur \cite{DNA2004}, or use thermally-matched shams. 

FDTD simulation of exposure geometry and sample heating is now usually possible with free, open-source software\footnote{PyTorch FDTD library from \cite{Highly2019} was used in our study with some modification, but \cite{CUDAbased2019} and \cite{openEMS} are also excellent} and should be performed. Metal ions from electrodes, UV exposure, and polymer surface effects must be ruled out by experimental design. The only artifact-free appear to be 1. Remote exposure, actively cooled OR pulsed exposure with sufficiently short pulses to rule out heating, with. 

Coherent control targeting small proteins seems to require very large fields. Coherent control of large-scale structural modes using ; however these results have failed replication.

If the effect exists, it appears to be impossible at any input power to externally drive a 10 GHz resonator in the lungs because of the layer of conductive muscle that shields the chest cavity. There is precedent for bronchoscopic \cite{Flexible2019}\cite{Antenna2018} or more invasive methods. Dispersion and Brillouin precursors do offer a significant improvement in loss through tissue, and the required electrical waveforms appear to be synthesizeable by variable-impedance or stub transmission-line pulse shaping\cite{Arbitrarya} . A sawtooth might be the optimal waveform. However, obtaining 10 GHz tones deep in the body is still very hard.

Although not sensitive to infectivity, as suggested by \cite{Quantification2020}, amplification-free fluorescence dsDNA detection using photon-counting plate readers using a 1:1 mix of viral solution with 1/4000 Biotium GelGreen is a very convenient (stable over long periods of time, requires no host culture) and sufficiently sensitive (titers down to $10^9$ of T4 (170 kbp dsDNA)) assay both for capsid damage and, by melting the capsid with an autoclave, for general virus titer estimation, far faster than even high-throughput plaque assays. More useful enveloped viral surrogates like Pseudomonas phage Phi 6 are RNA and a lower sensitivity can be expected.

While their bandwidth is narrow, very cheap 0.5" laser line thin-film filters still work great for both excitation and emission for fluorescence techniques. Blue LEDs like are then sufficient for excitation (though high CRI white LEDs emit more blue, green leakage is too high). Dichroic mirrors are unnecessary. Use a plastic fiber optic at right angles to decrease excitation light scattering. Beware material autofluorescence - custom Lexan cuvettes unexpectedly overwhelmed the DNA signal. (final set: [ in series with Thorlabs FL05532, 532nm, 10 nm FWHM] [Edmund 28432, 486 nm])

In both of our final experimental setups, non-grounded, single-sided coplanar waveguides were used as exposure chambers. A programmable syringe pumped 50 uL of viral fluid through one of the coplanar channels. Simulations showed a significant decrease in field. \cite{Microchamber2011} and \cite{Characterization2012} provide a good review of pulse exposure cells and a design which is much better than ours and should be used instead. Such cells should be used rather than our crude design. \footnote{the utility wcalc was used extensively}.

Electrorotation of viruses has been reported\cite{Analysis2004}\cite{New1999}\cite{comprehensive2001}, but as yet only in a non-destructive low-speed sub-synchronous regime for sensing with optical. Dielectrophoresis and electrostatic trapping of viruses has also been reported.

Data on the charge distribution of the virion can be obtained by and 2D-PAGE relative protein composition or perhaps directly via extraction from CryoEM maps.

Avalanche transistors are sufficient and perhaps the easiest way to produce high power sub 100 ps pulses. We stole an avalanche Marx design verbatim from Li (\cite{Development2016b} \cite{Design2018c}), which allows large pulses from lower voltage supplies (albeit requiring somewhat costly capacitors); but  If a fast laser is available, Auston switches appear to be equally effective. It is difficult to make spark gaps run than 200 ps without the Marx overvoltage.

We are particularly excited by the prospect of using picosecond time-domain measurements not only to perform steady state dielectric measurements or monitor the input pulse, as is now common practice, but to resolve the internal effects of a single high-voltage pulse on the virus in real time - a technique which has not, to our knowledge, previously been demonstrated. Scan-converter based transient digitizers, such as the Tektronix SCD-5000 series, appear to be the lowest-cost method of obtaining such THz sampling bandwidths. When high voltage pulses are available, miniaturized \ntilde 2 mm internal diameter pumped scan converter tubes like Fecher micro-oscillographs using direct-electron-detection in-vacuum CMOS image sensors appear to be an easy way to digitize 20+ GHz transients. The main bandwidth limitation appears to be the length of the Y deflection plate and the sensitivity of the recording medium.

Using Eppendorf tubes as. Impedimetric 


\end{document}