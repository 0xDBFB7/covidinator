%!TeX root = pure_electroporation
\documentclass[fleqn,10pt]{paper}

\usepackage[left=2cm,right=2cm,
top=0.5cm,
bottom=1.5cm,%
headheight=11pt,%
letterpaper]{geometry}



\usepackage[left=2cm,right=2cm,
			top=1.25cm,
			bottom=2.25cm,%
			headheight=11pt,%
			letterpaper]{geometry}
			
\frenchspacing			

\nonstopmode




\usepackage{lmodern}
\usepackage[T1]{fontenc}
\usepackage[utf8]{inputenc}



\usepackage{noweb}

\usepackage{multicol}
\usepackage{fancyhdr}
\usepackage{blindtext,graphicx}
\usepackage[absolute]{textpos}
%\usepackage[parfill]{parskip}
\usepackage{parskip}
\setlength{\parskip}{\baselineskip}

\usepackage[colorlinks=true,citecolor=brown]{hyperref}
\usepackage{gensymb}
\usepackage{csquotes}
\usepackage{amsmath}
\usepackage{fontawesome}
\usepackage{orcidlink}
\usepackage{standalone}
\usepackage{pdfpages}
\usepackage{subfiles}
\usepackage{svg}
\usepackage{sidecap}
\usepackage{float}
\usepackage{amssymb}
\usepackage{textcomp}
\usepackage{lettrine}
%\usepackage[T1]{fontenc}

\usepackage{soul}


%\usepackage{draftwatermark}
%\SetWatermarkText{DRAFT}
%\SetWatermarkScale{0.25}

\usepackage{booktabs,caption}
\usepackage[flushleft]{threeparttable}

%\usepackage{biblatex}
\usepackage[backend=bibtex8, sorting=none, style=chem-angew]{biblatex}

\let\cite\footfullcite

%\let\cite\footcite

\addbibresource{processed.bib}
%biblatex has a zoterordfxml
% might avoid the need for python bibtex_collections.py



\usepackage{etoolbox}
\AtBeginEnvironment{quote}{\small}




\usepackage{pifont}
\newcommand{\cmark}{\ding{51}}%
\newcommand{\xmark}{\ding{55}}%


\newcommand{\citationneeded}[1][]{\textsuperscript{[\color{blue}{\it \bf{citation needed}#1}]}}
\newcommand{\dubiousdiscuss}[1][]{\textsuperscript{\color{blue} [{\it \bf{dubious-discuss}}]} }

\newcommand{\light}[1]{\textcolor{gray}{#1}}

%
%
\usepackage{titlesec}
%
%% custom section


\titleformat{\section}
{\normalfont\LARGE\bfseries}{\thesection}{1em}{}
%\titleformat{\section}
%{\normalfont\LARGE\bfseries\PRLsep}
%{{{{\itshape \thesection\hskip 9pt\textpipe\hskip 9pt}}}}{0pt}{}
%
%% custom section
%\titleformat{\subsection}
%{\normalfont\Large\bfseries\PRLsep}
%{{{{\itshape \thesection\hskip 9pt\textpipe\hskip 9pt}}}}{0pt}{}
%
%
%


\newcommand{\Wsqm}{$\text{ W/m}^2$}

\newcommand{\ghfile}[1]{\href{https://github.com/0xDBFB7/covidinator/tree/master/#1}{\faGithub/\url{#1} }}

%\newcommand{\supercite}[1]{}
%\newcommand{\supercollect}[1]{}


\newlength{\PRLlen}
\newcommand*\PRLsep[1]{{\itshape \Large\settowidth{\PRLlen}{#1}\advance\PRLlen by -\textwidth\divide\PRLlen by -2\noindent\makebox[\the\PRLlen]{\resizebox{\the\PRLlen}{1pt}{$\blacktriangleleft$}}\raisebox{-.5ex}{#1}\makebox[\the\PRLlen]{\resizebox{\the\PRLlen}{1pt}{$\blacktriangleright$}}\bigskip}}


\renewcommand{\thefootnote}{\textcolor{gray}{\arabic{footnote}}}


\usepackage{graphicx}
\graphicspath{ {../media/} 
				{../firmware/eppenwolf/runs/sic_susceptor/} 
			}

\usepackage{tcolorbox}
\newtcolorbox{protocol}{colback=yellow!5!white,colframe=yellow!75!black}
\newtcolorbox{equipment}{colback=orange!5!white,colframe=orange!75!black}
\newtcolorbox{autem}{colback=red!5!white,colframe=red!75!black}
\newtcolorbox{toolchain}{colback=blue!5!white,colframe=blue!40!black!40}
\newtcolorbox{sidenote}{colback=cyan!5!white,colframe=blue!40!black!40}
%https://tex.stackexchange.com/questions/66154/how-to-construct-a-coloured-box-with-rounded-corners

%\usepackage[sfdefault,light]{roboto}

\setlength{\TPHorizModule}{1cm}
\setlength{\TPVertModule}{1cm}





%%%%********************************************************************
% fancy quotes
\definecolor{quotemark}{gray}{0.7}
\makeatletter
\def\fquote{%
	\@ifnextchar[{\fquote@i}{\fquote@i[]}%]
}%
\def\fquote@i[#1]{%
	\def\tempa{#1}%
	\@ifnextchar[{\fquote@ii}{\fquote@ii[]}%]
}%
\def\fquote@ii[#1]{%
	\def\tempb{#1}%
	\@ifnextchar[{\fquote@iii}{\fquote@iii[]}%]
}%
\def\fquote@iii[#1]{%
	\def\tempc{#1}%
	\vspace{1em}%
	\noindent%
	\begin{list}{}{%
			\setlength{\leftmargin}{0.1\textwidth}%
			\setlength{\rightmargin}{0.1\textwidth}%
		}%
		\item[]%
		\begin{picture}(0,0)%
		\put(-15,-5){\makebox(0,0){\scalebox{3}{\textcolor{quotemark}{``}}}}%
		\end{picture}%
		\begingroup\itshape}%
	%%%%********************************************************************
	\def\endfquote{%
		\endgroup\par%
		\makebox[0pt][l]{%
			\hspace{0.8\textwidth}%
			\begin{picture}(0,0)(0,0)%
			\put(15,15){\makebox(0,0){%
					\scalebox{3}{\color{quotemark}''}}}%
			\end{picture}}%
		\ifx\tempa\empty%
		\else%
		\ifx\tempc\empty%
		\hfill\rule{100pt}{0.5pt}\\\mbox{}\hfill\tempa,\ \emph{\tempb}%
		\else%
		\hfill\rule{100pt}{0.5pt}\\\mbox{}\hfill\tempa,\ \emph{\tempb},\ \tempc%
		\fi\fi\par%
		\vspace{0.5em}%
	\end{list}%
}%
\makeatother







%%%%********************************************************************
%title link to doi
\newbibmacro{string+doiurlisbn}[1]{%
	\iffieldundef{doi}{%
		\iffieldundef{url}{%
			\iffieldundef{isbn}{%
				\iffieldundef{issn}{%
					#1%
				}{%
					\href{http://books.google.com/books?vid=ISSN\thefield{issn}}{#1}%
				}%
			}{%
				\href{http://books.google.com/books?vid=ISBN\thefield{isbn}}{#1}%
			}%
		}{%
			\href{\thefield{url}}{#1}%
		}%
	}{%
		\href{https://doi.org/\thefield{doi}}{#1}%
	}%
}

\DeclareFieldFormat{journaltitle}{\usebibmacro{string+doiurlisbn}{\mkbibemph{#1}}}

\begin{document}




\title{Selective electroporation of the viral envelope}
\author{\footnotesize{Daniel Correia\orcidlink{0000-0002-9353-0216}}\footnote{therobotist@gmail.com, Ontario, Canada.}}
\date{\small{March 21 2021}}

\flushbottom 
%%\maketitle
%\begingroup
%\let\center\flushleft
%\let\endcenter\endflushleft
\maketitle
%\endgroup



\thispagestyle{empty}

\renewcommand{\abstractname}{Summary}    % clear the title

\begin{abstract}
	\noindent The phenomenon of electroporation occurs when extreme applied electric fields in the MV/m range drive transmembrane potentials, leading to pores through membranes; this is commonly harnessed for transfection into the cell in the laboratory, and has recently been shown to be safe and effective in the clinic for ablating tumors.\\
	
	\noindent As pulse duration is reduced from the microsecond to nanosecond regime, effects targeting smaller intracellular structures are expected and observed, and a degree of tunability arises. Indeed, pulses shorter than 10 ns have been observed to porate the nuclear envelope and organelle membranes while leaving the plasma membrane intact. \\
	
	fortuitous Dielectric properties apparently due to the 
	
	\noindent Recently-developed gigawatt pulse systems may offer a possible means of noninvasively propagating such pulses into deep tissues, such as within lungs. 
\end{abstract}

\lettrine{A}{s} an addendum to a technical report \cite{notes2021} by the author presenting experimental data on protein capsid adjacent technique; 

In a continued effort to which make ultracrepidarian physicist such as ourselves feel useful indirectly biological 

While both mechanisms hinge on time constants, those discussed in the previous paper referred to the viscous loss of energy into the surrounding medium and the normal modes of the viral envelope; here the time constants relate to capacitive charging, and Smoluchowski processes. The relevant damage quantities were mechanical displacement and membrane stress, whereas here we consider transmembrane voltage and pore diameter. 

The breakdown of lipid membranes has proved to be an From initial studies of crude membrane breakdown effects \cite{Reversible1979}

"...hit them [cells] with an electric field strong enough to knock over a horse, and they do enough things to justify international meetings, to fill a sizable book, and to lead one to speak of an entirely new technology for cell manipulation."

It is, in general, something of a singular outlier in that it is one of very few biological mechanisms relevant at practical electric field intensities. It also something like integration, which is

Electroinsertion of proteins into membranes\cite{Clinical1996}; to provoke a local immune response near diseased tissues, 



At the risk of self-plagarism, we will restate precedent 

The same electrostatic effects on lipid bilayers have been investigated for decades. Many extremely fascinating hydrodynamic and parametric 

The idea of inactivating viruses via electric fields is not new. Very recently, Osorio \cite{Receptor2021} have reported on a susceptibility of SARS-CoV-2's spike proteins to intense electric fields; however, in general, techniques targeting proteins appear to require truly extreme; in their study, fields of up to 2.9 GV/m were required. Dissocation of capsid proteins has been predicted by Man \cite{Picosecond2016b} at 2 GV/m. Similar techniques Tsen \cite{Studies2014}, but these have not been replicated \cite{No2011}. These regimes of intensity may possibly be accessible via ultrashort optical techniques.

on the other hand, the idea of shocking a virus is so basic that it can scarcely be believed 


Electroporation\cite{Electroporation1988}, now a common laboratory procedure, occurs when an enormous electric field causes ions in solution to diffuse across a cell membrane, capacitively charging it, provoking a change in lipid bilayer conformation\cite{Membrane2016}, reversibly rendering the membrane permeable. This is one among many other, more subtle mechanisms; the minutia of poration can become extraordinarily complex in certain circumstances and are beyond the author's current understanding\cite{Theoretical2007}. Hydrodynamic

Therapies based on irreversible electroporation\cite{Nonthermal2013}\cite{Lipid2017} have recently been used in the clinic\cite{Irreversible2013} for tumor ablation. There is much discussion on electroporation as an alternative to viral vectors, but we have found surprisingly little discussion on clinical treatment of viral infection with these conventional electroporation therapies.

However, long pulses at $\approx10$ MV/m appear to electroporate large enveloped viruses, similarly to host cells\cite{AC2017}

There is a single report of the electroporation of HIV in the literature. This one of very few reports involving the electroporation of viruses, reporting the ~95\% inactivation of HIV  using a standard commercial Bio-Rad electroporation system at 2.3 MV/m. However, the authors appear to commit such an egregious breach of human research ethics that we cannot cite them here.

 which we believe might hint that a regime of intensity, duration, and pulse count may exist to target the viral envelope. 
 
Despite considering the somewhat harder problem of electroporating while inside the cell, rather than in extracellular fluid, \cite{Electroporation2013} report that 250 nm liposomes should be porated without inducing damage to the host cell, and modelling shows that 100 nm liposomes may be porated by 4 ns pulses never exceeding 20 MV/m. 

The same field intensities do not appear to damage phages with protein capsids\cite{Manipulation2013}, in line with our negative results on T4.

Non-resonant capsid and envelope breakage with 120 pulses, 500 ns each, at 3 MV/m, has been previously observed\cite{Inactivation1990}, but the RNA damage suggests that this is an electrochemical artifact \cite{Formation1996} (c.f. \cite{Microwave1987}). 

The trapping experiments we use were conducted at 

As in the previous paper, we must stress that meaningful bioelectrics research hinges very strongly on experimental design and methodology, and very few results can be trusted implicitly. It appears to be very difficult to unambiguously determine cause and effect on a system as complex as a cell; especially when electrodes are in contact with solution and may release an energy-dependent dose of metal ions.




As in the previous paper, bulk dielectric properties of tissues are given by Gabriel. 


Pulse count and inter-pulse timing is another free variable. Joshi et al \cite{Selfconsistent2001} model that a carefully timed second pulse can have greater effects than a single pulse, in agreement with experiment on E. coli.

This seems to hint that electroporation occurs at the same order of magnitude field as host cells.

Of course, unlike localized tumors, viral infections are systemic, affecting many of the most favoured organs. However, either blood 



Note that, despite clear evidence in nanosecond review, microsecond in vivo and clinical treatment, nanosecond in vivo conduction, and, The NATO review of vivo data regarding non-contact, "electrodeless", even above the thresholds that might otherwise be expected to cause electroporation. has not yet found any evidence demonstrating electroporation via pulsed RF. de Seze, 

This property is 
\cite{Nanosecond2006b} 



This seems too simple.

Viruses have extraordinary mechanical properties\footnote{Why is this? What selection pressures drive the selection of this pressure?}, so it seems reasonable that their dielectric properties could differ substantially from that of other structures. 

In the extreme case of the bacteriophage, the genome is packaged so tightly into the protein capsid that the core is often described as a glassy crystal of DNA\cite{Conformation2007}. ssRNA viruses like coronaviridae are perhaps somewhat less. 

On the other hand, the cellular "structome" also contains countless lipid vesicles that might plausibly be damaged; small pockets of lipid in the endoplasmic reticula, Much like in drug discovery, we do not believe we can meaningfully predict all possible off-target effects.



At the risk of spending more time producing underpowered "scientifically useless" results, 

\subsection*{Basic scaling laws}

400x difference, reflected in the numerical steady state.

However, not field-time product [analytic, ]. 

Despite the two having nearly identical time constants, the course of the transmembrane voltage is clearly different.
[transmembrane plot here]

[gaussian pulse results, 1/2.7]. 

The most obvious difference is scale. Viruses are minute. The classic Schwan equation implies that the transmembrane voltage is proportional to 

There are two characteristic time constants; a low and a high-frequency one.

Some extremely simplified scaling laws for membrane charging effects due to short (sub Tc) pulses are given by Schoenbach \cite{Bioelectric2007}. This is based primarily on an equivalent circuit model, we largely neglect the electrodynamics that apply with short pulses and also the conductivity change due to pore formation. Electroporation is a highly nonlinear effect\cite{Letter1974}, and this .

The membrane charging time constant from a single-shell model \cite{Ultrashort2004} - simplified for the low-density case. As the fraction, and papers citing this usually mention that this may not be representative of charging time in tissue.

$$\tau_c = \left(1 \frac{\rho_{medium}}{2} + \rho_{internal}\right) C_m (D/2)$$

where $\rho_{medium}$ and $\rho_{internal}$ are the resistivities of the medium and internal fluid (the cytoplasm for cells, and assumed to be the 'core' for viruses). Note that this time constant is unrelated to the relaxation time constant discussed in the previous paper, which was purely mechanical. \cite{Bioelectric2007} eq. 15, 

$$\Delta V_M = \frac{3}{2}\frac{E\tau D}{2 \tau_c }$$

substituting,

$$\Delta V_M = \frac{3}{2}\frac{E \tau}{C_m ((\rho_1/2) + \rho_2)}$$

that is, all else being equal, the induced transmembrane voltage appears to be invariant with particle diameter - mildly contrary to Rjif's more detailed analysis.

The viral envelope capacitance per unit area, $C_m$, can be determined simply by parallel-plate (Ermolina \cite{Study2001}); $C_m = (k\epsilon_0) / d$, where d is the membrane thickness. The cell and nucleus membrane have similar capacitance, but greatly differ in conductivity. 

There may be an existing transmembrane potential that must be added - this is neglected in this case.

The dielectric properties of viruses have been determined by the fascinating technique of electrorotation, dielectrophoresis, and dielectric spectroscopy. To prevent fluid heating, these are often conducted in low osmotic strength media. Enveloped viruses are permeable to small ions, and so some results may not necessarily representative of conductivities in physiological conditions; \cite{Assessment}.

The internal core permittivity of viruses is often referred to as being about 5 to 3 (Brackley \cite{Electrostatic2020}, SARS-CoV-2; and Schnelle \cite{Trapping1996}, Sendai and Influenza A) - however, Gimsa \cite{New1999}, on influenza report that the internal $\epsilon_r$ is probably closer to 30. Hughes on HSV-1 originally suggest 80\cite{Manipulation1998}, then refine to 30\cite{Dielectrophoretic2001}. We have not investigated this discrepancy. The internal osmotic strength of some viruses depends direclt on the growth medium's osmotic strength\cite{Osmotic2003}, so results with different media may not be comparable. Aggregation of viruses may also be a relevant artifact. 

Via electrorotation, it is usually found that the viral shell has a permittivity of about 60 to 80, similar to the nuclear envelope (60 to 100) but massively unlike the cell membrane ($\epsilon_r\approx 10$). This is very remarkable, since the lipid bilayer itself is somewhat similar in composition to the host cell, and the proteins that are stuck into it permittivity is generally assumed to be only about 4. Envelope, spike and membrane proteins may perhaps account for the discrepancy?!?

On the other hand, \cite{Electrostatic2020a} use a membrane value of 4.

want to more accurately represent the - most importantly, the core permittivity seems like it should alter the reflection significantly at high frequencies, but isn't usually treated 

Influenza has a membrane thickness of about 4 nm (Harris \cite{Influenza2006}). However, Gimsa and Hughes use a shell thickness of 18 and 15 nm, respectively, to account for the membrane proteins and hemaglutinase. The $C_{envelope}$ may therefore be expected to fall within the range $2.9 \text{ to } 17.7 \text{ uF/cm}^2$.

Basic dry-weight measurements seem to suggest that influenza contains about 60\% water by weight \cite{lauffer1944biophysical}. Has diffusion coefficients and maintains a Donnan ion size equilibrium similar to that of lipid bilayers \cite{Effect2015b}.

% should this be noweb python? python isn't really unit aware though
% maybe sympy or something?
% ooh we could have the supplemental appendix written right inline with noweb, 
% also python could print the equation back to latex

% Viral envelope capacitance per area:
% >>> (80 * electricConstant) / 4 nanometers -> 17.7 uF/cm^2
% >>> (60 * electricConstant) / 18 nanometers -> 2.9514 uF/cm^2
% Actual envelope capacitance:
% >>> 17.7 uF/cm^2 * 4 * pi * (50 nm)^2 -> 5.56 fF
% >>> 2.95 uF/cm^2 * 4 * pi * (50 nm)^2 -> 0.92 fF
% Actual envelope resistance:
% 



Critically, Gimsa and Schnelle data, both on influenza, seems to suggest a core conductivity of 0.8 to 0.1 milliSiemens/m (1000 to 10000 $\Omega \cdot m$), using a medium conductivity of 74 mS/m and 3 mS/m, respectively. This is far less conductive than cytoplasm or nucleoplasm, each on the order of 1 S/m (1 $\Omega \cdot m$) \cite{Study2001}. 

% if this is correct, scaling linearly:
% like 60 % water in core  * (1 S/m / 0.074) would get us to 1 mS/m.

% absolute pessimistic, using 0.8 mS/m and Schnelle's medium conductivity, would get us 0.008 * 0.6 * (1/0.003) = 1.6 S/m

However, Hughes suggests 30 mS/m for HSV-1, a significant discrepancy. Unlike influenza, HSV-1 has a complete protein capsid rather than a helical nucleocapsid. Protein capsids are typically regarded as monolithic and impermeable, whereas enveloped viruses are permeable, and therefore the ionic strength of the medium may be very important. No other sources of data on permittivity ion mobility in enveloped viruses were found.  \footnote{Special thanks to Professor Jan Gimsa for replying to my email!}

The viral envelope appears to be slightly less conductive than the cell membrane (0.1 $\micro$S/m \cite{New1999} vs 10 $\micro$S/m \cite{Study2001}), and far less conductive than the nuclear envelope (0.1 $\text{m}$S/m \cite{Study2001}).
%This ratio is the relevant quantity to find the steady state transmembrane voltage for long pulses (before pore conduction becomes relevant). gotta look into "interfacial / surfaec conductance"

% is the relaxation time quoted in some papers the one we're looking for here? alfalfa is 600 ns, is that a better, more direct value?

The dielectric properties of viruses do appear to differ considerably from the host, the discrepancy mainly in the internal core conductivity and perhaps the envelope permittivity. Despite this, because of the small radius of the virus, the first time constant of the virus transmembrane might plausibly range anywhere between 50 ns and 90 us, compared to between 100 ns and  for the both the plasma membrane and the nucleus.

% Virus time constant:
% >>> (((1 ohm m) / 2) + (1000 ohm meter)) * 2.9 uF/cm^2 * 50 nanometer -> 1.45 us
% >>> (((1 ohm m) / 2) + (10000 ohm meter)) * 17.7 uF/cm^2 * 50 nanometer -> 44.25 ns
% Host time constant:
% (already given by others, 100 ns)
% Nucleus time constant:
% using data from feldman paper,
% (Must be very careful regarding S/m, mS/m, etc!)
% >>> (((1 ohm m) / 2) + (1 ohm meter)) * 1.5 uF/cm^2 * 3 um -> 67.5 ns
% that's interesting, because it conflicts somewhat with shoenbach's suggestion that the nucleus has a longer time constant.

%At first glance, this may seem unfortunate: the host will be electroporated far before the virus.
%However, if a monopolar pulse train with a pulse duration shorter than the cell time constant, an intensity too low for, and a repetition rate of some 3 us, (200 khz), 

\subsection{Electrodynamic considerations}

Kotnik note an important fact: all these permittivity and conductivity values will also depend on frequency, which is not included in their model. We have not examined how these factors can be extracted from. Also disregarded are the Poisson-boltzmann shielding space-charges that proved all-important in the last study. 

The necessity for the Laplace transform over the Fourier-based method used for the precursor propagation in the previous report is founded in the requirement for a fully transient solution with a definite t=0.

The question of optimizing a pulse is not novel, having been discussed by many groups in the past. Retelj use a series of 

\subsection{Optimal pulse}

This bears considerable similarity to e.g. radar waveform design and compressed sensing, and also problems of optimal control. In many cases, the optimal control is equal to the transfer function

The time-domain response can be obtained by the convolution integral with the step or impulse response function, or by manually integrating the differential equation via some integrator.

Lagrange multipliers. Unfortunately, while closed-form solutions appear to be readily available for some simplified cases (such as when the membrane is infinitesimally thin or when permittivity is neglected) [schwan paper], 

Kotnik discuss the transmembrane voltage due to several pulse shapes formed from the Laplace Transform of the original differential equation. Talele offer their own response due to an arbitrary waveform can be produced by 

The impulse response is the derivative of the step response. The convolution integral (or sum, in the discrete case) can be performed either with the derivative of the input function and the step response, or the impulse response and the input function as algebraic convenience demands. Unfortunately

There exists an alternative convolution technique, due to the Convolution Property of the fourier transform.

The fourier transform of both the impulse response and the step response exist, are analytic and of manageable size.

For use with GEKKO, the transfer function was converted into a system of differential equations:

For PSOPT, a further modification was needed, as the library does not support the derivative of the control during .

Initial guess of 0 produces a trap.

because the problem depends on the second derivative of sub-nanosecond curves, the scaling varies by about 30 orders of magnitude.
PSOPT [] with IPOPT\cite{implementation2006} solver.

A hard-to-detect lack of convergence would often produce a fallacious simulated output from PSOPT. To double-check that the issue was not numerical, the output was fed through the convolution integral.



% Is the effect replicable, or is it simply an electrochemical artifact? Will these nanopores be destructive to viruses - will the pores be open for sufficient time for the ssRNA to be released? Will the nucleocapsid restrain the genome sufficiently and prevent membrane damage from inactivating the virus? Are the apoptosis effects likely to be triggered before these? Are there similar structures in tissue?

% A serious danger would be that if a nuclear or organelle membrane were to be porated, viral ssRNA may be admitted to places it shouldn't be. The author does not have the biology experience to evaluate this risk.



\end{document}