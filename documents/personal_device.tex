%!TeX root = personal_device
\documentclass[paper.tex]{subfiles}
\begin{document}

\begin{autem}
What follows is intended to be a casual, first-principles e. Be very careful with its conclusions.
\end{autem}

{\color{red} speculation \{ } 

\clearpage
{\Large \it The plausibility of low-cost microwave masks}\\

With the most recent few generations of semiconductor processes, producing significant powers in the X-band has become inexpensive. Silicon transistors are limited in their output power and gain; Silicon-on-insulator topologies 

This is perhaps most readily illustrated with the HB100 radar module. This device implements a complete doppler radar with two RF MOSFETS and a dielectric resonator. It operates at 10.6 GHz, yet are produced for under \$3. The PCB is stock FR4.

Our crude experimentation has yielded about 10 mW with a single SiGe:C transistor, the Infineon BFP620 (\$ 0.34 apiece \@ 1000-of.). For the personal microwave sterilizer, given the threshold levels found by Yang, this corresponds to a cost per sterilized cubic centimeter of approx. \cite{BFP620H7764XTSA1}.




Trivially extending great work by \cite{Focusing} on passive reflectarrays, a 40-element active-antenna phased array, each element of which consists of a negative-resistance device such as the BFP620 transistor, a tuning varactor, a patch antenna, and a resonator, each contributing a pulsed tone of 500 nanoseconds length and 10 milliwatts of RF power, and a focal point spatial scan rate and pulse repetition frequency of 6 KHz. produces a field pattern which, when scanned, appears to be suitable to deny respiratory transmission in a 'face shield' configuration. 

Such a field would comply with all standards by at least a factor of 20.

%figure from ipynb

\begin{figure}[H]
	\captionsetup{singlelinecheck = false, justification=justified}
	\centering
	\includegraphics[width=\textwidth]{muller_test_cropped.jpg}
	\caption{\\ How simple an active antenna element can be. This is based on the feedback-loop oscillator of \cite{SmallSize2008}, and uses a single CEL CE3520K3 GaAs FET (\$2.5) and one 100pF capacitor. It oscillated with extreme instability at about 5 GHz.}
\end{figure}


\begin{figure}[H]
	\captionsetup{singlelinecheck = false, justification=justified}
	\centering
	\includegraphics[width=\textwidth]{my_photo-17.jpg}
	\caption{\\ A wideband tunable 7 GHz VCO based on \cite{TripleTuned2008}. This uses an Infineon BFP620 SiGe:C transistor (\$0.3) and a network of four varactor diodes. Unfortunately, the Si varactors did not appear to have a sufficiently high Q to prevent mode-hopping, and the output spectrum reflects the ignorance of its designer; however, it appeared to be capable of producing the 10 dBm required above.}
\end{figure}


% a prototype oscillator



Injection locking rather than discrete phase shifters saves a number of components.

Such a device would cost about \$12 in silicon at prototype prices, provide half a month of battery life on two alkaline AAA cells, 

We demonstrate this form factor because, superficially, there are fewer people than there are places. But is such a device really necessary or useful? Is it sufficiently superior to a fiber mask to justify spending energy developing it? It has the advantage of being self-cleaning; and hair, eyes; a hand brought near the face is immediately sterilized.


On the other hand, a personal device may present issues with participation and production volume. 

Interference

If HFSS or ADS are not available, coupled SPICE-FDTD methods appear to be particularly simple and effective for these types of problems.

The advantage of the phased array is in the free-space power combining and the near-field focusing.

In our crude experiments, the loss tangent and poor impedance control of inexpensive substrates did not appear to pose a problem in itself; standard FR-4 or even PET substrate and screen-printed metal techniques seen in high-volume RFID tag production may be of some use. 

However, the Q factor of tuning elements on such lossy materials appears to be too high to form useful oscillators. Air-dielectric coaxial, Barium titanate dielectric slab resonators, or various types of substrate-independent resonators may all be suitable for this purpose.

{\color{red} \} speculation } 





\end{document}