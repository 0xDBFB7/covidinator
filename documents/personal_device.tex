%!TeX root = personal_device
\documentclass[paper.tex]{subfiles}
\begin{document}

\begin{autem}
What follows is intended to be a casual, first-principles e. Be very careful with its conclusions.
\end{autem}

{\color{red} speculation \{ } 

Trivially extending great work by \cite{Focusing} on passive reflectarrays, a 40-element active-antenna phased array, each element of which consists of a negative-resistance device such as the BFP620 transistor, a tuning varactor, a patch antenna, and a resonator, each contributing a pulsed tone of 500 nanoseconds length and 10 milliwatts of RF power, and a focal point spatial scan rate and pulse repetition frequency of 6 KHz. produces a field pattern which, when scanned, appears to be suitable to deny respiratory transmission in a 'face shield' configuration. 

%figure from ipynb

% a prototype oscillator

Such a field would comply with all  standards by at least a factor of 20.

Injection locking rather than discrete phase shifters saves a number of components.

Such a device would cost about \$12 in silicon at prototype prices, provide half a month of battery life on two alkaline AAA cells, 

We demonstrate this form factor because, superficially, there are fewer people than there are places. But is such a device really necessary or useful? Is it sufficiently superior to a fiber mask to justify spending energy developing it? It has the advantage of being self-cleaning; and hair, eyes; a hand brought near the face is immediately sterilized.


On the other hand, a personal device may present issues with participation and production volume. 

Interference

If HFSS or ADS are not available, coupled SPICE-FDTD methods appear to be particularly simple and effective for these types of problems.

The advantage of the phased array is in the free-space power combining and the near-field focusing.

In our crude experiments, the loss tangent and poor impedance control of inexpensive substrates did not appear to pose a problem in itself; standard FR-4 or even PET substrate and screen-printed metal techniques seen in high-volume RFID tag production may be of some use. 

However, the Q factor of tuning elements on such lossy materials appears to be too high to form useful oscillators. Air-dielectric coaxial, Barium titanate dielectric slab resonators, or various types of substrate-independent resonators may all be suitable for this purpose.

{\color{red} \} speculation } 





\end{document}