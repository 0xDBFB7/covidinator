%!TeX root = brillouin_pulse_math
\documentclass[paper.tex]{subfiles}
\begin{document}
	
	
%http://web.physics.ucsb.edu/~fratus/phys103/LN/DHM.pdf
% section on Greens functions
% and also 
% and
% https://www.int.washington.edu/users/dbkaplan/228_01/green.pdf	


Pulse energy, rather than peak field, is not obviously the ideal constraint for this problem; however, it is the one most compatible with lagrange multiplier techniques.

The Euler-Lagrange equation is usually brought up in the case of the Brachistochrone problem, or the Principle of Least Action (see Feynman lectures). In this case, the path through space from endpoints a to b is replaced with the whole frequency domain from -$\inf$ to +$\inf$.

Aha! That's the ticket! 

Search terms: "first variation", gateaux derivative, 

As noted by Pozar, the complex conjugate arises from the concept of "adjointness". Franks.

Note the ni in the complex conjugate!

\end{document}