%!TeX root = brillouin_pulse_math
\documentclass[paper.tex]{subfiles}
\begin{document}
	
	
%http://web.physics.ucsb.edu/~fratus/phys103/LN/DHM.pdf
% section on Greens functions
% and also 
% and
% https://www.int.washington.edu/users/dbkaplan/228_01/green.pdf	




The Euler-Lagrange equation is usually brought up in the case of the Brachistochrone problem, or the Principle of Least Action (see Feynman lectures). In this case, the path through space, endpoints a to b, where space would seem to be replaced with the frequency domain. 

Aha! That's the ticket! 

Search terms: "first variation", gateaux derivative, 

As noted by Pozar, the complex conjugate arises from the concept of "adjointness". Franks.

Note the ni in the complex conjugate!

\end{document}