%!TeX root = brillouin_pulse_math
\documentclass[paper.tex]{subfiles}
\begin{document}

\pagebreak
\section{The Brillouin precursors}

Uzunoglu \cite{Theoretical2020} mention a handy way to circumvent this lossy-flesh problem: the Brillouin and Sommerfeld precursors.


The extremely high losses from such thin coaxial cables should be considered.

The original text of these papers are in German; translations can be found in Léon Brillouin's 1960 
book\cite{Wave1960} on the topic. \footnotemark

These are fascinating structures which arise as a result of two aspects (Albanese 
\cite{Shortrisetime1989}): 

The highly dispersive nature of tissue, which causes a peculiar distortion of short pulses,

The change in loss (the complex part of the permittivity, or the) and increase in penetration depth 

A third aspect leads to the formation of the prototypical brilluoin precursor: any "sharp" 
beginning or end of a tone, or other spatially confined structure in a waveform, includes harmonics 
up to these \footnotemark


Precursors have been used in practice, by Ong \cite{Detection2003} using simple square waves to detect respiration\footnotemark, Ossberger \cite{Noninvasive2004} for the same purpose; and various techniques for breast cancer detection.


Also another really good paper that I can't find.

\footnotetext{Well, Ong don't really observe the precursor itself, it seems.}




While it is clearly possible to produce this abnormal penetration (any sharp edge suffices), it was not obvious to us that a waveform 
could be constructed that would be an effective driver for the normal modes in question. 
Oughston ; if the pulses are too close together, they distort the precursor thus formed.



\footnotetext{it is interesting to note that a similar - if not equivalent - effect leads to the Heisenberg uncertainty principle. The more localized a }




At a depth of 8 cm in Cole-Cole muscle, an 8 GHz resonance, attack pattern alpha appears to achieve 4 picometers of oscillation amplitude, rather than 0.01 picometers with $\sin(t\ \omega_{resonance})$. The sharp edges are at approx. 250 GHz. 



At 9 GHz, assuming that a torso is about 10 penetration depths in radius and the Brillouin train 
drives the virus with an equally, implementing this would decrease the field at the skin required 
for complete eradication of the virus from the body to a perfectly practicable 1 kV/m, rather than 
an air-ionizing 6.6 MV/m ($\frac{1}{\sqrt{10}}=0.316$ versus $e^{-10}=4.5e^{-5}$). 

However, the energy contained in such a standard pulsed-sine or gaussian train decays very quickly, 


%http://web.physics.ucsb.edu/~fratus/phys103/LN/DHM.pdf
% section on Greens functions
% and also 
% and
% https://www.int.washington.edu/users/dbkaplan/228_01/green.pdf	





Note: this isn't truly transient - the fourier integral is still periodic, wrapping around every x cycles. It's just more transient than the steady-state.




attack pattern alpha


Unfortunately, despite NLTLs producing strong wideband components, in cursory simulations they do 
not appear to satisfy the requirements for Brillouin pulses. Our working hypothesis is that this is 
because of destructive interference from the falling edge that immediately follows; but this is 
mere speculation.
\begin{autem}
	This should be replicated.
\end{autem}

temporal Soliton-producing nonlinear transmission lines, and NLTL oscillators constructed therefrom, particularly well suited 


Transient solution

\pagebreak
A qualitative statement of the problem:

\begin{toolchain}
An arbitrary waveform propagates through a dispersive, lossy medium with a certain analytic complex 
permittivity. 

It then drives a damped harmonic oscillator that is initially at rest.\\

What waveform produces the peak transient amplitude on the oscillator?
\end{toolchain}

\subsection{Analytic techniques}


\paragraph{\textbf{Subharmonics}}\


\begin{fquote}[Wait][ \cite{Propagation1965}]
	In such cases, it may be feasible to evaluate the integral by a purely numerical procedure. With the wide availability of the digital computer, this is certainly fashionable at the moment. Consequently, one might say that the problem has been solved and no further discussion is needed. \\
	
	However, it would be a pity if one accepted this answer since all physical insight into the nature of transient processes has been ignored. 
\end{fquote}





\begin{autem}
Reflection from tissue interfaces?

Thermal noise will kick the oscillator around; it won't necessarily start at rest. Will this affect the 
\end{autem}

One might suggest try to find the optimal transient solution first, and then try to build an input waveform deconvolved through the  transfer function of the tissue. This hardly seems liable to produce a truly optimal solution, however.

Each problem has been solved separately. There are some nearly equivalent but not mathematically identical formulations for optimal control of quantum systems with dispersion.\footnotemark However, in the classical domain, this appears to be unprecedented in the literature.

\footnotetext{While not strictly related, \cite{Wavepacket1989} tune a light pulse to match the wavefunction of the desired chemical products in a photo-reaction, which is just pretty darned awesome}

Also, this seems similar to a pulse propagating through the tissue's Cole medium and then dissipating the maximum power in a uniform Lorentz medium, which might be another route to attacking it.

Another route might be to set up two ODEs for the wave equation and the oscillator and solve them simultaneously.

Progress was hindered by the ~6 conventions for Fourier normalization, the sign or handedness of the imaginary number. What self-respecting field has completely antipodal definitions of the same concept.

oh, on the same topic, can we finally settle the i / j convention. Also, everyone that doesn't explicitly note  

Optimal pulses for several situations in Debye and Lorentz media have been determined by Oughstun's asymptotic method, which appears to be highly satisfactory in terms of physical intution, and also avoids truncation issues that befall fourier-transform techniques. In general, these optimal pulses are simple gaussian monopulses or higher-order gaussian derivatives \cite{Optimal2017} \cite{Optimal2015}. 

Macke \cite{Simple2012} have come up with a number of other analytic dispersion representations for various signals, such as chirped gaussian pulses.


%\printendnotes

\subsubsection{Zhu et al 's dispersive waveform optimization propagation formulation}

The most promising analytical technique that we explored was essentially equivalent to Zhu, Hum, 
Costas formulation in \cite{Microwave2012b}, previously discussed by Pozar \cite{Waveform2003}, 
This is based on a lagrange multiplier technique and variational 
calculus\cite{Methods1989} \footnotetext{Based on personal communication with Professor Costas Sarris}.

$$ e_x(z,t) = \frac{1}{\sqrt{2 \pi}} \int_{-\infty}^{+\infty}{F(\omega) e^{- j (\omega/c_0)n(\omega)z}\ e^{j\omega t} d\omega} $$

where $\int e^{j \omega t} d\omega$ is the inverse fourier transform.

$$W = \frac{1}{\eta_0} \int_{-\infty}^{+\infty}{(n_r(\omega))\ |F(\omega)|^2}\ d\omega$$\\

Where $n(\omega)$ is a complex refractive index per Debye relaxation equation, $n_r()$ is the real part of the same, everything else is a real constant. Set up a functional with a lagrange multiplier\\

$$\xi = -e_x(z,t) + \lambda W$$\\

Take the functional ("variational") gradient with respect to $F(\omega)$.

$$ \nabla \xi = -\frac{1}{\sqrt{2\pi}} \left(e^{- j (\omega/c_0)n(\omega)z}\right)^\star \  e^{-j\omega T} + \lambda...$$

At this point, you could choose to perform a second integral over the time-domain Greens function solution to the harmonic oscillator\cite{Complex2020}, eq. 4 and 15, 

$$x(t) = \int_{-\infty}^{+\infty}{G(t,t')\ \frac{q\cdot e_x(z,t'))}{m}\ } dt'$$\\

(eq. 15, for the underdamped case)

$$G(t,t') = UnitStep(t-t') \frac{1}{\sqrt{\gamma^2-\omega_0^2}  }\sinh\left({\sqrt{\gamma^2-\omega_0^2}}\ (t-t')\right)$$

But a much more straightforward 

Where * is the complex conjugate.\\

The $\nabla H() = H()^\star$ thing seems to come from Franks, Signal Theory, 1969, page 140 onwards; I cannot make heads or tails of his formulation.

Neither Maxima, SageMath, Mathematica's VariationalD, Matlab's Fundiff, nor Maple's 
functionalDerivative seem to yield any meaningful result for grad xi.\\

Then set

$$x(t) = \int_{-\infty}^{+\infty}{G(t,t')\ \frac{q\cdot e_x(z,t'))}{m}\ } dt'$$\\

Anyhow, then modify the lagrange multiplier,

$$\xi = -x(t) + \lambda W$$\\

And now,

$$\nabla \xi = 0$$

Pulse energy, rather than peak field, is not obviously the ideal constraint for this problem; however, it is the one most compatible with lagrange multiplier techniques.

Tufts \cite{Optimum1964}

The Euler-Lagrange equation is usually brought up in the case of the Brachistochrone problem, or 
the Principle of Least Action (see Feynman lectures). In these types of problems, it appears that 
the path of integration through space from endpoints a to b is replaced with the whole frequency 
domain from -$\infty$ to +$\infty$. 

Search terms: "first variation", gateaux derivative, 

As noted by Pozar, the complex conjugate arises from the concept of "adjointness". Franks.

Note the ni specifically in the complex conjugate!


\subsection{Laziness prevails; a return to numerical optimization}

Numerically solving optimal control problems using CG was noted by Lasdon \cite{conjugate1967} and 
revisited by Kek \cite{Conjugate}. They reduce the computational workload by first computing the 
gradient function by hand.

Using an energy constraint as above appears to tend towards a square step function or a sharp impulse, perhaps as a result of Pontryagin's maximum principle\cite{Optimum1964}.

However, a maximum absolute value constraint produces fantastic results.


\begin{figure}[H]
	%	\makebox[\textwidth][c]{
	\includesvg[width=\textwidth]{muscle_pulse_8cm_minimize_test1}
	\caption{Second.}
\end{figure}

\begin{figure}[H]
	%	\makebox[\textwidth][c]{
	\includesvg[width=\textwidth]{muscle_pulse_8cm_minimize_test2}
	\caption{Second.}
\end{figure}

 Believed to be FFT window artifacts. Increasing the window size with the same sample rate caused CG convergence to stall.





\end{document}