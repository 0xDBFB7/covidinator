\documentclass[11pt]{article}
%Gummi|065|=)
\title{\textbf{Welcome to Gummi 0.6.6}}
\author{Alexander Van der Meij\\
		Wei-Ning Huang\\
		Dion Timmermann}
\date{}

\usepackage{physics}
\usepackage{bm}
\usepackage{listings}
\newcommand{\bstar}[1]{ {#1}^{\bm*} }




\begin{document}


\{ $a_{1V}$... \} etc have 


the a and b coefficients both have units     
       
\[ \{\text{meter}\cdot \text{siemens}^2, \  \text{farad}\cdot \text{meter}\cdot \text{siemens} ,\ \text{farad}^2 \cdot \text{meter}\} \]


in physical units, typical values for as an example, \\
\verb|Cell(0.3, 80, 0.005, 30, 1e-8, 60, 50e-9, 14e-9, t)| \

$$R = 2.5\times10^{-8}$$
$$a_1 = 6\times10^{-26}$$
$$a_2 = 9\times10^{-33}$$
$$a_3 = 2\times10^{-41}$$
$$b_1 = 4\times10^{-26}$$
$$b_2 = 1\times10^{-32}$$
$$b_3 = 4\times10^{-41}$$

An input control waveform u(t) (volts per meter), and two output waveforms, $x_V(t)$ and $x_H(t)$ in volts.\\


%virus.__dict__






First system of differential equations (named V):\\

$$\dv[2]{x_V}{t} = 
\frac{R_V a_{1V}}{b_{1V}} \dv[2]{u}{t} 
+ \frac{R_V a_{2V}}{b_{1V}}  \dv{u}{t} 
+ \frac{R_V a_{3V}}{b_{1V}}  u
- \frac{b_{2V}}{b_{1V}} \dv{x_V}{t}
- \frac{b_{3V}}{b_{1V}} x_V $$


In practice, the solver takes a DAE: the second differential equation is split into an explicit system of eqs:

$$u1 = \dv{}{t} u$$
$$u2 = \dv{}{t} u1$$


$$x1_V = \dv{}{t} x0_V$$
$$x2_V = \dv{}{t} x1_V$$


Another system is solved simultaneously (named H), of identical form but different coefficients:

$$\dv[2]{x_H}{t} = 
\frac{R_H a_{1H}}{b_{1H}} \dv[2]{u}{t} 
+ \frac{R_H a_{2H}}{b_{1H}}  \dv{u}{t} 
+ \frac{R_H a_{3H}}{b_{1H}}  u
- \frac{b_{2H}}{b_{1H}} \dv{x_H}{t}
- \frac{b_{3H}}{b_{1H}} x_H $$


%x2  = (R*a1*u2 + R*a2*u1 + R*a3*u0 - b2*x1 - b3*x0)/b1



\subsection*{Nondimensionalization}

Let $x_V^{\bm*} = \frac{x_V}{x_{VK}}$ and $t^{\bm*} = \frac{t}{t_{K}}$ where $_K$ denotes a nondimensionalization constant (standard convention of $t_0$ is changed to $t_K$ to avoid confusion with subscript 0s used in simulation). (the forcing functional u is dependent on t, but this is implied).\\ \\

$$\frac{d^n}{dt^n} = \frac{1}{{t_K}^n}\frac{d^n}{d{\bstar{t}}^n}$$\\

% https://math.stackexchange.com/questions/3435173/how-to-nondimensionalize-a-second-order-differential-equation
% http://math.colgate.edu/~wweckesser/math312Spring05/handouts/Nondim.pdf
% https://faculty.math.illinois.edu/~rdeville/teaching/558/nondimensionalization.pdf
% http://math.colgate.edu/~wweckesser/math312Spring05/handouts/MoreNondim.pdf


% Kronbichler et al suggest an interesting method of nondimensionalizing.

% https://www.math.colostate.edu/~bangerth/publications/2008-boussinesq.pdf



GEKKO's .dt() operator allows u0 to be chosen as the control. 
Otherwise, the DAE form requires u2 to be used as the control.




\end{document}
