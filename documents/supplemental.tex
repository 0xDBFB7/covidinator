%!TeX root = supplemental
\documentclass[paper.tex]{subfiles}
\begin{document}


\clearpage
%%%%%%%%%%%%%%%%%%%%%%%%%%%%%%%%%%%%%%%%%%%%%%%%
%%%%%%%%%%%%%%%%%%%%%%%%%%%%%%%%%%%%%%%%%%%%%%%%
{\Huge Captain's Log, Supplemental}\\
%%%%%%%%%%%%%%%%%%%%%%%%%%%%%%%%%%%%%%%%%%%%%%%%

\begin{quote}
The accumulated insight of a [] worker frequently merits recording when no documentation can be given.
\end{quote}

- W. G. Cochran et al, \cite{Statistical1953}





For in-vitro research, it seems to me that perhaps this field is running into the issues Feynman's rat-races \cite{Cargo}. There are a large number of very tricky variables. Requires analysis fo the very details of how the study was conducted, rather than simply its results. While the measurements 

Of course, we commit the very same mistakes and fall into the same traps, we are not immune from criticism.

In some cases, these are thoroughly characterized and described. UWB rat paper. 

But it seems to me that each worker (myself included!) is using their own test arrangement; and yet the studies are not so different as to require such customization.

It occurs to me that perhaps a set of exquisitely characterized standardized test cells and procedures would be helpful; and a requirement that any papers on the topic either use such a cell, or if their study requires some exceptional treatment, to follow an established procedure to add a new cell to the standard. 

Alternatively, methods such as the optical hetrodyne one,

If possible, very specific buffer compositions should be made universal.

What heating patterns can be expected from a certain microwave input? What modes can be introduced by deviations in machining tolerances?

Then, of course, are the questions of the excitation: what brand of attenuators are satisfactory?

Of course, one cell could not suffice for the whole range of microwave frequencies.


There is also the matter that the scientific publishing infrastructure could perhaps provide for a "scientific lineage". We often hear of a "scientific record": journals will retract a paper from "the record"; but where {\it is} this record? Where is the chain of trust upon which new results can be added? Each paper is forced to supply its own context; each worker is forced to re-dredge the entire chain of truth, to discover the controversy which clouds some papers; each offers its own interpretation of the literature. It is inevitable that some history will be lost; that the wrong starting point will be taken up; a veritable fork in the literature. Especially as the volume of scientific research increases as society proceeds. The only data is a few references per paper, and a date for each! Another such record exists in comprehensive books written by workers on the topic.

But is this not an artifact of the limitations of printed journals, whereby papers must necessarily be individual entities? Is there no way by which papers can be arranged so that they each supply a visible part of the jigsaw, where each paper brings with it the complete story of the field?

Scite.ai and similar services, but it feels like these are tacked-on workarounds to re-add metadata which should have existed in the first place.

Consider the non-thermal effects literature. It starts, perhaps, in the 1800s; 






\clearpage
%%%%%%%%%%%%%%%%%%%%%%%%%%%%%%%%%%%%%%%%%%%%%%%%
{\Large Optical centrifuge polarization chirp}\\
\begin{multicols}{1}
%%%%%%%%%%%%%%%%%%%%%%%%%%%%%%%%%%%%%%%%%%%%%%%%


\includepdf[
    %% Include all pages of the PDF
    pages=-,
    %% make this page have the usual page style
    %% (you can change it to plain etc). By default pdfpages
    %% sets the pagecommand to \pagestyle{empty}
    pagecommand={\thispagestyle{empty}},  
    %% Add a "section" entry to the ToC with the heading
    %% "Quilling Shapes" and the label "sec:shapes"
    addtotoc={1,section,1,Quilling Shapes,sec:shapes}]
%% The pdf file itself
{optical_centrifuge.pdf}



\includepdf[
%% Include all pages of the PDF
pages=-,
%% make this page have the usual page style
%% (you can change it to plain etc). By default pdfpages
%% sets the pagecommand to \pagestyle{empty}
pagecommand={\thispagestyle{empty}},  
%% Add a "section" entry to the ToC with the heading
%% "Quilling Shapes" and the label "sec:shapes"
addtotoc={1,section,1,Quilling Shapes,sec:shapes}]
%% The pdf file itself
{biology.pdf}


\end{multicols}



Contrary to some [Hardell 2017] reports, we do not find the FCC and ICNIRP to be significantly affected by special interest groups; their rationales appear to be transparent.  [FCC-19-126A1], for instance, is a particularly entertaining memorandum:

\begin{quote} 
Similarly, IEEE-ICES urges the Commission to adopt a higher SAR exposure limit of 2 W/kg averaged over 10 g. [snip] We are not persuaded that the IEC standard should be adopted at this time.", "Medtronic and the AAMI-CRMD recommend a more relaxed threshold of 20 mW. We decline to increase the 1-mW threshold.". 

\end{quote}

Things heating up at the Microwave Safety fandom.


\clearpage
\rule{\linewidth}{0.2pt}

We do not intend to denigrate the work or otherwise cast aspersions on the authors of the mentioned papers; their experiment extremely well-designed and performed. 

However, since the result is unexpected, we would like to mention a few means by which the effect described could be achieved without strong non-thermal coupling.

\begin{quote}

 To confirm that our observation is not due to the microwave thermal heating effect, we had monitored the sample temperature change during the microwave illumination experiments with a radiated power density of 486W/m2 at a frequency of 6GHz by using an infrared thermal imaging camera with a temperature accuracy of 0.05°C (CHCT, P384-20). The temperature rise after 15 minutes radiation was 7°C, from 27.5°C up to 34.5°C. We thus exclude the possible contribution of microwave thermal heating effect to inactivation H3N2 viruses under our experimental condition.

\end{quote}

The frequency-dependent inactivation result in Figure 4b of \cite{Efficient2015} is difficult to explain without invoking non-thermal mechanisms. Only one  It is not mentioned how this applied power level was measured. 

The amplifier in question is not available on the supplier's website \cite{Microwaved} has gain flatness of 3.5 dB over 6-12 GHz - just over 1 octave. The amplifier used in [Hung] is not mentioned; because the frequency range mentioned (6 GHz to 18 GHz) is the same as that of the amplifier, we assume the same amplifier was used. 

If the amplifier happened to be matched better at 8 GHz than 12 GHz.



QPJ-06183640

Similarly, the gain of the EM-6969 antenna\cite{EM6969} varies from 17 dBi to 21 dBi over the frequency range in Figure 3.

However, that the spectra of all these effects would combine to produce a peak at precisely the resonant frequency determined by the entirely different microwave cuvette is highly implausible; and so Figure 4b really does seem to be good evidence of such an effect.

\clearpage
\rule{\linewidth}{0.2pt}

There seems to be a discrepancy between \cite{Efficient2015} and our reading of [C95.1-2005].

They use a threshold in open public space of $100(f/3)^{1/5} \text{ W/m}^2$. The 115 W @ 6 GHz figure they provide corresponds to this equation with a coefficient of 100.

For Table 9, general public, the equation is $18.56 (f)^{0.699} \text{ W/m}^2$, or $64.93 \text{ W}/\text{m}^2$ @ 6 GHz. 

Different versions of the IEEE standard have used equations of equivalent form but with different coefficients [Wu 2015]; it is entirely possible that we have retrieved the wrong standard.

\rule{\linewidth}{0.2pt}


\PRLsep{{\itshape Phage lysis sensing }}



Phage releases several amino acids such as putrescine and spermadine when broken. These can be indicated at ppm scales with ninhydrin, or at nanogram scales via fluorescamine. 

Otherwise, GFP-flouresecent strains of E. coli, or phage which has been labelled with a 






Essentially the only method of determining the activity of phage is to have it infect a suitable host.

Typically, this is done either in a large tube or a flat-sided optical cuvette.

Because the variation in the optical density is quite small, this poses some rather. 

A lot of the basic facts of phage were determined in the 1930s; 

A small vial of liquid has a much larger surface area relative to its thermal mass. sis removed, the temperature drops very rapidly; if the temperature drops outside the "physiological range" of around 33-40 C, growth is stunted \cite{effect2003}. \cite{growth1946}. I'm not sure if this was the problem I encountered, 

As a faculative aerobe, E.coli requires a concentration of oxygen to be present in the mixture for speedy replication \cite{Effect1965}; and this supply must also be kept sterile. One method to ensure this is by using a bubbler aerator; Carolina recommends an aquarium pump with cotton-ball filters. 

Another method, often used in microfluidics\cite{Microfluidic}, is that of \cite{method1951}; the Manganese dioxide catalyzed decomposition of hydrogen peroxide. $MnO_2$ is readily obtained from alkaline batteries; and in this case no sterility issues are presented. However, it is not obvious how antimicrobial $H_2O_2$ vapor can be prevented from contacting the culture.

Bottles of zero air may be more effective.

In commercial fluid transfer machines, this is often resolved by HEPA filtering the input to an enclosing cabinet.

Shaking incubators, where the flasks are 

Ensuring that a great deal of headspace was present in a sealed tube (that is, culturing 2.5 mL in a 15 mL falcon tube) was sufficient for brief, low-density.



As the size of the well decreases, the Reynolds number also decreases \footnote{In fact, in the case of a circular well, it's not obvious that this is the case - the cross section decreases faster than the characteristic length}, and turbulent flow is greatly hampered.

Even slight rotation of the tube, such that the graduations 

"Chromogenic" substances \cite{Fluorogenic1991}, or the degradation of various dyes like Methylene Blue; tetrazolium dye methods;

Using 2.5 mm diameter x 3 mm high wells cut in polycarbonate and sealed with clear packing tape. However, the presence of small bubbles in the headspace of the wells prevented a consistent reading.

We tried simply adding 1 uL of culture to the microscope slide;

The 8-bit resolution and dynamic range of most cameras did not seem sufficient to discern the turbidity; and masking and correcting for 

\cite{Vision2016} use an ingenious technique. A background with sharp light-to-dark edges is used. Slicing the image into chunks only slightly larger than the object,


The 0.8 uL sample in each was not sufficient to produce a measurable change in turbidity in a 0.2 uL - or we were not experienced enough to perform the phage lysis protocol properly.

However, we were not able to make smaller volumes of culture enter the log phase.




Another very interesting technique is that of \cite{Study2003}, 



\PRLsep{{\itshape Wherein an oscillator is designed }}

Many 

Three incorrect pieces of information conspired to lead us to try to develop an inexpensive ultrawideband oscillator.

\begin{itemize}
	\item The assumption that a broadband frequency sweep would drastically improve the inactivation threshold, and moreover that 
	\item The ideal 
	\item The assumption that the only commercially available oscillators in this frequency range were expensive YIG-tuned devices.
\end{itemize}

There are really two different conflicting criteria: 

A low-cost, wide-band

Unfortunately, designing wideband microwave circuits is significantly more difficult than for a particular frequency. Quarter-wavelength 

Varactor diodes are a GaAs hyper-abrupt junction diode.

We were warned that this would be difficult: 



CEL did not supply a SPICE model for the GaAs FET device used in early prototypes. A FET was originally because a gate is ostensibly easier to bias than a base; but this turned out to be unfounded.

[Steenput 1999] has an interesting analytic method to synthesize a SPICE model suitable for a transient simulations from S-parameter measurements using negative resistances. However, this neglects the I/V characteristic. 

[Polyfet 1998] describes a simple optimization method to synthesize a SPICE model for an active device, and CEL appnote provides some details for GaAs devices.



\rule{\linewidth}{0.2pt}

OpenEMS is excellent, with Python bindings, some lumped components, and mesh refinement. However, embarrassingly, we were not able to resolve all the dependency issues in order to install it.





\paragraph{Biasing}\

In our simulations, the varactor-tuned feedback circuit appeared to be particularly sensitive to the introduction of bias-tees. 


The gate must be weakly pulled to ground, otherwise stray charge destroys the oscillation.

\fancyhead[C]{style 1 with thin line}



With 0.79 mm FR4 substrate and 0.2 mm (8 mil) wide traces, the maximum impedance achievable was about 115 ohms, which did not appear to be sufficient as an RF choke. Use of defected grounds can increase inductance, but this was not evaluated.

If suitably high-impedance traces are not available, a common technique is to use a quarter-wavelength line (approximately 6 mm long with the above parameters at 8 GHz) terminated with a stub to produce a virtual short or open circuit [Seo 2007]. 

However, reflections from these structures still appeared to distort the frequency/phase response beyond repair, even with ostensibly wideband stubs [Syrett 1980].

\noindent\fbox{\parbox{\linewidth}{
	Rebuke: despite this blather, many other papers have had success with bias-tees at these frequencies.
}}

Alternate methods evaluated, failures: 


In production, these could be accomodated by graphite-polymer printed resistors. 


Odd-pole varactor-loaded combline filters appeared to have excellent phase and frequency response; however, the geometry necessitates low-inductance via stitching to the ground plane.

%\noindent\fbox{\parbox{\linewidth}{
\begin{autem}
	{\it autem} Others have had great success with varactor-tuned comblines, especially non-grounded lines.
\end{autem}

[Tsuru 2008 fig. 10] is an excellent review of various oscillator designs.

The parasitic inductance of common varactors appears to become problematic at these frequencies (but not for non-wideband use).

\paragraph{\textbf{Spacing}}\

FDTD results still yielded some coupling at 2 mm isolation.

\paragraph{\textbf{Via stitching}}\

Rather than the almost universal technique of via stitching components immediately to the ground plane (a tedious process with our prototyping setup), a large ground pour on the component layer was used wherever ground was needed, as mentioned in [Hunter? combline]. Since we use microstrip rather than coplanar waveguide, 

Since the ground plane does not participate in the DC bias at all, no vias are required in the final device. This means that the drilling, electroless strike and electroplating steps are unnecessary for production.


\PRLsep{{\itshape The wideband oscillator }}

Not having access to any RF equipment, we needed an oscillator tunable over the X-band.

There are many oscillators. The HB100 uses a dielectric resonator, for instance; cavity-based methods with mechanical tuning; active antennas using. 


The literature is quite mature, partly due to the FCC's recent ultrawideband communications guides.

As a result of our gross incompetence, the oscillator was designed via an inane, roundabout, and fiendishly tedious manner, and our description of this technique only contributes to the field by being suitable for a dartboard; our analysis can only hold water when combined with paper mache; and a lesser invertebrate in posession of a copy of Grebennikov's excellent Microwave Oscillator Design would have accomplished the task faster.


Designing an oscillator of this type in one step with a few kindergarten equations appears to be well within the reach of modern network analytical techniques. Genetic algorithms are quite well matched to this problem, and many commercial software packages are  [computer microwave design book]. 


We began experimentally by building a scaled version of [Mueller 2008]'s active antenna crudely with copper tape, using a CEL [] GaAs FET.


We then manually tried a number of filter designs, using the manufacturer's S-parameters and QUCS' microstrip approximations. This intially appeared to yield good agreement with experiment. Peaks in the feedback voltage simulation corresponded approximately with peaks in the observed spectrum. [figures from LO prototype N]. 

A crude, inane, and extremely disorganized trial-and-error procedure was performed for many weeks. A wide variety of analytical methods for filter design were attempted [documents/global.ipynb], but simultaneously obtaining the correct phase shift and frequency response was somewhat difficult. Adding the parasitic inductance of the varactors always seemed to destroy our most perfect creations.

Using an SIR filter has the added advantage of confusing epidemiologists.

Eventually, QUCS, zonca/python-qucs, and scipy's basinhopper were used with a cost function somewhat similar to that described in [Kaplevich]:

\begin{verbatim}
    freq_coeff = 1
    phase_coeff = 1.5
    ratio_coeff = 0.5
    insertion_loss_coeff = 0.2
 
    frequency_cost = freq_coeff * (abs(desired_center_frequency-fb_peak_frequencies[0])/1e9)
    phase_cost = phase_coeff * abs(1.0 - phase_at_peak)
    ratio_cost = ratio_coeff * fb_peak_ratio
    insertion_loss_cost = insertion_loss_coeff*(1.0 - fb_peak_values[0])
    cost = frequency_cost + phase_cost + fb_peak_ratio + insertion_loss_cost
\end{verbatim}

Optimizing first for the high frequency, fixing the inductor and microstrip values, and then optimizing for the varactor values at the low frequency.

This produced a seemingly acceptable feedback loop (note the group delay / phase progression):

\includegraphics[scale=1]{LO_2_pole_test.png}

However, when this was built, varactor tuning performance was abysmal, with hops and peaks; and the tuning range was far smaller than expected.

It was thought that a transient simulation - to determine how the spectrum actually evolved - would improve the situation. 

As of 0.0.20, QUCS' microstrip models are not yet compatible with transient simulations; and some improved filter designs required simulating coupling between more than two microstrips, which QUCS did not yet support natively.

\begin{figure}[H]
\includegraphics[width=\textwidth]{3d_spectrum_2.png}
\end{figure}

The complete transient simulation matched reality very closely. 

\begin{figure}[H]
	\includesvg[width=\textwidth]{wideband_LO_2_simulated_tuning}
 	\caption{"Wideband VCO 2": BFP620  \\
 	 Oscillator tuning simulated with electronics/ngspice/oscillator\_designer\_2.py commit <>, eLabID 20200613-d28a8004c...}
\end{figure}


The oscillator design is a testament  This highlights that a computation is not a substitute for understanding.

The key sticking point seems to be that as frequency increases, using microstrip design techniques, the parasitics of any filter structure are so large that the small change in impedance that the varactors can provide does not tune the circuit by any meaningful amount.

The two varactors on each tuning circuit each contribute 0.7 nH and between 0.35 to 2.4 pF; each blocking capacitor contributes about 0.2 nH. The frequency range of the oscillator can be selected by adding between 0.5 nH to 1.5 nH symmetrically to both tuning circuits. 

"Using the lead inductances of the bipolar transistor and varactors provides the required value of the base inductance".

Indeed, [Tsuru 2008]'s "tuned circuit" in Fig 8 is, in fact, just the varactors, plus an almost invisible high-impedance line.


The final oscillator is a 'wideband double-tuned varactor VCO', based almost verbatim on [Tsuru 2008] and the reference design in Figure 8.36, p378 of [Grebennikov 2007]. 



The rest of the papers in the bibliography were highly enlightening regarding the principles of microwave oscillator design.

For ease of design and simulation, selecting a device with both Touchstone S-parameters and SPICE models is greatly preferable.

It is commonly claimed that FR4 is a 'slow' substrate, and that the high loss tangent of ~0.02 makes it unsuitable for microwave systems.

However, with this PCB substrate, the expected loss of only 0.026 dB/mm on signals of minimum 0 dBm is patently acceptable. As viruses do not have a discriminating palate, we are also only minimally concerned with $S_{11}$ reflections, noise, or spurs, so precise impedance control is not required; the wideband VCO sweep accomodates for any variations in resonant frequency due to wide manufacturing tolerances.


This topology of oscillator worked marvellously on essentially the first try.

Not having a good analytical understanding, we resorted to using a purely computational method.

In addition, the design of wideband feedback-loop VCOs is a relatively well-explored field, and many reference designs exist. 

Merely scaling designs like [] was not effective.





Several analytical filter design methods were 

This is apparently known as a "double-tuned" wideband.


Oscillators must meet the Barkhausen criterion:

\begin{itemize}

\item A 360 degree phase shift around the feedback loop (including the phase shift contribution from the amplifier, which itself varies greatly with frequency)
\item A loop gain $>1.$ 

\end{itemize}\
%
However, a third element is also required:
%
\begin{itemize}
\item A frequency-selective element that restricts oscillation modes and decreases phase noise.
\end{itemize}
%

With large feedback-loop structures, such as long PiN-switched phasing lines, proximity effects from nearby flesh would detune the oscillator significantly.

 design of a triple-tuned oscillator.


\paragraph{}
The buffer amplifiers 


\rule{\linewidth}{0.2pt}

In case you're wondering, like I was, why the voltage coefficient of capacitors 

\rule{\linewidth}{0.2pt}

Much of this project was done in 1.5 mm microstrip with a 2.5 to 3 mm routed isolation. However, it became apparent that the inductance of vias with the ground plane was too great. It has been suggested that low-inductance vias in prototypes \cite{Microwavee}

Grounded coplanar waveguide, where the primary ground plane is on the top side of the board, presents many advantages, and in later testing appeared to provide much better performance and easier design. connectors can interface with the top side; trace sizes are much more compatible with small SMD devices.



\rule{\linewidth}{0.2pt}

Again, inductive choke biasing in the feedback loop was practically impossible. Biasing PiN diodes with a 10kohm resistor, (with 330 ohm safety resistor) one to 48V bias and another through an 2N7002P N-channel mosfet worked fine. 

The oscillator ran fine with a PiN bias of 2.34 mA. [LO prototype N]. A high bias voltage of 48V was required to get sufficient current through the two 10K resistors and the PiN diode to obtain a low impedance while remaining delicate with the vfb.

The PiN diode used has a resting resistance of 500 ohms, 5 ohms at 2 mA and 2 ohms at ??. 

48V is a little tight on the 50V rated voltage of our DC blocking capacitors.

Each activated PiN diode should be biased separately, since putting 2x 2 mA through 20K would take an impractical 80V.

\rule{\linewidth}{0.2pt}

Conductors are represented by zeroing all components of the electric field in those regions. 

There are many different possible source geometries, each introducing their own distortions.

There are many ways of linking SPICE and FDTD. 

\rule{\linewidth}{0.2pt}

Bandpass filters can be designed by first designing a low-pass filter prototype (usually Chebychev) (or, in our case, using reference filter component tables), and then transforming this low-pass into a band-pass. [Hunter 2001] is an excellent overview of this process, with many design examples for different filter topologies. 

The coupling coefficient between two low-pass filters determines the band-pass bandwidth.[Hui 2012]

[Hunter 2001] also describes an analytical method to create a filter with the precise group delay - phase shift versus frequency - required for stable oscillation. However, simultaneously compensating for the group delay introduced by the amplifier itself (nearly 180 degrees over the frequency range for the CEL part) seemed complex.

Phase shift can be introduced either via a length of microstrip, or a high-pass/low-pass filter [Microwave101]. Adding a fixed microstrip line restricts the tuning range, however, and the filter inevitably affects the frequency response.  

\rule{\linewidth}{0.2pt}

NGSPICE's KSPICE coupled transmission lines require the capacitance and inductance per unit length in Maxwell matrix form, rather than the physical $C_{even}$/$L_{even}$ (each line's capacitance and inductance to ground) and $_{odd}$ (between elements) form provided by tools like wcalc. "matrix not positive definite". [Schutt-Aine] discusses this; we reproduce here for convienience.

\[ L_{11} = L_{22} = L_{even}  \]
\[ L_{12} = L_{21} = L_{odd}  \]
\[ C_{11} = C_{22} = C_{even}+C_{odd}  \]
\[ C_{12} = C_{21} {\it{(unused)}} = -C_{odd}  \]

\rule{\linewidth}{0.2pt}

In practice, the timestep required to obtain convergence in particularly tight corners of the SPICE simulation can drop to 1e-20, which is far below the Courant limit of the FDTD simulation. To eek out a bit more performance, a simple adaptive-timestep technique from [ a ] is used; we simply set the timestep so that the maximum change in voltage per timestep from the SPICE portion is less than 
some threshold.


\PRLsep{{\itshape Sampling time-domain spectrometry}}

\begin{quote}
A rapidly increasing voltage step V tŽ . is applied to the 0
line and recorded, along with the reflected voltage R tŽ .
returned from the sample and delayed by the cable propagation time Figure 2 . Any cable or instrument artifacts Ž .
are separated from the sample response due to the propagation delay, thus making them easy to identify and control. The entire frequency spectrum is captured at once,
thus eliminating drift and distortion between frequencies.
\end{quote}

The pulse is usually recorded in real-time with a many-GHz oscilloscope. However, by using a sampling oscilloscope, massively undersampling the signal,

Another advantage of this technique is that no high-Q filters are required to select the frequency;
discrimination is in the time-domain, and is spread out over many seconds. It is also much easier to generate a 10+ GHz fast edge than it is to generate a CW signal.



\PRLsep{{\itshape Power amplifier }}

The amplifier and oscillator are separate for ease of design; this way the phase response of the amplifier is of no consequence. 

Our target power for the test is about 0.01 W, mandating a $2 \sqrt(2)$ V peak to peak RF swing on the Z=50 output.

The $V_{CEO}$ breakdown of the bipolar transistor in question is only 2.3V. Above this value, current begins to flow from the emitter to the base, turning on the transistor and causing thermal runaway. If the base bias has a low enough impedance and can source current, this $V_{CEO}$ can be violated safely. [Toshiba, Bipolar Transistors, some other paper on the strong base bias]

One solution would be to match the output to a lower impedance [Leuschner 2010]:

\begin{quote}
Generating high RF power from low supply voltages poses also another problem: the optimum load impedance of the PA output stage transistors becomes very small. A matching network with a high transformation ratio is needed to transform the antenna impedance of 50 to the optimum load impedance. High transformation ratios usually result in small bandwidth and higher losses. Possible solutions are the use of either non-standard high power RF devices or special circuit topologies using only the available standard devices.
\end{quote}



Use of so-called stacked, HiVP, or collector-feedback cascode designs can also allow the RF voltage to be distributed over many devices.

It should also be noted that when stressed, these devices can fail with a slow power-output degradation over weeks, an effect not usually found in lower-frequency designs. 

\printbibliography[heading=none, title={}, keyword={amplifier}]




\clearpage
\PRLsep{ Wherein antennas are characterized}

Method-of-moments solvers like NEC-2 are the standard for simulating these sorts of antennas. However, most commonly-available packages don't seem to handle multiple dielectric constants, such as the air and substrate of a patch antenna.

The impedance of a structure over frequency can be determined in FDTD by:

\begin{itemize}
  \item Applying a gaussian pulse to a voltage source - in our case, applied to a via connecting the path (or probe)
  \item Running the simulation until the transients have all died out below some threshold, while logging the source voltage and current at each timestep
  \item Taking the fourier transform of the (real) excitation voltage and current (producing a complex result, mind you)
  \item Taking the magnitude of the ratio of the two complex spectra.
\end{itemize}

This is the computational equivalent of dropping a piano off a balcony to see which key is stuck.

See [Penney 1994], [Luebbers 1992], [Luebbers 1991], [Luk 1997]. 

Our implementation is in \ghfile{electronics/simple_fdtd/runs/U.py}.

A similar mismatch to the SPICE exists for fourier methods - but in the other direction.  The courant limit often demands fine timesteps, but since each FFT bin is $ f_{bin} = n_{bin} / (N_{simulation} \ dt) $ , the majority of the FFT bins exist into the hundreds or thousands of GHz, leaving no resolution in the low-frequency domain of interest unless $N_{simulation}$ is extremely large - even if all the transients in the simulation have died down, you still have to keep the sim running to make the FFT happy! 

There's more than enough 'information entropy' in 4000 FDTD points for most antennas. But you need some 30,000 points to get 10 bins below 20 GHz!

There are a few methods of changing the FFT bin size artificially, which [Bi 1992] reviews. You can "use a manual fourier integration over the frequency region of interest".

But, in a staggering turn of events which will presumably be familiar to statisticans and preposterous to everyone else, a far simpler method is to {\it discard} 95\% of the data, by down-sampling to 1/10th or so.

An equally simple method that seemed to produce better results in our case is to pad the voltage and current samples to the correct length. The jump discontinuity introduced by padding with zeros has a negligible effect.

It's so thumpingly unintuitive to me that adding 50,000 zeros to a 5000 value dataset can improve the resolution of a measurement by 300-fold.

It is important to remember to normalize the gaussian pulse, or else numerical noise will be introduced. The magnitude is not important - [Luebbers 1992] use 100v, others use 1v, etc.

Though uneven dt FFTs exist, the time step can be constant at the courant limit for this simulation.

A step impulse (1 first timestep, 0 otherwise) has been used in some works, though in our case performance was quite horrid.

A correction factor due to the staggered magnetic field of the Yee lattice must be introduced; [Fang 1994]. Their $Z_2$ equation (correcting for spatial inaccuracies, but not temporal) was sufficient.


This technique is equivalent to that used in electrochemistry, known as fourier impedance spectroscopy - except they seem to usually use a known impedance source rather than a hard source, presumably because ideal hard sources don't exist in reality.

Allowing the simulation to run for long enough that all transients dissipate is important for accuracy - deceptive dips in the current can cause early termination. A surprising amount of detail is contributed by even the smallest current levels. Our threshold is 1e-7 amps for 700 iterations.

[Samaras 2004] has a very useful set of experimental and FDTD data for calibration and comparison. Comparing a probe via source in the different positions, we obtained agreement of $\sim 7\%$ in impedance and $\sim 5\%$ in frequency.

The use of a hard source feed-port affects the number of timesteps required by introducing unphysical transients. Using a port with a virtual 50-ohm resistance reduces the computational requirements by a large factor; see [Luebbers 1996].

Simply monitoring the source current during the simulation is somewhat deceptive. Periodically monitoring the change in the fourier transform seems to be a better convergence metric.

The units of the FFT (unlike those of the continuous FT) remain in volts.

\rule{\linewidth}{0.2pt}

Most equations in papers on the FDTD method are supplied without making the scaling factors explicit; some need multiplying by the Courant number, For use with flaport/fdtd the H-field to current line integral in [] requires scaling by $\mu_0 * (dx/dt)$, where $\mu_0$ is the vacuum magnetic permittivity, dx the cell size, and dt the timestep - despite the equation already possessing a deceptive set of $dx$-es.

Naturally, if one is competent, this discrepancy will be immediately obvious. Those not dimensionally-intuitive, such as myself, find it useful to run dimensional analysis using a unit-aware calculator such as {\it sharkdp/insect}.

\rule{\linewidth}{0.2pt}



\begin{quote}
With four parameters I can fit an elephant, and with five I can make him wiggle his trunk.
\end{quote}
- von Neumann  

with six parameters, you can do it with only physical constants. Beware numerology.

\rule{\linewidth}{0.2pt}

Most passives (inductors, capacitors) companies supply S-parameters for their parts. Use them.

\rule{\linewidth}{0.2pt}

This has nothing to do with , but it's an amusing parlor trick that I accidentally ran into while trying to carbothermally reduce copper oxide powder. It led me to the SiC susceptor resonator, so it wasn't a total waste.

Add a thimblefull of fine copper powder to the bottom of a mason jar. Cover the mouth with a layer of Kapton tape. Next, inject pure argon gas for a few minutes, then seal the tape fully.

Put the whole shebang in a friend's microwave oven. 

Wait patiently. There will be a sizable bang and a terrific plasma.

My hypothesis is that this is due to the higher ionization energy of Ar. Air seems to slowly leak in, decreasing the 


\PRLsep{{\itshape The antenna itself }}

The impedance of a patch antenna varies with position on its surface. Probe feeds to a specific point on the patch

Unfortunately, patch antennas have a relatively narrow bandwidth.

\rule{\linewidth}{0.2pt}


\PRLsep{{\itshape Power sensing }}

The diode detector power sensor is based on the Infineon appnote [] and [], itself based on a circuit from The Art Of Electronics. 

A detailed description of various diode detectors are found in []. Because the steady-state current through the detector is very small, the microstrip can simply be terminated in the standard manner.

The Mini-Circuits ZX47-40-S+ would be an excellent low-cost COTS alternative.

\printbibliography[heading=none, title={}, keyword={rectifier}]

\PRLsep{{\itshape Microfluidics: small channels, big headaches }}

We had anticipated having to perform a large number of tests for optimization of the impulse. Automating tests also has the advantage of removing operator bias, and (in our case) minimizing artifacts from the sample holder. Since all the components were already computer-controlled, it seemed little extra effort.

We originally planned on using techniques from the rapidly accelerating field of centrifugal microfluidics, using a broadband patch antenna smaller than the cuvette, and measuring transmission through the cuvette.

Centrifugal microfluidics has the advantage of getting a pump for free; mixing can be effected by generating vortexes by accelerating and decelerating quickly; and because everything is circular, it is easier to design equidistant.

The most important feature is the consistent "down" vector; because capillary forces 

However, aligning seemed liable to produce artifacts; runout in the spindle;


Sterilization is often effected by cytotoxic gas []. UV we resorted to autoclaving, warping.

Th edges are charred; however, we have not found any discussion of toxicity of the laser cut edges, so we assume they're fine.


In a crude version of transmission welding. Tx welding usually relies on one material being transparent, and the other opaque - clear and black plastics, for instance. We obtained nearly useable results by scuffing one surface and rubbing in fine graphite powder.

Both the hydrophobicity / wetting angle of the substrate (useful for valves with a certain threshold pressure, among other things), the adhesive surface energy, and sterilization can all be effected by UVC exposure. Unfortunately, we were not confident in the homogeneity of the 

Not all plastics can be laser welded.

3DuF; this is an excellent program 



The final process was as follows:

Just using a single-sided tape over micromachined channels didn't work for us; the tape remains sticky and stuck to the bottom of the channel, blocking it.

\begin{itemize}
	\item 1/16" 003 size Viton O-rings McMaster-Carr \#9464K103, 
	\item 3M 467MP 0.051 mm double-sided adhesive transfer tape with 200MP acrylic adhesive, DigiKey 3M9726-ND.
	Various other tapes were tried, such as 9495MP; the laser was not powerful enough to cut these.
	While the intermittent temperature rating is over 121 C, the high-pressure steam environment does seem to degrade any exposed regions of tape.
	
	A CO2 laser is usually used to cut these otherwise transparent materials. We only had a diode laser. However, the front and back paper liners absorb sufficient laser power to melt the tape between. This produces horrible results, but was sufficient for this purpose. With our 2.5 W, ~0.15 mm spot size, 450 mm/min for the tape.
	
	\item MG Chemicals 416-T Laser Printer Transparency Film, 0.1 mm thick. Presumably a PET/Mylar material.
	
	\item A vinyl cutter was used to cut the openings in the top mylar cover slip.
	\item Frame machined from polycarbonate with a 1/16" two-flute carbide endmill, 24000 RPM, between 500 and 1000 mm/min, up to 1.5 mm step-down. G-code generated with MeshCAM. This seems to withstand the autoclave without any damage.
	\item Note: the tape preferentially sticks to the back liner, rather than the front.
	With both adhesive liners on, rub the thing hard with a rounded edge and rolling pin.
	Aluminum foil tape backing on the adhesive layer.
	Don't de-weed yet. 
	
	
\end{itemize}

Note that cutting many of these materials with a laser is a bad idea. The residue will 

\begin{itemize}
	\item Tape mylar to a cardstock backing with masking tape on the sides. Cut mylar on plotter with InkCut. 175 g force. 
	\item Apply aluminum foil tape to the rear backing (with "3M" text) of the adhesive transfer tape. This supports the cut structures, without being cut itself.
	\item Stick the adhesive to the laser bed with Kapton, foil-side down. It's not a good idea to use masking tape; it'll catch fire.
	\item 
\end{itemize}





Making this device ourselves was a grave mistake that potentially cost many lives. We should have used \cite{Straight}, a COTS version which we discovered halfway through. 

I'm sorry.

\paragraph{Lessons learned:}

When trying to characterize the response to some process parameters (say, laser power), iteratively trying different parameters in series can be somewhat useful, as each trial informs the next. In this instance, we would try a speed and power, unstick the material from the bed, observe the result, decide on the course of action, and repeat. However, if it is possible to try many parameters in parallel just as fast as in series, (that is, in our case, make a coupon using [] that tries all reasonable values simultaneously), it would be very dumb not to do this.

We are very dumb.

\printbibliography[title={General fluidics resources}, keyword={microfluidics}]

\printbibliography[title={Centrifugal}, keyword={centrifugal}]


\PRLsep{{\itshape aaaaa }}

One problem with the formal, academic paper format is that one starts to believe what is written, which is a death sentence for accurate reasoning. We try to avoid this by being unprofessional.

\label{para}
\ref{para}

\paragraph{Timeline}

\paragraph{Comments by others}



\Acrobatmenu{GoBack}{Back}


\end{document}