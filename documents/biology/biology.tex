\documentclass[11pt]{article}

    \usepackage[breakable]{tcolorbox}
    \usepackage{parskip} % Stop auto-indenting (to mimic markdown behaviour)
    
    \usepackage{iftex}
    \ifPDFTeX
    	\usepackage[T1]{fontenc}
    	\usepackage{mathpazo}
    \else
    	\usepackage{fontspec}
    \fi

    % Basic figure setup, for now with no caption control since it's done
    % automatically by Pandoc (which extracts ![](path) syntax from Markdown).
    \usepackage{graphicx}
    % Maintain compatibility with old templates. Remove in nbconvert 6.0
    \let\Oldincludegraphics\includegraphics
    % Ensure that by default, figures have no caption (until we provide a
    % proper Figure object with a Caption API and a way to capture that
    % in the conversion process - todo).
    \usepackage{caption}
    \DeclareCaptionFormat{nocaption}{}
    \captionsetup{format=nocaption,aboveskip=0pt,belowskip=0pt}

    \usepackage[Export]{adjustbox} % Used to constrain images to a maximum size
    \adjustboxset{max size={0.9\linewidth}{0.9\paperheight}}
    \usepackage{float}
    \floatplacement{figure}{H} % forces figures to be placed at the correct location
    \usepackage{xcolor} % Allow colors to be defined
    \usepackage{enumerate} % Needed for markdown enumerations to work
    \usepackage{geometry} % Used to adjust the document margins
    \usepackage{amsmath} % Equations
    \usepackage{amssymb} % Equations
    \usepackage{textcomp} % defines textquotesingle
    % Hack from http://tex.stackexchange.com/a/47451/13684:
    \AtBeginDocument{%
        \def\PYZsq{\textquotesingle}% Upright quotes in Pygmentized code
    }
    \usepackage{upquote} % Upright quotes for verbatim code
    \usepackage{eurosym} % defines \euro
    \usepackage[mathletters]{ucs} % Extended unicode (utf-8) support
    \usepackage{fancyvrb} % verbatim replacement that allows latex
    \usepackage{grffile} % extends the file name processing of package graphics 
                         % to support a larger range
    \makeatletter % fix for grffile with XeLaTeX
    \def\Gread@@xetex#1{%
      \IfFileExists{"\Gin@base".bb}%
      {\Gread@eps{\Gin@base.bb}}%
      {\Gread@@xetex@aux#1}%
    }
    \makeatother

    % The hyperref package gives us a pdf with properly built
    % internal navigation ('pdf bookmarks' for the table of contents,
    % internal cross-reference links, web links for URLs, etc.)
    \usepackage{hyperref}
    % The default LaTeX title has an obnoxious amount of whitespace. By default,
    % titling removes some of it. It also provides customization options.
    \usepackage{titling}
    \usepackage{longtable} % longtable support required by pandoc >1.10
    \usepackage{booktabs}  % table support for pandoc > 1.12.2
    \usepackage[inline]{enumitem} % IRkernel/repr support (it uses the enumerate* environment)
    \usepackage[normalem]{ulem} % ulem is needed to support strikethroughs (\sout)
                                % normalem makes italics be italics, not underlines
    \usepackage{mathrsfs}
    

    
    % Colors for the hyperref package
    \definecolor{urlcolor}{rgb}{0,.145,.698}
    \definecolor{linkcolor}{rgb}{.71,0.21,0.01}
    \definecolor{citecolor}{rgb}{.12,.54,.11}

    % ANSI colors
    \definecolor{ansi-black}{HTML}{3E424D}
    \definecolor{ansi-black-intense}{HTML}{282C36}
    \definecolor{ansi-red}{HTML}{E75C58}
    \definecolor{ansi-red-intense}{HTML}{B22B31}
    \definecolor{ansi-green}{HTML}{00A250}
    \definecolor{ansi-green-intense}{HTML}{007427}
    \definecolor{ansi-yellow}{HTML}{DDB62B}
    \definecolor{ansi-yellow-intense}{HTML}{B27D12}
    \definecolor{ansi-blue}{HTML}{208FFB}
    \definecolor{ansi-blue-intense}{HTML}{0065CA}
    \definecolor{ansi-magenta}{HTML}{D160C4}
    \definecolor{ansi-magenta-intense}{HTML}{A03196}
    \definecolor{ansi-cyan}{HTML}{60C6C8}
    \definecolor{ansi-cyan-intense}{HTML}{258F8F}
    \definecolor{ansi-white}{HTML}{C5C1B4}
    \definecolor{ansi-white-intense}{HTML}{A1A6B2}
    \definecolor{ansi-default-inverse-fg}{HTML}{FFFFFF}
    \definecolor{ansi-default-inverse-bg}{HTML}{000000}

    % commands and environments needed by pandoc snippets
    % extracted from the output of `pandoc -s`
    \providecommand{\tightlist}{%
      \setlength{\itemsep}{0pt}\setlength{\parskip}{0pt}}
    \DefineVerbatimEnvironment{Highlighting}{Verbatim}{commandchars=\\\{\}}
    % Add ',fontsize=\small' for more characters per line
    \newenvironment{Shaded}{}{}
    \newcommand{\KeywordTok}[1]{\textcolor[rgb]{0.00,0.44,0.13}{\textbf{{#1}}}}
    \newcommand{\DataTypeTok}[1]{\textcolor[rgb]{0.56,0.13,0.00}{{#1}}}
    \newcommand{\DecValTok}[1]{\textcolor[rgb]{0.25,0.63,0.44}{{#1}}}
    \newcommand{\BaseNTok}[1]{\textcolor[rgb]{0.25,0.63,0.44}{{#1}}}
    \newcommand{\FloatTok}[1]{\textcolor[rgb]{0.25,0.63,0.44}{{#1}}}
    \newcommand{\CharTok}[1]{\textcolor[rgb]{0.25,0.44,0.63}{{#1}}}
    \newcommand{\StringTok}[1]{\textcolor[rgb]{0.25,0.44,0.63}{{#1}}}
    \newcommand{\CommentTok}[1]{\textcolor[rgb]{0.38,0.63,0.69}{\textit{{#1}}}}
    \newcommand{\OtherTok}[1]{\textcolor[rgb]{0.00,0.44,0.13}{{#1}}}
    \newcommand{\AlertTok}[1]{\textcolor[rgb]{1.00,0.00,0.00}{\textbf{{#1}}}}
    \newcommand{\FunctionTok}[1]{\textcolor[rgb]{0.02,0.16,0.49}{{#1}}}
    \newcommand{\RegionMarkerTok}[1]{{#1}}
    \newcommand{\ErrorTok}[1]{\textcolor[rgb]{1.00,0.00,0.00}{\textbf{{#1}}}}
    \newcommand{\NormalTok}[1]{{#1}}
    
    % Additional commands for more recent versions of Pandoc
    \newcommand{\ConstantTok}[1]{\textcolor[rgb]{0.53,0.00,0.00}{{#1}}}
    \newcommand{\SpecialCharTok}[1]{\textcolor[rgb]{0.25,0.44,0.63}{{#1}}}
    \newcommand{\VerbatimStringTok}[1]{\textcolor[rgb]{0.25,0.44,0.63}{{#1}}}
    \newcommand{\SpecialStringTok}[1]{\textcolor[rgb]{0.73,0.40,0.53}{{#1}}}
    \newcommand{\ImportTok}[1]{{#1}}
    \newcommand{\DocumentationTok}[1]{\textcolor[rgb]{0.73,0.13,0.13}{\textit{{#1}}}}
    \newcommand{\AnnotationTok}[1]{\textcolor[rgb]{0.38,0.63,0.69}{\textbf{\textit{{#1}}}}}
    \newcommand{\CommentVarTok}[1]{\textcolor[rgb]{0.38,0.63,0.69}{\textbf{\textit{{#1}}}}}
    \newcommand{\VariableTok}[1]{\textcolor[rgb]{0.10,0.09,0.49}{{#1}}}
    \newcommand{\ControlFlowTok}[1]{\textcolor[rgb]{0.00,0.44,0.13}{\textbf{{#1}}}}
    \newcommand{\OperatorTok}[1]{\textcolor[rgb]{0.40,0.40,0.40}{{#1}}}
    \newcommand{\BuiltInTok}[1]{{#1}}
    \newcommand{\ExtensionTok}[1]{{#1}}
    \newcommand{\PreprocessorTok}[1]{\textcolor[rgb]{0.74,0.48,0.00}{{#1}}}
    \newcommand{\AttributeTok}[1]{\textcolor[rgb]{0.49,0.56,0.16}{{#1}}}
    \newcommand{\InformationTok}[1]{\textcolor[rgb]{0.38,0.63,0.69}{\textbf{\textit{{#1}}}}}
    \newcommand{\WarningTok}[1]{\textcolor[rgb]{0.38,0.63,0.69}{\textbf{\textit{{#1}}}}}
    
    
    % Define a nice break command that doesn't care if a line doesn't already
    % exist.
    \def\br{\hspace*{\fill} \\* }
    % Math Jax compatibility definitions
    \def\gt{>}
    \def\lt{<}
    \let\Oldtex\TeX
    \let\Oldlatex\LaTeX
    \renewcommand{\TeX}{\textrm{\Oldtex}}
    \renewcommand{\LaTeX}{\textrm{\Oldlatex}}
    % Document parameters
    % Document title
    \title{biology}
    
    
    
    
    
% Pygments definitions
\makeatletter
\def\PY@reset{\let\PY@it=\relax \let\PY@bf=\relax%
    \let\PY@ul=\relax \let\PY@tc=\relax%
    \let\PY@bc=\relax \let\PY@ff=\relax}
\def\PY@tok#1{\csname PY@tok@#1\endcsname}
\def\PY@toks#1+{\ifx\relax#1\empty\else%
    \PY@tok{#1}\expandafter\PY@toks\fi}
\def\PY@do#1{\PY@bc{\PY@tc{\PY@ul{%
    \PY@it{\PY@bf{\PY@ff{#1}}}}}}}
\def\PY#1#2{\PY@reset\PY@toks#1+\relax+\PY@do{#2}}

\expandafter\def\csname PY@tok@w\endcsname{\def\PY@tc##1{\textcolor[rgb]{0.73,0.73,0.73}{##1}}}
\expandafter\def\csname PY@tok@c\endcsname{\let\PY@it=\textit\def\PY@tc##1{\textcolor[rgb]{0.25,0.50,0.50}{##1}}}
\expandafter\def\csname PY@tok@cp\endcsname{\def\PY@tc##1{\textcolor[rgb]{0.74,0.48,0.00}{##1}}}
\expandafter\def\csname PY@tok@k\endcsname{\let\PY@bf=\textbf\def\PY@tc##1{\textcolor[rgb]{0.00,0.50,0.00}{##1}}}
\expandafter\def\csname PY@tok@kp\endcsname{\def\PY@tc##1{\textcolor[rgb]{0.00,0.50,0.00}{##1}}}
\expandafter\def\csname PY@tok@kt\endcsname{\def\PY@tc##1{\textcolor[rgb]{0.69,0.00,0.25}{##1}}}
\expandafter\def\csname PY@tok@o\endcsname{\def\PY@tc##1{\textcolor[rgb]{0.40,0.40,0.40}{##1}}}
\expandafter\def\csname PY@tok@ow\endcsname{\let\PY@bf=\textbf\def\PY@tc##1{\textcolor[rgb]{0.67,0.13,1.00}{##1}}}
\expandafter\def\csname PY@tok@nb\endcsname{\def\PY@tc##1{\textcolor[rgb]{0.00,0.50,0.00}{##1}}}
\expandafter\def\csname PY@tok@nf\endcsname{\def\PY@tc##1{\textcolor[rgb]{0.00,0.00,1.00}{##1}}}
\expandafter\def\csname PY@tok@nc\endcsname{\let\PY@bf=\textbf\def\PY@tc##1{\textcolor[rgb]{0.00,0.00,1.00}{##1}}}
\expandafter\def\csname PY@tok@nn\endcsname{\let\PY@bf=\textbf\def\PY@tc##1{\textcolor[rgb]{0.00,0.00,1.00}{##1}}}
\expandafter\def\csname PY@tok@ne\endcsname{\let\PY@bf=\textbf\def\PY@tc##1{\textcolor[rgb]{0.82,0.25,0.23}{##1}}}
\expandafter\def\csname PY@tok@nv\endcsname{\def\PY@tc##1{\textcolor[rgb]{0.10,0.09,0.49}{##1}}}
\expandafter\def\csname PY@tok@no\endcsname{\def\PY@tc##1{\textcolor[rgb]{0.53,0.00,0.00}{##1}}}
\expandafter\def\csname PY@tok@nl\endcsname{\def\PY@tc##1{\textcolor[rgb]{0.63,0.63,0.00}{##1}}}
\expandafter\def\csname PY@tok@ni\endcsname{\let\PY@bf=\textbf\def\PY@tc##1{\textcolor[rgb]{0.60,0.60,0.60}{##1}}}
\expandafter\def\csname PY@tok@na\endcsname{\def\PY@tc##1{\textcolor[rgb]{0.49,0.56,0.16}{##1}}}
\expandafter\def\csname PY@tok@nt\endcsname{\let\PY@bf=\textbf\def\PY@tc##1{\textcolor[rgb]{0.00,0.50,0.00}{##1}}}
\expandafter\def\csname PY@tok@nd\endcsname{\def\PY@tc##1{\textcolor[rgb]{0.67,0.13,1.00}{##1}}}
\expandafter\def\csname PY@tok@s\endcsname{\def\PY@tc##1{\textcolor[rgb]{0.73,0.13,0.13}{##1}}}
\expandafter\def\csname PY@tok@sd\endcsname{\let\PY@it=\textit\def\PY@tc##1{\textcolor[rgb]{0.73,0.13,0.13}{##1}}}
\expandafter\def\csname PY@tok@si\endcsname{\let\PY@bf=\textbf\def\PY@tc##1{\textcolor[rgb]{0.73,0.40,0.53}{##1}}}
\expandafter\def\csname PY@tok@se\endcsname{\let\PY@bf=\textbf\def\PY@tc##1{\textcolor[rgb]{0.73,0.40,0.13}{##1}}}
\expandafter\def\csname PY@tok@sr\endcsname{\def\PY@tc##1{\textcolor[rgb]{0.73,0.40,0.53}{##1}}}
\expandafter\def\csname PY@tok@ss\endcsname{\def\PY@tc##1{\textcolor[rgb]{0.10,0.09,0.49}{##1}}}
\expandafter\def\csname PY@tok@sx\endcsname{\def\PY@tc##1{\textcolor[rgb]{0.00,0.50,0.00}{##1}}}
\expandafter\def\csname PY@tok@m\endcsname{\def\PY@tc##1{\textcolor[rgb]{0.40,0.40,0.40}{##1}}}
\expandafter\def\csname PY@tok@gh\endcsname{\let\PY@bf=\textbf\def\PY@tc##1{\textcolor[rgb]{0.00,0.00,0.50}{##1}}}
\expandafter\def\csname PY@tok@gu\endcsname{\let\PY@bf=\textbf\def\PY@tc##1{\textcolor[rgb]{0.50,0.00,0.50}{##1}}}
\expandafter\def\csname PY@tok@gd\endcsname{\def\PY@tc##1{\textcolor[rgb]{0.63,0.00,0.00}{##1}}}
\expandafter\def\csname PY@tok@gi\endcsname{\def\PY@tc##1{\textcolor[rgb]{0.00,0.63,0.00}{##1}}}
\expandafter\def\csname PY@tok@gr\endcsname{\def\PY@tc##1{\textcolor[rgb]{1.00,0.00,0.00}{##1}}}
\expandafter\def\csname PY@tok@ge\endcsname{\let\PY@it=\textit}
\expandafter\def\csname PY@tok@gs\endcsname{\let\PY@bf=\textbf}
\expandafter\def\csname PY@tok@gp\endcsname{\let\PY@bf=\textbf\def\PY@tc##1{\textcolor[rgb]{0.00,0.00,0.50}{##1}}}
\expandafter\def\csname PY@tok@go\endcsname{\def\PY@tc##1{\textcolor[rgb]{0.53,0.53,0.53}{##1}}}
\expandafter\def\csname PY@tok@gt\endcsname{\def\PY@tc##1{\textcolor[rgb]{0.00,0.27,0.87}{##1}}}
\expandafter\def\csname PY@tok@err\endcsname{\def\PY@bc##1{\setlength{\fboxsep}{0pt}\fcolorbox[rgb]{1.00,0.00,0.00}{1,1,1}{\strut ##1}}}
\expandafter\def\csname PY@tok@kc\endcsname{\let\PY@bf=\textbf\def\PY@tc##1{\textcolor[rgb]{0.00,0.50,0.00}{##1}}}
\expandafter\def\csname PY@tok@kd\endcsname{\let\PY@bf=\textbf\def\PY@tc##1{\textcolor[rgb]{0.00,0.50,0.00}{##1}}}
\expandafter\def\csname PY@tok@kn\endcsname{\let\PY@bf=\textbf\def\PY@tc##1{\textcolor[rgb]{0.00,0.50,0.00}{##1}}}
\expandafter\def\csname PY@tok@kr\endcsname{\let\PY@bf=\textbf\def\PY@tc##1{\textcolor[rgb]{0.00,0.50,0.00}{##1}}}
\expandafter\def\csname PY@tok@bp\endcsname{\def\PY@tc##1{\textcolor[rgb]{0.00,0.50,0.00}{##1}}}
\expandafter\def\csname PY@tok@fm\endcsname{\def\PY@tc##1{\textcolor[rgb]{0.00,0.00,1.00}{##1}}}
\expandafter\def\csname PY@tok@vc\endcsname{\def\PY@tc##1{\textcolor[rgb]{0.10,0.09,0.49}{##1}}}
\expandafter\def\csname PY@tok@vg\endcsname{\def\PY@tc##1{\textcolor[rgb]{0.10,0.09,0.49}{##1}}}
\expandafter\def\csname PY@tok@vi\endcsname{\def\PY@tc##1{\textcolor[rgb]{0.10,0.09,0.49}{##1}}}
\expandafter\def\csname PY@tok@vm\endcsname{\def\PY@tc##1{\textcolor[rgb]{0.10,0.09,0.49}{##1}}}
\expandafter\def\csname PY@tok@sa\endcsname{\def\PY@tc##1{\textcolor[rgb]{0.73,0.13,0.13}{##1}}}
\expandafter\def\csname PY@tok@sb\endcsname{\def\PY@tc##1{\textcolor[rgb]{0.73,0.13,0.13}{##1}}}
\expandafter\def\csname PY@tok@sc\endcsname{\def\PY@tc##1{\textcolor[rgb]{0.73,0.13,0.13}{##1}}}
\expandafter\def\csname PY@tok@dl\endcsname{\def\PY@tc##1{\textcolor[rgb]{0.73,0.13,0.13}{##1}}}
\expandafter\def\csname PY@tok@s2\endcsname{\def\PY@tc##1{\textcolor[rgb]{0.73,0.13,0.13}{##1}}}
\expandafter\def\csname PY@tok@sh\endcsname{\def\PY@tc##1{\textcolor[rgb]{0.73,0.13,0.13}{##1}}}
\expandafter\def\csname PY@tok@s1\endcsname{\def\PY@tc##1{\textcolor[rgb]{0.73,0.13,0.13}{##1}}}
\expandafter\def\csname PY@tok@mb\endcsname{\def\PY@tc##1{\textcolor[rgb]{0.40,0.40,0.40}{##1}}}
\expandafter\def\csname PY@tok@mf\endcsname{\def\PY@tc##1{\textcolor[rgb]{0.40,0.40,0.40}{##1}}}
\expandafter\def\csname PY@tok@mh\endcsname{\def\PY@tc##1{\textcolor[rgb]{0.40,0.40,0.40}{##1}}}
\expandafter\def\csname PY@tok@mi\endcsname{\def\PY@tc##1{\textcolor[rgb]{0.40,0.40,0.40}{##1}}}
\expandafter\def\csname PY@tok@il\endcsname{\def\PY@tc##1{\textcolor[rgb]{0.40,0.40,0.40}{##1}}}
\expandafter\def\csname PY@tok@mo\endcsname{\def\PY@tc##1{\textcolor[rgb]{0.40,0.40,0.40}{##1}}}
\expandafter\def\csname PY@tok@ch\endcsname{\let\PY@it=\textit\def\PY@tc##1{\textcolor[rgb]{0.25,0.50,0.50}{##1}}}
\expandafter\def\csname PY@tok@cm\endcsname{\let\PY@it=\textit\def\PY@tc##1{\textcolor[rgb]{0.25,0.50,0.50}{##1}}}
\expandafter\def\csname PY@tok@cpf\endcsname{\let\PY@it=\textit\def\PY@tc##1{\textcolor[rgb]{0.25,0.50,0.50}{##1}}}
\expandafter\def\csname PY@tok@c1\endcsname{\let\PY@it=\textit\def\PY@tc##1{\textcolor[rgb]{0.25,0.50,0.50}{##1}}}
\expandafter\def\csname PY@tok@cs\endcsname{\let\PY@it=\textit\def\PY@tc##1{\textcolor[rgb]{0.25,0.50,0.50}{##1}}}

\def\PYZbs{\char`\\}
\def\PYZus{\char`\_}
\def\PYZob{\char`\{}
\def\PYZcb{\char`\}}
\def\PYZca{\char`\^}
\def\PYZam{\char`\&}
\def\PYZlt{\char`\<}
\def\PYZgt{\char`\>}
\def\PYZsh{\char`\#}
\def\PYZpc{\char`\%}
\def\PYZdl{\char`\$}
\def\PYZhy{\char`\-}
\def\PYZsq{\char`\'}
\def\PYZdq{\char`\"}
\def\PYZti{\char`\~}
% for compatibility with earlier versions
\def\PYZat{@}
\def\PYZlb{[}
\def\PYZrb{]}
\makeatother


    % For linebreaks inside Verbatim environment from package fancyvrb. 
    \makeatletter
        \newbox\Wrappedcontinuationbox 
        \newbox\Wrappedvisiblespacebox 
        \newcommand*\Wrappedvisiblespace {\textcolor{red}{\textvisiblespace}} 
        \newcommand*\Wrappedcontinuationsymbol {\textcolor{red}{\llap{\tiny$\m@th\hookrightarrow$}}} 
        \newcommand*\Wrappedcontinuationindent {3ex } 
        \newcommand*\Wrappedafterbreak {\kern\Wrappedcontinuationindent\copy\Wrappedcontinuationbox} 
        % Take advantage of the already applied Pygments mark-up to insert 
        % potential linebreaks for TeX processing. 
        %        {, <, #, %, $, ' and ": go to next line. 
        %        _, }, ^, &, >, - and ~: stay at end of broken line. 
        % Use of \textquotesingle for straight quote. 
        \newcommand*\Wrappedbreaksatspecials {% 
            \def\PYGZus{\discretionary{\char`\_}{\Wrappedafterbreak}{\char`\_}}% 
            \def\PYGZob{\discretionary{}{\Wrappedafterbreak\char`\{}{\char`\{}}% 
            \def\PYGZcb{\discretionary{\char`\}}{\Wrappedafterbreak}{\char`\}}}% 
            \def\PYGZca{\discretionary{\char`\^}{\Wrappedafterbreak}{\char`\^}}% 
            \def\PYGZam{\discretionary{\char`\&}{\Wrappedafterbreak}{\char`\&}}% 
            \def\PYGZlt{\discretionary{}{\Wrappedafterbreak\char`\<}{\char`\<}}% 
            \def\PYGZgt{\discretionary{\char`\>}{\Wrappedafterbreak}{\char`\>}}% 
            \def\PYGZsh{\discretionary{}{\Wrappedafterbreak\char`\#}{\char`\#}}% 
            \def\PYGZpc{\discretionary{}{\Wrappedafterbreak\char`\%}{\char`\%}}% 
            \def\PYGZdl{\discretionary{}{\Wrappedafterbreak\char`\$}{\char`\$}}% 
            \def\PYGZhy{\discretionary{\char`\-}{\Wrappedafterbreak}{\char`\-}}% 
            \def\PYGZsq{\discretionary{}{\Wrappedafterbreak\textquotesingle}{\textquotesingle}}% 
            \def\PYGZdq{\discretionary{}{\Wrappedafterbreak\char`\"}{\char`\"}}% 
            \def\PYGZti{\discretionary{\char`\~}{\Wrappedafterbreak}{\char`\~}}% 
        } 
        % Some characters . , ; ? ! / are not pygmentized. 
        % This macro makes them "active" and they will insert potential linebreaks 
        \newcommand*\Wrappedbreaksatpunct {% 
            \lccode`\~`\.\lowercase{\def~}{\discretionary{\hbox{\char`\.}}{\Wrappedafterbreak}{\hbox{\char`\.}}}% 
            \lccode`\~`\,\lowercase{\def~}{\discretionary{\hbox{\char`\,}}{\Wrappedafterbreak}{\hbox{\char`\,}}}% 
            \lccode`\~`\;\lowercase{\def~}{\discretionary{\hbox{\char`\;}}{\Wrappedafterbreak}{\hbox{\char`\;}}}% 
            \lccode`\~`\:\lowercase{\def~}{\discretionary{\hbox{\char`\:}}{\Wrappedafterbreak}{\hbox{\char`\:}}}% 
            \lccode`\~`\?\lowercase{\def~}{\discretionary{\hbox{\char`\?}}{\Wrappedafterbreak}{\hbox{\char`\?}}}% 
            \lccode`\~`\!\lowercase{\def~}{\discretionary{\hbox{\char`\!}}{\Wrappedafterbreak}{\hbox{\char`\!}}}% 
            \lccode`\~`\/\lowercase{\def~}{\discretionary{\hbox{\char`\/}}{\Wrappedafterbreak}{\hbox{\char`\/}}}% 
            \catcode`\.\active
            \catcode`\,\active 
            \catcode`\;\active
            \catcode`\:\active
            \catcode`\?\active
            \catcode`\!\active
            \catcode`\/\active 
            \lccode`\~`\~ 	
        }
    \makeatother

    \let\OriginalVerbatim=\Verbatim
    \makeatletter
    \renewcommand{\Verbatim}[1][1]{%
        %\parskip\z@skip
        \sbox\Wrappedcontinuationbox {\Wrappedcontinuationsymbol}%
        \sbox\Wrappedvisiblespacebox {\FV@SetupFont\Wrappedvisiblespace}%
        \def\FancyVerbFormatLine ##1{\hsize\linewidth
            \vtop{\raggedright\hyphenpenalty\z@\exhyphenpenalty\z@
                \doublehyphendemerits\z@\finalhyphendemerits\z@
                \strut ##1\strut}%
        }%
        % If the linebreak is at a space, the latter will be displayed as visible
        % space at end of first line, and a continuation symbol starts next line.
        % Stretch/shrink are however usually zero for typewriter font.
        \def\FV@Space {%
            \nobreak\hskip\z@ plus\fontdimen3\font minus\fontdimen4\font
            \discretionary{\copy\Wrappedvisiblespacebox}{\Wrappedafterbreak}
            {\kern\fontdimen2\font}%
        }%
        
        % Allow breaks at special characters using \PYG... macros.
        \Wrappedbreaksatspecials
        % Breaks at punctuation characters . , ; ? ! and / need catcode=\active 	
        \OriginalVerbatim[#1,codes*=\Wrappedbreaksatpunct]%
    }
    \makeatother

    % Exact colors from NB
    \definecolor{incolor}{HTML}{303F9F}
    \definecolor{outcolor}{HTML}{D84315}
    \definecolor{cellborder}{HTML}{CFCFCF}
    \definecolor{cellbackground}{HTML}{F7F7F7}
    
    % prompt
    \makeatletter
    \newcommand{\boxspacing}{\kern\kvtcb@left@rule\kern\kvtcb@boxsep}
    \makeatother
    \newcommand{\prompt}[4]{
        \ttfamily\llap{{\color{#2}[#3]:\hspace{3pt}#4}}\vspace{-\baselineskip}
    }
    

    
    % Prevent overflowing lines due to hard-to-break entities
    \sloppy 
    % Setup hyperref package
    \hypersetup{
      breaklinks=true,  % so long urls are correctly broken across lines
      colorlinks=true,
      urlcolor=urlcolor,
      linkcolor=linkcolor,
      citecolor=citecolor,
      }
    % Slightly bigger margins than the latex defaults
    
    \geometry{verbose,tmargin=1in,bmargin=1in,lmargin=1in,rmargin=1in}
    
    

\begin{document}
    
    \maketitle
    
    

    
    \begin{tcolorbox}[breakable, size=fbox, boxrule=1pt, pad at break*=1mm,colback=cellbackground, colframe=cellborder]
\prompt{In}{incolor}{1}{\boxspacing}
\begin{Verbatim}[commandchars=\\\{\}]
\PY{k+kn}{from} \PY{n+nn}{math} \PY{k+kn}{import} \PY{n}{pi}\PY{p}{,} \PY{n}{sqrt}
\PY{k+kn}{from} \PY{n+nn}{scipy}\PY{n+nn}{.}\PY{n+nn}{optimize} \PY{k+kn}{import} \PY{n}{curve\PYZus{}fit}
\PY{k+kn}{import} \PY{n+nn}{numpy} \PY{k}{as} \PY{n+nn}{np}
\PY{k+kn}{import} \PY{n+nn}{matplotlib}\PY{n+nn}{.}\PY{n+nn}{pyplot} \PY{k}{as} \PY{n+nn}{plt}
\PY{k+kn}{from} \PY{n+nn}{scipy}\PY{n+nn}{.}\PY{n+nn}{constants} \PY{k+kn}{import} \PY{n}{elementary\PYZus{}charge}
\PY{k+kn}{from} \PY{n+nn}{pytexit} \PY{k+kn}{import} \PY{n}{py2tex}
\PY{k+kn}{from} \PY{n+nn}{scipy}\PY{n+nn}{.}\PY{n+nn}{stats} \PY{k+kn}{import} \PY{n}{gaussian\PYZus{}kde}
\PY{k+kn}{from} \PY{n+nn}{tabulate} \PY{k+kn}{import} \PY{n}{tabulate}

\PY{n}{MDa} \PY{o}{=} \PY{l+m+mf}{1.6605e\PYZhy{}21} \PY{c+c1}{\PYZsh{}kg}
\PY{n}{nm} \PY{o}{=} \PY{l+m+mf}{1e\PYZhy{}9}
\end{Verbatim}
\end{tcolorbox}

    \begin{tcolorbox}[breakable, size=fbox, boxrule=1pt, pad at break*=1mm,colback=cellbackground, colframe=cellborder]
\prompt{In}{incolor}{2}{\boxspacing}
\begin{Verbatim}[commandchars=\\\{\}]
\PY{c+c1}{\PYZsh{}[Lauffer 1944]: diameter 109 +/\PYZhy{} 16.65 nm}
\PY{c+c1}{\PYZsh{}[Ruigrok 1984]: diameter 115 +/\PYZhy{} 12 nm}

\PY{n}{mean\PYZus{}diameter} \PY{o}{=} \PY{p}{(}\PY{l+m+mf}{109.0}\PY{o}{+}\PY{l+m+mf}{115.0}\PY{p}{)}\PY{o}{/}\PY{l+m+mf}{2.0}

\PY{n}{sigma\PYZus{}diameter} \PY{o}{=} \PY{p}{(}\PY{l+m+mf}{16.65}\PY{o}{+}\PY{l+m+mi}{12}\PY{p}{)}\PY{o}{/}\PY{l+m+mf}{2.0} \PY{c+c1}{\PYZsh{}average}

\PY{c+c1}{\PYZsh{}[Ruigrok 1984]: mass 160 MDa +/\PYZhy{} 17 MDa}
\PY{n}{sigma\PYZus{}mass} \PY{o}{=} \PY{p}{(}\PY{l+m+mf}{17.0}\PY{o}{+}\PY{l+m+mf}{24.0}\PY{p}{)}\PY{o}{/}\PY{l+m+mf}{2.0}

\PY{n}{mean\PYZus{}mass} \PY{o}{=} \PY{l+m+mi}{160}
\end{Verbatim}
\end{tcolorbox}

    \begin{tcolorbox}[breakable, size=fbox, boxrule=1pt, pad at break*=1mm,colback=cellbackground, colframe=cellborder]
\prompt{In}{incolor}{3}{\boxspacing}
\begin{Verbatim}[commandchars=\\\{\}]
\PY{c+c1}{\PYZsh{}check if variation in  variation in volume corresponds to the variation in mass}
\PY{c+c1}{\PYZsh{} mass is only }
\PY{n}{avg\PYZus{}radius} \PY{o}{=} \PY{l+m+mf}{57.5} 
\PY{n}{one\PYZus{}sigma\PYZus{}pos\PYZus{}rad} \PY{o}{=} \PY{n}{avg\PYZus{}radius} \PY{o}{+} \PY{p}{(}\PY{n}{sigma\PYZus{}diameter}\PY{o}{/}\PY{l+m+mf}{2.0}\PY{p}{)}
\PY{n}{volume\PYZus{}sigma} \PY{o}{=} \PY{p}{(}\PY{p}{(}\PY{p}{(}\PY{l+m+mi}{4}\PY{o}{/}\PY{l+m+mf}{3.0}\PY{p}{)} \PY{o}{*} \PY{n}{pi} \PY{o}{*} \PY{p}{(}\PY{n}{one\PYZus{}sigma\PYZus{}pos\PYZus{}rad}\PY{p}{)}\PY{o}{*}\PY{o}{*}\PY{l+m+mf}{2.0}\PY{p}{)}\PY{o}{/}\PY{p}{(}\PY{p}{(}\PY{l+m+mi}{4}\PY{o}{/}\PY{l+m+mf}{3.0}\PY{p}{)} \PY{o}{*} \PY{n}{pi} \PY{o}{*} \PY{p}{(}\PY{n}{avg\PYZus{}radius}\PY{p}{)}\PY{o}{*}\PY{o}{*}\PY{l+m+mf}{2.0}\PY{p}{)}\PY{p}{)}\PY{o}{\PYZhy{}}\PY{l+m+mf}{1.0}
\PY{n}{volume\PYZus{}sigma}
\PY{n}{avg\PYZus{}mass} \PY{o}{=} \PY{l+m+mf}{75.0}
\PY{n}{vol\PYZus{}one\PYZus{}sigma\PYZus{}mass} \PY{o}{=} \PY{n}{avg\PYZus{}mass}\PY{o}{*}\PY{n}{volume\PYZus{}sigma}
\PY{n}{vol\PYZus{}one\PYZus{}sigma\PYZus{}mass}
\PY{c+c1}{\PYZsh{}it doesn\PYZsq{}t}
\end{Verbatim}
\end{tcolorbox}

            \begin{tcolorbox}[breakable, size=fbox, boxrule=.5pt, pad at break*=1mm, opacityfill=0]
\prompt{Out}{outcolor}{3}{\boxspacing}
\begin{Verbatim}[commandchars=\\\{\}]
19.84851961247638
\end{Verbatim}
\end{tcolorbox}
        
    We know the variation in the diameter and therefore that in the volume.
There is, surprisingly, almost no variation in the total spike mass, so
we subtract that first. We would expect \(\sigma = 19.8\) MDa variation
if volume and mass were proportional, and see \(\sigma = 17\) MDa of
variation.

We therefore assume that mass and volume are proportional.

    \begin{tcolorbox}[breakable, size=fbox, boxrule=1pt, pad at break*=1mm,colback=cellbackground, colframe=cellborder]
\prompt{In}{incolor}{ }{\boxspacing}
\begin{Verbatim}[commandchars=\\\{\}]

\end{Verbatim}
\end{tcolorbox}

    \begin{tcolorbox}[breakable, size=fbox, boxrule=1pt, pad at break*=1mm,colback=cellbackground, colframe=cellborder]
\prompt{In}{incolor}{4}{\boxspacing}
\begin{Verbatim}[commandchars=\\\{\}]
\PY{c+c1}{\PYZsh{}Extracted from [Li 2011] using https://apps.automeris.io/wpd/,}

\PY{c+c1}{\PYZsh{} count = np.array([2.9816513761467904,}
\PY{c+c1}{\PYZsh{} 6.0091743119266035,}
\PY{c+c1}{\PYZsh{} 15.045871559633028,}
\PY{c+c1}{\PYZsh{} 21.14678899082569,}
\PY{c+c1}{\PYZsh{} 17.93577981651376,}
\PY{c+c1}{\PYZsh{} 11.10091743119266,}
\PY{c+c1}{\PYZsh{} 11.972477064220184,}
\PY{c+c1}{\PYZsh{} 11.972477064220184,}
\PY{c+c1}{\PYZsh{} 15.458715596330272,}
\PY{c+c1}{\PYZsh{} 8.119266055045873,}
\PY{c+c1}{\PYZsh{} 19.174311926605505,}
\PY{c+c1}{\PYZsh{} 9.908256880733944,}
\PY{c+c1}{\PYZsh{} 6.834862385321103,}
\PY{c+c1}{\PYZsh{} 5.045871559633027,}
\PY{c+c1}{\PYZsh{} 7.018348623853214,}
\PY{c+c1}{\PYZsh{} 2.018348623853214,}
\PY{c+c1}{\PYZsh{} 1.0091743119266046])}

\PY{n}{breaking\PYZus{}force\PYZus{}count} \PY{o}{=} \PY{n}{np}\PY{o}{.}\PY{n}{array}\PY{p}{(}\PY{p}{[} \PY{l+m+mf}{3.}\PY{p}{,}  \PY{l+m+mf}{6.}\PY{p}{,} \PY{l+m+mf}{15.}\PY{p}{,} \PY{l+m+mf}{21.}\PY{p}{,} \PY{l+m+mf}{18.}\PY{p}{,} \PY{l+m+mf}{11.}\PY{p}{,} \PY{l+m+mf}{12.}\PY{p}{,} \PY{l+m+mf}{12.}\PY{p}{,} \PY{l+m+mf}{15.}\PY{p}{,}  \PY{l+m+mf}{8.}\PY{p}{,} \PY{l+m+mf}{19.}\PY{p}{,} \PY{l+m+mf}{10.}\PY{p}{,}  \PY{l+m+mf}{7.}\PY{p}{,}
        \PY{l+m+mf}{5.}\PY{p}{,}  \PY{l+m+mf}{7.}\PY{p}{,}  \PY{l+m+mf}{2.}\PY{p}{,}  \PY{l+m+mf}{1.}\PY{p}{]}\PY{p}{)} \PY{c+c1}{\PYZsh{}rounded}


\PY{n}{breaking\PYZus{}force\PYZus{}bins} \PY{o}{=} \PY{p}{[}\PY{l+m+mf}{0.24179620034542312}\PY{p}{,}
\PY{l+m+mf}{0.33851468048359246}\PY{p}{,}
\PY{l+m+mf}{0.4421416234887737}\PY{p}{,}
\PY{l+m+mf}{0.5423143350604491}\PY{p}{,}
\PY{l+m+mf}{0.6528497409326425}\PY{p}{,}
\PY{l+m+mf}{0.753022452504318}\PY{p}{,}
\PY{l+m+mf}{0.8186528497409327}\PY{p}{,}
\PY{l+m+mf}{0.9395509499136444}\PY{p}{,}
\PY{l+m+mf}{1.0466321243523315}\PY{p}{,}
\PY{l+m+mf}{1.143350604490501}\PY{p}{,}
\PY{l+m+mf}{1.2469775474956826}\PY{p}{,}
\PY{l+m+mf}{1.3506044905008636}\PY{p}{,}
\PY{l+m+mf}{1.4300518134715028}\PY{p}{,}
\PY{l+m+mf}{1.5474956822107082}\PY{p}{,}
\PY{l+m+mf}{1.6165803108808294}\PY{p}{,}
\PY{l+m+mf}{1.7305699481865287}\PY{p}{,}
\PY{l+m+mf}{1.9412780656303976}\PY{p}{]} \PY{c+c1}{\PYZsh{} bin midpoints}
\end{Verbatim}
\end{tcolorbox}

    \begin{tcolorbox}[breakable, size=fbox, boxrule=1pt, pad at break*=1mm,colback=cellbackground, colframe=cellborder]
\prompt{In}{incolor}{5}{\boxspacing}
\begin{Verbatim}[commandchars=\\\{\}]
\PY{n}{N} \PY{o}{=} \PY{l+m+mi}{5000}
\end{Verbatim}
\end{tcolorbox}

    \hypertarget{fitting-breaking-force-with-gaussian---doesnt-work-particularly-well}{%
\subsubsection{Fitting breaking force with gaussian - doesn't work
particularly
well}\label{fitting-breaking-force-with-gaussian---doesnt-work-particularly-well}}

    \begin{tcolorbox}[breakable, size=fbox, boxrule=1pt, pad at break*=1mm,colback=cellbackground, colframe=cellborder]
\prompt{In}{incolor}{6}{\boxspacing}
\begin{Verbatim}[commandchars=\\\{\}]
\PY{k}{def} \PY{n+nf}{gaussian}\PY{p}{(}\PY{n}{x}\PY{p}{,} \PY{n}{mean}\PY{p}{,} \PY{n}{amplitude}\PY{p}{,} \PY{n}{standard\PYZus{}deviation}\PY{p}{)}\PY{p}{:}
    \PY{k}{return} \PY{n}{amplitude} \PY{o}{*} \PY{n}{np}\PY{o}{.}\PY{n}{exp}\PY{p}{(} \PY{o}{\PYZhy{}} \PY{p}{(}\PY{p}{(}\PY{n}{x} \PY{o}{\PYZhy{}} \PY{n}{mean}\PY{p}{)} \PY{o}{/} \PY{n}{standard\PYZus{}deviation}\PY{p}{)} \PY{o}{*}\PY{o}{*} \PY{l+m+mi}{2}\PY{p}{)}

\PY{n}{popt}\PY{p}{,} \PY{n}{\PYZus{}} \PY{o}{=} \PY{n}{curve\PYZus{}fit}\PY{p}{(}\PY{n}{gaussian}\PY{p}{,} \PY{n}{breaking\PYZus{}force\PYZus{}bins}\PY{p}{,} \PY{n}{breaking\PYZus{}force\PYZus{}count}\PY{p}{)}
\PY{n}{mean\PYZus{}breaking\PYZus{}force} \PY{o}{=} \PY{n}{popt}\PY{p}{[}\PY{l+m+mi}{0}\PY{p}{]}
\PY{n}{sigma\PYZus{}breaking\PYZus{}force} \PY{o}{=} \PY{n+nb}{abs}\PY{p}{(}\PY{n}{popt}\PY{p}{[}\PY{l+m+mi}{2}\PY{p}{]}\PY{p}{)} \PY{c+c1}{\PYZsh{}curve\PYZus{}fit can sometimes return negative.}

\PY{n+nb}{print}\PY{p}{(}\PY{l+s+s2}{\PYZdq{}}\PY{l+s+s2}{Liposome breaking force mean=}\PY{l+s+si}{\PYZob{}:.3f\PYZcb{}}\PY{l+s+s2}{ sigma=}\PY{l+s+si}{\PYZob{}:.3f\PYZcb{}}\PY{l+s+s2}{ pN}\PY{l+s+s2}{\PYZdq{}}\PY{o}{.}\PY{n}{format}\PY{p}{(}\PY{n}{mean\PYZus{}breaking\PYZus{}force}\PY{p}{,} \PY{n}{sigma\PYZus{}breaking\PYZus{}force}\PY{p}{)}\PY{p}{)}

\PY{c+c1}{\PYZsh{}this could also include some of the non\PYZhy{}infective mutants}
\end{Verbatim}
\end{tcolorbox}

    \begin{Verbatim}[commandchars=\\\{\}]
Liposome breaking force mean=0.839 sigma=0.733 pN
    \end{Verbatim}

    \begin{tcolorbox}[breakable, size=fbox, boxrule=1pt, pad at break*=1mm,colback=cellbackground, colframe=cellborder]
\prompt{In}{incolor}{7}{\boxspacing}
\begin{Verbatim}[commandchars=\\\{\}]
\PY{c+c1}{\PYZsh{}overriding with tweaked values because non\PYZhy{}bimodal}
\PY{c+c1}{\PYZsh{} mean\PYZus{}breaking\PYZus{}force = 0.839}
\PY{c+c1}{\PYZsh{} sigma\PYZus{}breaking\PYZus{}force = 0.45}
\end{Verbatim}
\end{tcolorbox}

    \begin{tcolorbox}[breakable, size=fbox, boxrule=1pt, pad at break*=1mm,colback=cellbackground, colframe=cellborder]
\prompt{In}{incolor}{8}{\boxspacing}
\begin{Verbatim}[commandchars=\\\{\}]
\PY{n}{weights} \PY{o}{=} \PY{p}{(}\PY{n}{breaking\PYZus{}force\PYZus{}count}\PY{o}{/}\PY{n}{np}\PY{o}{.}\PY{n}{sum}\PY{p}{(}\PY{n}{breaking\PYZus{}force\PYZus{}count}\PY{p}{)}\PY{p}{)}
\PY{n}{resamples} \PY{o}{=} \PY{n}{np}\PY{o}{.}\PY{n}{random}\PY{o}{.}\PY{n}{choice}\PY{p}{(}\PY{n}{breaking\PYZus{}force\PYZus{}bins}\PY{p}{,} \PY{n}{size}\PY{o}{=}\PY{n}{N}\PY{p}{,} \PY{n}{p}\PY{o}{=}\PY{n}{weights}\PY{p}{)}

\PY{n}{breaking\PYZus{}force\PYZus{}kernel} \PY{o}{=} \PY{n}{gaussian\PYZus{}kde}\PY{p}{(}\PY{n}{resamples}\PY{p}{)}
\PY{c+c1}{\PYZsh{} breaking\PYZus{}force\PYZus{}kernel.set\PYZus{}bandwidth(0.5)}
\end{Verbatim}
\end{tcolorbox}

    \begin{tcolorbox}[breakable, size=fbox, boxrule=1pt, pad at break*=1mm,colback=cellbackground, colframe=cellborder]
\prompt{In}{incolor}{9}{\boxspacing}
\begin{Verbatim}[commandchars=\\\{\}]
\PY{n}{mean\PYZus{}charge} \PY{o}{=} \PY{l+m+mf}{1e7}
\PY{n}{sigma\PYZus{}charge} \PY{o}{=} \PY{l+m+mf}{1e5} \PY{c+c1}{\PYZsh{}placeholder \PYZhy{} fixme}
\end{Verbatim}
\end{tcolorbox}

    Now we run into a problem. We don't know the dependence of k, the spring
constant, on any of the parameters in an actual virus. It is determined
by the strength of the intermolecular forces and is a free parameter.

From {[}Li 2011{]}, Figure 3a we see that the empty liposome stiffness
is negatively correlated with size; but the stiffness of the empty
liposome appears to be hundreds of times smaller than that of the full
virion.

From running ensembles with different values of k\_sigma, while the
distribution of the resonant frequencies is very sensitive to this
parameter, the inactivation threshold is not; so we guess a resonable
figure.

Figure 4b in Yang is dependent on k vs f\_res, and the overall
absorption cross-section data in Figure 3b includes the sum of all the
absorption cross-sections of all the variance. It is likely possible to
extract the remaining data from these plots; but, since none of this is
applicable to SARS-NCoV, this exceeds our patience.

The upshot is that all the resonant frequency distributions below are
just guesses.

    \begin{tcolorbox}[breakable, size=fbox, boxrule=1pt, pad at break*=1mm,colback=cellbackground, colframe=cellborder]
\prompt{In}{incolor}{10}{\boxspacing}
\begin{Verbatim}[commandchars=\\\{\}]
\PY{n}{mean\PYZus{}mu} \PY{o}{=} \PY{l+m+mf}{14.5} \PY{o}{*} \PY{n}{MDa} 
\PY{n}{mean\PYZus{}k} \PY{o}{=} \PY{p}{(}\PY{p}{(}\PY{l+m+mf}{2.0} \PY{o}{*} \PY{n}{pi} \PY{o}{*} \PY{l+m+mf}{8.2e9}\PY{p}{)}\PY{o}{*}\PY{o}{*}\PY{l+m+mf}{2.0}\PY{p}{)} \PY{o}{*} \PY{n}{mean\PYZus{}mu} \PY{c+c1}{\PYZsh{}from [Yang]}

\PY{n}{sigma\PYZus{}spring\PYZus{}constant} \PY{o}{=} \PY{p}{(}\PY{l+m+mf}{0.07}\PY{o}{*}\PY{n}{mean\PYZus{}k}\PY{p}{)}\PY{o}{/}\PY{n}{sigma\PYZus{}diameter} \PY{c+c1}{\PYZsh{} nN / nm}
\end{Verbatim}
\end{tcolorbox}

    \begin{tcolorbox}[breakable, size=fbox, boxrule=1pt, pad at break*=1mm,colback=cellbackground, colframe=cellborder]
\prompt{In}{incolor}{11}{\boxspacing}
\begin{Verbatim}[commandchars=\\\{\}]
\PY{n}{means} \PY{o}{=} \PY{p}{[}\PY{n}{mean\PYZus{}diameter}\PY{p}{,} \PY{n}{mean\PYZus{}mass}\PY{p}{,} \PY{n}{mean\PYZus{}breaking\PYZus{}force}\PY{p}{,} \PY{n}{mean\PYZus{}charge}\PY{p}{,} \PY{n}{mean\PYZus{}k}\PY{p}{]} \PY{c+c1}{\PYZsh{}oh he\PYZsq{}s got the means all right.}

\PY{n}{var\PYZus{}diameter} \PY{o}{=} \PY{p}{(}\PY{n}{sigma\PYZus{}diameter}\PY{o}{*}\PY{o}{*}\PY{l+m+mf}{2.0}\PY{p}{)} \PY{c+c1}{\PYZsh{}variance is the square of the standard deviation.}
\PY{n}{var\PYZus{}mass} \PY{o}{=} \PY{p}{(}\PY{n}{sigma\PYZus{}mass}\PY{o}{*}\PY{o}{*}\PY{l+m+mf}{2.0}\PY{p}{)} \PY{c+c1}{\PYZsh{}inertia is a property of matter.}
\PY{n}{var\PYZus{}breaking\PYZus{}force} \PY{o}{=} \PY{p}{(}\PY{n}{sigma\PYZus{}breaking\PYZus{}force}\PY{o}{*}\PY{o}{*}\PY{l+m+mf}{2.0}\PY{p}{)}
\PY{n}{var\PYZus{}charge} \PY{o}{=} \PY{p}{(}\PY{n}{sigma\PYZus{}charge}\PY{o}{*}\PY{o}{*}\PY{l+m+mf}{2.0}\PY{p}{)}
\PY{n}{var\PYZus{}spring\PYZus{}constant} \PY{o}{=} \PY{p}{(}\PY{n}{sigma\PYZus{}spring\PYZus{}constant}\PY{o}{*}\PY{o}{*}\PY{l+m+mf}{2.0}\PY{p}{)}


\PY{c+c1}{\PYZsh{}diameter varies with diameter, varies with mass, with breaking force, and with charge}
\PY{n}{cov\PYZus{}diameter} \PY{o}{=}        \PY{p}{[}\PY{n}{var\PYZus{}diameter}\PY{p}{,} \PY{n}{sigma\PYZus{}diameter}\PY{o}{*}\PY{n}{sigma\PYZus{}mass}\PY{o}{*}\PY{l+m+mf}{0.95}\PY{p}{,} \PY{l+m+mi}{0}\PY{p}{,} \PY{l+m+mi}{0}\PY{p}{,} \PY{l+m+mi}{0}\PY{p}{]}

\PY{n}{cov\PYZus{}mass} \PY{o}{=}            \PY{p}{[}\PY{n}{sigma\PYZus{}diameter}\PY{o}{*}\PY{n}{sigma\PYZus{}mass}\PY{o}{*}\PY{l+m+mf}{0.95}\PY{p}{,} \PY{n}{var\PYZus{}mass}\PY{p}{,} \PY{l+m+mi}{0}\PY{p}{,} \PY{l+m+mi}{0}\PY{p}{,} \PY{l+m+mi}{0}\PY{p}{]}

\PY{n}{cov\PYZus{}breaking\PYZus{}force} \PY{o}{=}  \PY{p}{[}\PY{l+m+mi}{0}\PY{p}{,} \PY{l+m+mi}{0}\PY{p}{,} \PY{n}{var\PYZus{}breaking\PYZus{}force}\PY{p}{,} \PY{l+m+mi}{0}\PY{p}{,} \PY{l+m+mi}{0}\PY{p}{]}

\PY{n}{cov\PYZus{}charge} \PY{o}{=}          \PY{p}{[}\PY{l+m+mi}{0}\PY{p}{,} \PY{l+m+mi}{0}\PY{p}{,} \PY{l+m+mi}{0}\PY{p}{,} \PY{n}{var\PYZus{}charge}\PY{p}{,} \PY{l+m+mi}{0}\PY{p}{]}

\PY{n}{cov\PYZus{}spring\PYZus{}constant} \PY{o}{=} \PY{p}{[}\PY{l+m+mi}{0}\PY{p}{,} \PY{l+m+mi}{0}\PY{p}{,} \PY{l+m+mi}{0}\PY{p}{,} \PY{l+m+mi}{0}\PY{p}{,} \PY{n}{var\PYZus{}spring\PYZus{}constant}\PY{p}{]}


\PY{n}{covariance\PYZus{}matrix} \PY{o}{=} \PY{p}{[}\PY{n}{cov\PYZus{}diameter}\PY{p}{,} 
                     \PY{n}{cov\PYZus{}mass}\PY{p}{,}
                     \PY{n}{cov\PYZus{}breaking\PYZus{}force}\PY{p}{,}
                     \PY{n}{cov\PYZus{}charge}\PY{p}{,}
                     \PY{n}{cov\PYZus{}spring\PYZus{}constant}\PY{p}{]}




\PY{n}{covariance\PYZus{}matrix} \PY{o}{=} \PY{n}{np}\PY{o}{.}\PY{n}{array}\PY{p}{(}\PY{n}{covariance\PYZus{}matrix}\PY{p}{)} \PY{c+c1}{\PYZsh{}must be symmetric positive semidefinite;}
\PY{c+c1}{\PYZsh{}eigenvalues must be }

\PY{n}{diameter\PYZus{}samples}\PY{p}{,}\PY{n}{mass\PYZus{}samples}\PY{p}{,}\PY{n}{breaking\PYZus{}force\PYZus{}samples}\PY{p}{,}\PY{n}{charge\PYZus{}samples}\PY{p}{,} \PY{n}{spring\PYZus{}constant\PYZus{}samples}\PYZbs{}
                \PY{o}{=} \PY{n}{np}\PY{o}{.}\PY{n}{random}\PY{o}{.}\PY{n}{multivariate\PYZus{}normal}\PY{p}{(}\PY{n}{means}\PY{p}{,} \PY{n}{covariance\PYZus{}matrix}\PY{p}{,} \PY{n}{N}\PY{p}{)}\PY{o}{.}\PY{n}{T}


\PY{n}{diameter\PYZus{}samples} \PY{o}{=} \PY{n}{diameter\PYZus{}samples}\PY{o}{.}\PY{n}{T} \PY{c+c1}{\PYZsh{}first .T lets us access with ,, operator, second .T gets us our 1d array back}
\PY{n}{mass\PYZus{}samples} \PY{o}{=} \PY{n}{mass\PYZus{}samples}\PY{o}{.}\PY{n}{T}
\PY{n}{breaking\PYZus{}force\PYZus{}samples} \PY{o}{=} \PY{n}{breaking\PYZus{}force\PYZus{}samples}\PY{o}{.}\PY{n}{T} 
\PY{n}{charge\PYZus{}samples} \PY{o}{=} \PY{n}{charge\PYZus{}samples}\PY{o}{.}\PY{n}{T} 
\PY{n}{spring\PYZus{}constant\PYZus{}samples} \PY{o}{=} \PY{n}{spring\PYZus{}constant\PYZus{}samples}\PY{o}{.}\PY{n}{T} 
\end{Verbatim}
\end{tcolorbox}

    The liposome breaking force is distributed in an obstinately
non-gaussian bimodal manner, so we override the covariance matrix and
introduce a gaussian KDE sampled from the force histogram in {[}Li
2011{]}.

    \begin{tcolorbox}[breakable, size=fbox, boxrule=1pt, pad at break*=1mm,colback=cellbackground, colframe=cellborder]
\prompt{In}{incolor}{12}{\boxspacing}
\begin{Verbatim}[commandchars=\\\{\}]
\PY{c+c1}{\PYZsh{}ignore the above covariant normal breaking forces, resample from KDE}
\PY{n}{breaking\PYZus{}force\PYZus{}samples} \PY{o}{=} \PY{n}{gaussian\PYZus{}kde}\PY{o}{.}\PY{n}{resample}\PY{p}{(}\PY{n}{breaking\PYZus{}force\PYZus{}kernel}\PY{p}{,} \PY{n}{N}\PY{p}{)}\PY{o}{.}\PY{n}{T}
\end{Verbatim}
\end{tcolorbox}

    \begin{tcolorbox}[breakable, size=fbox, boxrule=1pt, pad at break*=1mm,colback=cellbackground, colframe=cellborder]
\prompt{In}{incolor}{13}{\boxspacing}
\begin{Verbatim}[commandchars=\\\{\}]
\PY{c+c1}{\PYZsh{}choose only infectious particles [Lauffer 1944]}
\PY{n}{remove\PYZus{}indices} \PY{o}{=} \PY{n}{np}\PY{o}{.}\PY{n}{append}\PY{p}{(}\PY{n}{np}\PY{o}{.}\PY{n}{nonzero}\PY{p}{(}\PY{p}{(}\PY{n}{diameter\PYZus{}samples} \PY{o}{\PYZlt{}} \PY{l+m+mf}{80.0}\PY{p}{)}\PY{p}{)}\PY{p}{,} \PY{n}{np}\PY{o}{.}\PY{n}{nonzero}\PY{p}{(}\PY{p}{(}\PY{n}{diameter\PYZus{}samples} \PY{o}{\PYZgt{}} \PY{l+m+mf}{135.0}\PY{p}{)}\PY{p}{)}\PY{p}{)}

\PY{c+c1}{\PYZsh{}choose only positive breaking forces}
\PY{n}{remove\PYZus{}indices} \PY{o}{=} \PY{n}{np}\PY{o}{.}\PY{n}{append}\PY{p}{(}\PY{n}{remove\PYZus{}indices}\PY{p}{,} \PY{n}{np}\PY{o}{.}\PY{n}{nonzero}\PY{p}{(}\PY{n}{breaking\PYZus{}force\PYZus{}samples} \PY{o}{\PYZlt{}} \PY{l+m+mf}{0.0}\PY{p}{)}\PY{p}{)}

\PY{n}{remove\PYZus{}indices} \PY{o}{=} \PY{n}{np}\PY{o}{.}\PY{n}{unique}\PY{p}{(}\PY{n}{remove\PYZus{}indices}\PY{p}{)}

\PY{n}{diameter\PYZus{}samples} \PY{o}{=} \PY{n}{np}\PY{o}{.}\PY{n}{delete}\PY{p}{(}\PY{n}{diameter\PYZus{}samples}\PY{p}{,} \PY{n}{remove\PYZus{}indices}\PY{p}{)}
\PY{n}{mass\PYZus{}samples} \PY{o}{=} \PY{n}{np}\PY{o}{.}\PY{n}{delete}\PY{p}{(}\PY{n}{mass\PYZus{}samples}\PY{p}{,} \PY{n}{remove\PYZus{}indices}\PY{p}{)}
\PY{n}{breaking\PYZus{}force\PYZus{}samples} \PY{o}{=} \PY{n}{np}\PY{o}{.}\PY{n}{delete}\PY{p}{(}\PY{n}{breaking\PYZus{}force\PYZus{}samples}\PY{p}{,} \PY{n}{remove\PYZus{}indices}\PY{p}{)}
\PY{n}{charge\PYZus{}samples} \PY{o}{=} \PY{n}{np}\PY{o}{.}\PY{n}{delete}\PY{p}{(}\PY{n}{charge\PYZus{}samples}\PY{p}{,} \PY{n}{remove\PYZus{}indices}\PY{p}{)}
\PY{n}{spring\PYZus{}constant\PYZus{}samples} \PY{o}{=} \PY{n}{np}\PY{o}{.}\PY{n}{delete}\PY{p}{(}\PY{n}{spring\PYZus{}constant\PYZus{}samples}\PY{p}{,} \PY{n}{remove\PYZus{}indices}\PY{p}{)}


\PY{n}{plt}\PY{o}{.}\PY{n}{plot}\PY{p}{(}\PY{n}{diameter\PYZus{}samples}\PY{p}{,}\PY{n}{mass\PYZus{}samples}\PY{p}{,} \PY{l+s+s1}{\PYZsq{}}\PY{l+s+s1}{.}\PY{l+s+s1}{\PYZsq{}}\PY{p}{)}
\PY{n}{plt}\PY{o}{.}\PY{n}{figure}\PY{p}{(}\PY{p}{)}
\PY{n}{plt}\PY{o}{.}\PY{n}{plot}\PY{p}{(}\PY{n}{diameter\PYZus{}samples}\PY{p}{,}\PY{n}{breaking\PYZus{}force\PYZus{}samples}\PY{p}{,} \PY{l+s+s1}{\PYZsq{}}\PY{l+s+s1}{.}\PY{l+s+s1}{\PYZsq{}}\PY{p}{)}

\PY{c+c1}{\PYZsh{}\PYZgt{} \PYZdq{}More than 95\PYZpc{} puncture events occurred above 0.4 nN.\PYZdq{}}
\PY{n}{above\PYZus{}point\PYZus{}4} \PY{o}{=} \PY{n}{np}\PY{o}{.}\PY{n}{shape}\PY{p}{(}\PY{n}{np}\PY{o}{.}\PY{n}{where}\PY{p}{(}\PY{n}{breaking\PYZus{}force\PYZus{}samples} \PY{o}{\PYZgt{}} \PY{l+m+mf}{0.4}\PY{p}{)}\PY{p}{)}\PY{p}{[}\PY{l+m+mi}{1}\PY{p}{]} \PY{o}{/} \PY{n+nb}{len}\PY{p}{(}\PY{n}{breaking\PYZus{}force\PYZus{}samples}\PY{p}{)}
\PY{c+c1}{\PYZsh{} probably because of the clipped normal. np multivariate doesn\PYZsq{}t support beyond normal,}
\PY{c+c1}{\PYZsh{} so that\PYZsq{}ll have to be good enough.}
\PY{c+c1}{\PYZsh{} altered distribution to better fit breaking force; now it\PYZsq{}s 92\PYZpc{}}

\PY{n}{plt}\PY{o}{.}\PY{n}{figure}\PY{p}{(}\PY{p}{)}
\PY{n}{plt}\PY{o}{.}\PY{n}{hist}\PY{p}{(}\PY{n}{breaking\PYZus{}force\PYZus{}samples}\PY{p}{,}\PY{l+m+mi}{20}\PY{p}{)}

\PY{n}{plt}\PY{o}{.}\PY{n}{figure}\PY{p}{(}\PY{p}{)}
\PY{n}{plt}\PY{o}{.}\PY{n}{plot}\PY{p}{(}\PY{n}{diameter\PYZus{}samples}\PY{p}{,} \PY{n}{spring\PYZus{}constant\PYZus{}samples}\PY{p}{,} \PY{l+s+s1}{\PYZsq{}}\PY{l+s+s1}{.}\PY{l+s+s1}{\PYZsq{}}\PY{p}{)}

\PY{n}{above\PYZus{}point\PYZus{}4}
\end{Verbatim}
\end{tcolorbox}

            \begin{tcolorbox}[breakable, size=fbox, boxrule=.5pt, pad at break*=1mm, opacityfill=0]
\prompt{Out}{outcolor}{13}{\boxspacing}
\begin{Verbatim}[commandchars=\\\{\}]
0.9249410250911431
\end{Verbatim}
\end{tcolorbox}
        
    \begin{center}
    \adjustimage{max size={0.9\linewidth}{0.9\paperheight}}{output_17_1.png}
    \end{center}
    { \hspace*{\fill} \\}
    
    \begin{center}
    \adjustimage{max size={0.9\linewidth}{0.9\paperheight}}{output_17_2.png}
    \end{center}
    { \hspace*{\fill} \\}
    
    \begin{center}
    \adjustimage{max size={0.9\linewidth}{0.9\paperheight}}{output_17_3.png}
    \end{center}
    { \hspace*{\fill} \\}
    
    \begin{center}
    \adjustimage{max size={0.9\linewidth}{0.9\paperheight}}{output_17_4.png}
    \end{center}
    { \hspace*{\fill} \\}
    
    \begin{figure}
\centering
\includegraphics{attachment:Screenshot\%20from\%202020-06-29\%2013-11-21.png}
\caption{Screenshot\%20from\%202020-06-29\%2013-11-21.png}
\end{figure}

    \begin{tcolorbox}[breakable, size=fbox, boxrule=1pt, pad at break*=1mm,colback=cellbackground, colframe=cellborder]
\prompt{In}{incolor}{14}{\boxspacing}
\begin{Verbatim}[commandchars=\\\{\}]
\PY{c+c1}{\PYZsh{} convert everything to base SI}

\PY{n}{diameter\PYZus{}samples} \PY{o}{*}\PY{o}{=} \PY{l+m+mf}{1e\PYZhy{}9} \PY{c+c1}{\PYZsh{} nm}
\PY{n}{mass\PYZus{}samples} \PY{o}{*}\PY{o}{=} \PY{n}{MDa} \PY{c+c1}{\PYZsh{}MDa}
\PY{n}{breaking\PYZus{}force\PYZus{}samples} \PY{o}{*}\PY{o}{=} \PY{l+m+mf}{1e\PYZhy{}9} \PY{c+c1}{\PYZsh{}nN}
\PY{n}{charge\PYZus{}samples} \PY{o}{*}\PY{o}{=} \PY{n}{elementary\PYZus{}charge}

\PY{c+c1}{\PYZsh{} afm pressure}
\PY{n}{afm\PYZus{}tip\PYZus{}radius} \PY{o}{=} \PY{l+m+mf}{30e\PYZhy{}9} \PY{c+c1}{\PYZsh{}m}
\PY{n}{afm\PYZus{}area} \PY{o}{=} \PY{n}{pi}\PY{o}{*}\PY{p}{(}\PY{n}{afm\PYZus{}tip\PYZus{}radius}\PY{o}{*}\PY{o}{*}\PY{l+m+mf}{2.0}\PY{p}{)}

\PY{n}{breaking\PYZus{}stress\PYZus{}samples} \PY{o}{=} \PY{n}{breaking\PYZus{}force\PYZus{}samples} \PY{o}{/} \PY{n}{afm\PYZus{}area} \PY{c+c1}{\PYZsh{} now to Pa}
\end{Verbatim}
\end{tcolorbox}

    We very simply extend {[}Yang 2015{]}'s analysis to include the
distribution of viral particles.

We have to continue using Inf. A rather than the target SARS-NCoV-2
because of an apparent lack of AFM nanoindentation data. While both
virions have a lipid bilayer

\begin{verbatim}
The majority of virions (n = 78) may be described as spherical (axial ratio 1.2), the remainder are oval or   kidney-shaped with axial ratios up to 1.4 (n = 17) or more elongated with axial ratios as high as 7.7 (n = 15). The outer diameters of the spherical virions ranged from 84 to 170 nm (mean, 120 nm).
\end{verbatim}

Note that, per {[}Lauffer 1944{]}, of Inf A and B, ``the infectious
particles have diameters within the range 80 to 135{[}nm{]}''.

{[}Yang 2015{]}'s MDCK plaque assay is already sensitive to infectivity;
the extreme-sized non-infective mutants would already be filtered out.
Their PCR assay might not be; the DNA of all non-infectious mutants
would be considered.

{[}Yang 2015{]} use a minimum force of 400 pN, though their reference,
{[}Li 2011{]}, mention that

\begin{quote}
More than 95\% puncture events occurred above 0.4 nN.
\end{quote}

This probably accounts for the high 100\% threshold.

The distribution of breaking strengths in {[}Li 2011{]} (Figure 5b) is
bimodal, and a naive gaussian fit does not produce the above 95\%
figure. Rather than implement a unimodal+bimodal covariant sampling
function, we tweaked the gaussian parameters to accordingly. We hope
that the deities of statistical best practice are not displeased.

The mass, diameter, and breaking force are presumably not independently
distributed. The correlations between the three variables do not seem to
be documented in the literature.

{[}Li 2011{]}, the AFM data is from {[}H1N1{]} rather than the H3N1 or
H5N1.

We therefore need the CoV for NCoV.

There seems to be little variation in the mass of the spike proteins;
intact H3N1 is 161 MDa +/- 17, with spikes removed, 75 MDa +/- 17.
{[}Ruigrok 1984{]}

Influenza is found in two different morphologies; the one found in the
lab is almost entirely the spherical variant, whereas filamentous
variant is often associated with infections in humans. Because it is
rare in the lab, this is not part of the cause of {[}Yang 2015{]}'s
discrepancy; and NCoV does not seem to exhibit such pleomorphic
tendencies.

The average stiffness of the liposome varies approximately linearly from
0.025 to 0.015 from 80 to 135 nm {[}Li 2011{]}, plus a random
distribution with approx. 0.0025 sigma.

Size and weight are assumed to be uncorrelated with breaking strength.

From {[}Yang 2015{]}, eq. 7,

\[A = \frac{q E_0}{\mu \sqrt{((2 \pi f_{res})^2 - (2 \pi f_{excite})^2)^2 + \left(\frac{(2 \pi f_{res})(2 \pi f_{excite})}{Q}\right)^2}}\]

The resonant frequency varies as \(\sqrt{k/m^*}\).

Assuming (as Yang et al do) that the core and shell have 90\% and 10\%
of the mass,

    \begin{tcolorbox}[breakable, size=fbox, boxrule=1pt, pad at break*=1mm,colback=cellbackground, colframe=cellborder]
\prompt{In}{incolor}{15}{\boxspacing}
\begin{Verbatim}[commandchars=\\\{\}]
\PY{n}{reduced\PYZus{}masses\PYZus{}mu} \PY{o}{=} \PY{p}{(}\PY{p}{(}\PY{l+m+mf}{0.9}\PY{o}{*}\PY{n}{mass\PYZus{}samples}\PY{p}{)} \PY{o}{*} \PY{p}{(}\PY{l+m+mf}{0.1}\PY{o}{*}\PY{n}{mass\PYZus{}samples}\PY{p}{)}\PY{p}{)}\PY{o}{/}\PY{p}{(}\PY{n}{mass\PYZus{}samples}\PY{p}{)} 
\end{Verbatim}
\end{tcolorbox}

    \begin{tcolorbox}[breakable, size=fbox, boxrule=1pt, pad at break*=1mm,colback=cellbackground, colframe=cellborder]
\prompt{In}{incolor}{16}{\boxspacing}
\begin{Verbatim}[commandchars=\\\{\}]
\PY{n}{resonant\PYZus{}frequencies} \PY{o}{=} \PY{n}{np}\PY{o}{.}\PY{n}{sqrt}\PY{p}{(}\PY{n}{spring\PYZus{}constant\PYZus{}samples}\PY{o}{/}\PY{n}{reduced\PYZus{}masses\PYZus{}mu}\PY{p}{)}\PY{o}{/}\PY{p}{(}\PY{l+m+mf}{2.0}\PY{o}{*}\PY{n}{pi}\PY{p}{)}
\end{Verbatim}
\end{tcolorbox}

    \begin{tcolorbox}[breakable, size=fbox, boxrule=1pt, pad at break*=1mm,colback=cellbackground, colframe=cellborder]
\prompt{In}{incolor}{17}{\boxspacing}
\begin{Verbatim}[commandchars=\\\{\}]
\PY{n}{plt}\PY{o}{.}\PY{n}{hist}\PY{p}{(}\PY{n}{resonant\PYZus{}frequencies}\PY{p}{,} \PY{l+m+mi}{50}\PY{p}{)}
\PY{n}{plt}\PY{o}{.}\PY{n}{title}\PY{p}{(}\PY{l+s+s2}{\PYZdq{}}\PY{l+s+s2}{Predicted distribution of resonant frequencies of infective Inf. A}\PY{l+s+s2}{\PYZdq{}}\PY{p}{)}
\PY{n}{plt}\PY{o}{.}\PY{n}{ticklabel\PYZus{}format}\PY{p}{(}\PY{n}{style}\PY{o}{=}\PY{l+s+s1}{\PYZsq{}}\PY{l+s+s1}{sci}\PY{l+s+s1}{\PYZsq{}}\PY{p}{,} \PY{n}{axis}\PY{o}{=}\PY{l+s+s1}{\PYZsq{}}\PY{l+s+s1}{x}\PY{l+s+s1}{\PYZsq{}}\PY{p}{,} \PY{n}{scilimits}\PY{o}{=}\PY{p}{(}\PY{l+m+mi}{9}\PY{p}{,}\PY{l+m+mi}{9}\PY{p}{)}\PY{p}{)}
\PY{n}{plt}\PY{o}{.}\PY{n}{xlabel}\PY{p}{(}\PY{l+s+s2}{\PYZdq{}}\PY{l+s+s2}{Frequency}\PY{l+s+s2}{\PYZdq{}}\PY{p}{)}
\PY{n}{plt}\PY{o}{.}\PY{n}{ylabel}\PY{p}{(}\PY{l+s+s2}{\PYZdq{}}\PY{l+s+s2}{Count}\PY{l+s+s2}{\PYZdq{}}\PY{p}{)}
\end{Verbatim}
\end{tcolorbox}

            \begin{tcolorbox}[breakable, size=fbox, boxrule=.5pt, pad at break*=1mm, opacityfill=0]
\prompt{Out}{outcolor}{17}{\boxspacing}
\begin{Verbatim}[commandchars=\\\{\}]
Text(0, 0.5, 'Count')
\end{Verbatim}
\end{tcolorbox}
        
    \begin{center}
    \adjustimage{max size={0.9\linewidth}{0.9\paperheight}}{output_23_1.png}
    \end{center}
    { \hspace*{\fill} \\}
    
    Unfortunately, this distribution is very sensitive to the values of k

    A surprising source the sadists that drag cracked virions through
acrylamide gel and then irradiate them for sport.

Two dimensional gel electrophoresis resolves both the mass and the
isoelectric point (where the net charge and that of the solution are in
equilibrium) of large structures. {[}Privalsky 1978{]} provides us a
coarse. view of the charges and relative abundances in the virus.

    \begin{tcolorbox}[breakable, size=fbox, boxrule=1pt, pad at break*=1mm,colback=cellbackground, colframe=cellborder]
\prompt{In}{incolor}{18}{\boxspacing}
\begin{Verbatim}[commandchars=\\\{\}]
\PY{k+kn}{from} \PY{n+nn}{PIL} \PY{k+kn}{import} \PY{n}{Image}

\PY{n}{im} \PY{o}{=} \PY{n}{Image}\PY{o}{.}\PY{n}{open}\PY{p}{(}\PY{l+s+s1}{\PYZsq{}}\PY{l+s+s1}{../media/Inf\PYZus{}A\PYZus{}2d\PYZus{}gel\PYZus{}Privalsky.png}\PY{l+s+s1}{\PYZsq{}}\PY{p}{)} \PY{c+c1}{\PYZsh{} Can be many different formats.}
\PY{n}{pix} \PY{o}{=} \PY{n}{im}\PY{o}{.}\PY{n}{load}\PY{p}{(}\PY{p}{)}


\PY{n}{plt}\PY{o}{.}\PY{n}{imshow}\PY{p}{(}\PY{n}{im}\PY{p}{)}
\PY{n}{plt}\PY{o}{.}\PY{n}{title}\PY{p}{(}\PY{l+s+s2}{\PYZdq{}}\PY{l+s+s2}{2d gel electrophoresis [Privalsky 1978]}\PY{l+s+s2}{\PYZdq{}}\PY{p}{)}
\PY{n}{plt}\PY{o}{.}\PY{n}{xlabel}\PY{p}{(}\PY{l+s+s2}{\PYZdq{}}\PY{l+s+s2}{Isoelectric point (proportional to net charge)}\PY{l+s+se}{\PYZbs{}n}\PY{l+s+s2}{ \PYZlt{}\PYZhy{} Basic | Neutral | Acid \PYZhy{}\PYZgt{}}\PY{l+s+s2}{\PYZdq{}}\PY{p}{)}
\PY{n}{plt}\PY{o}{.}\PY{n}{ylabel}\PY{p}{(}\PY{l+s+s2}{\PYZdq{}}\PY{l+s+s2}{Molecular weight}\PY{l+s+s2}{\PYZdq{}}\PY{p}{)}

\PY{l+s+s2}{\PYZdq{}}\PY{l+s+s2}{NP has MW 60000 Da}\PY{l+s+s2}{\PYZdq{}}
\end{Verbatim}
\end{tcolorbox}

            \begin{tcolorbox}[breakable, size=fbox, boxrule=.5pt, pad at break*=1mm, opacityfill=0]
\prompt{Out}{outcolor}{18}{\boxspacing}
\begin{Verbatim}[commandchars=\\\{\}]
'NP has MW 60000 Da'
\end{Verbatim}
\end{tcolorbox}
        
    \begin{center}
    \adjustimage{max size={0.9\linewidth}{0.9\paperheight}}{output_26_1.png}
    \end{center}
    { \hspace*{\fill} \\}
    
    There's substantial hetrogeneity in the charge of the proteins with the
exception of the nucleocapsid protein NP. The M proteins have run off
the basic end. The followup work {[}Privalsky 1980{]} indeed finds the M
protein to be very basic. {[}Shtykova 2017{]} discusses the charge on
the M1 matrix protein.

confusingly, there are two different M proteins, sometimes called ``M''
and sometimes called ``M1''.

{[}Bosch 1984{]} discusses the charge hetrogenetity a bit, but not quite
relevant.

    \begin{figure}
\centering
\includegraphics{attachment:influenza-virion3.jpg}
\caption{influenza-virion3.jpg}
\end{figure}

From https://www.virology.ws/2009/04/30/structure-of-influenza-virus/,
CC-BY-3.0

    \begin{figure}
\centering
\includegraphics{attachment:512px-Coronavirus_virion_structure.svg.png}
\caption{512px-Coronavirus\_virion\_structure.svg.png}
\end{figure}

https://commons.wikimedia.org/wiki/File:Coronavirus\_virion\_structure.svg

    \begin{tcolorbox}[breakable, size=fbox, boxrule=1pt, pad at break*=1mm,colback=cellbackground, colframe=cellborder]
\prompt{In}{incolor}{ }{\boxspacing}
\begin{Verbatim}[commandchars=\\\{\}]

\end{Verbatim}
\end{tcolorbox}

    \begin{tcolorbox}[breakable, size=fbox, boxrule=1pt, pad at break*=1mm,colback=cellbackground, colframe=cellborder]
\prompt{In}{incolor}{19}{\boxspacing}
\begin{Verbatim}[commandchars=\\\{\}]
\PY{n}{excitation\PYZus{}frequency} \PY{o}{=} \PY{l+m+mf}{8.3e9}

\PY{n}{Q} \PY{o}{=} \PY{l+m+mf}{2.0} \PY{c+c1}{\PYZsh{} also assumed to be invariant}
\end{Verbatim}
\end{tcolorbox}

    

    \begin{tcolorbox}[breakable, size=fbox, boxrule=1pt, pad at break*=1mm,colback=cellbackground, colframe=cellborder]
\prompt{In}{incolor}{ }{\boxspacing}
\begin{Verbatim}[commandchars=\\\{\}]

\end{Verbatim}
\end{tcolorbox}

    \begin{tcolorbox}[breakable, size=fbox, boxrule=1pt, pad at break*=1mm,colback=cellbackground, colframe=cellborder]
\prompt{In}{incolor}{20}{\boxspacing}
\begin{Verbatim}[commandchars=\\\{\}]
\PY{k}{def} \PY{n+nf}{percentiles}\PY{p}{(}\PY{n}{inp}\PY{p}{)}\PY{p}{:}
    \PY{k}{for} \PY{n}{i} \PY{o+ow}{in} \PY{p}{[}\PY{l+m+mi}{6}\PY{p}{,} \PY{l+m+mi}{25}\PY{p}{,} \PY{l+m+mi}{38}\PY{p}{,} \PY{l+m+mi}{50}\PY{p}{,} \PY{l+m+mi}{63}\PY{p}{,} \PY{l+m+mi}{75}\PY{p}{,} \PY{l+m+mi}{90}\PY{p}{,} \PY{l+m+mf}{99.9}\PY{p}{]}\PY{p}{:}
        
        \PY{n+nb}{print}\PY{p}{(}\PY{l+s+s2}{\PYZdq{}}\PY{l+s+si}{\PYZob{}\PYZcb{}}\PY{l+s+s2}{th percentile: }\PY{l+s+si}{\PYZob{}:2f\PYZcb{}}\PY{l+s+s2}{\PYZdq{}}\PY{o}{.}\PY{n}{format}\PY{p}{(}\PY{n}{i}\PY{p}{,} \PY{n}{np}\PY{o}{.}\PY{n}{percentile}\PY{p}{(}\PY{n}{inp}\PY{p}{,} \PY{n}{i}\PY{p}{,} \PY{n}{axis}\PY{o}{=}\PY{l+m+mi}{0}\PY{p}{)}\PY{p}{)}\PY{p}{)}
\end{Verbatim}
\end{tcolorbox}

    \begin{tcolorbox}[breakable, size=fbox, boxrule=1pt, pad at break*=1mm,colback=cellbackground, colframe=cellborder]
\prompt{In}{incolor}{21}{\boxspacing}
\begin{Verbatim}[commandchars=\\\{\}]
\PY{n}{af} \PY{o}{=} \PY{l+m+mf}{2.0}\PY{o}{*}\PY{n}{pi}

\PY{k}{def} \PY{n+nf}{electric\PYZus{}field\PYZus{}threshold}\PY{p}{(}\PY{n}{i}\PY{p}{,} \PY{n}{excitation\PYZus{}frequency}\PY{p}{)}\PY{p}{:}
    \PY{n}{t1} \PY{o}{=} \PY{n}{breaking\PYZus{}stress\PYZus{}samples}\PY{p}{[}\PY{n}{i}\PY{p}{]} \PY{o}{*} \PY{n}{pi} \PY{o}{*} \PY{p}{(}\PY{p}{(}\PY{n}{diameter\PYZus{}samples}\PY{p}{[}\PY{n}{i}\PY{p}{]} \PY{o}{/} \PY{l+m+mf}{2.0}\PY{p}{)}\PY{o}{*}\PY{o}{*}\PY{l+m+mf}{2.0}\PY{p}{)}

    \PY{n}{t2} \PY{o}{=} \PY{n}{np}\PY{o}{.}\PY{n}{sqrt}\PY{p}{(}\PY{p}{(}\PY{n}{reduced\PYZus{}masses\PYZus{}mu}\PY{p}{[}\PY{n}{i}\PY{p}{]}\PY{o}{*}\PY{o}{*}\PY{l+m+mf}{2.0}\PY{p}{)} \PY{o}{*} \PY{p}{(}\PY{p}{(}\PY{p}{(}\PY{p}{(}\PY{p}{(}\PY{n}{resonant\PYZus{}frequencies}\PY{p}{[}\PY{n}{i}\PY{p}{]}\PY{o}{*}\PY{n}{af}\PY{p}{)}\PY{o}{*}\PY{o}{*}\PY{l+m+mf}{2.0}\PY{p}{)}\PY{p}{)} \PY{o}{\PYZhy{}} \PY{p}{(}\PY{p}{(}\PY{n}{excitation\PYZus{}frequency}\PY{o}{*}\PY{n}{af}\PY{p}{)}\PY{o}{*}\PY{o}{*}\PY{l+m+mf}{2.0}\PY{p}{)}\PY{p}{)}\PY{o}{*}\PY{o}{*}\PY{l+m+mf}{2.0}\PY{p}{)}\PYZbs{}
            \PY{o}{+} \PY{p}{(}\PY{p}{(}\PY{p}{(}\PY{p}{(}\PY{n}{resonant\PYZus{}frequencies}\PY{p}{[}\PY{n}{i}\PY{p}{]}\PY{o}{*}\PY{n}{af}\PY{p}{)}\PY{o}{*}\PY{n}{reduced\PYZus{}masses\PYZus{}mu}\PY{p}{[}\PY{n}{i}\PY{p}{]}\PY{p}{)}\PY{o}{/}\PY{n}{Q}\PY{p}{)}\PY{o}{*}\PY{o}{*}\PY{l+m+mf}{2.0}\PY{p}{)} \PY{o}{*} \PY{p}{(}\PY{p}{(}\PY{n}{excitation\PYZus{}frequency}\PY{o}{*}\PY{n}{af}\PY{p}{)}\PY{o}{*}\PY{o}{*}\PY{l+m+mf}{2.0}\PY{p}{)}\PY{p}{)}
    \PY{n}{b} \PY{o}{=} \PY{l+m+mf}{3.45} \PY{o}{*} \PY{n}{charge\PYZus{}samples}\PY{p}{[}\PY{n}{i}\PY{p}{]} \PY{o}{*} \PY{n}{reduced\PYZus{}masses\PYZus{}mu}\PY{p}{[}\PY{n}{i}\PY{p}{]} \PY{o}{*} \PY{p}{(}\PY{p}{(}\PY{n}{resonant\PYZus{}frequencies}\PY{p}{[}\PY{n}{i}\PY{p}{]}\PY{o}{*}\PY{n}{af}\PY{p}{)}\PY{p}{)}\PY{o}{*}\PY{o}{*}\PY{l+m+mf}{2.0}

    \PY{n}{electric\PYZus{}field\PYZus{}thresholds} \PY{o}{=} \PY{p}{(}\PY{n}{t1} \PY{o}{*} \PY{n}{t2}\PY{p}{)} \PY{o}{/} \PY{n}{b}
    
    \PY{k}{return} \PY{n}{electric\PYZus{}field\PYZus{}thresholds}

\PY{c+c1}{\PYZsh{}eq 12 in [Yang 2015]}
\PY{k}{def} \PY{n+nf}{electric\PYZus{}field\PYZus{}thresholds}\PY{p}{(}\PY{n}{excitation\PYZus{}frequency}\PY{p}{)}\PY{p}{:}

    \PY{n}{t1} \PY{o}{=} \PY{n}{breaking\PYZus{}stress\PYZus{}samples} \PY{o}{*} \PY{n}{pi} \PY{o}{*} \PY{p}{(}\PY{p}{(}\PY{n}{diameter\PYZus{}samples} \PY{o}{/} \PY{l+m+mf}{2.0}\PY{p}{)}\PY{o}{*}\PY{o}{*}\PY{l+m+mf}{2.0}\PY{p}{)}

    \PY{n}{t2} \PY{o}{=} \PY{n}{np}\PY{o}{.}\PY{n}{sqrt}\PY{p}{(}\PY{p}{(}\PY{n}{reduced\PYZus{}masses\PYZus{}mu}\PY{o}{*}\PY{o}{*}\PY{l+m+mf}{2.0}\PY{p}{)} \PY{o}{*} \PY{p}{(}\PY{p}{(}\PY{p}{(}\PY{p}{(}\PY{p}{(}\PY{n}{resonant\PYZus{}frequencies}\PY{o}{*}\PY{n}{af}\PY{p}{)}\PY{o}{*}\PY{o}{*}\PY{l+m+mf}{2.0}\PY{p}{)}\PY{p}{)} \PY{o}{\PYZhy{}} \PY{p}{(}\PY{p}{(}\PY{n}{excitation\PYZus{}frequency}\PY{o}{*}\PY{n}{af}\PY{p}{)}\PY{o}{*}\PY{o}{*}\PY{l+m+mf}{2.0}\PY{p}{)}\PY{p}{)}\PY{o}{*}\PY{o}{*}\PY{l+m+mf}{2.0}\PY{p}{)}\PYZbs{}
            \PY{o}{+} \PY{p}{(}\PY{p}{(}\PY{p}{(}\PY{p}{(}\PY{n}{resonant\PYZus{}frequencies}\PY{o}{*}\PY{n}{af}\PY{p}{)}\PY{o}{*}\PY{n}{reduced\PYZus{}masses\PYZus{}mu}\PY{p}{)}\PY{o}{/}\PY{n}{Q}\PY{p}{)}\PY{o}{*}\PY{o}{*}\PY{l+m+mf}{2.0}\PY{p}{)} \PY{o}{*} \PY{p}{(}\PY{p}{(}\PY{n}{excitation\PYZus{}frequency}\PY{o}{*}\PY{n}{af}\PY{p}{)}\PY{o}{*}\PY{o}{*}\PY{l+m+mf}{2.0}\PY{p}{)}\PY{p}{)}
    \PY{n}{b} \PY{o}{=} \PY{l+m+mf}{3.45} \PY{o}{*} \PY{n}{charge\PYZus{}samples} \PY{o}{*} \PY{n}{reduced\PYZus{}masses\PYZus{}mu} \PY{o}{*} \PY{p}{(}\PY{p}{(}\PY{n}{resonant\PYZus{}frequencies}\PY{o}{*}\PY{n}{af}\PY{p}{)}\PY{p}{)}\PY{o}{*}\PY{o}{*}\PY{l+m+mf}{2.0}

    \PY{n}{electric\PYZus{}field\PYZus{}thresholds} \PY{o}{=} \PY{p}{(}\PY{n}{t1} \PY{o}{*} \PY{n}{t2}\PY{p}{)} \PY{o}{/} \PY{n}{b}
    
    \PY{k}{return} \PY{n}{electric\PYZus{}field\PYZus{}thresholds}

\PY{n}{single\PYZus{}freq\PYZus{}elecric\PYZus{}field\PYZus{}thresholds} \PY{o}{=} \PY{n}{electric\PYZus{}field\PYZus{}thresholds}\PY{p}{(}\PY{n}{excitation\PYZus{}frequency}\PY{p}{)}

\PY{n}{plt}\PY{o}{.}\PY{n}{title}\PY{p}{(}\PY{l+s+s2}{\PYZdq{}}\PY{l+s+s2}{Predicted distribution of electric fields required to inactivate infective Inf. A}\PY{l+s+se}{\PYZbs{}n}\PY{l+s+s2}{\PYZdq{}}\PY{o}{+}\PYZbs{}
        \PY{l+s+s2}{\PYZdq{}}\PY{l+s+s2}{given excitation at a single frequency}\PY{l+s+s2}{\PYZdq{}}\PY{p}{)}
\PY{n}{plt}\PY{o}{.}\PY{n}{hist}\PY{p}{(}\PY{n}{single\PYZus{}freq\PYZus{}elecric\PYZus{}field\PYZus{}thresholds}\PY{p}{,} \PY{l+m+mi}{50}\PY{p}{)}
\PY{n}{plt}\PY{o}{.}\PY{n}{xlabel}\PY{p}{(}\PY{l+s+s2}{\PYZdq{}}\PY{l+s+s2}{Electric field (V/m)}\PY{l+s+s2}{\PYZdq{}}\PY{p}{)}
\PY{n}{plt}\PY{o}{.}\PY{n}{ylabel}\PY{p}{(}\PY{l+s+s2}{\PYZdq{}}\PY{l+s+s2}{Count}\PY{l+s+s2}{\PYZdq{}}\PY{p}{)}
\PY{n}{percentiles}\PY{p}{(}\PY{n}{single\PYZus{}freq\PYZus{}elecric\PYZus{}field\PYZus{}thresholds}\PY{p}{)}

\PY{c+c1}{\PYZsh{} Yang get:}
\PY{c+c1}{\PYZsh{} 100\PYZpc{}: 275 V/m}
\PY{c+c1}{\PYZsh{} 63\PYZpc{}: 171 V/m}
\PY{c+c1}{\PYZsh{} 38\PYZpc{}: 87 V/m}
\PY{c+c1}{\PYZsh{} 6\PYZpc{}: 68 V/m}
\end{Verbatim}
\end{tcolorbox}

    \begin{Verbatim}[commandchars=\\\{\}]
6th percentile: 104.519073
25th percentile: 178.738256
38th percentile: 226.608581
50th percentile: 272.961741
63th percentile: 329.662413
75th percentile: 395.087502
90th percentile: 512.626157
99.9th percentile: 870.175943
    \end{Verbatim}

    \begin{center}
    \adjustimage{max size={0.9\linewidth}{0.9\paperheight}}{output_35_1.png}
    \end{center}
    { \hspace*{\fill} \\}
    
    IEEE et al appear to specify power densities as those in free space, a
small distance from the body:

    \[S = \frac{|E|^2}{Z_0}\]

    \begin{tcolorbox}[breakable, size=fbox, boxrule=1pt, pad at break*=1mm,colback=cellbackground, colframe=cellborder]
\prompt{In}{incolor}{22}{\boxspacing}
\begin{Verbatim}[commandchars=\\\{\}]
\PY{n}{single\PYZus{}freq\PYZus{}power\PYZus{}densities} \PY{o}{=} \PY{p}{(}\PY{n}{single\PYZus{}freq\PYZus{}elecric\PYZus{}field\PYZus{}thresholds}\PY{o}{*}\PY{o}{*}\PY{l+m+mf}{2.0}\PY{p}{)} \PY{o}{/} \PY{l+m+mf}{377.0}

\PY{n}{plt}\PY{o}{.}\PY{n}{title}\PY{p}{(}\PY{l+s+s2}{\PYZdq{}}\PY{l+s+s2}{Predicted distribution of power densities required to inactivate infective Inf. A}\PY{l+s+se}{\PYZbs{}n}\PY{l+s+s2}{\PYZdq{}}\PY{o}{+}\PYZbs{}
        \PY{l+s+s2}{\PYZdq{}}\PY{l+s+s2}{given excitation at a single frequency}\PY{l+s+s2}{\PYZdq{}}\PY{p}{)}
\PY{n}{plt}\PY{o}{.}\PY{n}{hist}\PY{p}{(}\PY{n}{single\PYZus{}freq\PYZus{}power\PYZus{}densities}\PY{p}{,} \PY{l+m+mi}{50}\PY{p}{)}
\PY{n}{plt}\PY{o}{.}\PY{n}{xlabel}\PY{p}{(}\PY{l+s+s2}{\PYZdq{}}\PY{l+s+s2}{Power density (W/m\PYZca{}2)}\PY{l+s+s2}{\PYZdq{}}\PY{p}{)}
\PY{n}{plt}\PY{o}{.}\PY{n}{ylabel}\PY{p}{(}\PY{l+s+s2}{\PYZdq{}}\PY{l+s+s2}{Count}\PY{l+s+s2}{\PYZdq{}}\PY{p}{)}
\PY{n}{percentiles}\PY{p}{(}\PY{n}{single\PYZus{}freq\PYZus{}power\PYZus{}densities}\PY{p}{)}
\end{Verbatim}
\end{tcolorbox}

    \begin{Verbatim}[commandchars=\\\{\}]
6th percentile: 28.976758
25th percentile: 84.741023
38th percentile: 136.210741
50th percentile: 197.634249
63th percentile: 288.268717
75th percentile: 414.042797
90th percentile: 697.044008
99.9th percentile: 2008.529495
    \end{Verbatim}

    \begin{center}
    \adjustimage{max size={0.9\linewidth}{0.9\paperheight}}{output_38_1.png}
    \end{center}
    { \hspace*{\fill} \\}
    
    \begin{tcolorbox}[breakable, size=fbox, boxrule=1pt, pad at break*=1mm,colback=cellbackground, colframe=cellborder]
\prompt{In}{incolor}{23}{\boxspacing}
\begin{Verbatim}[commandchars=\\\{\}]
\PY{k}{for} \PY{n}{i} \PY{o+ow}{in} \PY{n}{np}\PY{o}{.}\PY{n}{random}\PY{o}{.}\PY{n}{choice}\PY{p}{(}\PY{n+nb}{range}\PY{p}{(}\PY{l+m+mi}{0}\PY{p}{,}\PY{n+nb}{len}\PY{p}{(}\PY{n}{resonant\PYZus{}frequencies}\PY{p}{)}\PY{p}{)}\PY{p}{,} \PY{n}{size} \PY{o}{=} \PY{l+m+mi}{20}\PY{p}{)}\PY{p}{:}

    \PY{n}{freqs} \PY{o}{=} \PY{n}{np}\PY{o}{.}\PY{n}{linspace}\PY{p}{(}\PY{l+m+mf}{1e9}\PY{p}{,}\PY{l+m+mf}{15e9}\PY{p}{)}
    \PY{n}{plt}\PY{o}{.}\PY{n}{plot}\PY{p}{(}\PY{n}{freqs}\PY{p}{,}\PY{n}{electric\PYZus{}field\PYZus{}threshold}\PY{p}{(}\PY{n}{i}\PY{p}{,} \PY{n}{freqs}\PY{p}{)}\PY{p}{,}\PY{l+s+s1}{\PYZsq{}}\PY{l+s+s1}{b}\PY{l+s+s1}{\PYZsq{}}\PY{p}{)}
    \PY{n}{plt}\PY{o}{.}\PY{n}{title}\PY{p}{(}\PY{l+s+s2}{\PYZdq{}}\PY{l+s+s2}{Electric field threshold of a sample of virions against excitation frequency}\PY{l+s+s2}{\PYZdq{}}\PY{p}{)}
    \PY{n}{plt}\PY{o}{.}\PY{n}{ticklabel\PYZus{}format}\PY{p}{(}\PY{n}{style}\PY{o}{=}\PY{l+s+s1}{\PYZsq{}}\PY{l+s+s1}{sci}\PY{l+s+s1}{\PYZsq{}}\PY{p}{,} \PY{n}{axis}\PY{o}{=}\PY{l+s+s1}{\PYZsq{}}\PY{l+s+s1}{x}\PY{l+s+s1}{\PYZsq{}}\PY{p}{,} \PY{n}{scilimits}\PY{o}{=}\PY{p}{(}\PY{l+m+mi}{9}\PY{p}{,}\PY{l+m+mi}{9}\PY{p}{)}\PY{p}{)}
    \PY{n}{plt}\PY{o}{.}\PY{n}{xlabel}\PY{p}{(}\PY{l+s+s2}{\PYZdq{}}\PY{l+s+s2}{Frequency (GHz)}\PY{l+s+s2}{\PYZdq{}}\PY{p}{)}
    \PY{n}{plt}\PY{o}{.}\PY{n}{ylabel}\PY{p}{(}\PY{l+s+s2}{\PYZdq{}}\PY{l+s+s2}{Theshold (V/m)}\PY{l+s+s2}{\PYZdq{}}\PY{p}{)}
    
    \PY{n}{optimal\PYZus{}excitation\PYZus{}frequency} \PY{o}{=} \PY{p}{(}\PY{n}{sqrt}\PY{p}{(}\PY{l+m+mf}{2.0} \PY{o}{*} \PY{n}{Q}\PY{o}{*}\PY{o}{*}\PY{l+m+mf}{2.0} \PY{o}{\PYZhy{}} \PY{l+m+mf}{1.0}\PY{p}{)} \PY{o}{*} \PY{p}{(}\PY{l+m+mf}{2.0}\PY{o}{*}\PY{n}{pi}\PY{o}{*}\PY{n}{resonant\PYZus{}frequencies}\PY{p}{[}\PY{n}{i}\PY{p}{]}\PY{p}{)}\PY{p}{)} \PY{o}{/} \PY{p}{(}\PY{p}{(}\PY{n}{sqrt}\PY{p}{(}\PY{l+m+mf}{2.0}\PY{p}{)}\PY{o}{*}\PY{n}{Q}\PY{p}{)}\PY{p}{)} \PY{o}{/} \PY{p}{(}\PY{l+m+mf}{2.0}\PY{o}{*}\PY{n}{pi}\PY{p}{)}
    \PY{n+nb}{print}\PY{p}{(}\PY{n}{optimal\PYZus{}excitation\PYZus{}frequency}\PY{p}{)}
\end{Verbatim}
\end{tcolorbox}

    \begin{Verbatim}[commandchars=\\\{\}]
7279284906.563067
7494657688.343927
7060094218.977478
7880199441.398486
7887076651.1556635
8391826356.039286
7687065355.281846
7674184760.335817
7851298608.500301
8105547455.286027
7543804559.687341
8037100367.448972
7135882289.73944
7124796958.367642
7740345854.416245
7688243255.571538
7908508087.189577
7891495836.990037
7598584570.812886
7735221662.281465
    \end{Verbatim}

    \begin{center}
    \adjustimage{max size={0.9\linewidth}{0.9\paperheight}}{output_39_1.png}
    \end{center}
    { \hspace*{\fill} \\}
    
    The resonant frequency does not necessarily correspond to the optimal
excitation frequency. The positive critical point of (12) is

\[w = \frac{\sqrt{2 Q^2 - 1} f_{res}}{\sqrt{2} \ Q}\]

    \begin{tcolorbox}[breakable, size=fbox, boxrule=1pt, pad at break*=1mm,colback=cellbackground, colframe=cellborder]
\prompt{In}{incolor}{24}{\boxspacing}
\begin{Verbatim}[commandchars=\\\{\}]
\PY{n}{optimal\PYZus{}excitation\PYZus{}frequencies} \PY{o}{=} \PY{p}{(}\PY{n}{sqrt}\PY{p}{(}\PY{l+m+mf}{2.0} \PY{o}{*} \PY{n}{Q}\PY{o}{*}\PY{o}{*}\PY{l+m+mf}{2.0} \PY{o}{\PYZhy{}} \PY{l+m+mf}{1.0}\PY{p}{)} \PY{o}{*} \PY{p}{(}\PY{l+m+mf}{2.0}\PY{o}{*}\PY{n}{pi}\PY{o}{*}\PY{n}{resonant\PYZus{}frequencies}\PY{p}{)}\PY{p}{)} \PY{o}{/} \PY{p}{(}\PY{p}{(}\PY{n}{sqrt}\PY{p}{(}\PY{l+m+mf}{2.0}\PY{p}{)}\PY{o}{*}\PY{n}{Q}\PY{p}{)}\PY{p}{)} \PY{o}{/} \PY{p}{(}\PY{l+m+mf}{2.0}\PY{o}{*}\PY{n}{pi}\PY{p}{)}

\PY{n}{plt}\PY{o}{.}\PY{n}{subplot}\PY{p}{(}\PY{p}{)}
\PY{n}{plt}\PY{o}{.}\PY{n}{hist}\PY{p}{(}\PY{n}{optimal\PYZus{}excitation\PYZus{}frequencies}\PY{p}{,} \PY{l+m+mi}{50}\PY{p}{)}
\PY{n}{plt}\PY{o}{.}\PY{n}{gca}\PY{p}{(}\PY{p}{)}\PY{o}{.}\PY{n}{ticklabel\PYZus{}format}\PY{p}{(}\PY{n}{style}\PY{o}{=}\PY{l+s+s1}{\PYZsq{}}\PY{l+s+s1}{sci}\PY{l+s+s1}{\PYZsq{}}\PY{p}{,} \PY{n}{axis}\PY{o}{=}\PY{l+s+s1}{\PYZsq{}}\PY{l+s+s1}{x}\PY{l+s+s1}{\PYZsq{}}\PY{p}{,} \PY{n}{scilimits}\PY{o}{=}\PY{p}{(}\PY{l+m+mi}{9}\PY{p}{,}\PY{l+m+mi}{9}\PY{p}{)}\PY{p}{)}
\end{Verbatim}
\end{tcolorbox}

    \begin{center}
    \adjustimage{max size={0.9\linewidth}{0.9\paperheight}}{output_41_0.png}
    \end{center}
    { \hspace*{\fill} \\}
    
    \begin{tcolorbox}[breakable, size=fbox, boxrule=1pt, pad at break*=1mm,colback=cellbackground, colframe=cellborder]
\prompt{In}{incolor}{ }{\boxspacing}
\begin{Verbatim}[commandchars=\\\{\}]

\end{Verbatim}
\end{tcolorbox}

    \begin{tcolorbox}[breakable, size=fbox, boxrule=1pt, pad at break*=1mm,colback=cellbackground, colframe=cellborder]
\prompt{In}{incolor}{25}{\boxspacing}
\begin{Verbatim}[commandchars=\\\{\}]
\PY{n}{sweep\PYZus{}thresholds} \PY{o}{=} \PY{n}{electric\PYZus{}field\PYZus{}thresholds}\PY{p}{(}\PY{n}{optimal\PYZus{}excitation\PYZus{}frequencies}\PY{p}{)}

\PY{c+c1}{\PYZsh{}plt.hist(thresholds, 50)}

\PY{c+c1}{\PYZsh{} percent\PYZus{}under(sweep\PYZus{}thresholds, 200.0)}

\PY{n}{plt}\PY{o}{.}\PY{n}{hist}\PY{p}{(}\PY{n}{single\PYZus{}freq\PYZus{}thresholds}\PY{o}{\PYZhy{}}\PY{n}{sweep\PYZus{}thresholds}\PY{p}{,} \PY{l+m+mi}{100}\PY{p}{)}

\PY{n+nb}{print}\PY{p}{(}\PY{l+s+s2}{\PYZdq{}}\PY{l+s+s2}{Given a frequency sweep in which each virion is excited with its frequency of maximum power dissipation,}\PY{l+s+s2}{\PYZdq{}}\PY{p}{)}
\PY{n+nb}{print}\PY{p}{(}\PY{l+s+s2}{\PYZdq{}}\PY{l+s+s2}{The following threshold percentiles would be expected:}\PY{l+s+s2}{\PYZdq{}}\PY{p}{)}
\PY{n}{percentiles}\PY{p}{(}\PY{n}{sweep\PYZus{}thresholds}\PY{p}{)}


\PY{n}{percentiles}\PY{p}{(}\PY{n}{single\PYZus{}freq\PYZus{}thresholds}\PY{o}{\PYZhy{}}\PY{n}{sweep\PYZus{}thresholds}\PY{p}{)}
\end{Verbatim}
\end{tcolorbox}

    \begin{Verbatim}[commandchars=\\\{\}]

        ---------------------------------------------------------------------------

        NameError                                 Traceback (most recent call last)

        <ipython-input-25-36c65eab0b34> in <module>
          5 \# percent\_under(sweep\_thresholds, 200.0)
          6 
    ----> 7 plt.hist(single\_freq\_thresholds-sweep\_thresholds, 100)
          8 
          9 print("Given a frequency sweep in which each virion is excited with its frequency of maximum power dissipation,")


        NameError: name 'single\_freq\_thresholds' is not defined

    \end{Verbatim}

    \begin{tcolorbox}[breakable, size=fbox, boxrule=1pt, pad at break*=1mm,colback=cellbackground, colframe=cellborder]
\prompt{In}{incolor}{ }{\boxspacing}
\begin{Verbatim}[commandchars=\\\{\}]

\end{Verbatim}
\end{tcolorbox}


    % Add a bibliography block to the postdoc
    
    
    
\end{document}
