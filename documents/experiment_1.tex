%!TeX root = experiment_1
\documentclass[paper.tex]{subfiles}
\begin{document}

\section{Experiment 1: Attempts at microwave absorption spectroscopy and inactivation in the microsecond pulsed 100 V/m regime}


\subsection{Microwave lysis feedback}

We postulate the following. When the virus is lysed, the resonator is destroyed. This should be reflected in the impedance spectrum.
We would expect a non-reversible decrease in the resonance peak after the application of power. This should also appear in the time-domain pulse envelope; we expect a "notch" on the rising edge, with a slow taper upwards, which does not repeat on the next pulse.

If this exists, it would provide a great deal of feedback. Not only would the virus be destroyed, but we would have an instant, quantitative determination of of how effective each treatment was. In practice, this could be done by applying two pulses in quick succession and monitoring the change in the reflected power; or perhaps by a time-domain pulse-shape method. Whether such a signature can be discriminated from background in a clinical setting remains to be seen.

At the time, this seemed important to establish, as it would allow the inactivation threshold to be determined without a time-intensive and potentially dangerous infectivity assay step (making it more plausible that we would get biosafety approval to obtain the microwave spectrum of NCoV). 

One could imagine it to be otherwise. For instance, the lysis that occurs after phage infection of bacteria does little harm to the cell; only the permeability of the membrane is damaged\cite{GROWTH}. Similarly, the results of the previous studies could perhaps have been obtained without necessarily destroying the virus' envelope.




















\end{document}