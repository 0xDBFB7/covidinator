\documentclass[fleqn,10pt]{article}




\usepackage[left=2cm,right=2cm,
			top=1.25cm,
			bottom=2.25cm,%
			headheight=11pt,%
			letterpaper]{geometry}
			
\frenchspacing			

\nonstopmode

\usepackage{multicol}
\usepackage{fancyhdr}
\usepackage{blindtext,graphicx}
\usepackage[absolute]{textpos}
\usepackage[parfill]{parskip}
\usepackage[colorlinks=true,citecolor=brown]{hyperref}
\usepackage{gensymb}
\usepackage{csquotes}
\usepackage{amsmath}
\usepackage{fontawesome}
\usepackage{orcidlink}
\usepackage{standalone}
\usepackage{pdfpages}
\usepackage{subfiles}
\usepackage{svg}
\usepackage{sidecap}
\usepackage{float}
\usepackage{amssymb}
\usepackage{textcomp}
\usepackage{lettrine}
%\usepackage[T1]{fontenc}

\usepackage{booktabs,caption}
\usepackage[flushleft]{threeparttable}

%\usepackage{biblatex}
\usepackage[backend=bibtex8, sorting=none]{biblatex}

\addbibresource{processed.bib}
%biblatex has a zoterordfxml
% might avoid the need for python bibtex_collections.py

\usepackage{etoolbox}
\AtBeginEnvironment{quote}{\small}

\usepackage{pifont}
\newcommand{\cmark}{\ding{51}}%
\newcommand{\xmark}{\ding{55}}%


\newcommand{\Wsqm}{$\text{ W/m}^2$}

\newcommand{\ghfile}[1]{\href{https://github.com/0xDBFB7/covidinator/tree/master/#1}{\faGithub/\url{#1} }}

%\newcommand{\supercite}[1]{}
%\newcommand{\supercollect}[1]{}


\newlength{\PRLlen}
\newcommand*\PRLsep[1]{{\itshape \Large\settowidth{\PRLlen}{#1}\advance\PRLlen by -\textwidth\divide\PRLlen by -2\noindent\makebox[\the\PRLlen]{\resizebox{\the\PRLlen}{1pt}{$\blacktriangleleft$}}\raisebox{-.5ex}{#1}\makebox[\the\PRLlen]{\resizebox{\the\PRLlen}{1pt}{$\blacktriangleright$}}\bigskip}}


\usepackage{graphicx}
\graphicspath{ {../media/} }

\usepackage{tcolorbox}
\newtcolorbox{autem}{colback=red!5!white,colframe=red!75!black}
\newtcolorbox{toolchain}{colback=blue!5!white,colframe=blue!40!black!40}
%https://tex.stackexchange.com/questions/66154/how-to-construct-a-coloured-box-with-rounded-corners

%\usepackage[sfdefault,light]{roboto}

\setlength{\TPHorizModule}{1cm}
\setlength{\TPVertModule}{1cm}




\title{On the application of microwave acoustic resonant viral inactivation
\thanks{I would be delighted to include any criticisms or comments anyone may have, both on substance and comprehensibility; preferably leave them on the GitHub issues page, or email therobotist@gmail.com, @0xDBFB7 on Twitter, or irc.0xDBFB7.com:6667 \#covid.}}
\date{May 2020}
\author{Based primarily on exceptional work by }


\begin{document}

\flushbottom 
\maketitle
\thispagestyle{empty}



\begin{textblock}{5}(1,1)
\noindent Please view the latest version at github.com/0xDBFB7/covidinator
\end{textblock}
%\begin{textblock}{5}(1,27)
%\end{textblock}

\null\begin{tabular}[t]{l@{}}
  {Daniel Correia}\ \orcidlink{0000-0002-9353-0216}  \\
  \textit{York University}
\end{tabular}



\begin{abstract}

We extend this landmark work rather trivially by:\\


Aims:

\begin{itemize}
  \item Establishing the time dependence of inactivation
  \item Demonstrating a modulation scheme that decreases the inactivation threshold to below current safety levels in surrogate bacteriophage
  \item Demonstrating a prototype emitter in an "electromagnetic mask" form-factor, costing about \$5 in prototype quantities, which can reasonably be produced in 10 million-of quantities
  \item Testing power thresholds in various conditions; biological fluids of various conductivities and pHs
  \item Using a coarse-grained molecular-dynamics simulation to optimize the impulse
  \item Using a virus-in-the-loop optimization with a centrifugal microfluidic system
  \item Discussing the biological basis for the safety of the device
  \item Showing that the deviation from the expected theshold could be explained by variance in the 
  
\end{itemize}

{Perhaps notable that a completely free and open-source toolchain was used for the entire project, including FDTD microwave antenna optimization. This work can be replicated with \$500 in equipment.}

\end{abstract}


\begin{autem}
	{\large  \it autem} \\
	
	This work was prepared by an undergraduate and has not been peer-reviewed. Additionally, the author has no prior experience with either biology or microwave design. \\

	While our study is very simple, biology is of such complexity that a tremendous amount of care must be taken before any conclusion can be drawn at all. It is not likely that we have taken sufficient care.\\ 

This is especially true with RF biophysics. The literature is littered with otherwise impeccably perormed research which was later demonstrated to be an artifact. \\

	Though the original research used Inf. A, our testing was only performed with a surrogate bacteriophage. All experiments must be repeated with SARS-NCoV-2. \\

	Additionally, other sterilization techniques may produce superior results and should be evaluated in the same context. Data on far UV [Buonanno 2017] indicate safety.\\

It has occurred before that otherwise striking resonances were detected and replication found them to be artifacts of the measurement equipment. [Foster 1987] even offers this disclaimer:

"To detect the DNA resonances with the probe
technique requires correction for system errors that are
potentially much larger than the effect to be studied and
lead to resonance-like artifacts. This is true even with the
more precise instrumentation used in this study. Such data
are easily misinterpreted, and we suggest this might have
happened in the former study."

The direct inactivation plaque and PCR data from [Yang], replicated by [Sun], is a somewhat convincing corroboration that this effect is not an artifact, but hundreds of convincing and  wrong papers exist in the literature. Our data is crude and imprecise, our equipment hastily constructed, and should not be considered validation of this effect even existing. \\


	Please consider all claims with appropriate skepticism.

\end{autem}
	

\begin{multicols}{1}



With typical 2.4 GHz microwave exposure, sterilization can only occur by heating of the fluid and tissue.

Of course, the temperature of the atoms of the virus undoubtedly increase; but the 

\begin{toolchain}
	{\it \bf [Yang 2015]'s toolchain}
	\begin{itemize}
	\item Envelope/liposome breaking strength and stiffness from AFM nanoindentation data
	\item Analytical expression assuming homogenous sphere for microwave absorption cross-section
	\item Experimental absorption data from microwave cuvette -> 
	\item COMSOL finite-element for illustration
	\end{itemize}
\end{toolchain}

\end{multicols}






%%%%%%%%%%%%%%%%%%%%%%%%%%%%%%%%%%%%%%%%%%%%%%%%
\clearpage
%%%%%%%%%%%%%%%%%%%%%%%%%%%%%%%%%%%%%%%%%%%%%%%%

The chain of literature Fruluch -> ([Sun ] | [Yang 2015]) establishes that spherical virions possess a weak dipole resonance mode which happens to fall within the microwave spectrum, at approximately the X-band (6-10 GHz).

This is a non-ionizing, non-thermal technique; 

This resonance mode 

The myth that 2.4 GHz was chosen because it "couples strongly with water" is largely false; but this is 


This is quite surprising, and should perhaps be considered with some trepidation, since 




\paragraph{\textbf{Time dependence}}\



Both Yang and [] used an apparently arbitrary 15-minute exposure in their tests - a very reasonable decision given the focus of their paper. 

The effectiveness against airborne particles, and to minimize the power required in a dwelling phased-array beam, we must first establish the required duration of exposure.

{\color{red} speculative hypothesizing \{ } 

In contrast to chemical inactivation, where the time dependence appears to be dominated by viscous fluid dynamic effects [Hirose 2017], or UV inactivation, where a certain quantized dose of photons must be absorbed, we expected RF to act instantaneously.

As a damped, driven oscillator, the ring-up time of the virus depends on the Q factor. Yang et al. state the Q of Inf. A as between 2 and 10, so at 8 GHz the steady-state amplitude should be reached in well under 100 nanoseconds.???????????FIXME

[] found a significant mechanical fatigue effect in phage capsids, where a small strain applied repetitively eventually causes a fracture. Such a mechanism could perhaps extend the exposure required to break the capsid or membrane. Other mechanisms could include some sort of lipid denaturation, requiring an absolute amount of energy absorption to break or twist bonds and modify the properties before the envelope fractures.


{\color{red}  \} } 

\footnote{All values have been converted to $W/m^2$ to avoid confusion.}




\subfile{safety}









\clearpage
%%%%%%%%%%%%%%%%%%%%%%%%%%%%%%%%%%%%%%%%%%%%%%%%
\begin{multicols}{1}
{\Large Biology}\\
%%%%%%%%%%%%%%%%%%%%%%%%%%%%%%%%%%%%%%%%%%%%%%%%

\paragraph{\textbf{Centrifugal microfluidics}}\

The field of centrifugal microfluidics is accelerating. 

Many CD microfluidics systems use standard CD molding techniques for the channels and machining techniques, using either acrylic or silicone. The turbidity sensor is most sensitive if the plastic is clear. Sterilization does not seem to be discussed. 

Polypropylene is the ideal material, being almost indefinitely autoclavable. It is quite difficult to machine.

\end{multicols}












\clearpage
%%%%%%%%%%%%%%%%%%%%%%%%%%%%%%%%%%%%%%%%%%%%%%%%
\begin{multicols}{1}
{\Large Modes of application}\\
%%%%%%%%%%%%%%%%%%%%%%%%%%%%%%%%%%%%%%%%%%%%%%%%

\paragraph{\textbf{Personal 'electromagnetic mask'}}\

Even with judicious use of phased-arrays, spatial power-combining, etc, each transistor can only reasonably sterilize approx. $ 1 \text{ m}^3 $.

We therefore demonstrate this form factor because, superficially, there are fewer people than there are places. 

On the other hand, a personal device may present issues with participation and production volume. 

\paragraph{\textbf{Direct treatments}}\

[Hand 1982] $E_{mag}=1/e$ (13.5\% power density) skin\footnote{electromagnetic skin, not tissue skin}\footnote{well, both, I suppose.} depth is approximately:

\begin{center}
\begin{tabular}{|l|l|l|}
\hline
F=10 GHz          & Dry tissue & Wet tissue \\ \hline
Penetration depth & 30 mm      & 5 mm \\ \hline
\end{tabular}
\end{center}

SARS is found widely distributed throughout the most favorable organs [Ding 2004], shielded by an average of 4 cm of chest wall [Schroeder 2013]; so safe external treatment of the body is unlikely.

However, destruction of lung tissue appears to be the primary cause of death via SARS [Nicholls 2006]. 

A bronchoscopic technique may therefore be effective, very similarly to that demonstrated by [Yuan 2019]: the bronchi are less than 2 mm thick [Theriault 2018] and the lungs themselves are only on the order of 7 mm thick [Chekan 2016]. 

There is considerable tolerance in the 

Further study is required to validate this method.

\paragraph{\textbf{More fanciful concepts}}\

Many systems operate in these X-band frequency ranges; and precision 

It is relatively easy to produce megawatts of power at these frequency ranges using Klystrons.

Existing marine, weather, and aviation radar systems often use the X-band; depending on the focusing power

\paragraph{}

Finally, there is one concern. Use of this technique will provide a selection bias towards immunity to electromagnetic fields, which could perhaps be effected by preferring extreme-sized mutants (shifting the resonance away from the applied field). We do not have the biological knowledge to know if this is plausible; it is simply worth mentioning.

\end{multicols}







\clearpage
%%%%%%%%%%%%%%%%%%%%%%%%%%%%%%%%%%%%%%%%%%%%%%%%
\begin{multicols}{1}
%%%%%%%%%%%%%%%%%%%%%%%%%%%%%%%%%%%%%%%%%%%%%%%%

\noindent\fbox{\parbox{\linewidth}{
	Toolchain:
	\begin{itemize}
	\item Failed oscillator feedback-loop optimization toolchain: QUCS 0.0.20 + python-qucs + scipy's 'basinhopper'
	\item Successful A slightly modified ngspice + ngspyce + pyEVTK
	\item gprMax for FDTD EM simulation
	\item KiCAD, wcalc, scikit-rf, ngspice
	\end{itemize}
}}
%


The oscillator is a 'wideband double-tuned varactor VCO', based almost verbatim on [Tsuru 2008] and the reference design in Figure 8.36, p378 of [Grebennikov 2007]. 

The rest of the papers in the bibliography were highly enlightening regarding the principles of microwave oscillator design.

This topology of oscillator worked marvellously on essentially the first try. 

The process by which we failed miserably to design our own oscillator topology is detailed in the supplemental.

The following properties: P1dB

For ease of design and simulation, a device with Touchstone S-parameters and SPICE files is greatly preferable.

It is commonly claimed that FR4 is a 'slow' substrate, and that the high loss tangent of ~0.02 makes it unsuitable for microwave systems.

However, with this PCB substrate, the expected loss of only 0.026 dB/mm on signals of minimum 0 dBm is patently acceptable. As viruses do not have a discriminating palate, we are also only minimally concerned with $S_{11}$ reflections, noise, or spurs, so precise impedance control is not required; the wideband VCO sweep accomodates for any variations in resonant frequency due to wide manufacturing tolerances.

As a result of our gross incompetence, the oscillator was designed via an inane, roundabout, and fiendishly tedious manner, and our description of this technique only contributes to the field by being suitable for a dartboard; our analysis can only hold water when combined with paper mache; and a lesser invertebrate in posession of a copy of Microwave Oscillator Design would have accomplished the task faster.

Designing an oscillator of this type in one step with a few kindergarten equations appears to be well within the reach of modern network analytical techniques. Genetic algorithms are quite well matched to this problem, and many commercial software packages are  [computer microwave design book]. 

Initially, 

Not having a good analytical understanding, we resorted to using a purely computational method.

In addition, the design of wideband feedback-loop VCOs is a relatively well-explored field, and many reference designs exist. 

The key sticking point - which we missed - seems to be that at these scales, using microstrip design techniques, the parasitics of any possible filter structure are so large that the small change in impedance that varactors can provide cannot tune the circuit by any meaningful amount.

"Using the lead inductances of the bipolar transistor and varactors provides the required value of the base inductance".

Indeed, [Tsuru 2008]'s "tuned circuit" in Fig 8 is, in fact, just the varactors, plus an almost invisible high-impedance line.





Several analytical filter design methods were 

This is apparently known as a "double-tuned" wideband.

%
\paragraph{\textbf{The feedback loop oscillator}}\

Oscillators must meet the Barkhausen criterion:

\begin{itemize}

\item A 360 degree phase shift around the feedback loop (including the phase shift contribution from the amplifier, which itself varies greatly with frequency)
\item A loop gain $>1.$ 

\end{itemize}\
%
However, a third element is also required:
%
\begin{itemize}
\item A frequency-selective element that restricts oscillation modes and decreases phase noise.
\end{itemize}
%

With large feedback-loop structures, such as long PiN-switched phasing lines, proximity effects from nearby flesh would detune the oscillator significantly.

 design of a triple-tuned oscillator.


\paragraph{}
The buffer amplifiers 

\paragraph{Biasing}\

In our simulations, the varactor-tuned feedback circuit appeared to be particularly sensitive to the introduction of bias-tees. 


The gate must be weakly pulled to ground, otherwise stray charge destroys the oscillation.

\fancyhead[C]{style 1 with thin line}



With 0.79 mm FR4 substrate and 0.2 mm (8 mil) wide traces, the maximum impedance achievable was about 115 ohms, which did not appear to be sufficient as an RF choke. Use of defected grounds can increase inductance, but this was not evaluated.

If suitably high-impedance traces are not available, a common technique is to use a quarter-wavelength line (approximately 6 mm long with the above parameters at 8 GHz) terminated with a stub to produce a virtual short or open circuit [Seo 2007]. 

However, reflections from these structures still appeared to distort the frequency/phase response beyond repair, even with ostensibly wideband stubs [Syrett 1980].

\noindent\fbox{\parbox{\linewidth}{
	Rebuke: despite this blather, many other papers have had success with bias-tees at these frequencies.
}}

Alternate methods evaluated, failures: 


In production, these could be accomodated by graphite-polymer printed resistors. 


Odd-pole varactor-loaded combline filters appeared to have excellent phase and frequency response; however, the geometry necessitates low-inductance via stitching to the ground plane.

%\noindent\fbox{\parbox{\linewidth}{
\begin{autem}
	{\it autem} Others have had great success with varactor-tuned comblines, especially non-grounded lines.
\end{autem}

[Tsuru 2008 fig. 10] is an excellent review of various oscillator designs.

The parasitic inductance of common varactors appears to become problematic at these frequencies (but not for non-wideband use).

\paragraph{\textbf{Spacing}}\

FDTD results still yielded some coupling at 2 mm isolation.

\paragraph{\textbf{Via stitching}}\

Rather than the almost universal technique of via stitching components immediately to the ground plane (a tedious process with our prototyping setup), a large ground pour on the component layer was used wherever ground was needed, as mentioned in [Hunter? combline]. Since we use microstrip rather than coplanar waveguide, 

Since the ground plane does not participate in the DC bias at all, no vias are required in the final device. This means that the drilling, electroless strike and electroplating steps are unnecessary for production.




\clearpage
%%%%%%%%%%%%%%%%%%%%%%%%%%%%%%%%%%%%%%%%%%%%%%%%
{\Large Mass production}
%%%%%%%%%%%%%%%%%%%%%%%%%%%%%%%%%%%%%%%%%%%%%%%%


If this 'electromagnetic mask' form is the ideal (not nearly ), a minimum of 5 GaAs or SiGe:C transistors will be required.

There are, for instance, 1 million hospital beds in the U.S. [AHA 2018]. 

It is difficult to determine the supply capacity for these modern semiconductor processes. A few fermi est

GaAs MMIC market of \$2.2 Bn USD / random sample of MMIC prices = 

3e9 wifi connected devices produced each year.

The largest RFID plants can produce


The techniques and equipment required to produce these are non-trivial. SiGe:C, for instance, requires, 


Figures are not forthcoming, 

Given the supply difficulties of comparatively simple materials such as Tyvek at pandemic scales, it is difficult to imagine that RF semiconductor production can be immediately re-tasked and scaled to this degree. 

\paragraph{\textbf{Vacuum RF triode}}\

One concern is the large filament heater power, which prevents the use of low-cost button cells for power. Use of cold-cathode field-emitter arrays would alleviate this issue, but at the cost of complexity.

Small tungsten incandescent lights are available down to 0.05 watts. With a suitable high-efficiency cathode coating, a pulsed heater power of less than 0.02



\end{multicols}







\clearpage
%%%%%%%%%%%%%%%%%%%%%%%%%%%%%%%%%%%%%%%%%%%%%%%%
\paragraph{Acknowledgements}\
%%%%%%%%%%%%%%%%%%%%%%%%%%%%%%%%%%%%%%%%%%%%%%%%

This paper is not novel enough to deserve this chapter, but those to be acknowledged are.

The authors of the original papers deserve the credit for this entire work;  it is rare that such a monumental finding is served on a silver platter.

Professor Shahramian's excellent video blog {\it The Signal Path} inspired this work. 

Dr. Dainis' videos on microbiological technique were also useful.

Thanks to the professors at York and beyond for returning my emails. 

Thanks to L.H. for wisdom regarding this project.

Thanks to P.B and H.P. for tolerating ramblings and M. for the hug.

Special thanks to all the people - most equally capable of writing this paper if provided the opportunity - who work menial shifts at the component suppliers that remained open that were so critical. Hopfstader be damned; despite the name below the abstract, this paper is of course entirely the result of circumstance and luck. 

Chiefly among the enablers are of course my parents, Doris and Brian, without whom this work would have been done by someone else, and who also supported this work financially. 



One Very Important Thought




\subfile{supplemental}







\end{document}
