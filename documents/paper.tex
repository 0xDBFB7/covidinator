\documentclass[fleqn,10pt]{article}




\usepackage[left=2cm,right=2cm,
			top=1.25cm,
			bottom=2.25cm,%
			headheight=11pt,%
			letterpaper]{geometry}
			
\frenchspacing			

\nonstopmode




\usepackage{lmodern}
\usepackage[T1]{fontenc}
\usepackage[utf8]{inputenc}



\usepackage{noweb}

\usepackage{multicol}
\usepackage{fancyhdr}
\usepackage{blindtext,graphicx}
\usepackage[absolute]{textpos}
%\usepackage[parfill]{parskip}
\usepackage{parskip}
\setlength{\parskip}{\baselineskip}

\usepackage[colorlinks=true,citecolor=brown]{hyperref}
\usepackage{gensymb}
\usepackage{csquotes}
\usepackage{amsmath}
\usepackage{fontawesome}
\usepackage{orcidlink}
\usepackage{standalone}
\usepackage{pdfpages}
\usepackage{subfiles}
\usepackage{svg}
\usepackage{sidecap}
\usepackage{float}
\usepackage{amssymb}
\usepackage{textcomp}
\usepackage{lettrine}
%\usepackage[T1]{fontenc}

\usepackage{soul}


%\usepackage{draftwatermark}
%\SetWatermarkText{DRAFT}
%\SetWatermarkScale{0.25}

\usepackage{booktabs,caption}
\usepackage[flushleft]{threeparttable}

%\usepackage{biblatex}
\usepackage[backend=bibtex8, sorting=none, style=chem-angew]{biblatex}

\let\cite\footfullcite

%\let\cite\footcite

\addbibresource{processed.bib}
%biblatex has a zoterordfxml
% might avoid the need for python bibtex_collections.py



\usepackage{etoolbox}
\AtBeginEnvironment{quote}{\small}




\usepackage{pifont}
\newcommand{\cmark}{\ding{51}}%
\newcommand{\xmark}{\ding{55}}%


\newcommand{\citationneeded}[1][]{\textsuperscript{[\color{blue}{\it \bf{citation needed}#1}]}}
\newcommand{\dubiousdiscuss}[1][]{\textsuperscript{\color{blue} [{\it \bf{dubious-discuss}}]} }

\newcommand{\light}[1]{\textcolor{gray}{#1}}

%
%
\usepackage{titlesec}
%
%% custom section


\titleformat{\section}
{\normalfont\LARGE\bfseries}{\thesection}{1em}{}
%\titleformat{\section}
%{\normalfont\LARGE\bfseries\PRLsep}
%{{{{\itshape \thesection\hskip 9pt\textpipe\hskip 9pt}}}}{0pt}{}
%
%% custom section
%\titleformat{\subsection}
%{\normalfont\Large\bfseries\PRLsep}
%{{{{\itshape \thesection\hskip 9pt\textpipe\hskip 9pt}}}}{0pt}{}
%
%
%


\newcommand{\Wsqm}{$\text{ W/m}^2$}

\newcommand{\ghfile}[1]{\href{https://github.com/0xDBFB7/covidinator/tree/master/#1}{\faGithub/\url{#1} }}

%\newcommand{\supercite}[1]{}
%\newcommand{\supercollect}[1]{}


\newlength{\PRLlen}
\newcommand*\PRLsep[1]{{\itshape \Large\settowidth{\PRLlen}{#1}\advance\PRLlen by -\textwidth\divide\PRLlen by -2\noindent\makebox[\the\PRLlen]{\resizebox{\the\PRLlen}{1pt}{$\blacktriangleleft$}}\raisebox{-.5ex}{#1}\makebox[\the\PRLlen]{\resizebox{\the\PRLlen}{1pt}{$\blacktriangleright$}}\bigskip}}


\renewcommand{\thefootnote}{\textcolor{gray}{\arabic{footnote}}}


\usepackage{graphicx}
\graphicspath{ {../media/} 
				{../firmware/eppenwolf/runs/sic_susceptor/} 
			}

\usepackage{tcolorbox}
\newtcolorbox{protocol}{colback=yellow!5!white,colframe=yellow!75!black}
\newtcolorbox{equipment}{colback=orange!5!white,colframe=orange!75!black}
\newtcolorbox{autem}{colback=red!5!white,colframe=red!75!black}
\newtcolorbox{toolchain}{colback=blue!5!white,colframe=blue!40!black!40}
\newtcolorbox{sidenote}{colback=cyan!5!white,colframe=blue!40!black!40}
%https://tex.stackexchange.com/questions/66154/how-to-construct-a-coloured-box-with-rounded-corners

%\usepackage[sfdefault,light]{roboto}

\setlength{\TPHorizModule}{1cm}
\setlength{\TPVertModule}{1cm}





%%%%********************************************************************
% fancy quotes
\definecolor{quotemark}{gray}{0.7}
\makeatletter
\def\fquote{%
	\@ifnextchar[{\fquote@i}{\fquote@i[]}%]
}%
\def\fquote@i[#1]{%
	\def\tempa{#1}%
	\@ifnextchar[{\fquote@ii}{\fquote@ii[]}%]
}%
\def\fquote@ii[#1]{%
	\def\tempb{#1}%
	\@ifnextchar[{\fquote@iii}{\fquote@iii[]}%]
}%
\def\fquote@iii[#1]{%
	\def\tempc{#1}%
	\vspace{1em}%
	\noindent%
	\begin{list}{}{%
			\setlength{\leftmargin}{0.1\textwidth}%
			\setlength{\rightmargin}{0.1\textwidth}%
		}%
		\item[]%
		\begin{picture}(0,0)%
		\put(-15,-5){\makebox(0,0){\scalebox{3}{\textcolor{quotemark}{``}}}}%
		\end{picture}%
		\begingroup\itshape}%
	%%%%********************************************************************
	\def\endfquote{%
		\endgroup\par%
		\makebox[0pt][l]{%
			\hspace{0.8\textwidth}%
			\begin{picture}(0,0)(0,0)%
			\put(15,15){\makebox(0,0){%
					\scalebox{3}{\color{quotemark}''}}}%
			\end{picture}}%
		\ifx\tempa\empty%
		\else%
		\ifx\tempc\empty%
		\hfill\rule{100pt}{0.5pt}\\\mbox{}\hfill\tempa,\ \emph{\tempb}%
		\else%
		\hfill\rule{100pt}{0.5pt}\\\mbox{}\hfill\tempa,\ \emph{\tempb},\ \tempc%
		\fi\fi\par%
		\vspace{0.5em}%
	\end{list}%
}%
\makeatother







%%%%********************************************************************
%title link to doi
\newbibmacro{string+doiurlisbn}[1]{%
	\iffieldundef{doi}{%
		\iffieldundef{url}{%
			\iffieldundef{isbn}{%
				\iffieldundef{issn}{%
					#1%
				}{%
					\href{http://books.google.com/books?vid=ISSN\thefield{issn}}{#1}%
				}%
			}{%
				\href{http://books.google.com/books?vid=ISBN\thefield{isbn}}{#1}%
			}%
		}{%
			\href{\thefield{url}}{#1}%
		}%
	}{%
		\href{https://doi.org/\thefield{doi}}{#1}%
	}%
}

\DeclareFieldFormat{journaltitle}{\usebibmacro{string+doiurlisbn}{\mkbibemph{#1}}}

%!TeX root = title
\documentclass[paper.tex]{subfiles}
\begin{document}

%\title{Maxwell's Silver Hammer: Musings and measurements on pulsed microwave acoustic resonant viral inactivation 

%Data and discussion?

\title{Maxwell's Silver Hammer: Pulsed microwave viral envelope disruption\footnotemark\\or, some book-keeping matters of minor consequence in the application of microwave-induced lysis}
\date{First published May 2020}
\author{Based primarily on excellent work by Liu et al, Yang et al, Hung et al, and at least 800 other groups.\\
	\\
		By that measure, this paper is 0.125\% by \\
		\\
		\small{{Daniel Correia}\ \orcidlink{0000-0002-9353-0216}, \textit{? York? FIXME}\footnotemark}}

\footnotetext{{ Responsibility for accuracy and completeness must always fall on the author. That said, I would be delighted to hear any criticisms or comments anyone may have, both on substance and comprehensibility; preferably leave them on the GitHub issues page, or email therobotist@gmail.com, @0xDBFB7 on Twitter, or irc.0xDBFB7.com:6667 \#covid.}}

\footnotetext{Also affiliated with SafeSump Incorporated.}

\flushbottom 
\maketitle
\thispagestyle{empty}


\null\begin{tabular}[t]{l@{}}
	  \\
	
\end{tabular}


\begin{textblock}{10}(5,0.5)
\noindent Data and software produced during this research are available at \\ \href{https://www.github.com/0xDBFB7/covidinator}{github.com/0xDBFB7/covidinator}. This document may be updated as information changes. Please view the latest version. This is V0.0.1.
\end{textblock}
%\begin{textblock}{5}(1,27)
%\end{textblock}

\begin{quotation} % abstract requires re-writing an introduction. Screw that.

Previous works appear to demonstrate that several species of viruses exhibit an anomalous dielectric relaxation time of $> 75 \text{ ps}$, as opposed to $< 20 \text{ ps}$ \footnotemark \dubiousdiscuss in the structures of human tissue.\\


Other studies then appear to demonstrate experimentally that, due to this effect and the favorable combination of envelope and genome net charge, envelope and capsid yield stress, mass distribution, and overall stiffness, when the virus is exposed to tones in the X-band, it rings up and the envelope shatters - much like a singer might shatter a wine glass.\\

Such a non-thermal mechanism is quite unprecedented and extremely dubious, oft-posited but never substantiated; so it has taken some convincing that this is not a particularly abstruse artifact. The structure of the virion appears to fill a singular gap in existing solvent damping theory.\footnotemark In fact, even in light of the fine evidence that the other groups have collected, and the poor-quality data we have collected, we are still not entirely convinced that this effect exists - but perhaps this skepticism is unjustified.\\

\footnotetext{we should verify, based on simple solvated sphere damping, whether the damping is of the right order of magnitude}

We contribute almost nothing to the record (much of our discussion has already been covered by []), except by apparently replicating the effect in a T4 coliphage surrogate, using both microwave feedback and a luminometric beta-galactosidase infectivity assay. We appear to demonstrate that inactivation does not require 15 minutes, but is complete in less than 500 nanoseconds. This provides a headroom of some $10^6$ to all safety limits, which can be traded for distance or depth. \\

This technique is unique in that it is selective, save for Far UV or cold-plasma sterilization. Its advantages over those is that it operates over large volumes with inexpensive emitters, but, most critically, can be made to penetrate tissue without harm.

We briefly review the biological basis for the safety. There is solid in vitro and limited in vivo evidence of safety in this regime.

\footnotetext{This value for human structures is the aggregate Cole-Cole relaxation time from Gabriel. This is an unfair comparison; the relaxation time of a heterogeneous mixture such as tissue often appears smaller than that of a pure sample. Additionally, it is probably not valid to compare a Cole-Cole tau. However, this value agrees well with the individual findings of cellular modes, so we are comfortable with this comparison. }

We briefly examine three applications. All depend significantly on what the microwave spectrum of SARS-NCoV-2 ends up being, which must be measured - or possibly reconstructed from existing whole-virion data or determined from MD simulation. 

With refinements, our inexpensive pulsed microwave spectrometer can obtain the required absorption and threshold data with sufficiently concentrated specimens; but proper dielectric spectroscopy labs (or, in a pinch, swept-EPR/ESR equipment \st{when used improperly}) will provide a much more detailed spectrum. A slightly safer low-titer primary specimen may be usable to avoid BSL-3 requirements. \\



First, free-space techniques are easily implemented to prevent respiratory transmission. A \$10 emitter appears to suffice on a personal level, and \$100 systems for room-scale.\\
 
%
Second, via cursory FDTD simulations, using off-the-shelf equipment and modified non-invasive microwave diathermy techniques, it appears to be possible to inactivate all virions in the outer 2 cm of the lungs and inner 1 cm of the bronchi without any effect on the surrounding tissue. 
\\

[] notes that with a bit more effort, by harnessing the dispersive nature of tissue using modulated Brillouin precursors, it may be possible to remove the virus from the body altogether.\\

The mechanism also provides a few distinctive microwave signatures that may be useful for detection; it remains to be seen whether these can be discerned in practice.

The claims above are extraordinary. We do not provide the extraordinary quality of evidence required to support them. Consider: is it more likely that an undergraduate in his parent's basement, with no experience in biology or microwave design, has misinterpreted a blip from a cobbled-together machine, or that 

expelling its viral contents into solution and 

Not to be confused with pseudoscientific 'resonance therapies', nor reports of a connection with 5G communications.

This means that the technique is {\it selective}.






Aims: this is primarily bookkeeping.\\

Legend:  \cmark $ = $ crude experiment, not nearly definitive, needs more work $\vert$ $\thicksim$ $ = $ crude experiment, not nearly definitive, needs more work $\vert$ \xmark \ $ = $ definitely not complete.\\

\begin{itemize}
  \item Establishing the time dependence of inactivation in surrogate bacteriophage $\vert$ \cmark
  \item Demonstrating a modulation scheme that decreases the inactivation threshold to below current safety levels in surrogate bacteriophage $\vert$ \cmark
  \item Demonstrating a prototype emitter in an "electromagnetic mask" form-factor, costing about \$5 in prototype quantities, which can reasonably be produced in 10 million-of quantities $\vert$ $\thicksim$
  \item Testing power thresholds in various conditions; biological fluids of various conductivities and pHs $\vert$ \xmark
  \item Synthesizing a coarse-grained molecular-dynamics simulation of the mechanism $\vert$ \xmark
  \item Discussing the biological basis for the safety of the device $\vert$ \cmark
  \item Showing that the deviation from the expected threshold can be explained by variance in the viron, and that  $\vert$ \cmark
  \item Taking 4 months to design a circuit with 2 transistors, thereby potentially leading to the maiming of hundreds of thousands of people and causing untold economic losses $\vert$ \cmark
  
\end{itemize}\


It is perhaps notable that a completely free and open-source toolchain was used for the entire project.

\end{quotation}

\end{document}

	

\clearpage

\paragraph{To-do list}

The following have to be done before production can be started:

\begin{itemize}
  \item Re-run the experiment with 
  \item 
  		
  \item 
  
\end{itemize}

\clearpage
\begin{multicols}{1}



\paragraph{{\Large the gist}}\

The chain of literature 
%
[Fr\"{o}hlich 1968] \textrightarrow \ [Fr\"{o}hlich 1980] \textrightarrow \ [Liu 2009] \textrightarrow \ ([Hung 2014] $\parallel$ [Yang 2015] $\parallel$ [Sun 2017])\footnote{We have not conducted a thorough review; many more papers are probably involved.} apparently culminates in the flagship work we focus on in this paper,

{\it Efficient Structure Resonance Energy Transfer from Microwaves to Confined Acoustic Vibrations in Viruses} [Yang 2015].
%

estabilishes that - unique to certain viruses, and apparently unlike human cellular structures, as we shall see - coincidentally have just the right size, shape, stiffness, and net charge distribution to form a weak (Q=2) spherical dipole resonance mode which couples well to the microwave spectrum at approximately the X-Band (6-10 GHz).

More critically, [Yang 2015] (and, in parallel, [Hung 2014]) theoretically model and then experimentally demonstrate in various strains of Influenza A that the electric field strength required to produce an amplitude of vibration sufficient to tear the lipid envelope is approximately 100 V/m, below safety limits.\footnote{See supplemental.}

They demonstrate this with both a plaque and PCR assay, with fantastic agreement with the theoretical model.\footnote{As we will discuss, there are a few issues with the experimental technique.; sham, blinding, and dosimetry demanded by [Vjl.] are not mentioned.}

Like pumping a tire swing, this effect allows an otherwise inconsequential field magnitude to store energy over a small number of cycles until the virus is destroyed. 

\lettrine{It} cannot be overstated how unexpected, unusual, and downright dubious this finding appears to be -at least, based on our limited research and experience to date. 

The RF safety limits set by standards organizations are themselves based on the assumption that no resonances exist in biological tissues, and (as we shall see), this claim is grounded in solid in vitro (albeit limited in vivo) evidence. 

It is difficult to reconcile this discrepancy. A failure tree is included below.


If this mechanism exists, it would seem to provide the following advantages over other methods such as UV or cold-plasma sterilization:

\begin{itemize}
  \item This is a non-ionizing, non-thermal, non-chemical technique.
  \item Far-field and moderately penetrating inactivation. 
  		
  \item 
  
\end{itemize}





\footnote{It should be noted that this {\it confined} acoustic resonance is subtly distinct from common-and-garden pipe-organ acoustic resonance; this is apparently not a strictly classical phenomenon. Besides standard Coulomb-like and Lennard-Jones-like interactions between constituent particles, if Fr\"{o}hlich is to be believed at these nanoscopic scales there are also wave-function interactions among the particles of the virus which can shift storage to modes not otherwise expected.

We confess to not yet understanding this phenomenon; fortunately, while helpful, the details of how this mode appears are not critical to implementing this technique.}


\footnote{``{For example, below about 6 GHz, where EMFs penetrate deep into tissue (and thus require depth to be considered), it is useful to describe this in terms of “specific energy absorption rate” (SAR), which is the power absorbed per unit mass $(W/kg)$. Conversely, above 6 GHz, where EMFs are absorbed more superficially (making depth less relevant), it is useful to describe exposure in terms of the density of absorbed power over area $W/m^2$, which we refer to as “absorbed power density”}'' [ICNIRP 2020 \faExternalLink] }



\footnote{All values have been converted to $\text{W/m}^2$ to avoid confusion. 100 $\text{ W/m}^2 = 10 \text{ mW/cm}^2 = 10 \text{ dBm/cm}^2$.}


With typical 2.4 GHz microwave exposure, sterilization can only occur by heating of the fluid and tissue.

Of course, the temperature of the atoms of the virus undoubtedly increase; but the 

\begin{toolchain}
	{\it \bf [Yang 2015]'s toolchain}
	\begin{itemize}
	\item Envelope/liposome breaking strength and stiffness from AFM nanoindentation data
	\item Analytical expression assuming homogenous sphere for microwave absorption cross-section
	\item Experimental absorption data from microwave cuvette -> 
	\item COMSOL finite-element for illustration
	\end{itemize}
\end{toolchain}

\end{multicols}

















\clearpage
{\Large \it Talkin' 'bout the Variation}\\

[Yang 2015] theoretically model the virus to determine the minimum electric field required to destroy it. 

They assume that the virus is a simple damped harmonic oscillator, where the 'core' and 'shell' oscillate in opposition.

They determine the net charge experimentally from microwave absorption measurements.

\begin{center}
 \begin{tabular}{||c c c c||} 
 \hline
 Col1 & Col2 & Col2 & Col3 \\ [0.5ex] 
 \hline\hline
 1 & 6 & 87837 & 787 \\ 
 \hline
 2 & 7 & 78 & 5415 \\
 \hline
 3 & 545 & 778 & 7507 \\
 \hline
 4 & 545 & 18744 & 7560 \\
 \hline
 5 & 88 & 788 & 6344 \\ [1ex] 
 \hline
\end{tabular}
\end{center}























\clearpage
\paragraph{\textbf{Time dependence}}\


Both [Yang 2015] and [Hung 2014] use an apparently arbitrary 15-minute exposure in their tests - a very reasonable decision, given the focus of their paper. 

The effectiveness against airborne particles, and to minimize the power required in a dwelling phased-array beam, we must establish the required duration of exposure.

{\color{red} speculative hypothesizing \{ } 

In contrast to chemical inactivation, where the time dependence appears to be dominated by viscous fluid dynamic effects [Hirose 2017], or UV inactivation, where a certain quantized dose of photons must be absorbed, we expected RF to act instantaneously.

As a damped, driven oscillator, the ring-up time of the virus depends on the Q factor. Yang et al. state the Q of Inf. A as between 2 and 10, so at 8 GHz the steady-state amplitude should be reached in well under 100 nanoseconds.???????????FIXME

[] found a significant mechanical fatigue effect in phage capsids, where a small strain applied repetitively eventually causes a fracture. Such a mechanism could perhaps extend the exposure required to break the capsid or membrane. Other mechanisms could include some sort of lipid denaturation, requiring an absolute amount of energy absorption to break or twist bonds and modify the properties before the envelope fractures.


{\color{red}  \} } 





\subfile{safety}









\clearpage
%%%%%%%%%%%%%%%%%%%%%%%%%%%%%%%%%%%%%%%%%%%%%%%%
\begin{multicols}{1}
{\Large The Experiment}\\
%%%%%%%%%%%%%%%%%%%%%%%%%%%%%%%%%%%%%%%%%%%%%%%%

\paragraph{\textbf{Centrifugal microfluidics}}\

The field of centrifugal microfluidics is accelerating. 

Many CD microfluidics systems use standard CD molding techniques for the channels and machining techniques, using either acrylic or silicone. The turbidity sensor is most sensitive if the plastic is clear. Sterilization does not seem to be discussed. 

Polypropylene is the ideal material, being almost indefinitely autoclavable. It is quite difficult to machine.

\end{multicols}












\clearpage
%%%%%%%%%%%%%%%%%%%%%%%%%%%%%%%%%%%%%%%%%%%%%%%%
\begin{multicols}{1}
{\Large Modes of application}\\
%%%%%%%%%%%%%%%%%%%%%%%%%%%%%%%%%%%%%%%%%%%%%%%%

\paragraph{\textbf{Personal 'electromagnetic mask'}}\

Even with judicious use of phased-arrays, spatial power-combining, etc, each transistor can only reasonably sterilize approx. $ 1 \text{ m}^3 $.

We therefore demonstrate this form factor because, superficially, there are fewer people than there are places. 

On the other hand, a personal device may present issues with participation and production volume. 

\paragraph{\textbf{Direct treatments}}\

[Hand 1982] $E_{mag}=1/e$ (13.5\% power density) skin\footnote{electromagnetic skin, not tissue skin}\footnote{well, both, I suppose.} depth is approximately:

\begin{center}
\begin{tabular}{|l|l|l|}
\hline
F=10 GHz          & Dry tissue & Wet tissue \\ \hline
Penetration depth & 30 mm      & 5 mm \\ \hline
\end{tabular}
\end{center}

SARS is found widely distributed throughout the most favorable organs [Ding 2004], shielded by an average of 4 cm of chest wall [Schroeder 2013]; so safe external treatment of the body is unlikely.

However, destruction of lung tissue appears to be the primary cause of death via SARS [Nicholls 2006]. 

A bronchoscopic technique may therefore be effective, very similarly to that demonstrated by [Yuan 2019]: in adults, the bronchi are less than 2 mm thick [Theriault 2018] and the lungs themselves are only on the order of 7 mm thick [Chekan 2016]. 

The main bronchus is about 8 mm in diameter, which is smaller than the patch antenna used here; a monopole (or multiple, phased monopoles) like is probably more suitable.


\paragraph{\textbf{More fanciful concepts}}\

Many systems operate in these X-band frequency ranges; and precision 

It is relatively easy to produce megawatts of power at these frequency ranges using Klystrons.

Existing marine, weather, and aviation radar systems often use the X-band; depending on the focusing power

\paragraph{}

Finally, there is one concern. Use of this technique will provide a selection bias towards immunity to electromagnetic fields, which could perhaps be effected by preferring extreme-sized mutants (shifting the resonance away from the applied field). We do not have the biological knowledge to know if this is plausible; it is simply worth mentioning.

\end{multicols}







\clearpage
%%%%%%%%%%%%%%%%%%%%%%%%%%%%%%%%%%%%%%%%%%%%%%%%
{\Large Microwave Musings}\\
\begin{multicols}{1}
%%%%%%%%%%%%%%%%%%%%%%%%%%%%%%%%%%%%%%%%%%%%%%%%

\noindent\fbox{\parbox{\linewidth}{
	Toolchain:
	\begin{itemize}
	\item Failed oscillator feedback-loop optimization toolchain: QUCS 0.0.20 + python-qucs + scipy's 'basinhopper'
	\item Successful A slightly modified ngspice + ngspyce + pyEVTK
	\item gprMax for FDTD EM simulation
	\item KiCAD, wcalc, scikit-rf, ngspice
	\end{itemize}
}}
%


The oscillator is a 'wideband double-tuned varactor VCO', based almost verbatim on [Tsuru 2008] and the reference design in Figure 8.36, p378 of [Grebennikov 2007]. 

The rest of the papers in the bibliography were highly enlightening regarding the principles of microwave oscillator design.

This topology of oscillator worked marvellously on essentially the first try. 

The process by which we failed miserably to design our own oscillator topology is detailed in the supplemental.

The following properties: P1dB

For ease of design and simulation, a device with Touchstone S-parameters and SPICE files is greatly preferable.

It is commonly claimed that FR4 is a 'slow' substrate, and that the high loss tangent of ~0.02 makes it unsuitable for microwave systems.

However, with this PCB substrate, the expected loss of only 0.026 dB/mm on signals of minimum 0 dBm is patently acceptable. As viruses do not have a discriminating palate, we are also only minimally concerned with $S_{11}$ reflections, noise, or spurs, so precise impedance control is not required; the wideband VCO sweep accomodates for any variations in resonant frequency due to wide manufacturing tolerances.

As a result of our gross incompetence, the oscillator was designed via an inane, roundabout, and fiendishly tedious manner, and our description of this technique only contributes to the field by being suitable for a dartboard; our analysis can only hold water when combined with paper mache; and a lesser invertebrate in posession of a copy of Microwave Oscillator Design would have accomplished the task faster.

Designing an oscillator of this type in one step with a few kindergarten equations appears to be well within the reach of modern network analytical techniques. Genetic algorithms are quite well matched to this problem, and many commercial software packages are  [computer microwave design book]. 

Initially, 

Not having a good analytical understanding, we resorted to using a purely computational method.

In addition, the design of wideband feedback-loop VCOs is a relatively well-explored field, and many reference designs exist. 

The key sticking point - which we missed - seems to be that at these scales, using microstrip design techniques, the parasitics of any possible filter structure are so large that the small change in impedance that varactors can provide cannot tune the circuit by any meaningful amount.

"Using the lead inductances of the bipolar transistor and varactors provides the required value of the base inductance".

Indeed, [Tsuru 2008]'s "tuned circuit" in Fig 8 is, in fact, just the varactors, plus an almost invisible high-impedance line.





Several analytical filter design methods were 

This is apparently known as a "double-tuned" wideband.

%
\paragraph{\textbf{The feedback loop oscillator}}\

Oscillators must meet the Barkhausen criterion:

\begin{itemize}

\item A 360 degree phase shift around the feedback loop (including the phase shift contribution from the amplifier, which itself varies greatly with frequency)
\item A loop gain $>1.$ 

\end{itemize}\
%
However, a third element is also required:
%
\begin{itemize}
\item A frequency-selective element that restricts oscillation modes and decreases phase noise.
\end{itemize}
%

With large feedback-loop structures, such as long PiN-switched phasing lines, proximity effects from nearby flesh would detune the oscillator significantly.

 design of a triple-tuned oscillator.


\paragraph{}
The buffer amplifiers 




\clearpage
%%%%%%%%%%%%%%%%%%%%%%%%%%%%%%%%%%%%%%%%%%%%%%%%
{\Large Mass production}
%%%%%%%%%%%%%%%%%%%%%%%%%%%%%%%%%%%%%%%%%%%%%%%%


If this 'electromagnetic mask' form is the ideal (not nearly ), a minimum of 5 GaAs or SiGe:C transistors will be required.

There are, for instance, 1 million hospital beds in the U.S. [AHA 2018]. 

It is difficult to determine the supply capacity for these modern semiconductor processes. A few fermi est

GaAs MMIC market of \$2.2 Bn USD / random sample of MMIC prices = 

3e9 wifi connected devices produced each year.

The largest RFID plants can produce


The techniques and equipment required to produce these are non-trivial. SiGe:C, for instance, requires, 


Figures are not forthcoming, 

Given the supply difficulties of comparatively simple materials such as Tyvek at pandemic scales, it is difficult to imagine that RF semiconductor production can be immediately re-tasked and scaled to this degree. 

\paragraph{\textbf{Vacuum RF triode}}\

One concern is the large filament heater power, which prevents the use of low-cost button cells for power. Use of cold-cathode field-emitter arrays would alleviate this issue, but at the cost of complexity.

Small tungsten incandescent lights are available down to 0.05 watts. With a suitable high-efficiency cathode coating, a pulsed heater power of less than 0.02



\end{multicols}





\subfile{acknowledgement}




\subfile{supplemental}







\end{document}
