\documentclass[fleqn,10pt]{article}





\usepackage[left=2cm,right=2cm,
			top=1.25cm,
			bottom=2.25cm,%
			headheight=11pt,%
			letterpaper]{geometry}
			
\frenchspacing			

\nonstopmode

\usepackage{multicol}
\usepackage{fancyhdr}
\usepackage{blindtext,graphicx}
\usepackage[absolute]{textpos}
\usepackage[parfill]{parskip}
\usepackage[colorlinks=true,citecolor=brown]{hyperref}
\usepackage{gensymb}
\usepackage{csquotes}
\usepackage{amsmath}
\usepackage{fontawesome}
\usepackage{orcidlink}
\usepackage{standalone}
\usepackage{pdfpages}
\usepackage{subfiles}
\usepackage{svg}
\usepackage{sidecap}
\usepackage{float}
\usepackage{amssymb}
\usepackage{textcomp}
\usepackage{lettrine}
%\usepackage[T1]{fontenc}

\usepackage{booktabs,caption}
\usepackage[flushleft]{threeparttable}

%\usepackage{biblatex}
\usepackage[backend=bibtex8, sorting=none]{biblatex}

\addbibresource{processed.bib}
%biblatex has a zoterordfxml
% might avoid the need for python bibtex_collections.py

\usepackage{etoolbox}
\AtBeginEnvironment{quote}{\small}

\usepackage{pifont}
\newcommand{\cmark}{\ding{51}}%
\newcommand{\xmark}{\ding{55}}%


\newcommand{\Wsqm}{$\text{ W/m}^2$}

\newcommand{\ghfile}[1]{\href{https://github.com/0xDBFB7/covidinator/tree/master/#1}{\faGithub/\url{#1} }}

%\newcommand{\supercite}[1]{}
%\newcommand{\supercollect}[1]{}


\newlength{\PRLlen}
\newcommand*\PRLsep[1]{{\itshape \Large\settowidth{\PRLlen}{#1}\advance\PRLlen by -\textwidth\divide\PRLlen by -2\noindent\makebox[\the\PRLlen]{\resizebox{\the\PRLlen}{1pt}{$\blacktriangleleft$}}\raisebox{-.5ex}{#1}\makebox[\the\PRLlen]{\resizebox{\the\PRLlen}{1pt}{$\blacktriangleright$}}\bigskip}}


\usepackage{graphicx}
\graphicspath{ {../media/} }

\usepackage{tcolorbox}
\newtcolorbox{autem}{colback=red!5!white,colframe=red!75!black}
\newtcolorbox{toolchain}{colback=blue!5!white,colframe=blue!40!black!40}
%https://tex.stackexchange.com/questions/66154/how-to-construct-a-coloured-box-with-rounded-corners

%\usepackage[sfdefault,light]{roboto}

\setlength{\TPHorizModule}{1cm}
\setlength{\TPVertModule}{1cm}



\begin{document}



\subfile{title}




\section{Figures which have nothing to do with our conclusion but are pretty nonetheless}

\begin{figure}[H]
	\captionsetup{singlelinecheck = false, justification=justified}
	\centering
	\includegraphics[width=\textwidth]{pulse_exposure_setup.JPG}
	\caption{\\ \textit{Chengxiang} pulse exposure jig. pulser seen off the left. Autosampler Pulse shaping plate not shown. Aluminum disc belongs to the scan-conversion digitizer, not used in this study.}
\end{figure}
	
\begin{figure}[H]
	\captionsetup{singlelinecheck = false, justification=justified}
	\centering
	\includegraphics[width=\textwidth]{eppenwolf_2.jpg}
	\caption{\\ \textit{Eppenwolf} misguided pulsed X-Band microwave spectrometer used in this study, pictured with the near-field cuvette installed.}
\end{figure}

\begin{figure}[H]
	\captionsetup{singlelinecheck = false, justification=justified}
	\centering
	\includegraphics[width=\textwidth]{bronch_9GHz_500W_2_transparent.png}
	\caption{\\ \textit{Eppenwolf} misguided pulsed X-Band microwave spectrometer used in this study, pictured with the near-field cuvette installed.}
\end{figure}




\begin{figure}[H]
	\makebox[\textwidth][c]{
		\includegraphics[width=\textwidth*2]{chunk_4_line_2_1.png}
	}%
	\caption{Chunk 4, it.is Founda}
\end{figure}


\clearpage


\begin{figure}[H]
	\captionsetup{singlelinecheck = false, justification=justified}
	\centering
	\includegraphics[width=\textwidth]{CPWG_sim_pretty_2}
	\caption{FDTD simulation geometry of the 0.3 mm coplanar waveguide exposure cell. Red object is the fluid channel. Visualized with Paraview.
		\ghfile{covidinator/electronics/simple_fdtd/run/microfluidic_coplanar.py}}
\end{figure}


\begin{figure}[H]
	\captionsetup{singlelinecheck = false, justification=justified}
	\centering
	\includesvg[width=\textwidth]{compChargePlot}
	\caption{
		\light{
			\\
			We've got a thing\\
			that's called\\
			{\it Radar Love}\\}
		We've got a wave\\
		in the air}
\end{figure}



such a waveform may seem implausible but can probably be produced by e.g. an Auston switch feeding a pulse shaping line.


\clearpage

\subfile{background}














\clearpage





%\subfile{safety}





\subfile{brillouin_pulse_math}



%\paragraph{}


%\subfile{acknowledgement}


\subfile{charge}

\subfile{viscosity}

\subfile{experiment_1}

\subfile{experiment_2}

\subfile{molecular_dynamics}

\section{Appendix on microwave design}

\subfile{supplemental}

\section{Appendix on microbiology}

\subfile{sic_susceptor}

\subfile{detection}

\subfile{fluorescence}

\subfile{lessons_learned}

\subfile{acknowledgement}

\nocite{*}

% LC_ALL=C tr -dc '\0-\177' <references.bib >newfile && mv newfile references.bib 

%\printbibliography[title={General fluidics resources}]

\end{document}
