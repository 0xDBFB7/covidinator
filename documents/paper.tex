
\documentclass[fleqn,10pt]{article}

\usepackage[left=2cm,right=2cm,
			top=1.25cm,
			bottom=2.25cm,%
			headheight=11pt,%
			letterpaper]{geometry}
			
\frenchspacing
			
\usepackage{multicol}
\usepackage{fancyhdr}
\usepackage{blindtext,graphicx}
\usepackage[absolute]{textpos}
\usepackage[parfill]{parskip}

%\usepackage[sfdefault,light]{roboto}

\setlength{\TPHorizModule}{1cm}
\setlength{\TPVertModule}{1cm}

\title{On the application of microwave-acoustic resonant viral inactivation
\thanks{I would be delighted to include any criticisms or comments you may have; preferably leave them on the GitHub issues page, or email therobotist@gmail.com, @0xDBFB7 on Twitter, or irc.0xDBFB7.com:6667 \#covid.}}
\date{May 2020}
\author{Based primarily on exceptional work by }


\begin{document}



\flushbottom 
\maketitle
\thispagestyle{empty}

This work was prepared by an undergraduate and has not been peer-reviewed. Additionally, the author has no prior experience with either biology or microwave design. 
Please consider all claims with appropriate skepticism.


\begin{textblock}{5}(1,1)
\noindent Please view the latest version at github.com/0xDBFB7/covidinator
\end{textblock}
%\begin{textblock}{5}(1,27)
%\end{textblock}

\null\begin{tabular}[t]{l@{}}
  {Daniel Correia} \\
  \textit{York University}
\end{tabular}



\begin{abstract}
We extend this landmark work rather trivially by:
\begin{itemize}
  \item Establishing the time dependence of inactivation
  \item Demonstrating a modulation scheme that decreases the inactivation threshold to below current safety levels in surrogate bacteriophage
  \item Demonstrating a prototype emitter in an "electromagnetic mask" form-factor, costing about \$5 in prototype quantities, which can reasonably be produced in 10 million-of quantities
  \item Testing power thresholds in various conditions; biological fluids of various conductivities and pHs
  \item Using a coarse-grained molecular-dynamics simulation to optimize the impulse
  \item Using a virus-in-the-loop optimization with a centrifugal microfluidic system
  \item Discussing the biological basis for the safety of the device
  \item Showing that the deviation from the expected theshold could be explained by variance in the 
\end{itemize}
\end{abstract}









\begin{multicols}{1}

With typical 2.4 GHz microwave exposure, sterilization occurs by heating of the fluid and tissue.

Of course, the temperature of the atoms of the virus undoubtedly increase; but the 

\paragraph{\textbf{Time dependence}}\


Both Yang and [] (apparently arbitrarily) used a 15-minute exposure in their tests - a very reasonable decision given the focus of their paper. 

To determine the effectiveness against airborne particles, and to minimize the power required in a dwelling phased-array beam, we must first establish the required duration of exposure. 

As opposed to chemical inactivation, where the time dependence is dominated by viscous fluid dynamic effects [Hirose 2017], or UV inactivation, where a certain dose of photons must be absorbed, one might expect RF 

As a damped, driven oscillator, the ring-up time of the virus depends on the Q factor. Yang et al. state the Q of Inf. A as between 2 and 10, so at 8 GHz the steady-state amplitude should be reached in well under 1 us. 

[] have found a significant mechanical fatigue effect in phage capsids, where a small strain applied repetitively eventually causes a fracture. At low field strengths, such a mechanism could perhaps extend the exposure required to break the capsid or membrane. 



\end{multicols}

\clearpage


\begin{multicols}{1}

\noindent\fbox{\parbox{\linewidth}{
	Toolchain:
	\begin{itemize}
	\item QUCS 0.0.20 for AC simulation, with python-qucs and scipy's 'basinhopper' for optimization
	\item KiCAD for PCB design and schematic capture
	\item gprMax for FDTD EM simulation
	\item 
	\end{itemize}
}}
%
The following properties: P1dB

For ease of design and simulation, a device with Touchstone S-parameters and SPICE files is greatly preferable.
%
\paragraph{\textbf{The feedback loop oscillator}}\

Oscillators must meet the Barkhausen criterion:

\begin{itemize}

\item A 360 degree phase shift around the feedback loop (including the phase shift contribution from the amplifier, which itself varies greatly with frequency)
\item A loop gain $>1.$ 

\end{itemize}\
%
Generally a third element is also used:
%
\begin{itemize}
\item A frequency-selective element that restricts oscillation modes and decreases phase noise.
\end{itemize}
%
We encountered proximity effects from reflections and near-field coupling off nearby flesh; if this instablity, mode-hopping and de-tuning becomes an issue, it could be mitigated by way of shielding (adding a manufacturing step), or a tight filter around the desired mode.



 design of a triple-tuned oscillator.


\paragraph{}
The buffer amplifiers 

\paragraph{Biasing}\

In our simulations, the varactor-tuned feedback circuit appeared to be particularly sensitive to the introduction of bias-tees. 


The gate must be weakly pulled to ground, otherwise stray charge destroys the oscillation.

\fancyhead[C]{style 1 with thin line}



With 0.79 mm FR4 substrate and 0.2 mm (8 mil) wide traces, the maximum impedance achievable was about 115 ohms. 

If suitably high-impedance traces are not available, a common technique is to use a quarter-wavelength line (approximately 6 mm with the above parameters) terminated with a stub to produce a virtual short or open circuit [Seo 2007]. 

However, reflections from these structures still appeared to distort the frequency/phase response beyond repair, even with ostensibly wideband stubs [Syrett 1980].

\noindent\fbox{\parbox{\linewidth}{
	Rebuke: despite this blather, many other papers have had success with bias-tees at these frequencies.
}}

In production, these could be accomodated by graphite-polymer printed resistors. 


Odd-pole varactor-loaded combline filters appeared to have excellent phase and frequency response; however, the geometry necessitates low-inductance via stitching to the ground plane.

\noindent\fbox{\parbox{\linewidth}{
	Rebuke: Others have had great success with varactor-tuned comblines, especially non-grounded lines.
}}

[Tsuru 2008 fig. 10] is an excellent review of various oscillator designs.

The parasitic inductance of common varactors appears to become problematic at these frequencies (but not for non-wideband use).


\clearpage


{\Large Captain's Log, Supplemental}

\paragraph{Timeline}

\paragraph{Comments}

\paragraph{Lessons Learned}

When iteratively testing with simulations, it may be helpful to automatically save a package with images of all the components (schematics, graphs) of each distinct test. 

Version control alone isn't quite enoughjust having a simulation setup file somewhere in the commit history isn't "discoverable" - that is, you should be able to see what the input and output was without re-running the simulation; and manually taking notes tended to disrupt the flow of testing.






\end{multicols}

\end{document}
