\documentclass[fleqn,10pt]{article}





\usepackage[left=2cm,right=2cm,
			top=1.25cm,
			bottom=2.25cm,%
			headheight=11pt,%
			letterpaper]{geometry}
			
\frenchspacing			

\nonstopmode




\usepackage{lmodern}
\usepackage[T1]{fontenc}
\usepackage[utf8]{inputenc}



\usepackage{noweb}

\usepackage{multicol}
\usepackage{fancyhdr}
\usepackage{blindtext,graphicx}
\usepackage[absolute]{textpos}
%\usepackage[parfill]{parskip}
\usepackage{parskip}
\setlength{\parskip}{\baselineskip}

\usepackage[colorlinks=true,citecolor=brown]{hyperref}
\usepackage{gensymb}
\usepackage{csquotes}
\usepackage{amsmath}
\usepackage{fontawesome}
\usepackage{orcidlink}
\usepackage{standalone}
\usepackage{pdfpages}
\usepackage{subfiles}
\usepackage{svg}
\usepackage{sidecap}
\usepackage{float}
\usepackage{amssymb}
\usepackage{textcomp}
\usepackage{lettrine}
%\usepackage[T1]{fontenc}

\usepackage{soul}


%\usepackage{draftwatermark}
%\SetWatermarkText{DRAFT}
%\SetWatermarkScale{0.25}

\usepackage{booktabs,caption}
\usepackage[flushleft]{threeparttable}

%\usepackage{biblatex}
\usepackage[backend=bibtex8, sorting=none, style=chem-angew]{biblatex}

\let\cite\footfullcite

%\let\cite\footcite

\addbibresource{processed.bib}
%biblatex has a zoterordfxml
% might avoid the need for python bibtex_collections.py



\usepackage{etoolbox}
\AtBeginEnvironment{quote}{\small}




\usepackage{pifont}
\newcommand{\cmark}{\ding{51}}%
\newcommand{\xmark}{\ding{55}}%


\newcommand{\citationneeded}[1][]{\textsuperscript{[\color{blue}{\it \bf{citation needed}#1}]}}
\newcommand{\dubiousdiscuss}[1][]{\textsuperscript{\color{blue} [{\it \bf{dubious-discuss}}]} }

\newcommand{\light}[1]{\textcolor{gray}{#1}}

%
%
\usepackage{titlesec}
%
%% custom section


\titleformat{\section}
{\normalfont\LARGE\bfseries}{\thesection}{1em}{}
%\titleformat{\section}
%{\normalfont\LARGE\bfseries\PRLsep}
%{{{{\itshape \thesection\hskip 9pt\textpipe\hskip 9pt}}}}{0pt}{}
%
%% custom section
%\titleformat{\subsection}
%{\normalfont\Large\bfseries\PRLsep}
%{{{{\itshape \thesection\hskip 9pt\textpipe\hskip 9pt}}}}{0pt}{}
%
%
%


\newcommand{\Wsqm}{$\text{ W/m}^2$}

\newcommand{\ghfile}[1]{\href{https://github.com/0xDBFB7/covidinator/tree/master/#1}{\faGithub/\url{#1} }}

%\newcommand{\supercite}[1]{}
%\newcommand{\supercollect}[1]{}


\newlength{\PRLlen}
\newcommand*\PRLsep[1]{{\itshape \Large\settowidth{\PRLlen}{#1}\advance\PRLlen by -\textwidth\divide\PRLlen by -2\noindent\makebox[\the\PRLlen]{\resizebox{\the\PRLlen}{1pt}{$\blacktriangleleft$}}\raisebox{-.5ex}{#1}\makebox[\the\PRLlen]{\resizebox{\the\PRLlen}{1pt}{$\blacktriangleright$}}\bigskip}}


\renewcommand{\thefootnote}{\textcolor{gray}{\arabic{footnote}}}


\usepackage{graphicx}
\graphicspath{ {../media/} 
				{../firmware/eppenwolf/runs/sic_susceptor/} 
			}

\usepackage{tcolorbox}
\newtcolorbox{protocol}{colback=yellow!5!white,colframe=yellow!75!black}
\newtcolorbox{equipment}{colback=orange!5!white,colframe=orange!75!black}
\newtcolorbox{autem}{colback=red!5!white,colframe=red!75!black}
\newtcolorbox{toolchain}{colback=blue!5!white,colframe=blue!40!black!40}
\newtcolorbox{sidenote}{colback=cyan!5!white,colframe=blue!40!black!40}
%https://tex.stackexchange.com/questions/66154/how-to-construct-a-coloured-box-with-rounded-corners

%\usepackage[sfdefault,light]{roboto}

\setlength{\TPHorizModule}{1cm}
\setlength{\TPVertModule}{1cm}





%%%%********************************************************************
% fancy quotes
\definecolor{quotemark}{gray}{0.7}
\makeatletter
\def\fquote{%
	\@ifnextchar[{\fquote@i}{\fquote@i[]}%]
}%
\def\fquote@i[#1]{%
	\def\tempa{#1}%
	\@ifnextchar[{\fquote@ii}{\fquote@ii[]}%]
}%
\def\fquote@ii[#1]{%
	\def\tempb{#1}%
	\@ifnextchar[{\fquote@iii}{\fquote@iii[]}%]
}%
\def\fquote@iii[#1]{%
	\def\tempc{#1}%
	\vspace{1em}%
	\noindent%
	\begin{list}{}{%
			\setlength{\leftmargin}{0.1\textwidth}%
			\setlength{\rightmargin}{0.1\textwidth}%
		}%
		\item[]%
		\begin{picture}(0,0)%
		\put(-15,-5){\makebox(0,0){\scalebox{3}{\textcolor{quotemark}{``}}}}%
		\end{picture}%
		\begingroup\itshape}%
	%%%%********************************************************************
	\def\endfquote{%
		\endgroup\par%
		\makebox[0pt][l]{%
			\hspace{0.8\textwidth}%
			\begin{picture}(0,0)(0,0)%
			\put(15,15){\makebox(0,0){%
					\scalebox{3}{\color{quotemark}''}}}%
			\end{picture}}%
		\ifx\tempa\empty%
		\else%
		\ifx\tempc\empty%
		\hfill\rule{100pt}{0.5pt}\\\mbox{}\hfill\tempa,\ \emph{\tempb}%
		\else%
		\hfill\rule{100pt}{0.5pt}\\\mbox{}\hfill\tempa,\ \emph{\tempb},\ \tempc%
		\fi\fi\par%
		\vspace{0.5em}%
	\end{list}%
}%
\makeatother







%%%%********************************************************************
%title link to doi
\newbibmacro{string+doiurlisbn}[1]{%
	\iffieldundef{doi}{%
		\iffieldundef{url}{%
			\iffieldundef{isbn}{%
				\iffieldundef{issn}{%
					#1%
				}{%
					\href{http://books.google.com/books?vid=ISSN\thefield{issn}}{#1}%
				}%
			}{%
				\href{http://books.google.com/books?vid=ISBN\thefield{isbn}}{#1}%
			}%
		}{%
			\href{\thefield{url}}{#1}%
		}%
	}{%
		\href{https://doi.org/\thefield{doi}}{#1}%
	}%
}

\DeclareFieldFormat{journaltitle}{\usebibmacro{string+doiurlisbn}{\mkbibemph{#1}}}

\begin{document}



\subfile{title}




\section{Figures which have nothing to do with our conclusion but are pretty nonetheless}

\begin{figure}[H]
	\captionsetup{singlelinecheck = false, justification=justified}
	\centering
	\includegraphics[width=\textwidth]{pulse_exposure_setup.JPG}
	\caption{\\ \textit{Chengxiang} pulse exposure jig. pulser seen off the left. Autosampler Pulse shaping plate not shown. Aluminum disc belongs to the scan-conversion digitizer, not used in this study.}
\end{figure}
	
\begin{figure}[H]
	\captionsetup{singlelinecheck = false, justification=justified}
	\centering
	\includegraphics[width=\textwidth]{eppenwolf_2.jpg}
	\caption{\\ \textit{Eppenwolf} misguided pulsed X-Band microwave spectrometer used in this study, pictured with the near-field cuvette installed.}
\end{figure}

\begin{figure}[H]
	\captionsetup{singlelinecheck = false, justification=justified}
	\centering
	\includegraphics[width=\textwidth]{bronch_9GHz_500W_2_transparent.png}
	\caption{\\ \textit{Eppenwolf} misguided pulsed X-Band microwave spectrometer used in this study, pictured with the near-field cuvette installed.}
\end{figure}




\begin{figure}[H]
	\makebox[\textwidth][c]{
		\includegraphics[width=\textwidth*2]{chunk_4_line_2_1.png}
	}%
	\caption{Chunk 4, it.is Founda}
\end{figure}


\clearpage


\begin{figure}[H]
	\captionsetup{singlelinecheck = false, justification=justified}
	\centering
	\includegraphics[width=\textwidth]{CPWG_sim_pretty_2}
	\caption{FDTD simulation geometry of the 0.3 mm coplanar waveguide exposure cell. Red object is the fluid channel. Visualized with Paraview.
		\ghfile{covidinator/electronics/simple_fdtd/run/microfluidic_coplanar.py}}
\end{figure}


\begin{figure}[H]
	\captionsetup{singlelinecheck = false, justification=justified}
	\centering
	\includesvg[width=\textwidth]{compChargePlot}
	\caption{
		\light{
			\\
			We've got a thing\\
			that's called\\
			{\it Radar Love}\\}
		We've got a wave\\
		in the air}
\end{figure}



such a waveform may seem implausible but can probably be produced by e.g. an Auston switch feeding a pulse shaping line.


\clearpage

\subfile{background}














\clearpage





%\subfile{safety}





\subfile{brillouin_pulse_math}



%\paragraph{}


%\subfile{acknowledgement}


\subfile{charge}

\subfile{viscosity}

\subfile{experiment_1}

\subfile{experiment_2}

\subfile{molecular_dynamics}

\section{Appendix on microwave design}

\subfile{supplemental}

\section{Appendix on microbiology}

\subfile{sic_susceptor}

\subfile{detection}

\subfile{fluorescence}

\subfile{lessons_learned}

\subfile{acknowledgement}

\nocite{*}

% LC_ALL=C tr -dc '\0-\177' <references.bib >newfile && mv newfile references.bib 

%\printbibliography[title={General fluidics resources}]

\end{document}
