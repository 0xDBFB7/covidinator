\documentclass[fleqn,10pt]{article}





\usepackage[left=2cm,right=2cm,
			top=1.25cm,
			bottom=2.25cm,%
			headheight=11pt,%
			letterpaper]{geometry}
			
\frenchspacing			

\nonstopmode




\usepackage{lmodern}
\usepackage[T1]{fontenc}
\usepackage[utf8]{inputenc}



\usepackage{noweb}

\usepackage{multicol}
\usepackage{fancyhdr}
\usepackage{blindtext,graphicx}
\usepackage[absolute]{textpos}
%\usepackage[parfill]{parskip}
\usepackage{parskip}
\setlength{\parskip}{\baselineskip}

\usepackage[colorlinks=true,citecolor=brown]{hyperref}
\usepackage{gensymb}
\usepackage{csquotes}
\usepackage{amsmath}
\usepackage{fontawesome}
\usepackage{orcidlink}
\usepackage{standalone}
\usepackage{pdfpages}
\usepackage{subfiles}
\usepackage{svg}
\usepackage{sidecap}
\usepackage{float}
\usepackage{amssymb}
\usepackage{textcomp}
\usepackage{lettrine}
%\usepackage[T1]{fontenc}

\usepackage{soul}


%\usepackage{draftwatermark}
%\SetWatermarkText{DRAFT}
%\SetWatermarkScale{0.25}

\usepackage{booktabs,caption}
\usepackage[flushleft]{threeparttable}

%\usepackage{biblatex}
\usepackage[backend=bibtex8, sorting=none, style=chem-angew]{biblatex}

\let\cite\footfullcite

%\let\cite\footcite

\addbibresource{processed.bib}
%biblatex has a zoterordfxml
% might avoid the need for python bibtex_collections.py



\usepackage{etoolbox}
\AtBeginEnvironment{quote}{\small}




\usepackage{pifont}
\newcommand{\cmark}{\ding{51}}%
\newcommand{\xmark}{\ding{55}}%


\newcommand{\citationneeded}[1][]{\textsuperscript{[\color{blue}{\it \bf{citation needed}#1}]}}
\newcommand{\dubiousdiscuss}[1][]{\textsuperscript{\color{blue} [{\it \bf{dubious-discuss}}]} }

\newcommand{\light}[1]{\textcolor{gray}{#1}}

%
%
\usepackage{titlesec}
%
%% custom section


\titleformat{\section}
{\normalfont\LARGE\bfseries}{\thesection}{1em}{}
%\titleformat{\section}
%{\normalfont\LARGE\bfseries\PRLsep}
%{{{{\itshape \thesection\hskip 9pt\textpipe\hskip 9pt}}}}{0pt}{}
%
%% custom section
%\titleformat{\subsection}
%{\normalfont\Large\bfseries\PRLsep}
%{{{{\itshape \thesection\hskip 9pt\textpipe\hskip 9pt}}}}{0pt}{}
%
%
%


\newcommand{\Wsqm}{$\text{ W/m}^2$}

\newcommand{\ghfile}[1]{\href{https://github.com/0xDBFB7/covidinator/tree/master/#1}{\faGithub/\url{#1} }}

%\newcommand{\supercite}[1]{}
%\newcommand{\supercollect}[1]{}


\newlength{\PRLlen}
\newcommand*\PRLsep[1]{{\itshape \Large\settowidth{\PRLlen}{#1}\advance\PRLlen by -\textwidth\divide\PRLlen by -2\noindent\makebox[\the\PRLlen]{\resizebox{\the\PRLlen}{1pt}{$\blacktriangleleft$}}\raisebox{-.5ex}{#1}\makebox[\the\PRLlen]{\resizebox{\the\PRLlen}{1pt}{$\blacktriangleright$}}\bigskip}}


\renewcommand{\thefootnote}{\textcolor{gray}{\arabic{footnote}}}


\usepackage{graphicx}
\graphicspath{ {../media/} 
				{../firmware/eppenwolf/runs/sic_susceptor/} 
			}

\usepackage{tcolorbox}
\newtcolorbox{protocol}{colback=yellow!5!white,colframe=yellow!75!black}
\newtcolorbox{equipment}{colback=orange!5!white,colframe=orange!75!black}
\newtcolorbox{autem}{colback=red!5!white,colframe=red!75!black}
\newtcolorbox{toolchain}{colback=blue!5!white,colframe=blue!40!black!40}
\newtcolorbox{sidenote}{colback=cyan!5!white,colframe=blue!40!black!40}
%https://tex.stackexchange.com/questions/66154/how-to-construct-a-coloured-box-with-rounded-corners

%\usepackage[sfdefault,light]{roboto}

\setlength{\TPHorizModule}{1cm}
\setlength{\TPVertModule}{1cm}





%%%%********************************************************************
% fancy quotes
\definecolor{quotemark}{gray}{0.7}
\makeatletter
\def\fquote{%
	\@ifnextchar[{\fquote@i}{\fquote@i[]}%]
}%
\def\fquote@i[#1]{%
	\def\tempa{#1}%
	\@ifnextchar[{\fquote@ii}{\fquote@ii[]}%]
}%
\def\fquote@ii[#1]{%
	\def\tempb{#1}%
	\@ifnextchar[{\fquote@iii}{\fquote@iii[]}%]
}%
\def\fquote@iii[#1]{%
	\def\tempc{#1}%
	\vspace{1em}%
	\noindent%
	\begin{list}{}{%
			\setlength{\leftmargin}{0.1\textwidth}%
			\setlength{\rightmargin}{0.1\textwidth}%
		}%
		\item[]%
		\begin{picture}(0,0)%
		\put(-15,-5){\makebox(0,0){\scalebox{3}{\textcolor{quotemark}{``}}}}%
		\end{picture}%
		\begingroup\itshape}%
	%%%%********************************************************************
	\def\endfquote{%
		\endgroup\par%
		\makebox[0pt][l]{%
			\hspace{0.8\textwidth}%
			\begin{picture}(0,0)(0,0)%
			\put(15,15){\makebox(0,0){%
					\scalebox{3}{\color{quotemark}''}}}%
			\end{picture}}%
		\ifx\tempa\empty%
		\else%
		\ifx\tempc\empty%
		\hfill\rule{100pt}{0.5pt}\\\mbox{}\hfill\tempa,\ \emph{\tempb}%
		\else%
		\hfill\rule{100pt}{0.5pt}\\\mbox{}\hfill\tempa,\ \emph{\tempb},\ \tempc%
		\fi\fi\par%
		\vspace{0.5em}%
	\end{list}%
}%
\makeatother







%%%%********************************************************************
%title link to doi
\newbibmacro{string+doiurlisbn}[1]{%
	\iffieldundef{doi}{%
		\iffieldundef{url}{%
			\iffieldundef{isbn}{%
				\iffieldundef{issn}{%
					#1%
				}{%
					\href{http://books.google.com/books?vid=ISSN\thefield{issn}}{#1}%
				}%
			}{%
				\href{http://books.google.com/books?vid=ISBN\thefield{isbn}}{#1}%
			}%
		}{%
			\href{\thefield{url}}{#1}%
		}%
	}{%
		\href{https://doi.org/\thefield{doi}}{#1}%
	}%
}

\DeclareFieldFormat{journaltitle}{\usebibmacro{string+doiurlisbn}{\mkbibemph{#1}}}

\begin{document}

\subfile{title}
	
\subfile{introduction}

\clearpage



\paragraph{To-do list}

We see the following steps that must be undertaken before production can be started:

\begin{itemize}
  \item Re-run the experiment with 
  \item Verify 
  \item Failure-tree to ensure that power levels can never go above specified values.
  \item Obtain special permission from the FCC?
\end{itemize}

\clearpage
\begin{multicols}{1}

\supercite{Yang Efficient 2016}[1789 \href{https://doi.org/10.1038/srep18030}{\faExternalLink}]
\supercite{Yang Efficient 2016}[1789 \href{https://doi.org/10.1038/srep18030}{\faExternalLink}]

\supercite{Yang Efficient 2016}[1789 \href{https://doi.org/10.1038/srep18030}{\faExternalLink}]



[Frolich 1968] [Frolich 1980] \textrightarrow \ ([Hung 2014] $\parallel$ [Yang 2015] $\parallel$ [Sun 2017])\footnote{We have not conduc98 \href{https://doi.org/10.2514/3.29363}{\faExternalLink}ently unlike human cellular structures, as we shall see - coincidentally have just the right size, shape, stiffness, and net charge distribution to form a weak (Q=2) spherical dipole resonance mode which couples well to the microwave spectrum at approximately 8 GHz.

More critically, [Yang 2015] (and, in parallel, [Hung 2014]) theoretically model and then experimentally validate in various strains of Influenza A that - due to this acoustic-resonance effect - the power density levels required to crack the lipid envelope are near the safety limits for continuous exposure to humans.\footnote{Sort of. See below and supplemental.}

They demonstrate this with both a plaque and PCR assay, finding good agreement with the theoretical model.\footnote{As we will discuss, there are a few issues with the experimental technique.; sham, blinding, and dosimetry demanded by [Vjl.] are not mentioned.}

Like pumping a swing, this effect allows an otherwise inconsequential field magnitude to store energy over a small number of cycles until the virus is destroyed. \\
\\\\

\paragraph{Ramifications}\

If this mechanism exists, it would seem to provide significant advantages over existing UV or cold-plasma sterilization. 

If the extensions made in our paper are valid, this is a non-ionizing, non-thermal
%
\footnote{It may be useful to define 'non-thermal'; it caused us some confusion. Certainly the proteins of the virion locally absorb energy and increase in temperature. The key is that, when excited in this manner, the energies in the envelope are poorly modelled by a Maxwell-Boltzmann distribution; they are not given sufficient time to 'thermalize'. In contrast, with typical 2.4 GHz microwave exposure, sterilization can only occur by aggregate heating of the fluid and tissue.} 
%
, non-chemical technique, harmless for continuous exposure to tissue, which can sterilize air and surfaces alike, including skin, eyes, and within hair; it evolves no ozone, can readily be generated with \$1 USD - scale devices, acts instantaneously, and - perhaps most critically - could be made to act {\bf within infected tissues}.

The X-band is also minimally absorbed by air, allowing action in the far-field and scaling to square-kilometer areas with single installations.

\begin{autem}

{\it autem}\\
The previous paragraph may cut an impressive figure; but it hinges on all the rest of this paper being correct.

\end{autem}

\lettrine{It} cannot be overstated how unexpected and positively dubious this finding appears to be -at least, based on our limited research and experience to date.

For even the RF power limits set by standards organizations appear to be based on the observation that few significant resonance modes exist in biological tissues, and (as we shall see), this is grounded in solid {\it in vitro} (albeit more limited {\it in vivo}) evidence. 

This may account for why this paper has been ignored.

The precise structure and charge of the virion appears to be a singular anomaly in this otherwise catagorical non-existence.


\footnote{It should be noted that this {\it confined} acoustic resonance is subtly distinct from common-and-garden pipe-organ acoustic resonance; this is apparently not a strictly classical phenomenon. Besides standard Coulomb-like and Lennard-Jones-like interactions between constituent particles, if Fr\"{o}hlich is to be believed at these nanoscopic scales there are also wave-function interactions among the particles of the virus which can shift storage to modes not otherwise expected.

We confess to not yet understanding this phenomenon; fortunately, while helpful, the details of how this mode appears are not critical to implementing this technique.}

\footnote{``{[B]elow about 6 GHz, where EMFs penetrate deep into tissue (and thus require depth to be considered), it is useful to describe this in terms of “specific energy absorption rate” (SAR), which is the power absorbed per unit mass $(W/kg)$. Conversely, above 6 GHz, where EMFs are absorbed more superficially (making depth less relevant), it is useful to describe exposure in terms of the density of absorbed power over area $W/m^2$, which we refer to as “absorbed power density”}'' [ICNIRP 2020 \faExternalLink] }



\footnote{All values have been converted to $\text{W/m}^2$ to avoid confusion. 100 $\text{ W/m}^2 = 10 \text{ mW/cm}^2 = 10 \text{ dBm/cm}^2$.}



\end{multicols}
\clearpage
%%%%%%%%%%%%%%%%%%%%%%%%%%%%%%%%%%%%%%%%%%%%%%%%%%%%%%%%%%%%%%%%%%%%%%%%%%



A poignant summary of the consensus is found in [IEEE C95.1-2005], Annex B, "Identification of levels of RF exposure responsible for adverse effects: summary of the literature".

\begin{quote}
%%%%%%%%%%%
Further examination of the RF literature reveals no reproducible low level (non-thermal) effect that would
occur even under extreme environmental exposures. The scientific consensus is that there are no accepted
theoretical mechanisms that would suggest the existence of such effects. This consensus further supports the
analysis presented in this section, i.e., that harmful effects are and will be due to excessive absorption of
energy, resulting in heating that can result in a detrimentally elevated temperature. The accepted mechanism
is RF energy absorbed by the biological system through interaction with polar molecules (dielectric relaxation) or interactions with ions (ohmic loss) is rapidly dispersed to all modes of the system leading to an
average energy rise or temperature elevation. Since publication of ANSI C95.1-1982 [B6], significant
advances have been made in our knowledge of the biological effects of exposure to RF energy. This
increased knowledge strengthens the basis for and confidence in the statement that the MPEs and BRs in this
standard are protective against established adverse health effects with a large margin of safety.

\end{quote}

Unfortunately, the buried lede is that the power levels shown by [Yang 2015] are about twice the continuous safety levels. 

However, there is an obvious avenue of optimization. If the virus is destroyed in a few dozen nanoseconds via an electric field amplitude incidentally caused by an instantaneous power $\text{P}$, but tissue damage requires a temperature rise due to an energy deposition $\text{P} \  \text{dt}$, then we should minimize dt.\

We thus turn the CW microwave signal into a series of effectively instantaneous pulses.

The safety standards of [IEEE] and [ICNIRP] account for such time-domain modulation by specifying both an average power limit over a 6 minute period, and a time-integrated energy deposition limit. All known non-thermal and thermal physiological effects, including changes in the permeability of membranes, direct nerve stimulation, etc, are accounted for by obeying those two limits.

To drive home this point, apparently the best-quality evidence available fulfills sham requirements [Vijayalaxmi 2006] provide in-vitro. They expose blood lymphocytes to pulse power in precisely the regime required in this work; 8.2 GHz, 8 ns duration and 50 khz repetition rate, a whopping pulse power density of 250,000 $\text{W/m}^2$\footnote{computed from average power / duty cycle} (2500x the time-averaged power density safety limit), average power density 100 $\text{W/m}^2$ (the safety limit), for 2 hours, finding no change in any of the measured quantities. 

Simlarly, [Chemeris 2004] use 8.8 GHz, 180 ns pulse width, peak power 65,000 W, repetition rate 50 Hz, exposure duration 40 min, on frog etheyrocytes, and find genotoxicity only from the temperature rise.

So it appears that this is not merely rules-lawyering to exploit a loophole in the regulations; it is a physically-informed effect.

\begin{autem}

These papers (the best quality we are aware of) were extensively cherry-picked from the literature based on the results of a meta-analysis [Vj. 2019], listing the requirements for reliable in-vitro RF experiments.\\

To illustrate how sensitive this research is, [Chemeris 2004] mention:

\begin{quote}

The increase in DNA damage after exposure of cells to HPPP EMF shown in Table 2 was due to the temperature rise in the cell suspension by $3.5\pm0.1^{\circ}  $C. This was confirmed in sham-exposure experiments and experiments with incubation of cells for 40 min under the corresponding temperature conditions."

\end{quote}

There are a substantial number of papers in the literature which show positive effect sizes; these are listed in the bibliography. \\

The literature reviews conducted by standards organizations.\\

We are not aware of any past uses of these microwave en-masse. We do not have the 

\end{autem}

\cite{POLAROGRAPHICOXYGENSENSORS}

\collect{Flagship}

\clearpage
{\Large \it Talkin' 'bout the Variation}\\

To re-cap, [Yang 2015] theoretically model the virus to determine the minimum electric field required to destroy it. 

They assume that the virus is a simple damped harmonic oscillator, where the 'core' and 'shell' oscillate in opposition. 

They determine the net charge experimentally from microwave absorption measurements.

Since [Yang] try to compute the {\it threshold}, they use a value of 400 pN for the breaking strength of the envelope, obtained from [Li 2011]. However,  

"95%.".

\begin{center}
 \begin{tabular}{||c c c c||} 
 \hline
 Col1 & Col2 & Col2 & Col3 \\ [0.5ex] 
 \hline\hline
 1 & 6 & 87837 & 787 \\ 
 \hline
 2 & 7 & 78 & 5415 \\
 \hline
 3 & 545 & 778 & 7507 \\
 \hline
 4 & 545 & 18744 & 7560 \\
 \hline
 5 & 88 & 788 & 6344 \\ [1ex] 
 \hline
\end{tabular}
\end{center}





Of course, this application is hardly much better than an N95 mask, except that it is non-disposable and provides protection for eyes and skin.

Vomit


[Vj] mention the importance of precise dosimetry. Even simple structures can produce hot-spots 
Yang et al use both a plastic cuvette, and a single drop of solution on a glass slide; in either case, a sharp change dielectric constant is present.

To the extreme, some papers have used














\clearpage
\paragraph{\textbf{Time dependence}}\

The fact that the viral inactivation is non-thermal

Both [Yang 2015] and [Hung 2014] use an apparently arbitrary 15-minute exposure in their tests - a very reasonable decision, given the focus of their paper. 

The effectiveness against airborne particles, and to minimize the power required in a dwelling phased-array beam, we must establish the required duration of exposure.

{\color{red} speculative hypothesizing \{ } 

In contrast to chemical inactivation, where the time dependence appears to be dominated by viscous fluid dynamic effects [Hirose 2017], or UV inactivation, where a certain quantized dose of photons must be absorbed, we expected RF to act instantaneously.

As a damped, driven oscillator, the ring-up time of the virus depends on the Q factor. Yang et al. state the Q of Inf. A as between 2 and 10, so at 8 GHz the steady-state amplitude should be reached in well under 100 nanoseconds.???????????FIXME

[] found a significant mechanical fatigue effect in phage capsids, where a small strain applied repetitively eventually causes a fracture. Such a mechanism could perhaps extend the exposure required to break the capsid or membrane. Other mechanisms could include some sort of lipid denaturation, requiring an absolute amount of energy absorption to break or twist bonds and modify properties before the envelope fractures.


{\color{red}  \} } 





\subfile{safety}









\clearpage
%%%%%%%%%%%%%%%%%%%%%%%%%%%%%%%%%%%%%%%%%%%%%%%%
\begin{multicols}{1}
{\Large The Experiment}\\
%%%%%%%%%%%%%%%%%%%%%%%%%%%%%%%%%%%%%%%%%%%%%%%%

\paragraph{\textbf{Centrifugal microfluidics}}\

The field of centrifugal microfluidics is accelerating. 

Many CD microfluidics systems use standard CD molding techniques for the channels and machining techniques, using either acrylic or silicone. The turbidity sensor is most sensitive if the plastic is clear. Sterilization does not seem to be discussed. 

Polypropylene is the ideal material, being almost indefinitely autoclavable. It is quite difficult to machine.

\end{multicols}












\clearpage
%%%%%%%%%%%%%%%%%%%%%%%%%%%%%%%%%%%%%%%%%%%%%%%%
\begin{multicols}{1}
{\Large Modes of application}\\
%%%%%%%%%%%%%%%%%%%%%%%%%%%%%%%%%%%%%%%%%%%%%%%%

\paragraph{\textbf{Personal 'electromagnetic mask'}}\

Even with judicious use of phased-arrays, spatial power-combining, etc, each transistor can only reasonably maintain sterility in approx. $ 0.1 \text{ m}^3 $.

We therefore demonstrate this form factor because, superficially, there are fewer people than there are places. 

On the other hand, a personal device may present issues with participation and production volume. 

\paragraph{\textbf{Direct treatments at the fundamental}}\

Via [Hand 1982], the $E_{mag}=1/e$ penetration depth\footnote{"skin skin depth?"} is approximately 40 mm for 'dry' tissues and 5 mm for 'wet'. 



SARS is found widely distributed throughout many of the most favoured organs [Ding 2004], shielded by an average of 4 cm of chest wall [Schroeder 2013]; so safe external treatment of the body is unlikely.

However, destruction of lung tissue appears to be significant in the lethality of SARS [Nicholls 2006]. 

A bronchoscopic technique may therefore be effective, very similarly to that demonstrated by [Yuan 2019]: in adults, the bronchi are less than 2 mm thick [Theriault 2018] and the lungs themselves are only on the order of 7 mm thick [Chekan 2016]. 

The main bronchus is about 8 mm in diameter, which is smaller than the patch antenna used here; a monopole (or multiple, phased monopoles) like is probably more suitable.

\paragraph{\textbf{Subharmonics}}\

The 8 GHz wave need not be the fundamental. A modulated carrier wave with a more deeply penetrating wavelength which then locally interferes to produce the 8 GHz tone within the tissue could be useful.

The penetration depth shortens as frequency increases, until the so-called 'optical windows' in the absorption spectra of tissues, particularly in the near-IR range. 

\paragraph{\textbf{More fanciful concepts}}\

Many systems operate in these X-band frequency ranges; and precision 

It is easy to produce megawatts of power at these frequency ranges using Klystrons. Assuming an appropriate antenna, a single 50MW SLAC XL-class klystron with a PRF of 180 Hz could sanitize almost a square kilometer every second. \footnote{Giving 'asceptic field' new meaning.}

On the other hand, if this is conducted in outdoor free-space, sunlight

Existing marine, weather, and aviation radar systems often use the X-band; depending on the focusing power

\paragraph{Sensing}

Surprisingly, there is little discussion on microwave virus detection, with the notable exception of [Mehrotra 2019]. Such scholarly silence is usually an indicator that a technique is impossible for reasons so obvious they scarcely bear repeating. 

However, [Oberoi 2012] find that - in controlled conditions - they can detect a single E. Coli bacterium in 50 uL of broth via a microwave cavity. 

Of course, discriminating miniscule returns in free-space is not easy; but with precise knowledge of this resonance mode, it may become somewhat easier. Frequency-domain measurements can be made with incredible precision; and there are many parameters by which virions could be discriminated; ring-up time of the resonance, excitation non-linearity, etc.

\paragraph{}

It has been shown that virions can effect large-scale changes in the electric charges of their []. 

Use of this technique will provide a selection bias towards immunity to electromagnetic fields, which could perhaps be effected by preferring extreme-sized mutants (shifting the resonance away from the applied field). 

We do not have the biological knowledge to know if this is plausible; it simply seemed worth mentioning. 

\end{multicols}







\clearpage
%%%%%%%%%%%%%%%%%%%%%%%%%%%%%%%%%%%%%%%%%%%%%%%%
{\Large Microwave Musings}\\
\begin{multicols}{1}
%%%%%%%%%%%%%%%%%%%%%%%%%%%%%%%%%%%%%%%%%%%%%%%%

\noindent\fbox{\parbox{\linewidth}{
	Toolchain:
	\begin{itemize}
	\item Failed oscillator feedback-loop optimization toolchain: QUCS 0.0.20 + python-qucs + scipy's 'basinhopper'
	\item Successful A slightly modified ngspice + ngspyce + pyEVTK
	\item gprMax for FDTD EM simulation
	\item KiCAD, wcalc, scikit-rf, ngspice
	\end{itemize}
}}
%


Microwave design has a reputation for being the purview of wizards. Modern RF software packages like HFSS, Microwave Office, Agilent's ADS, and Mathworks' RF toolbox, for which; component models 

Luckily, this project falls perfectly within the subset of microwave technologies that do not require a goatee. The vast majority of the behavior of most circuits can be accurately modelled with the very same SPICE tools as one would use at low frequencies. With judicious application of reference designs, it seems to be possible to design 

Before about 2005, however, it seems to have been somewhat the norm to write a simple numerical code to solve the problem at hand based on underlying principles.

We are also aided by the fact that what took a \$150,000 computing cluster a single decade ago can now be done in a few minutes by a single budget GPU.


\end{multicols}


\clearpage
%%%%%%%%%%%%%%%%%%%%%%%%%%%%%%%%%%%%%%%%%%%%%%%%
{\Large Interference}\\
\begin{multicols}{1}
%%%%%%%%%%%%%%%%%%%%%%%%%%%%%%%%%%%%%%%%%%%%%%%%



%%%%%%%%%%%%%%%%%%%%%%%%%%%%%%%%%%%%%%%%%%%%%%%%
{\Large Mass production}
%%%%%%%%%%%%%%%%%%%%%%%%%%%%%%%%%%%%%%%%%%%%%%%%



Almost all components on the device can be replicated with fully vertically-integrated first-principles. Capacitors can be

If this 'electromagnetic mask' form is the ideal (not nearly), 

The largest RFID plants can produce

a minimum of 5 GaAs or SiGe:C transistors will be required. Without SOI or, it does not appear that 

As a lower bound, there are 1 million hospital beds in the U.S. [AHA 2018]; and as an upper bound obtaining global herd immunity would take 1.75 billion units.

It is difficult to determine the supply capacity for these semiconductor processes; information is not forthcoming from the manufacturers. Fermi estimates \footnote{GaAs MMIC market of \$2.2 Bn USD / random sample of MMIC prices = about 2 Bil devices / yr, 3e9 wifi connected devices produced each year,}



It is possible to use common Si-based devices at these frequencies, especially with second-harmonic techniques [Winch 1982]. However, obtaining the required gain and output power is not a trivial matter. 

The techniques and equipment required to produce these devices are extremely complex; load-lock UHV,  


Figures are not forthcoming, 

Given the supply difficulties of comparatively simple materials such as Tyvek at pandemic scales, it is difficult to imagine that RF semiconductor production can be immediately re-tasked and scaled to this degree. 

\paragraph{\textbf{Vacuum RF triode}}\



Especially combined with an integrated titanium sorption pump,

One concern is the large filament heater power, which prevents the use of low-cost button cells for power. Use of cold-cathode field-emitter arrays would alleviate this issue, but at the cost of complexity.

Small tungsten incandescent lights are available down to 0.05 watts. With a suitable high-efficiency cathode coating, a pulsed heater power of less than 0.02



\end{multicols}





\subfile{acknowledgement}




\subfile{supplemental}







\bibliography{1.bib}{}
\bibliographystyle{plain}
\end{document}
