\documentclass[fleqn,10pt]{article}

\usepackage[left=2cm,right=2cm,
			top=1.25cm,
			bottom=2.25cm,%
			headheight=11pt,%
			letterpaper]{geometry}
			
\frenchspacing			


\usepackage{multicol}
\usepackage{fancyhdr}
\usepackage{blindtext,graphicx}
\usepackage[absolute]{textpos}
\usepackage[parfill]{parskip}
\usepackage[colorlinks]{hyperref}
\usepackage{hyperref}
\usepackage{gensymb}
\usepackage{csquotes}

\usepackage{graphicx}
\graphicspath{ {../media/} }

\usepackage{tcolorbox}
\newtcolorbox{autem}{colback=red!5!white,colframe=red!75!black}
\newtcolorbox{toolchain}{colback=blue!5!white,colframe=blue!40!black!40}
%https://tex.stackexchange.com/questions/66154/how-to-construct-a-coloured-box-with-rounded-corners

%\usepackage[sfdefault,light]{roboto}

\setlength{\TPHorizModule}{1cm}
\setlength{\TPVertModule}{1cm}






\title{On the application of microwave acoustic resonant viral inactivation
\thanks{I would be delighted to include any criticisms or comments anyone may have; preferably leave them on the GitHub issues page, or email therobotist@gmail.com, @0xDBFB7 on Twitter, or irc.0xDBFB7.com:6667 \#covid.}}
\date{May 2020}
\author{Based primarily on exceptional work by }


\begin{document}

\flushbottom 
\maketitle
\thispagestyle{empty}




\begin{textblock}{5}(1,1)
\noindent Please view the latest version at github.com/0xDBFB7/covidinator
\end{textblock}
%\begin{textblock}{5}(1,27)
%\end{textblock}

\null\begin{tabular}[t]{l@{}}
  {Daniel Correia} \\
  \textit{York University}
\end{tabular}



\begin{abstract}

We extend this landmark work rather trivially by:


Aims:

\begin{itemize}
  \item Establishing the time dependence of inactivation
  \item Demonstrating a modulation scheme that decreases the inactivation threshold to below current safety levels in surrogate bacteriophage
  \item Demonstrating a prototype emitter in an "electromagnetic mask" form-factor, costing about \$5 in prototype quantities, which can reasonably be produced in 10 million-of quantities
  \item Testing power thresholds in various conditions; biological fluids of various conductivities and pHs
  \item Using a coarse-grained molecular-dynamics simulation to optimize the impulse
  \item Using a virus-in-the-loop optimization with a centrifugal microfluidic system
  \item Discussing the biological basis for the safety of the device
  \item Showing that the deviation from the expected theshold could be explained by variance in the 
\end{itemize}
\end{abstract}


\footnote{footnotes working fine}

This work was prepared by an undergraduate and has not been peer-reviewed. 

Additionally, the author has no prior experience with either biology or microwave design. 

Testing was only performed with a surrogate bacteriophage. All experiments must be repeated with Inf. A and then SARS-NCoV-2.

Please consider all claims with appropriate skepticism.


\begin{multicols}{1}


With typical 2.4 GHz microwave exposure, sterilization occurs by heating of the fluid and tissue.

Of course, the temperature of the atoms of the virus undoubtedly increase; but the 

\begin{toolchain}
	{\it \bf [Yang 2015]'s toolchain}
	\begin{itemize}
	\item Envelope/liposome breaking strength and stiffness from AFM nanoindentation data
	\item Analytical expression assuming homogenous sphere for microwave absorption cross-section
	\item Experimental data from microwave cuvette to 
	\item COMSOL finite-element for illustration
	\end{itemize}
\end{toolchain}

\end{multicols}

\clearpage

\paragraph{\textbf{Time dependence}}\

Both Yang and [] used an apparently arbitrary 15-minute exposure in their tests - a very reasonable decision given the focus of their paper. 

The effectiveness against airborne particles, and to minimize the power required in a dwelling phased-array beam, we must first establish the required duration of exposure.

{\color{red} speculative hypothesizing \{ } 

In contrast to chemical inactivation, where the time dependence appears to be dominated by viscous fluid dynamic effects [Hirose 2017], or UV inactivation, where a certain dose of photons must be absorbed, one might expect RF 

As a damped, driven oscillator, the ring-up time of the virus depends on the Q factor. Yang et al. state the Q of Inf. A as between 2 and 10, so at 8 GHz the steady-state amplitude should be reached in well under 1 us. 

[] found a significant mechanical fatigue effect in phage capsids, where a small strain applied repetitively eventually causes a fracture. Such a mechanism could perhaps extend the exposure required to break the capsid or membrane. Other mechanisms could include some sort of lipid denaturation, requiring an absolute amount of energy absorption to break or twist bonds and modify the properties before the envelope fractures.

{\color{red}  \} } 



\clearpage
\begin{multicols}{1}
{\Large Biology}\\

\paragraph{\textbf{Centrifugal microfluidics}}\

The field of centrifugal microfluidics is accelerating. 

Many CD microfluidics systems use standard CD molding techniques for the channels and machining techniques, using either acrylic or silicone. The turbidity sensor is most sensitive if the plastic is clear. Sterilization does not seem to be discussed. 

Polypropylene is the ideal material, being almost indefinitely autoclavable. It is quite difficult to machine.

\end{multicols}


\clearpage

\begin{multicols}{1}

\noindent\fbox{\parbox{\linewidth}{
	Toolchain:
	\begin{itemize}
	\item QUCS 0.0.20 for AC simulation, with python-qucs and scipy's 'basinhopper' for optimization
	\item gprMax for FDTD EM simulation
	\item KiCAD, wcalc, scikit-rf, ngspice
	\end{itemize}
}}
%
The following properties: P1dB

For ease of design and simulation, a device with Touchstone S-parameters and SPICE files is greatly preferable.
%
\paragraph{\textbf{The feedback loop oscillator}}\

Oscillators must meet the Barkhausen criterion:

\begin{itemize}

\item A 360 degree phase shift around the feedback loop (including the phase shift contribution from the amplifier, which itself varies greatly with frequency)
\item A loop gain $>1.$ 

\end{itemize}\
%
However, a third element is also required:
%
\begin{itemize}
\item A frequency-selective element that restricts oscillation modes and decreases phase noise.
\end{itemize}
%
Without this element, proximity effects from nearby flesh, startup instablity, mode-hopping, and de-tuning were all encountered. 



 design of a triple-tuned oscillator.


\paragraph{}
The buffer amplifiers 

\paragraph{Biasing}\

In our simulations, the varactor-tuned feedback circuit appeared to be particularly sensitive to the introduction of bias-tees. 


The gate must be weakly pulled to ground, otherwise stray charge destroys the oscillation.

\fancyhead[C]{style 1 with thin line}



With 0.79 mm FR4 substrate and 0.2 mm (8 mil) wide traces, the maximum impedance achievable was about 115 ohms, which did not appear to be sufficient as an RF choke.

If suitably high-impedance traces are not available, a common technique is to use a quarter-wavelength line (approximately 6 mm long with the above parameters) terminated with a stub to produce a virtual short or open circuit [Seo 2007]. 

However, reflections from these structures still appeared to distort the frequency/phase response beyond repair, even with ostensibly wideband stubs [Syrett 1980].

\noindent\fbox{\parbox{\linewidth}{
	Rebuke: despite this blather, many other papers have had success with bias-tees at these frequencies.
}}

Alternate methods evaluated, failures: 


In production, these could be accomodated by graphite-polymer printed resistors. 


Odd-pole varactor-loaded combline filters appeared to have excellent phase and frequency response; however, the geometry necessitates low-inductance via stitching to the ground plane.

%\noindent\fbox{\parbox{\linewidth}{
\begin{autem}
	{\it autem} Others have had great success with varactor-tuned comblines, especially non-grounded lines.
\end{autem}

[Tsuru 2008 fig. 10] is an excellent review of various oscillator designs.

The parasitic inductance of common varactors appears to become problematic at these frequencies (but not for non-wideband use).


\clearpage

{\Large Mass production}


\clearpage

\paragraph{Acknowledgements}\

This paper is not novel enough to deserve this chapter, but those to be acknowledged are.

The authors of the original paper on 

Professor Shahramian's excellent video blog {\it The Signal Path} inspired this work.

Hopfstader be damned; despite the name below the abstract, this paper is of course entirely the result of circumstance and luck. Any comparable dimwit to the author would have accomplished the same under the same conditions. 

It is infinitely to be regretted that such equally capable people were stuck in warehouses doing menial 15-hour shifts or assembling smartphones.

Chiefly among the enablers are of course my parents, Doris and Brian, without which this work would have been done by someone else[] - but who also supported this work financially. 



One Very Important Thought


\clearpage

{\Large Captain's Log, Supplemental}\\

CEL did not supply a SPICE model for the GaAs FET device used in early prototypes. A FET was chosen because the a gate is ostensibly easier to bias than a base.

[Steenput 1999] has an interesting analytic method to synthesize a SPICE model suitable for a transient simulations from S-parameter measurements using negative resistances. However, this neglects the I/V characteristic. 

[Polyfet 1998] describes a simple optimization method to synthesize a SPICE model; it would be useful.

However, since a Si-process device would be required for mass production anyhow, we decided to accept the difficulty in biasing and use a bipolar transistor.

\rule{\linewidth}{0.2pt}

OpenEMS is excellent, with Python bindings, some lumped components, and mesh refinement. However, embarassingly, we were not able to resolve all the dependency issues in order to install it.

\rule{\linewidth}{0.2pt}

Simulating the oscillator with an AC sim using the manufacturer's S-parameters and QUCS' microstrip approximations [qucs/optimize\_filter\_1] appeared to yield good agreement with experiment. The peaks in the feedback voltage simulation corresponded approximately with peaks in the observed spectrum. [figures from LO prototype N]. 
 
However, unexpected dips and peaks were found with varactor tuning; and the tuning range was far smaller than expected.

A filter design with apparently ideal phase and frequency properties was designed using this process, but again poor agreement was found with experiment.

\hspace*{-0.7cm}   \includegraphics[scale=0.6]{LO_2_pole_test.png}

Because we lacked a spectrum analyzer that could monitor the higher poles of the filter until the bootstrap LO was designed.

It was thought that a transient simulation - to determine how the spectrum actually evolved - would improve the situation. 

QUCS' microstrip models are not yet compatible with transient simulations; and some improved filter designs required simulating coupling between more than two microstrips, which QUCS did not yet support natively. Coupling can be emulated by introducing discrete coupling capacitors; but 


\rule{\linewidth}{0.2pt}

Again, inductive choke biasing in the feedback loop was practically impossible. Biasing PiN diodes with a 10kohm resistor, (with 330 ohm safety resistor) one to 48V bias and another through an N-channel mosfet worked fine. The oscillator ran fine with a PiN bias of 2.34 mA. [LO prototype N]. Such a high bias voltage was required to get sufficient current through the PiN while remaining delicate with the vfb.

\rule{\linewidth}{0.2pt}

Conductors are represented by zeroing all components of the electric field in those regions. 

There are many different possible source geometries, each introducing their own distortions.

It's possible to link SPICE and FDTD in two main ways

\rule{\linewidth}{0.2pt}

Bandpass filters can be designed by first designing a low-pass filter prototype (usually Chebychev) (or, in our case, using reference filter component tables), and then transforming this low-pass into a band-pass. [Hunter 2001] is an excellent overview of this process, with many design examples for different filter topologies. 

The coupling coefficient between two low-pass filters determines the band-pass bandwidth.[Hui 2012]

[Hunter 2001] also describes an analytical method to create a filter with the precise group delay - phase shift versus frequency - required for stable oscillation. However, simultaneously compensating for the group delay introduced by the amplifier itself (nearly 180 degrees over the frequency range for the CEL part) seemed complex.

 


\label{para}
\ref{para}

\paragraph{Timeline}

\paragraph{Comments by others}

\paragraph{Lessons Learned} \


\rule{\linewidth}{0.2pt}

It may be helpful to treat simulations very similarly to IRL experiments. 

This is obvious to all compentent, but when rapidly iteratively testing with simulations, it may be helpful to automatically save a package with images of all the components (schematics, graphs) of each distinct test. Including a picture with a webcam of the assembled board is also helpful. 

Version control alone isn't quite enough. Just having a simulation setup file somewhere in the commit history isn't "discoverable" - that is, you should be able to see what the input and output was without re-running the simulation. 

Manually taking notes tended to disrupt the flow of testing; and in any case, just noting "SIR filter has appropriate phase response" is almost useless. What {\it was} the phase response? Plot it!

\rule{\linewidth}{0.2pt}

A great deal of time was spent trying to resolve version conflicts and dependency hells with the numerous libraries used by all the simulation programs. This also wastes developer time - some fraction of issues raised are due to library version conflicts. OpenFOAM's Docker installation is excellent.

Running chroot with the original developer's Linux distribution is also of some utility, but clunky. 

Packaged binaries help this slightly, but of course don't help if modifications are required, and managing shared libraries is still a tricky matter.

In some situations (especially where the library has a permissive licence) perhaps it could be useful to consider packaging a complete, batteries-included 'known good' repository, either with the source of all the correct library versions included, or with a script to clone and compile the specific version used, and integrating all the libraries with the build system. 

For instance, this was done with the PDB reader. 

\rule{\linewidth}{0.2pt}


Write a paper to be understood; to be as clear and helpful as possible to the reader.

\paragraph{Hall of Hubris} \

Lest our hats stop fitting

\rule{\linewidth}{0.2pt}

An inane remark:

\begin{displayquote}
We believe once we have the P. Syringae host, we from environmental samples
\end{displayquote}

Truly the depths of Dunning-Kruger.

\rule{\linewidth}{0.2pt}

An expert and distinguished gentleman that we contacted regarding assistance resolving transients in our microstrip VCO stated the following:



This was a perfectly sensible remark; it is almost always the case that (at X-band, no less!) a custom IC would have been needed to build a VCO.

Also, if taken in the context of a tired, overworked PI getting an unsolicited email from an excessively verbose undergraduate at a different university, I hardly think I would have replied differently.

However, I think this person may have missed out. Learning 

So perhaps it is wise to ponder the ideas of fools for skeet; one always learns target practice, if only an example what not to do, and occasionally one learns positively. I have often found that I learn greatly from working on the projects of others.

We present this only as a cautionary tale in the hopes that someday I will listen, rather than talk.

\rule{\linewidth}{0.2pt}

\end{multicols}

\Acrobatmenu{GoBack}{Back}
\end{document}
