\documentclass[fleqn,10pt]{article}

\usepackage[left=2cm,right=2cm,
			top=1.25cm,
			bottom=2.25cm,%
			headheight=11pt,%
			letterpaper]{geometry}
			
\frenchspacing			


\usepackage{multicol}
\usepackage{fancyhdr}
\usepackage{blindtext,graphicx}
\usepackage[absolute]{textpos}
\usepackage[parfill]{parskip}
\usepackage[colorlinks]{hyperref}
\usepackage{hyperref}
\usepackage{gensymb}
\usepackage{csquotes}
\usepackage{amsmath}
\usepackage{fontawesome}

\usepackage{graphicx}
\graphicspath{ {../media/} }

\usepackage{tcolorbox}
\newtcolorbox{autem}{colback=red!5!white,colframe=red!75!black}
\newtcolorbox{toolchain}{colback=blue!5!white,colframe=blue!40!black!40}
%https://tex.stackexchange.com/questions/66154/how-to-construct-a-coloured-box-with-rounded-corners

%\usepackage[sfdefault,light]{roboto}

\setlength{\TPHorizModule}{1cm}
\setlength{\TPVertModule}{1cm}






\title{On the application of microwave acoustic resonant viral inactivation
\thanks{I would be delighted to include any criticisms or comments anyone may have; preferably leave them on the GitHub issues page, or email therobotist@gmail.com, @0xDBFB7 on Twitter, or irc.0xDBFB7.com:6667 \#covid.}}
\date{May 2020}
\author{Based primarily on exceptional work by }


\begin{document}

\flushbottom 
\maketitle
\thispagestyle{empty}




\begin{textblock}{5}(1,1)
\noindent Please view the latest version at github.com/0xDBFB7/covidinator
\end{textblock}
%\begin{textblock}{5}(1,27)
%\end{textblock}

\null\begin{tabular}[t]{l@{}}
  {Daniel Correia} \\
  \textit{York University}
\end{tabular}



\begin{abstract}

We extend this landmark work rather trivially by:


Aims:

\begin{itemize}
  \item Establishing the time dependence of inactivation
  \item Demonstrating a modulation scheme that decreases the inactivation threshold to below current safety levels in surrogate bacteriophage
  \item Demonstrating a prototype emitter in an "electromagnetic mask" form-factor, costing about \$5 in prototype quantities, which can reasonably be produced in 10 million-of quantities
  \item Testing power thresholds in various conditions; biological fluids of various conductivities and pHs
  \item Using a coarse-grained molecular-dynamics simulation to optimize the impulse
  \item Using a virus-in-the-loop optimization with a centrifugal microfluidic system
  \item Discussing the biological basis for the safety of the device
  \item Showing that the deviation from the expected theshold could be explained by variance in the 
\end{itemize}
\end{abstract}



This work was prepared by an undergraduate and has not been peer-reviewed. 

Additionally, the author has no prior experience with either biology or microwave design. 

Though the original research used Inf. A, our testing was only performed with a surrogate bacteriophage. All experiments must be repeated with SARS-NCoV-2.

Additionally, other mechanisms may produce superior results and should be evaluated in the same context. Recent data on far UVC [Buonanno 2017] indicate safety.

Please consider all claims with appropriate skepticism.


\begin{multicols}{1}


With typical 2.4 GHz microwave exposure, sterilization occurs by heating of the fluid and tissue.

Of course, the temperature of the atoms of the virus undoubtedly increase; but the 

\begin{toolchain}
	{\it \bf [Yang 2015]'s toolchain}
	\begin{itemize}
	\item Envelope/liposome breaking strength and stiffness from AFM nanoindentation data
	\item Analytical expression assuming homogenous sphere for microwave absorption cross-section
	\item Experimental absorption data from microwave cuvette -> 
	\item COMSOL finite-element for illustration
	\end{itemize}
\end{toolchain}

\end{multicols}






%%%%%%%%%%%%%%%%%%%%%%%%%%%%%%%%%%%%%%%%%%%%%%%%
\clearpage
%%%%%%%%%%%%%%%%%%%%%%%%%%%%%%%%%%%%%%%%%%%%%%%%
\paragraph{\textbf{Time dependence}}\


Both Yang and [] used an apparently arbitrary 15-minute exposure in their tests - a very reasonable decision given the focus of their paper. 

The effectiveness against airborne particles, and to minimize the power required in a dwelling phased-array beam, we must first establish the required duration of exposure.

{\color{red} speculative hypothesizing \{ } 

In contrast to chemical inactivation, where the time dependence appears to be dominated by viscous fluid dynamic effects [Hirose 2017], or UV inactivation, where a certain quantized dose of photons must be absorbed, we expected RF to act instantaneously.

As a damped, driven oscillator, the ring-up time of the virus depends on the Q factor. Yang et al. state the Q of Inf. A as between 2 and 10, so at 8 GHz the steady-state amplitude should be reached in well under 1 us. 

[] found a significant mechanical fatigue effect in phage capsids, where a small strain applied repetitively eventually causes a fracture. Such a mechanism could perhaps extend the exposure required to break the capsid or membrane. Other mechanisms could include some sort of lipid denaturation, requiring an absolute amount of energy absorption to break or twist bonds and modify the properties before the envelope fractures.


{\color{red}  \} } 


The medium (not specified in the paper, but ATCC recommends that MDCK cells are grown in Eagle's MEM + 10\% fetal bovine serum).




[IEEE C95.1-2019]

[IEEE C95.1-2005]

\footnote{All values have been converted to $W/m^2$ to avoid confusion.}



\clearpage
%%%%%%%%%%%%%%%%%%%%%%%%%%%%%%%%%%%%%%%%%%%%%%%%
\begin{multicols}{1}
{\Large Safety}\\
%%%%%%%%%%%%%%%%%%%%%%%%%%%%%%%%%%%%%%%%%%%%%%%%



Because this is an ostensibly novel and niche mechanism, it may behoove us to breifly review the biological basis for the safety limits set by standards organizations. \footnote{The idea of biological microwave resonances appears to have originated in [Frohlich 1980].}

Contrary to some [Hardell 2017] reports, we do not find the FCC and ICNIRP to be significantly affected by special interest groups; their rationales appear to be transparent. 
\footnote{[FCC-19-126A1] an FCC memorandum wherein the FCC tells groups to respectfully screw themselves no fewer than three times: ''Similarly, IEEE-ICES urges the Commission to
adopt a higher SAR exposure limit of 2 W/kg averaged over 10 g. [snip] We are not persuaded that the IEC standard should be adopted at this time.", "Medtronic and the AAMI-CRMD recommend a more relaxed threshold of 20 mW. We decline to increase the 1-mW threshold.". Things heating up at the Microwave Safety fandom.}


The IEEE standard which the flagship paper [Yang 2015] references [IEEE C95.1-2005] was recently overhauled [IEEE C95.1-2019] with new metrics.

For instance, we wondered why we might not see such resonances in bodily tissues. Approximately similar nanoscopic structures exist in humans; for instance, nuclear pores (120 nm) and the famous microtubules (10x50 nm). 

While from which epidemiological data could be derived.\footnote{The cloud service [scite.ai] was irreplaceable during this. It's like a mirror to an alternate universe where literature made sense.}

What we find is a messy affair. Biology is hard.

\rule{\linewidth}{0.2pt}

There are some 20,000 papers on the topic of RF safety, with solid in vitro, in vivo and epidemiological evidence of safety in the 2.4 GHz and 5.8 GHz bands.

However, "There are limited experimental human data upon which to set limits on exposures above 6 GHz" [Chan 2019]. Only 2\% of the above papers are on frequencies $>6$ GHz [Vijayalaxmi 2018]. 

In addition, what data remains in this category is of poor quality. As has been starkly demonstrated in the recent hydroxychloroquine contradictions [Gautret 2020] [Geleris 2020], biological research must be essentially perfect to have any meaning at all.

[Vijayalaxmi 2018] is a remarkable meta-analysis of in vitro data. They synthesize a quality score, rating bindedness, sham controls, dosimetry, and sample size. Quality was inversely, monotonically related to the effect size. An example of the subtle effects that can call into question superfically reasonable results are [

They additionally find a publication bias of incredible magnitude in the positive (harm) direction in this field, further degrading the usefulness of statistical analyeses.

An example of a relevant study that they grade as "quality 1" [Karaca 2011]. This study's conclusion is that However, in figure 5, one can see that only 1 out of 11 genes tested had a statistically significant difference in expression. Given the N=6 cultures, 

\rule{\linewidth}{0.2pt}

[Adair 2002] demonstrate theoretically that acoustic resonances are not possible in biological solvents. All resonance modes are strongly overdamped by the surrounding solvent, and no amplitude amplification can take place.

This analysis would seem to disagree with the viral resonance apparently conclusively demonstrated by [Yang 2015]; the PCR result seems especially inarguable. We cannot explain this discrepancy. 

[Liu 2009] cite [Edwards, Swicord 1984], saying: "It was proposed that a hydration layer surrounding DNA molecules could lower the viscoelastic transition
frequency, raise the quality factor of confined acoustic vibrations, and result in a microwave resonant absorption". But [Adair 2002]'s [Foster 1987] seem to quite conclusively put the kibosh on that idea, demonstrating no resonances in DNA with 20x higher precision and showing previous results to be an artifact of the measurement equipment.

The supplemental material to [Vijayalaxmi 2018] has a table of results.

\paragraph{[Manikowska 1979]} \

9.4 GHz pulsed / in vivo, mouse, $0.1-10 \text{mW}/\text{m}^2$ over the whole body for 2 weeks. N=16.

This is a particularly concerning result, especially given their $p<0.001$. [Servantie 1989] discusses the (admittedly circuituious) route by which birth defects could occur.

Work by [Manikowska 1985] in the 2.4 GHz range has been contradicted [Beechey 1986], but the 9 GHz range does not appear to have been repeated.

\rule{\linewidth}{0.2pt}

More modern research [Juutilainen 2011], [Chemeris], [Vijayalaxmi 2006],  does not tend to agree with these concerns. As cited by [ICNIRP 2020]'s excellent literature review, expose lymphocyte blood cells to 8 ns pulsed 8.2 GHz radiation for 2 hours at an average power of $10 \text{mW/cm}^2$. They find no change in any of the parameters measured, including various chromosomal parameters. They further discuss previous results.

\rule{\linewidth}{0.2pt}

In the frequency range of interest, there seems to be good-quality in-vitro evidence using lymphocytes of lack of harm. We are not aware of any high-quality in vivo or epidemeological data from which useful conclusions can be drawn. 

We must defer to health experts for interpretation of these data.

\rule{\linewidth}{0.2pt}

{\it Please contact the author if you are aware of other salient research, or have a different interpretation.}

\footnote{``{\it{For example, below about 6 GHz, where EMFs penetrate deep into tissue (and thus require depth to be considered), it is useful to describe this in terms of “specific energy absorption rate” (SAR), which is the power absorbed per unit mass $(W/kg)$. Conversely, above 6 GHz, where EMFs are absorbed more superficially (making depth less relevant), it is useful to describe exposure in terms of the density of absorbed power over area $W/m^2$, which we refer to as “absorbed power density”}}'' [ICNIRP 2020 \faExternalLink]}

\end{multicols}















\clearpage
%%%%%%%%%%%%%%%%%%%%%%%%%%%%%%%%%%%%%%%%%%%%%%%%
\begin{multicols}{1}
{\Large Biology}\\
%%%%%%%%%%%%%%%%%%%%%%%%%%%%%%%%%%%%%%%%%%%%%%%%

\paragraph{\textbf{Centrifugal microfluidics}}\

The field of centrifugal microfluidics is accelerating. 

Many CD microfluidics systems use standard CD molding techniques for the channels and machining techniques, using either acrylic or silicone. The turbidity sensor is most sensitive if the plastic is clear. Sterilization does not seem to be discussed. 

Polypropylene is the ideal material, being almost indefinitely autoclavable. It is quite difficult to machine.

\end{multicols}








\clearpage
%%%%%%%%%%%%%%%%%%%%%%%%%%%%%%%%%%%%%%%%%%%%%%%%
\begin{multicols}{1}
{\Large Modes of application}\\
%%%%%%%%%%%%%%%%%%%%%%%%%%%%%%%%%%%%%%%%%%%%%%%%

\paragraph{\textbf{Personal 'electromagnetic mask'}}\

We demonstrate this form factor because, superficially, there are fewer people than there are places. Even with judicious use of phased-arrays, power-combining, etc, each transistor can only reasonably sterilize approx. $ 1 \text{ m}^3 $.

On the other hand, a personal device may present issues with compliance\footnote{that sounds evil, but it really isn't} and production volume.

\paragraph{\textbf{Direct treatments}}\

[Hand 1982] $E_{mag}=1/e$ (13.5\% power density) skin\footnote{electromagnetic skin, not tissue skin}\footnote{well, both, I suppose.} depth is approximately:

\begin{center}
\begin{tabular}{|l|l|l|}
\hline
F=10 GHz          & Dry tissue & Wet tissue \\ \hline
Penetration depth & 30 mm      & 5 mm \\ \hline
\end{tabular}
\end{center}

SARS is found widely distributed throughout the most desirable organs [Ding 2004], shielded by an average of 4 cm of chest wall [Schroeder 2013]; so safe external treatment of the body is unlikely. 

However, destruction of lung tissue appears to be the primary cause of death via SARS [Nicholls 2006]. 

A minimally-invasive bronchoscopic technique may be effective when mounted on a modified brochoscope and periodically inserted into the main bronchus, a la [Yuan 2019].

The bronchi are less than 2 mm thick [Theriault 2018] and the lungs themselves are only on the order of 
7 mm thick [Chekan 2016].

Further study is required to validate this method.

\paragraph{\textbf{More fanciful concepts}}\

Many weather radar systems operate in these X-band frequency ranges. 

It is relatively easy to produce megawatts of power at these frequency ranges using Klystrons.



\end{multicols}







\clearpage
%%%%%%%%%%%%%%%%%%%%%%%%%%%%%%%%%%%%%%%%%%%%%%%%
\begin{multicols}{1}
%%%%%%%%%%%%%%%%%%%%%%%%%%%%%%%%%%%%%%%%%%%%%%%%

\noindent\fbox{\parbox{\linewidth}{
	Toolchain:
	\begin{itemize}
	\item QUCS 0.0.20 for AC simulation, with python-qucs and scipy's 'basinhopper' for optimization
	\item gprMax for FDTD EM simulation
	\item KiCAD, wcalc, scikit-rf, ngspice
	\end{itemize}
}}
%
The following properties: P1dB

For ease of design and simulation, a device with Touchstone S-parameters and SPICE files is greatly preferable.
%
\paragraph{\textbf{The feedback loop oscillator}}\

Oscillators must meet the Barkhausen criterion:

\begin{itemize}

\item A 360 degree phase shift around the feedback loop (including the phase shift contribution from the amplifier, which itself varies greatly with frequency)
\item A loop gain $>1.$ 

\end{itemize}\
%
However, a third element is also required:
%
\begin{itemize}
\item A frequency-selective element that restricts oscillation modes and decreases phase noise.
\end{itemize}
%
Without this element, proximity effects from nearby flesh, startup instablity, mode-hopping, and de-tuning were all encountered. 



 design of a triple-tuned oscillator.


\paragraph{}
The buffer amplifiers 

\paragraph{Biasing}\

In our simulations, the varactor-tuned feedback circuit appeared to be particularly sensitive to the introduction of bias-tees. 


The gate must be weakly pulled to ground, otherwise stray charge destroys the oscillation.

\fancyhead[C]{style 1 with thin line}



With 0.79 mm FR4 substrate and 0.2 mm (8 mil) wide traces, the maximum impedance achievable was about 115 ohms, which did not appear to be sufficient as an RF choke.

If suitably high-impedance traces are not available, a common technique is to use a quarter-wavelength line (approximately 6 mm long with the above parameters at 8 GHz) terminated with a stub to produce a virtual short or open circuit [Seo 2007]. 

However, reflections from these structures still appeared to distort the frequency/phase response beyond repair, even with ostensibly wideband stubs [Syrett 1980].

\noindent\fbox{\parbox{\linewidth}{
	Rebuke: despite this blather, many other papers have had success with bias-tees at these frequencies.
}}

Alternate methods evaluated, failures: 


In production, these could be accomodated by graphite-polymer printed resistors. 


Odd-pole varactor-loaded combline filters appeared to have excellent phase and frequency response; however, the geometry necessitates low-inductance via stitching to the ground plane.

%\noindent\fbox{\parbox{\linewidth}{
\begin{autem}
	{\it autem} Others have had great success with varactor-tuned comblines, especially non-grounded lines.
\end{autem}

[Tsuru 2008 fig. 10] is an excellent review of various oscillator designs.

The parasitic inductance of common varactors appears to become problematic at these frequencies (but not for non-wideband use).






\clearpage
%%%%%%%%%%%%%%%%%%%%%%%%%%%%%%%%%%%%%%%%%%%%%%%%
{\Large Mass production}
%%%%%%%%%%%%%%%%%%%%%%%%%%%%%%%%%%%%%%%%%%%%%%%%


If this 'electromagnetic mask' form is the ideal (not nearly ), a minimum of 5 GaAs or SiGe:C transistors will be required.

There are, for instance, 1 million hospital beds in the U.S. [AHA 2018]. 

It is difficult to determine the supply capacity for these modern semiconductor processes. A few fermi est

GaAs MMIC market of \$2.2 Bn USD / random sample of MMIC prices = 

3e9 wifi connected devices produced each year.

The largest RFID plants can produce


The techniques and equipment required to produce these are non-trivial. SiGe:C, for instance, requires, 








\clearpage
%%%%%%%%%%%%%%%%%%%%%%%%%%%%%%%%%%%%%%%%%%%%%%%%
{\Large Molecular dynamics simulation}\\
\begin{multicols}{1}
%%%%%%%%%%%%%%%%%%%%%%%%%%%%%%%%%%%%%%%%%%%%%%%%


aa





\end{multicols}




\clearpage
%%%%%%%%%%%%%%%%%%%%%%%%%%%%%%%%%%%%%%%%%%%%%%%%
{\Large Optical centrifuge polarization chirp}\\
\begin{multicols}{1}
%%%%%%%%%%%%%%%%%%%%%%%%%%%%%%%%%%%%%%%%%%%%%%%%


aa





\end{multicols}







\clearpage
%%%%%%%%%%%%%%%%%%%%%%%%%%%%%%%%%%%%%%%%%%%%%%%%
\paragraph{Acknowledgements}\
%%%%%%%%%%%%%%%%%%%%%%%%%%%%%%%%%%%%%%%%%%%%%%%%
This paper is not novel enough to deserve this chapter, but those to be acknowledged are.

The authors of the original paper on 

Professor Shahramian's excellent video blog {\it The Signal Path} inspired this work. Alex Dainis' videos on sterile technique were also useful.

Hopfstader be damned; despite the name below the abstract, this paper is of course entirely the result of circumstance and luck. Any comparable dimwit to the author would have accomplished the same under the same conditions. 

It is infinitely to be regretted that such equally capable people were stuck in warehouses doing menial 15-hour shifts or assembling smartphones.

Chiefly among the enablers are of course my parents, Doris and Brian, without which this work would have been done by someone else[] - but who also supported this work financially. 



One Very Important Thought


\clearpage

{\Large Captain's Log, Supplemental}\\


\rule{\linewidth}{0.2pt}

There seems to be a discrepancy between [Yang 2015] and our reading of C95.1-2005. This is very likely a misreading on our part, and so calls into question our understanding of the standard.

They say: "Based on the IEEE Microwave Safety Standard, the spatial averaged value of the power density in air in open public space shall not exceed the equivalent power density of $100(f/3)^{1/5} \text{ W/m}^2$ at frequencies between 3 and 96GHz ($f$ is in GHz). This corresponds to $115 \text{ W}/\text{m}^2$ at 6 GHz". The 115 W @ 6 GHz value correctly corresponds to this equation with a coefficient of 100.

The only reference to a 3 - 96 GHz limit we can find is that of Section 4.6, {\it Relaxation of the power density MPEs for localized exposures}, where the equation is $200 (f/3)^{1/5} \text{ W/m}^2$ [IEEE C95.1-2005]; but this is for controlled, occupational exposures only, referring to Table 8. For Table 9, general public, the equation is $18.56 (f)^{0.699} \text{ W/m}^2$, or $64.93 \text{ W}/\text{m}^2$ @ 6 GHz. 

We don't think this changes the substance of [Yang 2015]'s paper, if it is an error; the relaxed localized exposure is probably the correct threshold to use, and it just means the mechanism is more useful for healthcare workers who acknowledge the risks.

Different versions of the IEEE standard have used equations of equivalent form but with different coefficients [Wu 2015]; it is possible that we have retrieved the wrong standard.

\rule{\linewidth}{0.2pt}



CEL did not supply a SPICE model for the GaAs FET device used in early prototypes. A FET was chosen because the a gate is ostensibly easier to bias than a base.

[Steenput 1999] has an interesting analytic method to synthesize a SPICE model suitable for a transient simulations from S-parameter measurements using negative resistances. However, this neglects the I/V characteristic. 

[Polyfet 1998] describes a simple optimization method to synthesize a SPICE model; it would be useful.

However, since a Si-process device would be required for mass production anyhow, we decided to accept the difficulty in biasing and use a bipolar transistor.

\rule{\linewidth}{0.2pt}

OpenEMS is excellent, with Python bindings, some lumped components, and mesh refinement. However, embarassingly, we were not able to resolve all the dependency issues in order to install it.

\rule{\linewidth}{0.2pt}

Simulating the oscillator with an AC sim using the manufacturer's S-parameters and QUCS' microstrip approximations [qucs/optimize\_filter\_1] intially appeared to yield good agreement with experiment. Peaks in the feedback voltage simulation corresponded approximately with peaks in the observed spectrum. [figures from LO prototype N]. 
 
However, unexpected dips and peaks were found with varactor tuning; and the tuning range was far smaller than expected.

A feedback loop filter design with apparently ideal phase and frequency properties was designed using this process; but the behavior of this oscillator did not correspond to reality.

\hspace*{-0.7cm}   \includegraphics[scale=0.6]{LO_2_pole_test.png}

Because we lacked a spectrum analyzer that could monitor the higher poles of the filter until the bootstrap LO was designed, we had to rely on in-silico analyeses.

It was thought that a transient simulation - to determine how the spectrum actually evolved - would improve the situation. 

As of 0.0.20, QUCS' microstrip models are not yet compatible with transient simulations; and some improved filter designs required simulating coupling between more than two microstrips, which QUCS did not yet support natively. Coupling can be emulated by introducing discrete coupling capacitors

Transient simulations with ngspice matched experiment far more closely.

\rule{\linewidth}{0.2pt}

Again, inductive choke biasing in the feedback loop was practically impossible. Biasing PiN diodes with a 10kohm resistor, (with 330 ohm safety resistor) one to 48V bias and another through an N-channel mosfet worked fine. The oscillator ran fine with a PiN bias of 2.34 mA. [LO prototype N]. Such a high bias voltage was required to get sufficient current through the PiN while remaining delicate with the vfb.

\rule{\linewidth}{0.2pt}

Conductors are represented by zeroing all components of the electric field in those regions. 

There are many different possible source geometries, each introducing their own distortions.

It's possible to link SPICE and FDTD in two main ways

\rule{\linewidth}{0.2pt}

Bandpass filters can be designed by first designing a low-pass filter prototype (usually Chebychev) (or, in our case, using reference filter component tables), and then transforming this low-pass into a band-pass. [Hunter 2001] is an excellent overview of this process, with many design examples for different filter topologies. 

The coupling coefficient between two low-pass filters determines the band-pass bandwidth.[Hui 2012]

[Hunter 2001] also describes an analytical method to create a filter with the precise group delay - phase shift versus frequency - required for stable oscillation. However, simultaneously compensating for the group delay introduced by the amplifier itself (nearly 180 degrees over the frequency range for the CEL part) seemed complex.

Phase shift can be introduced either via a length of microstrip, or a high-pass/low-pass filter [Microwave101]. Adding a fixed microstrip line restricts the tuning range, however, and the filter inevitably affects the frequency response.  

\rule{\linewidth}{0.2pt}

NGSPICE's KSPICE coupled transmission lines require the capacitance and inductance per unit length in Maxwell matrix form, rather than the physical $C_{even}$/$L_{even}$ (each line's capacitance and inductance to ground) and $_{odd}$ (between elements) form provided by tools like wcalc. "matrix not positive definite". [Schutt-Aine] discusses this; we reproduce here for convienience.

\[ L_{11} = L_{22} = L_{even}  \]
\[ L_{12} = L_{21} = L_{odd}  \]
\[ C_{11} = C_{22} = C_{even}+C_{odd}  \]
\[ C_{12} = C_{21} {\it{(unused)}} = -C_{odd}  \]

\rule{\linewidth}{0.2pt}



\label{para}
\ref{para}

\paragraph{Timeline}

\paragraph{Comments by others}

\paragraph{Lessons Learned} \


\rule{\linewidth}{0.2pt}

It may be helpful to think of simulations very similarly to IRL experiments. 

A simulation is just a universe in a bottle that you can examine more closely; as with real-world, you can learn much from observation, but 

But this need for documentation conflicts with the rapid, iterative cycle necessary for productivity.

This is obvious to all compentent, but when rapidly iteratively testing with simulations, it may be helpful to automatically save a package with images of all the components (schematics, graphs, input files) of each distinct test. 

For comparison to experiment, having a webcam take an image of the assembled board is also helpful. 

Version control alone isn't quite enough. Just having a simulation setup file somewhere in the commit history isn't "discoverable" - that is, you must be able to see what the input and output was without re-running the simulation. 

Manually taking notes tended to disrupt the flow of testing; and in any case, just noting "SIR filter has appropriate phase response" is almost useless. What {\it was} the phase response? Plot it!

Software such as Sumatra, Sacred, recipy, and others. In our case, we used eLabFTW's elabapy bindings.

\rule{\linewidth}{0.2pt}

A great deal of time was spent trying to resolve version conflicts and dependency hells with the numerous libraries used by all the simulation programs. This also wastes developer time - some fraction of issues raised are due to library version conflicts. OpenFOAM's Docker installation is excellent.

Running chroot with the original developer's Linux distribution is also of some utility, but clunky. 

Packaged binaries help this slightly, but of course don't help if modifications are required, and managing shared libraries is still a tricky matter.

In some situations (especially where the library has a permissive licence) perhaps it could be useful to consider packaging a complete, batteries-included 'known good' repository, either with the source of all the correct library versions included, or with a script to clone and compile the specific version used, and integrating all the libraries with the build system. 

For instance, this was done with the PDB reader. 

\rule{\linewidth}{0.2pt}


Write a paper to be understood; to be as clear and helpful as possible to the reader.

\rule{\linewidth}{0.2pt}

We have always encountered wasting extreme amounts of time on subtle assembly mistakes in hardware prototypes. 

In one example, many hours were spent because the enamel insulation on a bodge wire had not burned off completely in the solder joint, leading to a high-impedance connection.

The same has occurred in previous projects; in one example, many days were spent debugging software to fix an apparently slow hardware interrupt, which ended up being the result of a poor solder paste stencil leading to a hidden high-impedance connection to a leadless package.

Many failures are perhaps the result of carelessness in modification and a lack of inspection; but others would only have been found by a 100\% electrical test.

If flying-probe or bed-of-nails tests can be made sufficiently rapid and closely coupled with the existing toolchain, 

In {\it 2001: A space odyssey}, an automated system is shown guiding the troubleshooting of an assembly, apparently generating a fault tree of all the 

\rule{\linewidth}{0.2pt}



\paragraph{Hall of Hubris} \

Lest our hats stop fitting

\rule{\linewidth}{0.2pt}

An inane remark:

\begin{displayquote}
We believe once we have the P. Syringae host, we from environmental samples
\end{displayquote}

Truly the depths of Dunning-Kruger.

\rule{\linewidth}{0.2pt}

An expert and distinguished gentleman that we contacted regarding assistance resolving transients in our microstrip VCO stated the following:



This was a perfectly sensible remark; it is almost always the case that (at X-band, no less!) a custom IC would have been needed to build a VCO.

Also, if taken in the context of a tired, overworked PI getting an unsolicited email from an excessively verbose undergraduate at a different university, I hardly think I would have replied differently.

However, I think this person may have missed out. Learning 

So perhaps it is wise to ponder the ideas of fools for skeet; one always learns target practice, if only an example what not to do, and occasionally one learns positively. I have often found that I learn greatly from working on the projects of others.

We present this only as a cautionary tale in the hopes that someday I will listen.

\rule{\linewidth}{0.2pt}

\end{multicols}

\Acrobatmenu{GoBack}{Back}
\end{document}
