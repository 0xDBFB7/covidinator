
\documentclass[fleqn,10pt]{article}

\usepackage[left=2cm,right=2cm,
			top=2.25cm,
			bottom=2.25cm,%
			headheight=11pt,%
			letterpaper]{geometry}
			
			
\usepackage{multicol}
\usepackage{fancyhdr}
\usepackage{blindtext,graphicx}
\usepackage[absolute]{textpos}
\setlength{\TPHorizModule}{1cm}
\setlength{\TPVertModule}{1cm}


\title{The Triangulation of Titling Data in Non-Linear Gaussian Fashion via $\rho$ Series\thanks{No procrastination}}
\date{2017\\ December}
\author{John Doe\\ Magic Department\thanks{I am no longer a member of this department}, Richard Miles University 
\and Richard Row, \LaTeX\ Academy}


\begin{document}

\flushbottom 
\maketitle
\thispagestyle{empty}

This work was prepared by an undergraduate and has not been peer-reviewed. 
Please consider all claims with the appropriate skepticism.


\begin{textblock}{5}(14,1)
\noindent University of Nowhere
\end{textblock}
\begin{textblock}{5}(1,27)
\end{textblock}

\null\hfill\begin{tabular}[t]{l@{}}
  \textbf{My e} \\
  \textit{My University}
\end{tabular}



\begin{abstract}
We extend this landmark work rather trivially by:
\begin{itemize}
  \item Testing the time dependence of inactivation
  \item Demonstrating a prototype emitter in an "electromagnetic mask" form-factor, costing about \$5 in prototype quantities, which can be produced in 10 million-of quantities using existing PCB infrastructure  
  \item Testing power thresholds in various conditions; biological fluids of various conductivities and pHs
  \item Using a coarse-grained molecular-dynamics simulation to optimize the impulse
  \item Using a virus-in-the-loop optimization with a centrifugal microfluidic system
  \item Discussing the biological basis for the safety of the device
  \item Showing that the deviation from the expected theshold could be explained by variance in the 
\end{itemize}
\end{abstract}











With typical 2.4 GHz microwave exposure, sterilization occurs by heating of the fluid and tissue.

Of course, the temperature of the atoms of the virus undoubtedly increase; but the 



\clearpage





\begin{multicols}{1}


The following properties: P1dB

For ease of design and simulation, a device with Touchstone S-parameters and SPICE files is greatly preferable.


\paragraph{} Feedback-loop-oscillator


\paragraph{}
The buffer amplifiers 

In our simulations, the varactor-tuned feedback circuit appeared to be particularly sensitive to the introduction of bias-tees. 


 gate must be weakly biased to ground, otherwise stray charge destroys the oscillation.

\fancyhead[C]{style 1 with thin line}



With 0.79 mm FR4 substrate and 0.2 mm (8 mil) wide traces, the maximum impedance achievable was about 115 ohms. 

If suitably high-impedance traces are not available, a common technique is to use a quarter-wavelength line (approximately 6 mm at these frequencies) terminated with a stub to produce a virtual short or open circuit [Seo 2007]. 

However, reflections from these structures still appeared to distort signals beyond repair, even with wideband half-moon stubs [Syrett 1980].

\noindent\fbox{\parbox{\linewidth}{
	Rebuke: despite this blather, many other papers have had success with VCOs at these frequencies.

}}

In production, these could be accomodated by graphite-polymer resistors.


Timeline


\end{multicols}

\end{document}
