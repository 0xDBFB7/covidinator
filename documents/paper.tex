\documentclass[fleqn,10pt]{article}




\usepackage[left=2cm,right=2cm,
			top=1.25cm,
			bottom=2.25cm,%
			headheight=11pt,%
			letterpaper]{geometry}
			
\frenchspacing			

\nonstopmode

\usepackage{multicol}
\usepackage{fancyhdr}
\usepackage{blindtext,graphicx}
\usepackage[absolute]{textpos}
\usepackage[parfill]{parskip}
\usepackage[colorlinks=true,citecolor=brown]{hyperref}
\usepackage{gensymb}
\usepackage{csquotes}
\usepackage{amsmath}
\usepackage{fontawesome}
\usepackage{orcidlink}
\usepackage{standalone}
\usepackage{pdfpages}
\usepackage{subfiles}
\usepackage{svg}
\usepackage{sidecap}
\usepackage{float}
\usepackage{amssymb}
\usepackage{textcomp}
\usepackage{lettrine}
%\usepackage[T1]{fontenc}

\usepackage{booktabs,caption}
\usepackage[flushleft]{threeparttable}

%\usepackage{biblatex}
\usepackage[backend=bibtex8, sorting=none]{biblatex}

\addbibresource{processed.bib}
%biblatex has a zoterordfxml
% might avoid the need for python bibtex_collections.py

\usepackage{etoolbox}
\AtBeginEnvironment{quote}{\small}

\usepackage{pifont}
\newcommand{\cmark}{\ding{51}}%
\newcommand{\xmark}{\ding{55}}%


\newcommand{\Wsqm}{$\text{ W/m}^2$}

\newcommand{\ghfile}[1]{\href{https://github.com/0xDBFB7/covidinator/tree/master/#1}{\faGithub/\url{#1} }}

%\newcommand{\supercite}[1]{}
%\newcommand{\supercollect}[1]{}


\newlength{\PRLlen}
\newcommand*\PRLsep[1]{{\itshape \Large\settowidth{\PRLlen}{#1}\advance\PRLlen by -\textwidth\divide\PRLlen by -2\noindent\makebox[\the\PRLlen]{\resizebox{\the\PRLlen}{1pt}{$\blacktriangleleft$}}\raisebox{-.5ex}{#1}\makebox[\the\PRLlen]{\resizebox{\the\PRLlen}{1pt}{$\blacktriangleright$}}\bigskip}}


\usepackage{graphicx}
\graphicspath{ {../media/} }

\usepackage{tcolorbox}
\newtcolorbox{autem}{colback=red!5!white,colframe=red!75!black}
\newtcolorbox{toolchain}{colback=blue!5!white,colframe=blue!40!black!40}
%https://tex.stackexchange.com/questions/66154/how-to-construct-a-coloured-box-with-rounded-corners

%\usepackage[sfdefault,light]{roboto}

\setlength{\TPHorizModule}{1cm}
\setlength{\TPVertModule}{1cm}



%!TeX root = title
\documentclass[paper.tex]{subfiles}
\begin{document}

%\title{Maxwell's Silver Hammer: Musings and measurements on pulsed microwave acoustic resonant viral inactivation 

%Data and discussion?

% Maxwell's Silver Hammer: Pulsed microwave viral envelope disruption\footnotemark\\or, some book-keeping matters of minor consequence in the application of microwave-induced lysis\\or 
% On pulsed microwave viral envelope disruption

% electropermeabilization 
% irreversible electroporation

% A haphazard exploration of the plausiblity of dispersive shaped-pulse electropermeabilization of the viral envelope
% A haphazard traipse through pulsed viral envelope disruption
%  the viral envelope via shaped dispersive pulse
\title{On the sub-nanosecond electropermeabilization of viruses}
%\subtitle{A shadowy flight into the dangerous world of a viral inactivation mechanism which should not exist.}
\date{}

%\footnotetext{Also affiliated with SafeSump Incorporated.}

\flushbottom 
\maketitle
\thispagestyle{empty}

\renewcommand{\abstractname}{Summary}    % clear the title

\begin{abstract}
	
	
	\footnote{Blame \small{{Daniel Correia}\ \orcidlink{0000-0002-9353-0216}}}
	
	\footnote{{I would be delighted to hear any criticisms anyone may have, both on substance and comprehensibility; preferably leave them on the GitHub issues page, or email therobotist@gmail.com, @0xDBFB7 on Twitter, or irc.0xDBFB7.com:6667 \#covid.}}
	
	In particular, a long relaxation time 
	
	Work by the Sun group (Liu 2009, Yang 2015, Sun 2017) and (Dykeman 2009) and 
	(Burkhartsmeyer 2020) suggests that certain species of viruses may have remarkable dielectric 
	properties which are not found in other biological systems. More importantly, they appear to 
	demonstrate that, due to this effect, high power densities in the X-band can tear the envelope, 
	inactivating the virus. \\
	
	We have attempted to replicate with a slightly improved thermal sham on Bacteriophage T4 in the 6 - 11.5 GHz band with Xu's slick amplification-free fluorescence assay. T4 does not appear to exhibit this effect, but this is to be expected as T4's protein capsid is $\approx 5 \times$ more resilient than the lipid envelope of Inf. A et al. T4 is not a good surrogate.\\
	
	 We therefore contribute nothing to the scientific record; most of our discussion is not novel and is primarily qualitative.\\

	 Experiments in the field of microwave biophysics pose peculiar challenges in obtaining a meaningful result. In the process of performing a literature review for this paper, a variety of existential crises regarding we lost confidence in the concept of reliable measurement itself.\\
	 
	 We therefore have some vague reservations regarding aspects of the previous research and recommend great care in interpretation. With some imagination, the results can be ascribed purely to characteristic artifacts in this field. We also believe there may be inconsistencies in the observed protein charge and coupling by many orders of magnitude. However, unlike historical reports, these effects do not appear to be ruled out by basic classical physics.\\
	 
	It seems important to note that most published papers in this field are wrong. There is no reason to believe that this is any different.
	
	Sufficient high-quality evidence does not appear to exist to definitively confirm or refute 
	this effect.
	
	There does not appear to be sufficient clinical evidence to determine safety. What evidence does exist in this regime suggests few acute symptoms, but carcinogenicity as a possible long-term side-effect. 
	
	As almost a dozen clinical trials using ionizing radiation therapies are being undertaken, possible carcinogenicity of treatment does not seem to be a great impediment. \cite{LowDose2020}

	Is sensitive to parameters for which we do not have even an order-of-magnitude guess.

	Unduly preoccupied with matters of instrumentation and methodology - things which have nothing to do with viruses; and so no base truths have been arrived at. 
	
	muster a field of about 100 kV/m. 
	
	there are fundamental gaps in understanding
	
	400 times as effective at penetrating tissue. 
	
	TRL 0 and no resources should be redirected.
	
	This project has been an abject failure in many interesting and notable ways, particularly in redirecting effort and resources from more plausible mechanisms - or even the local community - towards some half-baked pseudoscientific nonsense. 
	
\end{abstract}


\null\begin{tabular}[t]{l@{}}
	  \\
	
\end{tabular}


%\begin{textblock*}{\textwidth}(5,0.5)
%\noindent Data and the garbage codebase produced during this research are available at 
%\href{https://www.github.com/0xDBFB7/covidinator}{github.com/0xDBFB7/covidinator}. This document 
%may be updated as information changes. Please view the latest version. 
%\end{textblock*}
%\begin{textblock}{5}(1,27)
%\end{textblock}

\begin{quotation}\

H. H. Williams: Furious activity is no substitute for understanding.

Given that every route persued consumes precious resources and attention that might be more effective in 
other routes...



Based on artifacts previously observed in the nonthermal effects literature, we can create scenarios in which all of the methods are simultaneously wrong. 

\section{Aims}




\begin{autem}
	
	{\Large {This has \textbf{nothing to do} with 5G communications.}}\\
	
	I am comfortable making such a definitive statement.
	
	The power levels required to even begin to observe this inactivation effect in practice are so high that, if conditions are not specifically tailored for clinical use, extensive burns will result within seconds. Even in this case, the effect is to eviscerate the virus, not to aid its spread. 
	
	Moreover, these frequency bands have been in almost continuous use not long after Jagadish Chandra Bose 1894; the world has been bathed in high-power 10 GHz pulses at least since the commissioning of the first centimeter-wave weather radars in 1959. 
	
	Countless far more prosaic explanations exist in the modern world, such as air travel.
	
	If you have any concerns regarding this statement, please contact me. I would be happy to discuss this.
	
\end{autem}




Legend:  \cmark $ = $ crude experiment, not nearly definitive, needs more work $\vert$ $\thicksim$ $ = $ crude experiment, not nearly definitive, needs more work $\vert$ \xmark \ $ = $ definitely not complete.\\

\begin{itemize}
  \item Establishing the time dependence of inactivation in surrogate bacteriophage $\vert$ \cmark
  \item Demonstrating a modulation scheme that decreases the inactivation threshold to below current safety levels in surrogate bacteriophage $\vert$ \cmark
  \item Demonstrating a prototype emitter in an "electromagnetic mask" form-factor, costing about \$5 in prototype quantities, which can reasonably be produced in 10 million-of quantities $\vert$ $\thicksim$
  \item Testing power thresholds in various conditions; biological fluids of various conductivities and pHs $\vert$ \xmark
  \item Synthesizing a coarse-grained molecular-dynamics simulation of the mechanism $\vert$ \xmark
  \item Discussing the biological basis for the safety of the device $\vert$ \cmark
  \item Showing that the deviation from the expected threshold can be explained by variance in the viron $\vert$ \cmark
  
\end{itemize}\


\tableofcontents


It is perhaps notable that a completely free and open-source toolchain was used for the entire project (with one exception).

\end{quotation}

\end{document}

	
%!TeX root = introduction
\documentclass[paper.tex]{subfiles}
\begin{document}


\flushbottom 
\thispagestyle{empty}



\begin{autem}
	{\large  \it autem} \\
	
	This work was prepared by an undergraduate and has not been peer-reviewed. Additionally, the author has no prior experience with either biology or microwave design. \\

	While our study is very simple, biological research is of such complexity that a tremendous amount of care must be taken before any conclusion can be drawn at all. It is not likely that we have taken sufficient care.\\ 

This is especially true with RF biophysics. The literature is littered with otherwise impeccably perormed research which is both difficult to find fault in, and faulty. \\

This paper does not contribute to the state of the art at all, except perhaps in the obvious avenue of time-domain modulation; [Hung 2014] have already demonstrated a reflectarray, and the rest of our discussion is anything but novel.

Though the original research used Inf. A, our testing was only performed with a surrogate bacteriophage. All experiments must be repeated with SARS-NCoV-2. \\


Accurate microwave measurements are rife with parasitics and sensitivities. It has occurred before that otherwise striking resonances were detected, and further replication found them to be artifacts of the measurement equipment. [Foster 1987] even offers this disclaimer:\\

\begin{quote}

"To detect the DNA resonances with the probe
technique requires correction for system errors that are
potentially much larger than the effect to be studied and
lead to resonance-like artifacts. This is true even with the
more precise instrumentation used in this study. Such data
are easily misinterpreted".

\end{quote}

The direct inactivation plaque and PCR data in the referenced papers are a somewhat convincing corroboration that this effect is not a thermal artifact; but hundreds of convincing and wrong papers exist in the literature.\\

We discuss some possible artifacts in the supplemental, but nothing seems plausible. It should be noted that all three papers that reference this technique use similar methods to determine the resonance mode.\\

Then again, our incredulity is itself probably unfounded. The data these papers provide are excellent; they passed peer-review in Nature Srep, which is a far closer scrutiny than we could possibly provide.

Our data is crude and imprecise, our equipment hastily constructed, and should not be considered validation of this effect even existing. \\

Even if this effect exists and is as effective as it appears, it could still be impractical to apply for any number of reasons.\\


In essence, please consider all claims with appropriate skepticism.

\end{autem}

\end{document}

\clearpage

\paragraph{To-do list}

We see the following steps that must be undertaken before production can be started:

\begin{itemize}
  \item Re-run the experiment with 
  \item Verify 
  \item Failure-tree to ensure that power levels can never go above specified values.
  \item Obtain special permission from the FCC?
\end{itemize}

\clearpage
\begin{multicols}{1}



\paragraph{{\Large the gist}}\

The chain of literature 
%
[Fr\"{o}hlich 1968] \textrightarrow \ [Fr\"{o}hlich 1980] \textrightarrow \ [Liu 2009] \textrightarrow \ ([Hung 2014] $\parallel$ [Yang 2015] $\parallel$ [Sun 2017])\footnote{We have not conducted a thorough review; many more papers are probably involved.} apparently culminates in the flagship work we focus on in this paper,

{\it Efficient Structure Resonance Energy Transfer from Microwaves to Confined Acoustic Vibrations in Viruses} [Yang 2015].
%

estabilishes that - unique to certain viruses, and apparently unlike human cellular structures, as we shall see - coincidentally have just the right size, shape, stiffness, and net charge distribution to form a weak (Q=2) spherical dipole resonance mode which couples well to the microwave spectrum at approximately the X-Band, around 7-10 GHz.

More critically, [Yang 2015] (and, in parallel, [Hung 2014]) theoretically model and then experimentally validate in various strains of Influenza A that - due to this acoustic-resonance effect - the power levels required to crack the lipid envelope are on the order of magnitude of the safety limits even for continous point-blank exposure.\footnote{See supplemental.}

They demonstrate this with both a plaque and PCR assay, finding good agreement with the theoretical model.\footnote{As we will discuss, there are a few issues with the experimental technique.; sham, blinding, and dosimetry demanded by [Vjl.] are not mentioned.}

Like pumping a swing, this effect allows an otherwise inconsequential field magnitude to store energy over a small number of cycles until the virus is destroyed. 

\lettrine{It} cannot be overstated how unexpected and dubious this finding appears to be -at least, based on our limited research and experience to date.

For even the RF power limits set by standards organizations appear to be based on the observation that few significant resonance modes exist in biological tissues, and (as we shall see), this is grounded in solid {\it in vitro} (albeit limited {\it in vivo}) evidence. 

This may account for why this paper has been ignored.

It is difficult to reconcile this discrepancy. A failure tree is included below.


If this mechanism exists, it would seem to provide the following advantages over other methods such as UV or cold-plasma sterilization:

\begin{itemize}
  \item This is a non-ionizing, non-thermal
%
\footnote{It may be useful to define 'non-thermal'; it caused us some confusion. Certainly the proteins of the virion locally absorb energy and increase in temperature. The key is that, when excited in this manner, the energies in the envelope are poorly modelled by a Maxwell-Boltzmann distribution; they are not given sufficient time to 'thermalize'. In contrast, with typical 2.4 GHz microwave exposure, sterilization can only occur by aggregate heating of the fluid and tissue.}
%
, non-chemical technique, harmless for continuous exposure to tissue, which produces no ozone, can be effective inside bodily tissues, and can readily be produced with \$1 - scale devices.
  
  \item Potentially far-field and moderately penetrating inactivation. 
  \item 
  
\end{itemize}





\footnote{It should be noted that this {\it confined} acoustic resonance is subtly distinct from common-and-garden pipe-organ acoustic resonance; this is apparently not a strictly classical phenomenon. Besides standard Coulomb-like and Lennard-Jones-like interactions between constituent particles, if Fr\"{o}hlich is to be believed at these nanoscopic scales there are also wave-function interactions among the particles of the virus which can shift storage to modes not otherwise expected.

We confess to not yet understanding this phenomenon; fortunately, while helpful, the details of how this mode appears are not critical to implementing this technique.}


\footnote{``{[B]elow about 6 GHz, where EMFs penetrate deep into tissue (and thus require depth to be considered), it is useful to describe this in terms of “specific energy absorption rate” (SAR), which is the power absorbed per unit mass $(W/kg)$. Conversely, above 6 GHz, where EMFs are absorbed more superficially (making depth less relevant), it is useful to describe exposure in terms of the density of absorbed power over area $W/m^2$, which we refer to as “absorbed power density”}'' [ICNIRP 2020 \faExternalLink] }



\footnote{All values have been converted to $\text{W/m}^2$ to avoid confusion. 100 $\text{ W/m}^2 = 10 \text{ mW/cm}^2 = 10 \text{ dBm/cm}^2$.}


\begin{toolchain}
	{\it \bf [Yang 2015]'s toolchain}
	\begin{itemize}
	\item Envelope/liposome breaking strength and stiffness from AFM nanoindentation data
	\item Analytical expression assuming homogenous sphere for microwave absorption cross-section
	\item Experimental absorption data from microwave cuvette -> 
	\item COMSOL finite-element for illustration
	\end{itemize}
\end{toolchain}

\end{multicols}

\clearpage

[IEEE C95.1-2005], Annex B, "Identification of levels of RF exposure responsible for adverse
effects: summary of the literature",

\begin{quote}
Further examination of the RF literature reveals no reproducible low level (non-thermal) effect that would
occur even under extreme environmental exposures. The scientific consensus is that there are no accepted
theoretical mechanisms that would suggest the existence of such effects. This consensus further supports the
analysis presented in this section, i.e., that harmful effects are and will be due to excessive absorption of
energy, resulting in heating that can result in a detrimentally elevated temperature. The accepted mechanism
is RF energy absorbed by the biological system through interaction with polar molecules (dielectric relaxation) or interactions with ions (ohmic loss) is rapidly dispersed to all modes of the system leading to an
average energy rise or temperature elevation. Since publication of ANSI C95.1-1982 [B6], significant
advances have been made in our knowledge of the biological effects of exposure to RF energy. This
increased knowledge strengthens the basis for and confidence in the statement that the MPEs and BRs in this
standard are protective against established adverse health effects with a large margin of safety.

\end{quote}

The human body contains proteins with all sorts of net charges. 

Other physiological effects include changes in the permeability of membranes, direct nerve stimulation, and others. These are all accounted for in the time-domain equation and the 6-minute averaging period.

This would seem to open an obvious route of optimization. If the virus is destroyed in tens of nanoseconds - essentially governed by instantaneous power. But tissue damage requires *energy deposition*. 

Obviously, the solution is thus to turn the CW microwave signal into an effectively instantaneous pulse.


\clearpage
{\Large \it Talkin' 'bout the Variation}\\

To re-cap, [Yang 2015] theoretically model the virus to determine the minimum electric field required to destroy it. 

They assume that the virus is a simple damped harmonic oscillator, where the 'core' and 'shell' oscillate in opposition. 

They determine the net charge experimentally from microwave absorption measurements.

Since [Yang] try to compute the {\it threshold}, they use a value of 400 pN for the breaking strength of the envelope, obtained from [Li 2011]. However,  

"95%.".

\begin{center}
 \begin{tabular}{||c c c c||} 
 \hline
 Col1 & Col2 & Col2 & Col3 \\ [0.5ex] 
 \hline\hline
 1 & 6 & 87837 & 787 \\ 
 \hline
 2 & 7 & 78 & 5415 \\
 \hline
 3 & 545 & 778 & 7507 \\
 \hline
 4 & 545 & 18744 & 7560 \\
 \hline
 5 & 88 & 788 & 6344 \\ [1ex] 
 \hline
\end{tabular}
\end{center}





Of course, this application is hardly much better than an N95 mask, except that it is non-disposable and provides protection for eyes and skin.

Vomit


[Vj] mention the importance of precise dosimetry. Even simple structures can produce hot-spots 
Yang et al use both a plastic cuvette, and a single drop of solution on a glass slide; in either case, a sharp change dielectric constant is present.

To the extreme, some papers have used














\clearpage
\paragraph{\textbf{Time dependence}}\

The fact that the viral inactivation is non-thermal

Both [Yang 2015] and [Hung 2014] use an apparently arbitrary 15-minute exposure in their tests - a very reasonable decision, given the focus of their paper. 

The effectiveness against airborne particles, and to minimize the power required in a dwelling phased-array beam, we must establish the required duration of exposure.

{\color{red} speculative hypothesizing \{ } 

In contrast to chemical inactivation, where the time dependence appears to be dominated by viscous fluid dynamic effects [Hirose 2017], or UV inactivation, where a certain quantized dose of photons must be absorbed, we expected RF to act instantaneously.

As a damped, driven oscillator, the ring-up time of the virus depends on the Q factor. Yang et al. state the Q of Inf. A as between 2 and 10, so at 8 GHz the steady-state amplitude should be reached in well under 100 nanoseconds.???????????FIXME

[] found a significant mechanical fatigue effect in phage capsids, where a small strain applied repetitively eventually causes a fracture. Such a mechanism could perhaps extend the exposure required to break the capsid or membrane. Other mechanisms could include some sort of lipid denaturation, requiring an absolute amount of energy absorption to break or twist bonds and modify the properties before the envelope fractures.


{\color{red}  \} } 





\subfile{safety}









\clearpage
%%%%%%%%%%%%%%%%%%%%%%%%%%%%%%%%%%%%%%%%%%%%%%%%
\begin{multicols}{1}
{\Large The Experiment}\\
%%%%%%%%%%%%%%%%%%%%%%%%%%%%%%%%%%%%%%%%%%%%%%%%

\paragraph{\textbf{Centrifugal microfluidics}}\

The field of centrifugal microfluidics is accelerating. 

Many CD microfluidics systems use standard CD molding techniques for the channels and machining techniques, using either acrylic or silicone. The turbidity sensor is most sensitive if the plastic is clear. Sterilization does not seem to be discussed. 

Polypropylene is the ideal material, being almost indefinitely autoclavable. It is quite difficult to machine.

\end{multicols}












\clearpage
%%%%%%%%%%%%%%%%%%%%%%%%%%%%%%%%%%%%%%%%%%%%%%%%
\begin{multicols}{1}
{\Large Modes of application}\\
%%%%%%%%%%%%%%%%%%%%%%%%%%%%%%%%%%%%%%%%%%%%%%%%

\paragraph{\textbf{Personal 'electromagnetic mask'}}\

Even with judicious use of phased-arrays, spatial power-combining, etc, each transistor can only reasonably maintain sterility in approx. $ 0.1 \text{ m}^3 $.

We therefore demonstrate this form factor because, superficially, there are fewer people than there are places. 

On the other hand, a personal device may present issues with participation and production volume. 

\paragraph{\textbf{Direct treatments}}\

[Hand 1982] $E_{mag}=1/e$ (13.5\% power density) skin\footnote{electromagnetic skin, not tissue skin}\footnote{well, both, I suppose.} depth is approximately:

\begin{center}
\begin{tabular}{|l|l|l|}
\hline
F=10 GHz          & Dry tissue & Wet tissue \\ \hline
Penetration depth & 30 mm      & 5 mm \\ \hline
\end{tabular}
\end{center}

SARS is found widely distributed throughout the most favorable organs [Ding 2004], shielded by an average of 4 cm of chest wall [Schroeder 2013]; so safe external treatment of the body is unlikely.

However, destruction of lung tissue appears to be the primary cause of death via SARS [Nicholls 2006]. 

A bronchoscopic technique may therefore be effective, very similarly to that demonstrated by [Yuan 2019]: in adults, the bronchi are less than 2 mm thick [Theriault 2018] and the lungs themselves are only on the order of 7 mm thick [Chekan 2016]. 

The main bronchus is about 8 mm in diameter, which is smaller than the patch antenna used here; a monopole (or multiple, phased monopoles) like is probably more suitable.


\paragraph{\textbf{More fanciful concepts}}\

Many systems operate in these X-band frequency ranges; and precision 

It is easy to produce megawatts of power at these frequency ranges using Klystrons. Assuming an appropriate antenna, a single 50MW SLAC XL-class klystron with a PRF of 180 Hz could sanitize almost a square kilometer every second. \footnote{Giving 'asceptic field' new meaning.}

On the other hand, if this is conducted in outdoor free-space, sunlight

Existing marine, weather, and aviation radar systems often use the X-band; depending on the focusing power

\paragraph{Sensing}


Surprisingly, there is little discussion on microwave virus detection, with the notable exception of [Mehrotra 2019]. Such scholarly silence is usually an indicator that a technique is impossible for reasons so obvious they scarcely bear repeating. 

However, [Oberoi 2012] find that they can detect a single E. Coli bacterium in 50 uL of broth via a microwave cavity. This is so unbelievable that we are inclined to look for methodological issues. 

Frequency-domain measurements can be made with incredible precision; and there are many parameters by which virions could be discriminated; ring-up time of the resonance, excitation non-linearity, precise microwave spectroscopy, etc.

\paragraph{}

Finally, there is one concern. It has been shown that virions can effect large-scale changes in the electric charges of their []. 

Use of this technique will provide a selection bias towards immunity to electromagnetic fields, which could perhaps be effected by preferring extreme-sized mutants (shifting the resonance away from the applied field). We do not have the biological knowledge to know if this is plausible; it is simply worth mentioning. 

\end{multicols}







\clearpage
%%%%%%%%%%%%%%%%%%%%%%%%%%%%%%%%%%%%%%%%%%%%%%%%
{\Large Microwave Musings}\\
\begin{multicols}{1}
%%%%%%%%%%%%%%%%%%%%%%%%%%%%%%%%%%%%%%%%%%%%%%%%

\noindent\fbox{\parbox{\linewidth}{
	Toolchain:
	\begin{itemize}
	\item Failed oscillator feedback-loop optimization toolchain: QUCS 0.0.20 + python-qucs + scipy's 'basinhopper'
	\item Successful A slightly modified ngspice + ngspyce + pyEVTK
	\item gprMax for FDTD EM simulation
	\item KiCAD, wcalc, scikit-rf, ngspice
	\end{itemize}
}}
%


Microwave design has a reputation for being the purview of wizards. Modern RF software packages like HFSS, Microwave Office, Agilent's ADS, and Mathworks' RF toolbox, for which; component models 

Luckily, this project falls perfectly within the subset of microwave technologies that do not require a goatee. The vast majority of the behavior of most circuits can be modelled with the very same SPICE tools as one would use at low frequencies. With judicious application of reference designs, it seems to be possible to design 

Before about 2005, however, it seems to have been somewhat the norm to write a simple numerical code to solve the problem at hand based on underlying principles.

We are also aided by the fact that what took a \$150,000 computing cluster a single decade ago can now be done in a few minutes by a single budget GPU.


\end{multicols}


\clearpage
%%%%%%%%%%%%%%%%%%%%%%%%%%%%%%%%%%%%%%%%%%%%%%%%
{\Large Interference}\\
\begin{multicols}{1}
%%%%%%%%%%%%%%%%%%%%%%%%%%%%%%%%%%%%%%%%%%%%%%%%



%%%%%%%%%%%%%%%%%%%%%%%%%%%%%%%%%%%%%%%%%%%%%%%%
{\Large Mass production}
%%%%%%%%%%%%%%%%%%%%%%%%%%%%%%%%%%%%%%%%%%%%%%%%

We've got a thing 
that's called 
Radar Love
We've got a wave
in the air
Radar Love

Almost all components on the device can be replicated with fully vertically-integrated first-principles. Capacitors can be

If this 'electromagnetic mask' form is the ideal (not nearly), 

The largest RFID plants can produce

a minimum of 5 GaAs or SiGe:C transistors will be required. Without SOI or, it does not appear that 

As a lower bound, there are 1 million hospital beds in the U.S. [AHA 2018]; and as an upper bound obtaining global herd immunity would take 1.75 billion units.

It is difficult to determine the supply capacity for these semiconductor processes; information is not forthcoming from the manufacturers. Fermi estimates \footnote{GaAs MMIC market of \$2.2 Bn USD / random sample of MMIC prices = about 2 Bil devices / yr, 3e9 wifi connected devices produced each year,}



It is possible to use common Si-based devices at these frequencies, especially with second-harmonic techniques [Winch 1982]. However, obtaining the required gain and output power is not a trivial matter. 

The techniques and equipment required to produce these devices are extremely complex; load-lock UHV,  


Figures are not forthcoming, 

Given the supply difficulties of comparatively simple materials such as Tyvek at pandemic scales, it is difficult to imagine that RF semiconductor production can be immediately re-tasked and scaled to this degree. 

\paragraph{\textbf{Vacuum RF triode}}\



Especially combined with an integrated titanium sorption pump,

One concern is the large filament heater power, which prevents the use of low-cost button cells for power. Use of cold-cathode field-emitter arrays would alleviate this issue, but at the cost of complexity.

Small tungsten incandescent lights are available down to 0.05 watts. With a suitable high-efficiency cathode coating, a pulsed heater power of less than 0.02



\end{multicols}





\subfile{acknowledgement}




\subfile{supplemental}







\end{document}
