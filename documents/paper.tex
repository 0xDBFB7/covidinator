\documentclass[fleqn,10pt]{article}





\usepackage[left=2cm,right=2cm,
			top=1.25cm,
			bottom=2.25cm,%
			headheight=11pt,%
			letterpaper]{geometry}
			
\frenchspacing			

\nonstopmode




\usepackage{lmodern}
\usepackage[T1]{fontenc}
\usepackage[utf8]{inputenc}



\usepackage{noweb}

\usepackage{multicol}
\usepackage{fancyhdr}
\usepackage{blindtext,graphicx}
\usepackage[absolute]{textpos}
%\usepackage[parfill]{parskip}
\usepackage{parskip}
\setlength{\parskip}{\baselineskip}

\usepackage[colorlinks=true,citecolor=brown]{hyperref}
\usepackage{gensymb}
\usepackage{csquotes}
\usepackage{amsmath}
\usepackage{fontawesome}
\usepackage{orcidlink}
\usepackage{standalone}
\usepackage{pdfpages}
\usepackage{subfiles}
\usepackage{svg}
\usepackage{sidecap}
\usepackage{float}
\usepackage{amssymb}
\usepackage{textcomp}
\usepackage{lettrine}
%\usepackage[T1]{fontenc}

\usepackage{soul}


%\usepackage{draftwatermark}
%\SetWatermarkText{DRAFT}
%\SetWatermarkScale{0.25}

\usepackage{booktabs,caption}
\usepackage[flushleft]{threeparttable}

%\usepackage{biblatex}
\usepackage[backend=bibtex8, sorting=none, style=chem-angew]{biblatex}

\let\cite\footfullcite

%\let\cite\footcite

\addbibresource{processed.bib}
%biblatex has a zoterordfxml
% might avoid the need for python bibtex_collections.py



\usepackage{etoolbox}
\AtBeginEnvironment{quote}{\small}




\usepackage{pifont}
\newcommand{\cmark}{\ding{51}}%
\newcommand{\xmark}{\ding{55}}%


\newcommand{\citationneeded}[1][]{\textsuperscript{[\color{blue}{\it \bf{citation needed}#1}]}}
\newcommand{\dubiousdiscuss}[1][]{\textsuperscript{\color{blue} [{\it \bf{dubious-discuss}}]} }

\newcommand{\light}[1]{\textcolor{gray}{#1}}

%
%
\usepackage{titlesec}
%
%% custom section


\titleformat{\section}
{\normalfont\LARGE\bfseries}{\thesection}{1em}{}
%\titleformat{\section}
%{\normalfont\LARGE\bfseries\PRLsep}
%{{{{\itshape \thesection\hskip 9pt\textpipe\hskip 9pt}}}}{0pt}{}
%
%% custom section
%\titleformat{\subsection}
%{\normalfont\Large\bfseries\PRLsep}
%{{{{\itshape \thesection\hskip 9pt\textpipe\hskip 9pt}}}}{0pt}{}
%
%
%


\newcommand{\Wsqm}{$\text{ W/m}^2$}

\newcommand{\ghfile}[1]{\href{https://github.com/0xDBFB7/covidinator/tree/master/#1}{\faGithub/\url{#1} }}

%\newcommand{\supercite}[1]{}
%\newcommand{\supercollect}[1]{}


\newlength{\PRLlen}
\newcommand*\PRLsep[1]{{\itshape \Large\settowidth{\PRLlen}{#1}\advance\PRLlen by -\textwidth\divide\PRLlen by -2\noindent\makebox[\the\PRLlen]{\resizebox{\the\PRLlen}{1pt}{$\blacktriangleleft$}}\raisebox{-.5ex}{#1}\makebox[\the\PRLlen]{\resizebox{\the\PRLlen}{1pt}{$\blacktriangleright$}}\bigskip}}


\renewcommand{\thefootnote}{\textcolor{gray}{\arabic{footnote}}}


\usepackage{graphicx}
\graphicspath{ {../media/} 
				{../firmware/eppenwolf/runs/sic_susceptor/} 
			}

\usepackage{tcolorbox}
\newtcolorbox{protocol}{colback=yellow!5!white,colframe=yellow!75!black}
\newtcolorbox{equipment}{colback=orange!5!white,colframe=orange!75!black}
\newtcolorbox{autem}{colback=red!5!white,colframe=red!75!black}
\newtcolorbox{toolchain}{colback=blue!5!white,colframe=blue!40!black!40}
\newtcolorbox{sidenote}{colback=cyan!5!white,colframe=blue!40!black!40}
%https://tex.stackexchange.com/questions/66154/how-to-construct-a-coloured-box-with-rounded-corners

%\usepackage[sfdefault,light]{roboto}

\setlength{\TPHorizModule}{1cm}
\setlength{\TPVertModule}{1cm}





%%%%********************************************************************
% fancy quotes
\definecolor{quotemark}{gray}{0.7}
\makeatletter
\def\fquote{%
	\@ifnextchar[{\fquote@i}{\fquote@i[]}%]
}%
\def\fquote@i[#1]{%
	\def\tempa{#1}%
	\@ifnextchar[{\fquote@ii}{\fquote@ii[]}%]
}%
\def\fquote@ii[#1]{%
	\def\tempb{#1}%
	\@ifnextchar[{\fquote@iii}{\fquote@iii[]}%]
}%
\def\fquote@iii[#1]{%
	\def\tempc{#1}%
	\vspace{1em}%
	\noindent%
	\begin{list}{}{%
			\setlength{\leftmargin}{0.1\textwidth}%
			\setlength{\rightmargin}{0.1\textwidth}%
		}%
		\item[]%
		\begin{picture}(0,0)%
		\put(-15,-5){\makebox(0,0){\scalebox{3}{\textcolor{quotemark}{``}}}}%
		\end{picture}%
		\begingroup\itshape}%
	%%%%********************************************************************
	\def\endfquote{%
		\endgroup\par%
		\makebox[0pt][l]{%
			\hspace{0.8\textwidth}%
			\begin{picture}(0,0)(0,0)%
			\put(15,15){\makebox(0,0){%
					\scalebox{3}{\color{quotemark}''}}}%
			\end{picture}}%
		\ifx\tempa\empty%
		\else%
		\ifx\tempc\empty%
		\hfill\rule{100pt}{0.5pt}\\\mbox{}\hfill\tempa,\ \emph{\tempb}%
		\else%
		\hfill\rule{100pt}{0.5pt}\\\mbox{}\hfill\tempa,\ \emph{\tempb},\ \tempc%
		\fi\fi\par%
		\vspace{0.5em}%
	\end{list}%
}%
\makeatother







%%%%********************************************************************
%title link to doi
\newbibmacro{string+doiurlisbn}[1]{%
	\iffieldundef{doi}{%
		\iffieldundef{url}{%
			\iffieldundef{isbn}{%
				\iffieldundef{issn}{%
					#1%
				}{%
					\href{http://books.google.com/books?vid=ISSN\thefield{issn}}{#1}%
				}%
			}{%
				\href{http://books.google.com/books?vid=ISBN\thefield{isbn}}{#1}%
			}%
		}{%
			\href{\thefield{url}}{#1}%
		}%
	}{%
		\href{https://doi.org/\thefield{doi}}{#1}%
	}%
}

\DeclareFieldFormat{journaltitle}{\usebibmacro{string+doiurlisbn}{\mkbibemph{#1}}}

\begin{document}



\subfile{title}


\begin{table}[h!]
	\centering
	\begin{tabular}{ |c|c|c|c| } 
		\hline
		col1 & col2 & col3 \\
		\hline
		& cell2 & cell3 \\ 
		& cell5 & cell6 \\ 
		& Negative & 5 \\ 
		\hline
	\end{tabular}

	\caption{Data from testing on bacteriophage. Data rounded to the nearest thousand photon counts. Not thoroughly characterized. Repeatability between test conditions   Background was no. 
	The flurometer printed a constant stream of the integration, all within the random error between capture windows $\sigma=0.5$ kcounts. }
\end{table}


\begin{figure}[H]
	\captionsetup{singlelinecheck = false, justification=justified}
	\centering
	\includegraphics[width=\textwidth]{pulse_exposure_setup.JPG}
	\caption{\\ \textit{Chengxiang} pulse exposure jig. pulser seen off the left. Autosampler Pulse shaping plate not shown. Aluminum disc belongs to the scan-conversion digitizer, not used in this study.}
\end{figure}
	
\begin{figure}[H]
	\captionsetup{singlelinecheck = false, justification=justified}
	\centering
	\includegraphics[width=\textwidth]{eppenwolf_2.jpg}
	\caption{\\ \textit{Eppenwolf} misguided pulsed X-Band microwave spectrometer used in this study, pictured with the near-field cuvette installed.}
\end{figure}

\clearpage

\subfile{background}













\begin{figure}[H]
	\makebox[\textwidth][c]{
		\includegraphics[width=\textwidth*2]{chunk_4_line_2_1.png}
	}%
	\caption{Chunk 4, it.is Founda}
\end{figure}


\clearpage

\begin{figure}[H]
	\captionsetup{singlelinecheck = false, justification=justified}
	\centering
	\includegraphics[width=\textwidth]{chunk_4}
	\caption{
		\light{
		\\
		We've got a thing\\
		that's called\\
		{\it Radar Love}\\}
		We've got a wave\\
		in the air}
\end{figure}






\paragraph{To-do list}

We see the following steps that must be undertaken before production can be started:

\begin{itemize}
  \item Re-run the experiment with 
  \item Verify 
  \item Failure-tree to ensure that power levels can never go above specified values.
  \item Obtain special permission from the FCC?
\end{itemize}

\clearpage	
\begin{multicols}{1}




which establishes that - unique to certain viruses, and apparently unlike human cellular structures, as we shall see - coincidentally have just the right size, shape, stiffness, and net charge distribution to form a weak (Q=2) spherical dipole resonance mode which couples well to the microwave spectrum at approximately 8 GHz.

More critically, [] (and, in parallel, [Hung 2014]) theoretically model and then experimentally validate in various strains of Influenza A that - due to this acoustic-resonance effect - the power density levels required to crack the lipid envelope are near the safety limits for continuous exposure to humans.\footnote{Sort of. See below and supplemental.}

They demonstrate this with both a plaque and PCR assay, finding good agreement with the theoretical model.\footnote{As we will discuss, there are a few issues with the experimental technique.; sham, blinding, and dosimetry demanded by [Vjl.] are not mentioned.}

Like pumping a swing, this effect allows an otherwise inconsequential field magnitude to store energy over a small number of cycles until the virus is destroyed. \\
\\\\

\paragraph{Ramifications}\



\begin{autem}

{\it autem}\\
The previous paragraph may cut an impressive figure; but it hinges on all the rest of this paper being correct.

\end{autem}

\lettrine{It} cannot be overstated how unexpected and positively dubious this finding appears to be - that is, based on our limited research and inexperience to date.

For even the RF power limits set by standards organizations like \cite{ICNIRP2020} \cite{IEEE2006} appear to be based on the observation that no significant resonance modes exist in biological tissues due to damping forces from the surrounding tissue \cite{Vibrational2002}. As we shall see, this is grounded in solid {\it in vitro} (albeit more limited {\it in vivo}) evidence. 

This may account for why this paper has been largely ignored.

The precise structure and charge of the virion appears to be a singular anomaly in this otherwise categorical non-existence.





\end{multicols}
\clearpage
%%%%%%%%%%%%%%%%%%%%%%%%%%%%%%%%%%%%%%%%%%%%%%%%%%%%%%%%%%%%%%%%%%%%%%%%%%

%
%\begin{table} 
%	\centering
%	\begin{threeparttable}
%		\caption{Safety limits from various standards}
%		\begin{tabular}{lllll}
%			\toprule
%			Stubhead & \( df \) & \( f \) & \( \Wsqm \) & \( p \) \\
%			\midrule
%			&     \multicolumn{4}{c}{Spanning text}     \\
%			Row 1    & 1        & 0.67    & 0.55       & 0.41    \\
%			Row 2    & 2        & 0.02    & 0.01       & 0.39    \\
%			Row 3    & 3        & 0.15    & 0.33       & 0.34    \\
%			Row 4    & 4        & 1.00    & 0.76       & 0.54    \\
%			\bottomrule
%		\end{tabular}
%		\begin{tablenotes}
%			\small
%			\item This is where authors provide additional information about
%			the data, including whatever notes are needed.
%		\end{tablenotes}
%	\end{threeparttable}
%\end{table}

All pertinent restrictions for frequency $<$ 6 GHz, for general public.
ICNIRP 2020, Table 2, "Basic restrictions for electromagnetic field exposure",
20 \Wsqm, averaged over a 6 min period, averaged over a square $4 \text{cm}^2$ surface area of the body.

Whole-body average SAR $<$ 0.08 W/kg

And, additionally, 

ICNIRP 2020, Table 3, Basic Restrictions


\clearpage
{\Large \it The contenders}\\



\clearpage



\paragraph{Time-domain modulation}

However, there is an obvious avenue of optimization. If the virus is destroyed in a few dozen nanoseconds via an electric field amplitude incidentally caused by an instantaneous power $\text{P}$, but tissue damage requires a temperature rise due to an energy deposition $\text{P} \  \text{dt}$, then we should minimize dt.\

We thus turn the CW microwave signal into a series of effectively instantaneous pulses.

Note that contrary to the limits at lower frequencies, neither ICNIRP nor IEEE appear to explicitly specify a maximum electric field; only the maximum power is specified (thereby implicitly restricting the electric field magnitude).

The safety standards of [IEEE] and [ICNIRP] account for such time-domain modulation by specifying both an average power limit over a 6 minute period, and a time-integrated energy deposition limit. All known non-thermal and thermal physiological effects, including changes in the permeability of membranes, direct nerve stimulation, etc, are accounted for by obeying those two limits.

To drive home this point, apparently the best-quality evidence available fulfills sham requirements provide in-vitro. They expose blood lymphocytes to pulse power in precisely the regime required in this work; 8.2 GHz, 8 ns duration and 50 khz repetition rate, a whopping pulse power density of 250,000 $\text{W/m}^2$\footnote{computed from average power / duty cycle} (2500x the time-averaged power density safety limit), average power density 100 $\text{W/m}^2$ (the safety limit), for 2 hours, finding no change in any of the measured quantities. 

Similarly, [Chemeris 2004] use 8.8 GHz, 180 ns pulse width, peak power 65,000 W, repetition rate 50 Hz, exposure duration 40 min, and find genotoxicity purely due to the temperature rise.

So it appears that this is not merely rules-lawyering to exploit a loophole in the regulations; it is a physically-informed effect.

\begin{autem}

These papers (the best quality we are aware of) were extensively cherry-picked from the literature based on the results of a meta-analysis [Vj. 2019], listing the requirements for reliable in-vitro RF experiments.\\


There are a substantial number of papers in the literature which show positive effect sizes; a select few of these are listed in the bibliography. \\

The literature reviews conducted by standards organizations.\\

We are not aware of any past uses of these microwave en-masse. We do not have the 

Moreover, 
\begin{quote}
	
In the absence of publication bias, since the sampling error is random, all studies will be distributed symmetrically around the mean d value.

\end{quote}


\end{autem}

Applying microwave power to the body is already used extensively in diathermy (the internal heating of tissue for therapeutic effect) and hyperthermy (the overheating or cauterization of tissue for destructive effect). While most diathermy systems use much lower frequencies for deeper penetration, several papers have explored the 10 GHz band. 

Industry Canada only appears to offer an instantaneous limit at $1.35 \times 10^-4$.

6 cm of l

The 

Fat and 

The penetration of microwave power is very sensitive to 

Conduction effects. Because the transmission line voltage only goes as the to the square root of the input power

There seems to be some sort of focusing effect which makes internal bronchoscopic treatment slightly more effective than might be expected.

\clearpage
%\printbibliography[heading=none, title={}, keyword={Flagship}]
\printbibliography[heading=none, title={}, keyword={standards}]



\clearpage
{\Large \it Time-dependence}\\

The fact that the viral inactivation is non-thermal

Both [Yang 2015] and [Hung 2014] use an apparently arbitrary 15-minute exposure in their tests - a very reasonable decision, given the focus of their paper. 

The effectiveness against airborne particles, and to minimize the power required in a dwelling phased-array beam, we must establish the required duration of exposure.

{\color{red} speculative hypothesizing \{ } 

In contrast to chemical inactivation, where the time dependence appears to be dominated by viscous fluid dynamic effects [Hirose 2017], or UV inactivation, where a certain quantized dose of photons must be absorbed, we expected RF to act instantaneously.

As a damped, driven oscillator, the ring-up time of the virus depends on the Q factor. Yang et al. state the Q of Inf. A as between 2 and 10, so at 8 GHz the steady-state amplitude should be reached in well under 100 nanoseconds.???????????FIXME

[] found a significant mechanical fatigue effect in phage capsids, where a small strain applied repetitively eventually causes a fracture. Such a mechanism could perhaps extend the exposure required to break the capsid or membrane. Other mechanisms could include some sort of lipid denaturation, requiring an absolute amount of energy absorption to break or twist bonds and modify properties before the envelope fractures.


{\color{red}  \} } 




\clearpage



Of course, this application is hardly much better than an N95 mask, except that it is non-disposable and provides protection for eyes and skin.

Vomit



To the extreme, some papers have used

















\subfile{safety}



















\clearpage
%%%%%%%%%%%%%%%%%%%%%%%%%%%%%%%%%%%%%%%%%%%%%%%%
\begin{multicols}{1}
{\Large Modes of application}\\
%%%%%%%%%%%%%%%%%%%%%%%%%%%%%%%%%%%%%%%%%%%%%%%%

\section{Sensing}

There is little discussion on microwave virus detection, with the notable exception of [Mehrotra 2019]. 

However, \cite{Microwave2012} find that - in controlled conditions - they can detect a single E. Coli bacterium in 50 uL of broth via a microwave cavity. 

Havelka \cite{Highfrequency2011} discuss microtubules.

With an extremely high power excitation pulse, the problem may become somewhat easier; especially if the virus has a ring-down that can be subjected to autocorrelation.

However, the dipole radiation power expected in even the most implausibly optimistic case would be about as follows:

$$P={\frac {\mu _{0}(2 \pi \cdot 20 GHz) ^{4}(0.5 \text{nm} * 10^6 e^-)^{2}}{12\pi c}} = 7.1 \times10^{-16} \text{W} $$

which for most reasonable conditions would $\approx 0.5 Janskies$, meaning that the power radiated from the virus would be considerably overpowered by the milky way.

% (magneticConstant*(2 pi * 20 GHz)^4 * (0.1 nm * 1000 * electronCharge)^2)/(12*pi * c) -> W



\end{multicols}

{\color{red} speculation \} } 



\subfile{brillouin_pulse_math}



%\paragraph{}


%\subfile{acknowledgement}


\subfile{charge}

\subfile{viscosity}

\subfile{molecular_dynamics}

\section{Appendix on microwave design}

\subfile{supplemental}

\section{Appendix on microbiology}

\subfile{fluorescence}

\subfile{detection}

\subfile{lessons_learned}

\nocite{*}

%\printbibliography[]
%\printbibliography[title={All references}]

\end{document}
