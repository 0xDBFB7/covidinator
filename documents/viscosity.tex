%!TeX root = viscosity
\documentclass[paper.tex]{subfiles}
\begin{document}


\begin{autem}
	Burkheartsmeier and others have already computed this. We add this to satisfy our own curiosity, and because it is a central point and must be correct.
\end{autem}


We use Persson's description of purely classical Navier-Stokes drag \cite{nature1986} on a homogeneous sphere in a viscous liquid. The first breathing mode is used for a conservative estimate.

To compare with Yang's results, Influenza A. Density is equal to that of the surrounding solution: $\rho_{liquid} = \rho_{solid} = 1000 \text{kg/m}^3$. R = 50 nm. An approximate c=1000 m/s is used. Viscosity of $\mu=10e-3 Ns/m^2$ is used.

$$ b = \frac{4\mu}{R^2 \rho_{solid}} = \frac{4 \cdot 1\times 10^{-3}\  \text{N}\cdot \text{s} / \text{m}^2 }{(50 \text{nm})^2 1000 \text{kg /m}^2} = 1.6 \times 10^9\ \text{N}\cdot \text{s} /( \text{m}\cdot\text{kg})$$

Where $\mu$ is the viscosity of the fluid, R is the radius of the sphere, $\rho_{solid}$. \footnotemark

\footnotetext{I must remember to disambiguate denominators by adding brackets. Is $1/R^2\rho$ = $1/(R^2 \rho)$ or $(1/R^2)\cdot\rho$ ?}


$$m = \rho_{solid}  (4/3) \pi r^3 = 5.2 \times 10^{-19} \text{ kg} $$

(In the case of Influenza A, this yields 313 MDa, which is a 1/3 overestimate). 

The exponential damping time constant (units 1 / s ) is then

$$\gamma = \frac{b}{2m} = $$

Finally,

$$ \omega_{res} = \sqrt{\frac{(\rho_{solid} / \rho_{liquid})c^2}{R^2}} $$










\end{document}