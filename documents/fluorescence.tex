%!TeX root = fluroescence
\documentclass[paper.tex]{subfiles}
\begin{document}

If time-domain filtering is sufficient, 

Low background is required.

Xu's method \cite{Quantification2020}

Performance of this arrangement was satisfactory. A 10-second integration time, with, produced a background fluorescence signal of ~1500 counts, with per-sample stability of approximately \pm 1500 counts. 



Contrary to standard microscopy, you want as little excitation light to enter the objective as possible. Using the existing 



Nanodrop, using 280 nm absorbance. They're also \$10,000.




the input edge is considered a clock; the minimum pulse width limits apply, but are not clearly specified in the datasheet. In this case (1/155 mhz) = 6 ns.

However, we had no luck with precipitation.


Luckily, a recent paper has the answer: direct fluorescent detection of DNA in solution, outside using dyes that bind to (intercalate into) DNA. GelGreen doesn’t penetrate.



As noted by [xi?], this is a saturating effect; if too many flurophores intercalate into the DNA the fluorescence is weaker. A “1/4000” solution of GG in distilled water mixed 50/50 with the sample (“1/8000”) worked great in our case.

(Phi6 uses an RNA - many dyes have different responses to single-stranded (ss)DNA, dsDNA, or 





With the setup we’re using and the small quantities, the excitation light is somewhere around ~10^5 times as powerful as the emission. This doesn’t seem to be a big issue gel-docs, picking out bands on gels - they don’t usually seem to use excitation filters; however, to get the excitation bleed-through low enough to do this quantitative assay, the bleed-through must be really low, and in our testing proper dielectric filters are required on both the excitation and emission sides. 

There are a few sources of noise:

PMT dark counts
Some be filtered out by judicious use of comparator pulse height threshold, the lock-in takes care of it. A non-issue in our case.
Ambient light leakage
Using microscope optics and the lock-in, essentially a non-issue for use, surprisingly. Some fluorescence microscopes use micron-size apertures to limit the depth of field to avoid Putting 
Figure out the average current limit for your PMT, the expected voltage based on your load resistor value and watch that it never exceeds it. It’ll be fine. Even with the room lights


Note that PMTs are happy to endure very high pulse currents, like flashbulbs for exciting. 


No shutters or anything required. 


The PMT was at maximum sensitivity (in fact, slightly above max voltage)

Gel-docs , with spectra specified using the Wratten scale. These do not seem to have optical density (OD - logarithmic value of transmission in the blocking band) values sufficient for this qfluor. 

Epiflurorescence means the Most commercial microscopes use dichroic beamsplitters, allowing the excitation beam beam to go through the same objective. The dichroic only removes the excitation to the 1\% level - you still need the dielectric filters, and they’re really expensive. With fiber optic excitation at right angles to the objective, scattering was low enough that the dichroic was completely unnecessary. 

You might be wondering why the filters are even required - why not just subtract the excitation bleed-through with a control? Unfortunately, any miniscule variation in the scattering of the excitation will be orders of magnitude larger than the fluorescence signal you’re looking for.

Some papers discuss adding a third chopper or gate period or photomultiplier or to measure the drift of the excitation light source.

Gating the photomultiplier HV is another technique.

PCR is another method


Silicon photomultipliers like ON's C−Series are almost certainly sufficient for this application, eliminating the HV requirements of PMTs at the cost of a higher dark count and slightly smaller active area.

half-life is 0.693 times the average lifetime. 


I thought the pulse height

An even better Phase-shift time domain fluorimetry. Iwata use a 20 MHz DSO to measure a 5 ns $\tau$ fluorophore. However, this involves a light source with a fast modulation bandwidth; and in the implementation they describe, the PMT must be in the voltage mode.

Called time-correlated photon counting.

Another neat technique is to add

Then your 

Cree recommends staying below 300\% of the continuous power when modulating an LED.

While there are ways to deconvolve a slow falling edge, there’s another problem: per time interval, the time spent in the excitation-off, flurophore exponential decay time is proportional to the frequency of the excitation light. If you’re only getting 1000 photons from the sample per second, the dye lifetime is 5 ns, and you’re turning the light on and off at 1 MHz, you’re only getting a [] photons from the exponential decay region. Even with a long exposure, that doesn’t seem to be enough to pick up the decay. 

Another way to pick up this phase shift is to make your lock-in amplifier phase-sensitive - very simple (a 2x clock put into a flip-flop divider, then the middle). This also wasn’t nearly good enough, completely useless. 

Simple spectrally-filtered intensity is good enough.

Contrary to luminescence, you have control over when the excitation and emission light turns on, so you can do a lock-in. This doesn’t subtract the excitation bleed-through, but it 

Polarization is a neat way to filter; linearly polarize the excitation, the emission comes out whatever orientation the DNA happens to be, which is usually random. Apparently



\end{document}