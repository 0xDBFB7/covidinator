%!TeX root = application
\documentclass[paper.tex]{subfiles}
\begin{document}


Applying microwave power to the body is already used extensively in diathermy (the internal heating of tissue for therapeutic effect) and hyperthermy (the overheating or cauterization of tissue for destructive effect). While the majority of diathermy systems use much lower frequencies  for deeper heating penetration, several papers have explored the 10 GHz band. 

Industry Canada limits on the instantaneous electric field anywhere inside the body are $1.35 \times 10^{-4}$ V/m $\times$ Hz (\cite{RSS1022015}, Table 4) - that is, $0.81 \times 10^6$ V/m @ 6 GHz to $1.2 \times 10^6$ V/m @ 9 GHz. It should be noted that more recent standards have abandoned instantaneous electric field limits, finding that they have little biological rationale (\cite{C95}, B.4.3, {\it Rationale for pulsed RF field limits}).

\footnote{In our case, it seems to make the most sense to discuss the attenuation of the electric field, V/m, rather than the field power (also referred to as the {\it intensity}),\Wsqm, since the inactivation seems most closely related to the former. Penetration depth refers to the depth at which the power is reduced by 1/$e$; skin depth, where the electric field is 1/$e$. The penetration depth is simply half the skin depth.\cite{Penetration2019a}}

Considering the electric field skin depth of average lung tissue\cite{gabriel1996compilation} (which changes considerably based on )

\footnote{It should also be noted that at very specific pulsed regimes at millimeter wavelengths (far beyond our study) mere adherence to the letter of the law may not be sufficient for protection\cite{Limitations}.}



6 cm of l

The 

Fat and 

The penetration of microwave power is very sensitive to 

Conduction effects. Because the transmission line voltage only goes as the to the square root of the input power

There seems to be some sort of focusing effect which makes internal bronchoscopic treatment slightly more effective than might be expected.




My kingdom for a Bragg peak!

\clearpage
\printbibliography[heading=none, title={}, keyword={Flagship}]
\printbibliography[heading=none, title={}, keyword={standards}]


\end{document}