%!TeX root = experiment_1_data
\documentclass[paper.tex]{subfiles}
\begin{document}
	
\begin{table}[h!]
	\centering
	\begin{tabular}{ |c|c|c| } 
		\hline
		col1 & kilocounts per 10s integration \\
		\hline
		Autoclaved (0.2 mL) & 32 \\ 
		Autoclaved (50 uL, sham) & 30 \\ 
		Autoclaved (0.2 mL) & 29 \\ 
		PG1 'blank' & 8 \\ 
		Dist. water & 0.5 \\
		\hline
		\multicolumn{2}{|c|}{\ntilde 0.25 MV/m, 0.2 uL/s at 500 Hz  } \\
		\hline 

		& cell5  \\ 
		& Negative  \\ 
		\hline
	\end{tabular}
	

\end{table}

Data from testing on bacteriophage. Data rounded to the nearest thousand photon counts per 10 seconds. Not thoroughly characterized. 

Field strengths are very approximate. 

Background ('blank', in notes) readings measure the amount of DNA present in the phage culture tube before any treatment.

Repeatability between test conditions was not excellent. The source of the ~7000 count variation between backgrounds is not known.


The flurometer printed a constant stream of results, all within the random error between capture windows $\sigma=0.5$ kcounts. Only after most of the data was taken was it realized how dumb this was.



\end{document}