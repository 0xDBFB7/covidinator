



\title{Maxwell's Silver Hammer: On the application of pulsed microwave acoustic resonant viral inactivation
\thanks{I would be delighted to include any criticisms or comments anyone may have, both on substance and comprehensibility; preferably leave them on the GitHub issues page, or email therobotist@gmail.com, @0xDBFB7 on Twitter, or irc.0xDBFB7.com:6667 \#covid.}}
\date{May 2020}
\author{Based primarily on exceptional work by }


\begin{document}

\flushbottom 
\maketitle
\thispagestyle{empty}



\begin{textblock}{5}(1,1)
\noindent Please view the latest version at github.com/0xDBFB7/covidinator
\end{textblock}
%\begin{textblock}{5}(1,27)
%\end{textblock}

\null\begin{tabular}[t]{l@{}}
  {Daniel Correia}\ \orcidlink{0000-0002-9353-0216}  \\
  \textit{????????? York??????}
\end{tabular}



\begin{abstract}

We extend this landmark work rather trivially by:\\


Aims:

\begin{itemize}
  \item Establishing the time dependence of inactivation
  \item Demonstrating a modulation scheme that decreases the inactivation threshold to below current safety levels in surrogate bacteriophage
  \item Demonstrating a prototype emitter in an "electromagnetic mask" form-factor, costing about \$5 in prototype quantities, which can reasonably be produced in 10 million-of quantities
  \item Testing power thresholds in various conditions; biological fluids of various conductivities and pHs
  \item Using a coarse-grained molecular-dynamics simulation to optimize the impulse
  \item Using a virus-in-the-loop optimization with a centrifugal microfluidic system
  \item Discussing the biological basis for the safety of the device
  \item Showing that the deviation from the expected theshold could be explained by variance in the 
  
\end{itemize}

{Perhaps notable that a completely free and open-source toolchain was used for the entire project, including FDTD microwave antenna optimization. This work can be replicated with \$500 in equipment.}

\end{abstract}


\begin{autem}
	{\large  \it autem} \\
	
	This work was prepared by an undergraduate and has not been peer-reviewed. Additionally, the author has no prior experience with either biology or microwave design. \\

	While our study is very simple, biology is of such complexity that a tremendous amount of care must be taken before any conclusion can be drawn at all. It is not likely that we have taken sufficient care.\\ 

This is especially true with RF biophysics. The literature is littered with otherwise impeccably perormed research which was later demonstrated to be an artifact. \\

	Though the original research used Inf. A, our testing was only performed with a surrogate bacteriophage. All experiments must be repeated with SARS-NCoV-2. \\

	Other sterilization methods may produce superior results with less faffing about, and should be evaluated in the same context. Recent data on far UV [Buonanno 2017] indicate safety; and existing cold-plasma\\

It has occurred before that otherwise striking resonances were detected and replication found them to be artifacts of the measurement equipment. [Foster 1987] even offers this disclaimer:

\

"To detect the DNA resonances with the probe
technique requires correction for system errors that are
potentially much larger than the effect to be studied and
lead to resonance-like artifacts. This is true even with the
more precise instrumentation used in this study. Such data
are easily misinterpreted, and we suggest this might have
happened in the former study."

\

The direct inactivation plaque and PCR data from [Yang], replicated by [Sun], is a somewhat convincing corroboration that this effect is not an artifact, but hundreds of convincing and  wrong papers exist in the literature. Our data is crude and imprecise, our equipment hastily constructed, and should not be considered validation of this effect even existing. \\


	Please consider all claims with appropriate skepticism.

\end{autem}