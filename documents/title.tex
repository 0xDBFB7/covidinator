%!TeX root = title
\documentclass[paper.tex]{subfiles}
\begin{document}

%\title{Maxwell's Silver Hammer: Musings and measurements on pulsed microwave acoustic resonant viral inactivation 

%Data and discussion?

\title{Maxwell's Silver Hammer: Pulsed microwave viral envelope disruption\footnotemark\\or, some book-keeping matters of minor consequence in the application of microwave-induced lysis}
\date{First published May 2020}
\author{Based primarily on excellent work by Liu et al, Yang et al, Hung et al, and at least 800 other groups.\\
	\\
		By that measure, this paper is 0.125\% by \\
		\\
		\small{{Daniel Correia}\ \orcidlink{0000-0002-9353-0216}, \textit{????????? York??????,}\footnotemark}}

\footnotetext{{ Responsibility for accuracy and completeness must always fall on the author. That said, I would be delighted to hear any criticisms or comments anyone may have, both on substance and comprehensibility; preferably leave them on the GitHub issues page, or email therobotist@gmail.com, @0xDBFB7 on Twitter, or irc.0xDBFB7.com:6667 \#covid.}}

\footnotetext{Also affiliated with SafeSump Incorporated.}

\flushbottom 
\maketitle
\thispagestyle{empty}


\null\begin{tabular}[t]{l@{}}
	  \\
	
\end{tabular}


\begin{textblock}{10}(0.5,0.5)
\noindent This document may be updated regularly. Please view the latest version at \href{https://www.github.com/0xDBFB7/covidinator}{github.com/0xDBFB7/covidinator}. This is V0.0.1.
\end{textblock}
%\begin{textblock}{5}(1,27)
%\end{textblock}





\begin{abstract}




Previous studies appear to demonstrate that several species of viruses exhibit an anomalous relaxation time of $>$ y, as opposed to $<< x$\footnotemark seen in tissue structures.


Then, net charge, rings up and shatters, much like a singer might a wine glass. Such a non-thermal inactivation mechanism is quite unprecedented and extremely dubious, oft-posited but never substantiated; so it has taken some convincing that this is not a particularly abstruse artifact.\\


We extend this landmark work very trivially by apparently replicating the effect in a T4 coliphage surrogate, using both microwave feedback and a fluroimetric beta-galactosidase infectivity assay. We appear to determine that inactivation does not require 15 minutes, but is complete in less than 500 nanoseconds. This provides a headroom of some $10^6$ to all safety limits. 

\footnotetext{This value for human structures is the aggregate Cole-Cole relaxation time. This is an unfair comparison; the relaxation time of a heterogeneous mixture such as tissue often appears smaller than that of a pure sample. However, this also agrees with individual findings of a lack of resonance modes, so we are comfortable with this comparison}


Via FDTD simulations, using off-the-shelf equipment and modified non-invasive microwave diathermy techniques, it appears to be possible to inactivate all virions in the outer 2 cm of the lungs without any effect on the surrounding tissue. 

With a bit more effort, by harnessing the dispersive nature of tissue using modulated Brillouin pulses, it may be possible to remove the virus from the body altogether.


expelling its viral contents into solution and 

Not to be confused with pseudoscientific 'resonance therapies', nor reports of a connection with 5G communications.

This means that the technique is {\it selective}.






Aims: this is primarily bookkeeping.\\

Legend:  \cmark $ = $ crude experiment, not nearly definitive, needs more work $\vert$ $\thicksim$ $ = $ crude experiment, not nearly definitive, needs more work $\vert$ \xmark \ $ = $ definitely not complete.\\

\begin{itemize}
  \item Establishing the time dependence of inactivation in surrogate bacteriophage $\vert$ \cmark
  \item Demonstrating a modulation scheme that decreases the inactivation threshold to below current safety levels in surrogate bacteriophage $\vert$ \cmark
  \item Demonstrating a prototype emitter in an "electromagnetic mask" form-factor, costing about \$5 in prototype quantities, which can reasonably be produced in 10 million-of quantities $\vert$ $\thicksim$
  \item Testing power thresholds in various conditions; biological fluids of various conductivities and pHs $\vert$ \xmark
  \item Synthesizing a coarse-grained molecular-dynamics simulation of the mechanism $\vert$ \xmark
  \item Discussing the biological basis for the safety of the device $\vert$ \cmark
  \item Showing that the deviation from the expected threshold can be explained by variance in the viron, and that  $\vert$ \cmark
  \item Taking 4 months to design a circuit with 2 transistors, thereby potentially leading to the maiming of hundreds of thousands of people and causing untold economic losses $\vert$ \cmark
  
\end{itemize}\

With refinements, our inexpensive pulsed microwave spectrometer can obtain the required data with sufficiently concentrated specimens; but certain swept-EPR or ESR equipment may provide a much more sensitive spectrum \st{when used improperly} with modifications in technique. \\

It is perhaps notable that a completely free and open-source toolchain was used for the entire project.

\end{abstract}

\end{document}
