



\title{Maxwell's Silver Hammer: On the application of pulsed microwave acoustic resonant viral inactivation
\thanks{{\small I would be delighted to include any criticisms or comments anyone may have, both on substance and comprehensibility; preferably leave them on the GitHub issues page, or email therobotist@gmail.com, @0xDBFB7 on Twitter, or irc.0xDBFB7.com:6667 \#covid.}}}
\date{May 2020}
\author{Based primarily on exceptional work by }


\begin{document}

\flushbottom 
\maketitle
\thispagestyle{empty}



\begin{textblock}{5}(1,1)
\noindent Please view the latest version at github.com/0xDBFB7/covidinator \\This is V0.0.1
\end{textblock}
%\begin{textblock}{5}(1,27)
%\end{textblock}

\null\begin{tabular}[t]{l@{}}
  {Daniel Correia}\ \orcidlink{0000-0002-9353-0216}  \\
  \textit{????????? York??????,}
\end{tabular}



\begin{abstract}

We extend this landmark work rather trivially by:\\


Aims:

\begin{itemize}
  \item Establishing the time dependence of inactivation
  \item Demonstrating a modulation scheme that decreases the inactivation threshold to below current safety levels in surrogate bacteriophage
  \item Demonstrating a prototype emitter in an "electromagnetic mask" form-factor, costing about \$5 in prototype quantities, which can reasonably be produced in 10 million-of quantities
  \item Testing power thresholds in various conditions; biological fluids of various conductivities and pHs
  \item Using a coarse-grained molecular-dynamics simulation to optimize the impulse
  \item Using a virus-in-the-loop optimization with a centrifugal microfluidic system
  \item Discussing the biological basis for the safety of the device
  \item Showing that the deviation from the expected theshold could be explained by variance in the 
  
\end{itemize}

{Perhaps notable that a completely free and open-source toolchain was used for the entire project, including FDTD microwave antenna optimization. This work can be replicated with \$500 in equipment.}

\end{abstract}

