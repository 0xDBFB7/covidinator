%!TeX root = title
\documentclass[paper.tex]{subfiles}
\begin{document}

%\title{Maxwell's Silver Hammer: Musings and measurements on pulsed microwave acoustic resonant viral inactivation 

%Data and discussion?

% Maxwell's Silver Hammer: Pulsed microwave viral envelope disruption\footnotemark\\or, some book-keeping matters of minor consequence in the application of microwave-induced lysis\\or 

\title{On pulsed microwave viral envelope disruption}

%\subtitle{A shadowy flight into the dangerous world of a viral inactivation mechanism which should not exist.}
\date{}
\author{Based primarily on excellent work by the Sun et al group and at least 800 other groups.\\
	\\
		By that measure, this paper is 0.125\% by \\
		\\
		\small{{Daniel Correia}\ \orcidlink{0000-0002-9353-0216}\footnotemark}}

\footnotetext{{I would be delighted to hear any criticisms anyone may have, both on substance and comprehensibility; preferably leave them on the GitHub issues page, or email therobotist@gmail.com, @0xDBFB7 on Twitter, or irc.0xDBFB7.com:6667 \#covid.}}

%\footnotetext{Also affiliated with SafeSump Incorporated.}

\flushbottom 
\maketitle
\thispagestyle{empty}

\begin{abstract}
	Work by the Sun group (Liu 2009, Yang 2015, Sun 2017) and (Burkhartsmeyer 2020) suggests that certain species of viruses may have remarkable dielectric properties which are not found in other biological systems. More importantly, they appear to demonstrate that, due to this effect, high power densities in the X-band can tear the envelope, inactivating the virus. \\
	
	We have attempted to replicate with a slightly improved thermal sham on Bacteriophage T4 in the 6 - 11.5 GHz band with Xie's slick amplification-free fluorescence assay. T4 does not appear to exhibit this effect, but this is to be expected as T4's proteinaceous capsid is $\approx 5 \times$ stronger than the lipid envelope of Inf. A et al. T4 is not a good surrogate.\\
	
	 We therefore contribute nothing to the scientific record; most of our discussion is not novel and is primarily qualitative.\\

	 Experiments in the field of microwave biophysics pose peculiar challenges in obtaining a meaningful result. In the process of performing a literature review for this paper, a variety of existential crises regarding we lost confidence in the concept of reliable measurement itself.\\
	 
	 We therefore have some vague reservations regarding aspects of the previous research and recommend great care in interpretation. With some imagination, the results can be ascribed purely to characteristic artifacts in this field. We also believe there may be inconsistencies in the observed protein charge and coupling by many orders of magnitude. However, unlike historical reports, these effects do not appear to be ruled out by basic classical physics.\\
	 
	  If this effect exists, and inactivation is found to take place in less than$~10^6$ cycles, it appears to offer a straightforward treatment for SARS-NCoV-2, for which sufficient evidence of safety exists, particularly in the sub-microsecond pulse regime, and especially when driven by Brillouin precursor trains as generated by some GaAs NLTL comb generators.
	  
	  
	  Sufficient high-quality evidence does not appear to exist to definitively confirm or refute this effect.
\end{abstract}


\null\begin{tabular}[t]{l@{}}
	  \\
	
\end{tabular}


\begin{textblock}{10}(5,0.5)
\noindent Data and the garbage codebase produced during this research are available at \href{https://www.github.com/0xDBFB7/covidinator}{github.com/0xDBFB7/covidinator}. This document may be updated as information changes. Please view the latest version. This is V0.0.1.
\end{textblock}
%\begin{textblock}{5}(1,27)
%\end{textblock}

\begin{quotation}\



Given that every route persued consumes precious resources and attention that might be more effective in 
other routes...



Based on artifacts previously observed in the nonthermal effects literature, we can create scenarios in which all of the methods are simultaneously wrong. 

\section{Aims}




Legend:  \cmark $ = $ crude experiment, not nearly definitive, needs more work $\vert$ $\thicksim$ $ = $ crude experiment, not nearly definitive, needs more work $\vert$ \xmark \ $ = $ definitely not complete.\\

\begin{itemize}
  \item Establishing the time dependence of inactivation in surrogate bacteriophage $\vert$ \cmark
  \item Demonstrating a modulation scheme that decreases the inactivation threshold to below current safety levels in surrogate bacteriophage $\vert$ \cmark
  \item Demonstrating a prototype emitter in an "electromagnetic mask" form-factor, costing about \$5 in prototype quantities, which can reasonably be produced in 10 million-of quantities $\vert$ $\thicksim$
  \item Testing power thresholds in various conditions; biological fluids of various conductivities and pHs $\vert$ \xmark
  \item Synthesizing a coarse-grained molecular-dynamics simulation of the mechanism $\vert$ \xmark
  \item Discussing the biological basis for the safety of the device $\vert$ \cmark
  \item Showing that the deviation from the expected threshold can be explained by variance in the viron $\vert$ \cmark
  
\end{itemize}\


\tableofcontents


It is perhaps notable that a completely free and open-source toolchain was used for the entire project (with one exception).

\end{quotation}

\end{document}
