%!TeX root = title
\documentclass[paper.tex]{subfiles}
\begin{document}

%\title{Maxwell's Silver Hammer: Musings and measurements on pulsed microwave acoustic resonant viral inactivation 

%Data and discussion?

% Maxwell's Silver Hammer: Pulsed microwave viral envelope disruption\footnotemark\\or, some book-keeping matters of minor consequence in the application of microwave-induced lysis\\or 
% On pulsed microwave viral envelope disruption

% electropermeabilization 
% irreversible electroporation

% A haphazard exploration of the plausiblity of dispersive shaped-pulse electropermeabilization of the viral envelope
% A haphazard traipse through pulsed viral envelope disruption
%  the viral envelope via shaped dispersive pulse
% On the sub-nanosecond electropermeabilization of viruses
\title{Some notes and experiments on electropermeabilization of viral membranes}
%\subtitle{A shadowy flight into the dangerous world of a viral inactivation mechanism which should not exist.}
\date{}

%\footnotetext{Also affiliated with SafeSump Incorporated.}

\flushbottom 
\maketitle
\thispagestyle{empty}

\renewcommand{\abstractname}{Summary}    % clear the title

\begin{abstract}
	
	
	\footnote{Blame \small{{Daniel Correia}\ \orcidlink{0000-0002-9353-0216}}}
	
	\footnote{{I would be delighted to hear any criticisms anyone may have, both on substance and comprehensibility; preferably leave them on the GitHub issues page, or email therobotist@gmail.com, @0xDBFB7 on Twitter, or irc.0xDBFB7.com:6667 \#covid.}}
	
	It has previously been suggested that viruses may possess interesting dielectric properties; in particular, a substantial charge distribution and a relatively long relaxation time. 
	
	If the duration of exposure is sufficiently short that thermal and membrane breakdown effects can be neglected, tissue appears to withstand peak electric fields of great magnitude - above 1 megavolt/meter, even in in-vivo studies - without severe acute effects.
	
	with at our disposal, is there produce mechanical forces sufficient to disrupt the without acutely damaging the surrounding tissue.
	
	thoroughly inconclusive. 
	
	There is no reason to believe that the required field is not impractically high.
	
	physical basis is extremely tenuous at best
	
	In a series of experiments on bacteriophage T4, we have been unable to replicate previous research demonstrating the release of the genome after exposure to pulses at $0.7 \times 10^6 V/m \pm ~0.4 \times 10^6 V/m$.
	
	
	
	amplification-free fluorescence assay with sub-nanogram sensitivity; 
	
	Our use of un-characterized equipment 
	
	Work by the Sun group (Liu 2009, Yang 2015, Sun 2017) and (Dykeman 2009) and 
	(Burkhartsmeyer 2020) suggests that certain species of viruses may have remarkable dielectric 
	properties which are not found in other biological systems. More importantly, they appear to 
	demonstrate that, due to this effect, high power densities in the X-band can tear the envelope, 
	inactivating the virus. \\
	
	We have attempted to replicate with a slightly improved thermal sham on Bacteriophage T4 in the 6 - 11.5 GHz band with Xu's slick amplification-free fluorescence assay. T4 does not appear to exhibit this effect, but this is to be expected as T4's protein capsid is $\approx 5 \times$ more resilient than the lipid envelope of Inf. A et al. T4 is not a good surrogate.\\
	
	 We therefore contribute nothing to the scientific record; most of our discussion is not novel and is primarily qualitative.\\

	 Experiments in the field of microwave biophysics pose peculiar challenges in obtaining a meaningful result. In the process of performing a literature review for this paper, a variety of existential crises regarding we lost confidence in the concept of reliable measurement itself.\\
	 
	 Because this is a TRL 0 technique, 
	 

	 
	 We therefore have some vague reservations regarding aspects of the previous research and recommend great care in interpretation. With some imagination, the results can be ascribed purely to characteristic artifacts in this field. We also believe there may be inconsistencies in the observed protein charge and coupling by many orders of magnitude. However, unlike historical reports, these effects do not appear to be ruled out by basic classical physics.\\
	 
	 Adjacent to a field of pathological science. There is little reason to expect any of the mechanisms to exist.
	 
	 In the context of the literature, any discussion of non-thermal effects merits extraordinary care. Every other alternative explanation should be considered.
	 
	It seems important to note that most published papers in this field are wrong. There is no reason to believe that this is any different.
	
	Sufficient high-quality evidence does not appear to exist to definitively confirm or refute 
	this effect.
	
	There does not appear to be sufficient clinical evidence to determine safety. What evidence does exist in this regime suggests few acute symptoms, but carcinogenicity as a possible long-term side-effect. 
	
	As almost a dozen clinical trials using ionizing radiation therapies are being undertaken, possible carcinogenicity of treatment does not seem to be a great impediment. \cite{LowDose2020}

	Is sensitive to parameters for which we do not have even an order-of-magnitude guess.

	Unduly preoccupied with matters of instrumentation and methodology - things which have nothing to do with viruses; and so no base truths have been arrived at. 
	
	muster a field of about 100 kV/m. 
	
	there are fundamental gaps in understanding
	amateurish
	
	
	Telling that we only the easiest problems from each field. It concerns microbiological technique, but only to the extent of Delbruck; it concerns coherent control, but with only purely classical assumptions; microwave engineering, but without the difficult low-noise bits.
	
	400 times as effective at penetrating tissue. 
	
	TRL 0 and no resources should be redirected.
	
	This project has been an abject failure in many interesting and notable ways, particularly in redirecting effort and resources from more plausible mechanisms - or even the local community - towards some half-baked pseudoscientific nonsense. 
	
	None of what we say will be of any interest to the experienced investigator in this field.
	
	We have worked too slowly to be of any use to the apocalypse. 
	
	If they are not careful, a researcher in this field can easily succumb to crack-pottery. It is difficult to tell whether this has already happened to us.
	
	
	One interesting result is that the minimum-amplitude pulse to drive an oscillator through a lossy dispersive medium appears to be a simple sawtooth.
	
\end{abstract}


\null\begin{tabular}[t]{l@{}}
	  \\
	
\end{tabular}


%\begin{textblock*}{\textwidth}(5,0.5)
%\noindent Data and the garbage codebase produced during this research are available at 
%\href{https://www.github.com/0xDBFB7/covidinator}{github.com/0xDBFB7/covidinator}. This document 
%may be updated as information changes. Please view the latest version. 
%\end{textblock*}
%\begin{textblock}{5}(1,27)
%\end{textblock}

%\begin{quotation}\

H. H. Williams: Furious activity is no substitute for understanding.

Given that every route persued consumes precious resources and attention that might be more effective in 
other routes...



Based on artifacts previously observed in the nonthermal effects literature, we can create scenarios in which all of the methods are simultaneously wrong. 





\begin{autem}
	
	{\Large {This has \textbf{nothing to do} with 5G communications.}}\\
	
	I am comfortable making such a definitive statement.
	
	The power levels required to even begin to observe this inactivation effect in practice are so high that, if conditions are not specifically tailored for clinical use, extensive burns will result within seconds. Even in this case, the effect is to eviscerate the virus, not to aid its spread. 
	
	Moreover, these frequency bands have been in almost continuous use not long after Jagadish Chandra Bose 1894; the world has been bathed in high-power 10 GHz pulses at least since the commissioning of the first centimeter-wave weather radars in 1959. 
	
	Countless far more prosaic explanations exist in the modern world, such as air travel.
	
	If you have any concerns regarding this statement, please contact me. I would be happy to discuss this.
	
\end{autem}



%
%Legend:  \cmark $ = $ crude experiment, not nearly definitive, needs more work $\vert$ $\thicksim$ $ = $ crude experiment, not nearly definitive, needs more work $\vert$ \xmark \ $ = $ definitely not complete.\\
%
%\begin{itemize}
%  \item Establishing the time dependence of inactivation in surrogate bacteriophage $\vert$ \cmark
%  \item Demonstrating a modulation scheme that decreases the inactivation threshold to below current safety levels in surrogate bacteriophage $\vert$ \cmark
%  \item Demonstrating a prototype emitter in an "electromagnetic mask" form-factor, costing about \$5 in prototype quantities, which can reasonably be produced in 10 million-of quantities $\vert$ $\thicksim$
%  \item Testing power thresholds in various conditions; biological fluids of various conductivities and pHs $\vert$ \xmark
%  \item Synthesizing a coarse-grained molecular-dynamics simulation of the mechanism $\vert$ \xmark
%  \item Discussing the biological basis for the safety of the device $\vert$ \cmark
%  \item Showing that the deviation from the expected threshold can be explained by variance in the viron $\vert$ \cmark
%  
%\end{itemize}\
%

\tableofcontents


It is perhaps notable that a completely free and open-source toolchain was used for the entire project (with one exception).

\end{document}
