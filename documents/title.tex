%!TeX root = title
\documentclass[paper.tex]{subfiles}
\begin{document}

%\title{Maxwell's Silver Hammer: Musings and measurements on pulsed microwave acoustic resonant viral inactivation 

%Data and discussion?

% Maxwell's Silver Hammer: Pulsed microwave viral envelope disruption\footnotemark\\or, some book-keeping matters of minor consequence in the application of microwave-induced lysis\\or 

\title{A shadowy flight into the dangerous world of a viral inactivation mechanism which should not exist.}
\date{First published May 2020}
\author{Based primarily on excellent work by Liu et al, Yang et al, Hung et al, and at least 800 other groups.\\
	\\
		By that measure, this paper is 0.125\% by \\
		\\
		\small{{Daniel Correia}\ \orcidlink{0000-0002-9353-0216}, \textit{TODO: get York affiliation}\footnotemark}}

\footnotetext{{ Responsibility for accuracy and completeness must always fall on the author. That said, I would be delighted to hear any criticisms or comments anyone may have, both on substance and comprehensibility; preferably leave them on the GitHub issues page, or email therobotist@gmail.com, @0xDBFB7 on Twitter, or irc.0xDBFB7.com:6667 \#covid.}}

\footnotetext{Also affiliated with SafeSump Incorporated.}

\flushbottom 
\maketitle
\thispagestyle{empty}


\null\begin{tabular}[t]{l@{}}
	  \\
	
\end{tabular}


\begin{textblock}{10}(5,0.5)
\noindent Data and the garbage codebase produced during this research are available at \\ \href{https://www.github.com/0xDBFB7/covidinator}{github.com/0xDBFB7/covidinator}. This document may be updated as information changes. Please view the latest version. This is V0.0.1.
\end{textblock}
%\begin{textblock}{5}(1,27)
%\end{textblock}

\begin{quotation} % abstract requires re-writing an introduction. Screw that.

Previous works (Liu et al, ) appear to demonstrate that several species of viruses exhibit an anomalous dielectric relaxation time of $> 75 \text{ ps}$, as opposed to $< 20 \text{ ps}$ \footnotemark measured in the structures of human tissue.\\


Other studies (Yang et al, Hung et al, Sun et al) then appear to demonstrate experimentally that - due to this effect and the favorable combination of envelope and genome net charge, envelope and capsid yield stress, mass distribution, and overall stiffness - certain viruses can  apparently be made to ring up, leading their envelope to shatter, when exposed to intense but ostensibly safe CW tones in the X-band - very much like a singer might shatter a wine glass.\\

Such a non-thermal mechanism is quite unprecedented and extremely dubious, oft-posited but never substantiated; so it has taken some convincing that this is not a particularly abstruse artifact, especially when considered in the context of the broader non-thermal effects literature. For instance, the very paper (Edwards) which spurred development of the hydration-layer damping hypothesis used to explain these results, is has been quite\cite{Resonances1987} definitively\cite{Microwave1993a} shown to be an artifact, one which we unfortunately have not excluded in our study.\footnotemark \ In fact, even in light of the fine evidence that the other groups have collected, and the poor-quality data we have collected, we are still not entirely convinced that this effect exists - but perhaps this skepticism is unjustified.\\

\footnotetext{we should verify, based on simple solvated sphere damping, whether the damping is of the right order of magnitude}

We contribute almost nothing to the record (much of our discussion has already been covered by []), except by apparently replicating the effect in a T4 bacteriophage surrogate, using both microwave feedback and a luminometric beta-galactosidase infectivity assay. We appear to demonstrate that inactivation does not require 15 minutes, but is complete in less than 500 nanoseconds. This provides a headroom of some $10^6$ to all safety limits imposed by all known harm mechanisms, which can be traded for depth in tissue or distance in space. \\

This technique appears to be unique in that it is selective, save for Far UV or cold-plasma sterilization. Its advantages over those is that it operates over large volumes with inexpensive emitters, but, most critically, can be made to penetrate tissue without harm.

We briefly review the biological basis for the safety. There is solid in vitro and limited in vivo evidence of safety in this regime.

\footnotetext{This value for human structures is the aggregate Cole-Cole relaxation time from Gabriel, in the case of breast tissue (which has the highest value). This is somewhat an unfair comparison; the relaxation time of a heterogeneous mixture such as tissue often appears smaller than that of a pure sample. However, this value agrees well with the individual findings of cellular modes (for instance, the damping from Adair's estimates is approx. 10 ps)}

We briefly examine three applications. All depend significantly on what the microwave spectrum of SARS-NCoV-2 ends up being, which must be measured - or possibly reconstructed from existing whole-virion data or determined from coarse-grained MD simulation. 

With refinements, our inexpensive pulsed microwave spectrometer can obtain the required absorption and threshold data with sufficiently concentrated specimens; but proper dielectric spectroscopy labs (or, in a pinch, swept-EPR/ESR equipment \st{when used improperly}) will provide a much more detailed spectrum. A slightly safer low-titer primary specimen may be usable to avoid BSL-3 requirements. \\



First, free-space techniques are easily implemented to prevent respiratory transmission. A \$10 emitter appears to suffice on a personal level, and \$100 systems for room-scale.\\
 
%
Second, via cursory FDTD simulations, using off-the-shelf equipment and modified non-invasive microwave diathermy techniques, it appears to be possible to inactivate all virions in the outer 2 cm of the lungs and inner 1 cm of the bronchi without any effect on the surrounding tissue. 
\\

[] notes that with a bit more effort, by harnessing the dispersive nature of tissue using modulated Brillouin precursors, it may be possible to remove the virus from the body altogether.\\

The mechanism also provides a few distinctive microwave signatures that may be useful for detection; it remains to be seen whether these can be discerned in practice.

The claims above are extraordinary. We do not provide the extraordinary quality of evidence required to support them. Consider: is it more likely that an undergraduate in his parent's basement, with no experience in biology or microwave design, has misinterpreted a blip from a cobbled-together machine, or that 

expelling its viral contents into solution and 

Not to be confused with pseudoscientific 'resonance therapies', nor reports of a connection with 5G communications.

This means that the technique is {\it selective}.





\section{Aims}


Legend:  \cmark $ = $ crude experiment, not nearly definitive, needs more work $\vert$ $\thicksim$ $ = $ crude experiment, not nearly definitive, needs more work $\vert$ \xmark \ $ = $ definitely not complete.\\

\begin{itemize}
  \item Establishing the time dependence of inactivation in surrogate bacteriophage $\vert$ \cmark
  \item Demonstrating a modulation scheme that decreases the inactivation threshold to below current safety levels in surrogate bacteriophage $\vert$ \cmark
  \item Demonstrating a prototype emitter in an "electromagnetic mask" form-factor, costing about \$5 in prototype quantities, which can reasonably be produced in 10 million-of quantities $\vert$ $\thicksim$
  \item Testing power thresholds in various conditions; biological fluids of various conductivities and pHs $\vert$ \xmark
  \item Synthesizing a coarse-grained molecular-dynamics simulation of the mechanism $\vert$ \xmark
  \item Discussing the biological basis for the safety of the device $\vert$ \cmark
  \item Showing that the deviation from the expected threshold can be explained by variance in the viron, and that  $\vert$ \cmark
  \item Taking 4 months to design a circuit with 2 transistors, thereby potentially leading to the maiming of hundreds of thousands of people and causing untold economic losses $\vert$ \cmark
  
\end{itemize}\


\tableofcontents


It is perhaps notable that a completely free and open-source toolchain was used for the entire project (with one exception).

\end{quotation}

\end{document}
