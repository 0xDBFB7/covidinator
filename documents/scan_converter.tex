%!TeX root = scan_converter
\documentclass[paper.tex]{subfiles}
\begin{document}

\begin{autem}
This is an example of how the course of action can be distorted by engineering. In a non-pandemic scenario such a diversion might have been a reasonable. Indefensible. I am consoled that it did not account for more than 2 to 3 weeks of work.
\end{autem}


\begin{autem}
	This section is deliberately kept vague and without reference to construction details to avoid export restrictions. In any case, it is hard to imagine a technique fully developed and published in the 1940s would be of great secrecy now. 
\end{autem}

Because of the small sizes of viruses, all interesting time-domain spectroscopy results require resolving signals with greater than 10 GHz bandwidth.

So called "equivalent-time" or "sampling" oscilloscopes are less expensive; but it was hoped that it would be possible to resolve the breaking of the capsid - a transient, single-shot phenomenon -   A digital oscilloscope capable of real-time sampling such a transient is a tall order, and it did not appear that an undergraduate would be allowed access.

Gating time. Photoconductive gating is a particularly interesting technique for this.

There is also the concern that, when measuring a system producing kilovolt pulses using an instrument with an extremely electrically delicate 1V maximum input, some harm might come to a machine worth more than the experimenter. 

This turned out to be not unfounded - the kilovolt pulser used was observed to cause the RPi controlling the experiment to 

However, transient recordings of sub-nanosecond phenomena were made as early as 1938 by Manfred Von Ardenne using a "micro-oscillograph". By 1946, such oscillographs - now with bandwidths in excess of 10 GHz - were readily available commercially\cite{3beam1946}, and by 1949 found extensive use for quantitative measurements of ultrafast phenomena by Fletcher\cite{Production1949} and others. 

"Direct access" tubes to the logical conclusion. The design of Fletcher puts the cathode ray deflection plates essentially inside the.

The main practical limitations on oscilloscope sweep speed are discussed by Mackay\cite{New1948}, and are imposed by the Y axis deflection plate electron transit time and bandwidth (the X-axis speed is of less importance), and the beam current and sensitivity of the phosphor (faster sweeps reduce the electron flux and brightness of the dot).


\begin{figure}[H]
	%	\makebox[\textwidth][c]{
	\centering
	\subfloat[Damage to the sensor, believed to be due to. ]{
	\includegraphics[width=0.3\textwidth]{e_beam_damage_2}
	
}
	\hfill
	\subfloat[]{
		\includegraphics[width=0.3\textwidth]{e_beam_damage}
		
	}
		\hfill

	\subfloat[Direct electron beam detection]{
	\includegraphics[width=0.3\textwidth]{direct_electron_beam}
}
	\caption{Second.}
		\hfill

\end{figure}



\begin{figure}[H]
	\captionsetup{singlelinecheck = false, justification=justified}
	\centering
	
	\caption{}
\end{figure}


\begin{figure}[H]
	\captionsetup{singlelinecheck = false, justification=justified}
	\centering
	
	\caption{}
\end{figure}


The very high voltage actually makes this much easier

means that low-sensitivity diagnostics, like lithium niobate optical probes. Not having a suitable optical bench, an electron-beam probe, similar to a streak camera, itself similar in principle to scan-conversion digitizer tubes in Tek SCD and 7912 series transient digitizers.

If a low voltage sensitivity is required, and because modern high-pixel-density CMOS sensors allow the required deflection (and therefore the entire structure) to be minimized, these are simple to design and inexpensive (much of the complexity is in the slow-wave or travelling-wave structure which is usually required to match the signal velocity to the - slowing down light without affecting its properties or drawing power is an extremely difficult problem) and can offer sample rates in the terasamples/s and analog bandwidths of many dozens of GHz, albeit with very low vertical resolution. 

original von Ardenne 


\begin{fquote}[Livermore Operations][\cite{Scan2008}]
An internal CCD recorder would provide high sensitivity to the swept electron beam, but a CCD in a sealed vacuum tube has never been attempted or demonstrated in the phototube facility at Livermore Operations (LO). A prototype tube would most likely be built with a phosphor screen.
\end{fquote}


Possible routes:

make the plates super small like Von Ardenne (thin wires). see the papers on thin wires.

make the plates slow-wave structures matched to the speed of the electron; travelling-wave deflection plates.
unfortunately, travelling wave structures necessarily absorb power and affect the impedance of the circuit in question


multiple beams with different path lengths to have different phases?

put the x-axis at a 45 degree angle?


radiation damage to sensor 

Possibly some optical method. 

however, recording such transients is quite a solved problem, having been accomplished to the required degree 

taking "direct access" tube to its logical conclusion.

since the sensor will inevitably outgas, a sealed tube is impractical.


The distortion due to transit-time phasing is

$$ \frac{\sin(\frac{2\pi F \tau}{2})}{(2\pi F \tau)/2} $$

where $\tau$ is the transit time of the electron through the deflection region 

Image intensifier microchannel plates where angled holes cause secondary electron 

\end{document}