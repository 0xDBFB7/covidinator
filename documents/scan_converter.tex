


The very high voltage means that low-sensitivity diagnostics, like lithium niobate optical probes. Not having a suitable optical bench, an electron-beam probe, similar to a streak camera, itself similar in principle to scan-conversion digitizer tubes in Tek SCD and 7912 series transient digitizers.

If a low voltage sensitivity is required, and because modern high-pixel-density CMOS sensors allow the required deflection and the entire structure to be minimized, these are simple to design and inexpensive (much of the complexity being in the slow-wave or travelling-wave structure which is usually required to match the signal velocity to the - slowing down light without affecting its properties or drawing power is an extremely difficult problem) and can offer sample rates in the terasamples/s and analog bandwidths , albeit with low vertical resolution. 

original von Ardenne 



We would like real-time rather than equivalent-time sampling, 


Possible routes:

make the plates super small like Von Ardenne (thin wires). see the papers on thin wires.

make the plates slow-wave structures matched to the speed of the electron; travelling-wave deflection plates.
unfortunately, travelling wave structures necessarily absorb power and affect the impedance of the circuit in question


multiple beams with different path lengths to have different phases?

put the x-axis at a 45 degree angle?


radiation damage to sensor 

A digital oscilloscope capable of real-time sampling such a transient is no small feat, and it was deemed unlikely that we would be 
able to obtain access, There is also the concern that, when working with a system producing kilovolt 
pulses and an input of 5v, some harm might come to a machine worth more than the experimenter.

Possibly some optical method. 

however, recording such transients is quite a solved problem, having been accomplished to the required degree 

taking "direct access" tube to its logical conclusion.

since the sensor will inevitably outgas, a sealed tube is impractical.


The distortion due to transit-time phasing is

$$ \frac{\sin(\frac{2\pi F \tau}{2})}{(2\pi F \tau)/2} $$

where \tau is the transit time of the electron through the deflection region 

Image intensifier microchannel plates where angled holes cause secondary electron 

