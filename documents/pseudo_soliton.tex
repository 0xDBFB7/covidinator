%!TeX root = pseudo_soliton
\documentclass[paper.tex]{subfiles}
\begin{document}




Precursor waves do not appear to be explicitly mentioned in C95- but are implicitly governed by the Fourier-spectrum clauses. Ultra-short wavepackets of a similar nature are widely used in MRI systems with a repetition rate up to 300 MHz. We can make some mechanistic claims.

At 9 GHz, assuming that a torso is about 10 penetration depths in radius, implementing this would decrease the required surface field for complete eradication of the virus from the body to only 1 kV/m, rather than 6.6 MV/m ($1/\sqrt{10}=0.316$ rather than $e^{-10}=4.5e-5$). 





There are some extremely tricky subtleties in the simulation. See \cite{Comments1993}. Frankly, I can't make heads or tails of any of these. 

As saliently noted by Adair:

\begin{quote}
	“the energy of the pulse at a depth of 1 cm is about 0.8% of the initial energy;
	the maximum electric field strength, E is reduced by a factor of about 12, and the maximum rate-ofchange of the field, dE/dt, is reduced by a factor of 20. Every characteristic of the pulse at depth is
	contained, albeit attenuated, in the initial pulse. Consequently, we do not find it credible that the
	strongly attenuated pulse – albeit with a different shape – can induce biological effects beyond that of
	the initial pulse.”
\end{quote}


The 2018 NATO Research Task Group 189 report \cite{treatyelectromagnetic} is an excellent source of data for these. 

\begin{quote}
	Other basic bioeffects research [50], [52], [53], [66], [81], [83] has failed to provide support for Albanese’s multiple	theoretical postulates.
\end{quote}

The real killer 





temporal Soliton-producing nonlinear transmission lines, and NLTL oscillators constructed therefrom, particularly well suited 




\footnotemark

\footnotetext{Some of the initial work on these precursors (or "forerunners", as they were known at the time) apparently neglected some of the higher-order terms, leading to the conclusion that the signals were "very weak" (Sommerfield). However, the amplitude is often greater than that of the carrier.}

\footnotemark
\footnotetext{Honestly, how cool is this? Like, this unimportant factor (as simple as the rise time!) connects so deeply with Schrodinger's wave-packets - the very same functions that apply for probability distributions - and produces a result so striking - a complete change }



The "asymptotic method", and the fourier series method. Unfortunately, the fourier coefficients and phases provide little intuitive sense for how different input pulses propagate. 

%The Somerfield precursor is high-frequency, 





\end{document}
