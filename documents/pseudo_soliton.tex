%!TeX root = pseudo_soliton
\documentclass[paper.tex]{subfiles}
\begin{document}


Uzunoglu \cite{Theoretical2020} cite a very handy way to circumvent this lossy-flesh problem: the Brilloin and Somerfield precursors.



At 9 GHz, assuming that a torso is about 10 penetration depths in radius and the brilluoin train drives the virus with an equally, implementing this would decrease the surface field required for complete eradication of the virus from the body to a perfectly practicable 1 kV/m, rather than an air-ionizing 6.6 MV/m ($\frac{1}{\sqrt{10}}=0.316$ versus $e^{-10}=4.5e-5$). 



Precursor waves do not appear to be explicitly mentioned in IEEE standards, but are implicitly governed by the Fourier-spectrum clauses and more recent FCC ultrawideband regulations. Ultra-short wavepackets of a similar nature are widely used in MRI systems with a repetition rate up to 300 MHz. We can make some mechanistic claims.


\footnotetext{
	Sincere thanks go to the legendary Kurt Oughstun for taking the time to thoroughly answer some of our questions.
}

The prototypical Vircator (not varactor) pinched electron beam tubes 

step recovery diodes

\begin{sidenote}
There are some extremely tricky subtleties in the simulation.  \cite{propagation1992}. In \cite{Comments1993}, Leubbers says FDTD and FFT both are better than Asympt, Ogh says no there's no problem, Asymptotic is really accurate. Frankly, I can't make heads or tails of any of these, not understanding the asymptotic method at all.
\end{sidenote}
\begin{tab}
	content
\end{tab}

The 2018 NATO Research Task Group 189 report \cite{treatyelectromagnetic} is an excellent source of data for these. 

\begin{quote}
	Other basic bioeffects research [50], [52], [53], [66], [81], [83] has failed to provide support for Albanese’s multiple	theoretical postulates.
\end{quote}

The real killer 





temporal Soliton-producing nonlinear transmission lines, and NLTL oscillators constructed therefrom, particularly well suited 




\footnotemark

\footnotetext{Some of the initial work on these precursors (or "forerunners", as they were known at the time) apparently neglected some of the higher-order terms, leading to the conclusion that the signals were "very weak" (Sommerfield). However, the amplitude is often greater than that of the carrier.}

\footnotemark
\footnotetext{Honestly, how cool is this? Like, this unimportant factor (as simple as the rise time!) connects so deeply with Schrodinger's wave-packets - the very same functions that apply for probability distributions - and produces a result so striking - a complete change }



The "asymptotic method", and the fourier series method. Unfortunately, the fourier coefficients and phases provide little intuitive sense for how different input pulses propagate. 

%The Somerfield precursor is high-frequency, 





\end{document}
