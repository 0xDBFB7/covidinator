%!TeX root = introduction
\documentclass[paper.tex]{subfiles}
\begin{document}


\flushbottom 
\thispagestyle{empty}



\begin{autem}
	{\large  \it autem} \\
	
	This work was prepared by an undergraduate and has not been peer-reviewed. Additionally, the author has no prior experience with either biology or microwave design. \\

	While our study is very simple, biological research is of such complexity that a tremendous amount of care must be taken before any conclusion can be drawn at all. It is not likely that we have taken sufficient care.\\ 

This is especially true with RF biophysics. The literature is littered with otherwise impeccably perormed research which is both difficult to find fault in, and faulty. \\

This paper does not contribute to the state of the art at all, except perhaps in the obvious avenue of time-domain modulation; [Hung 2014] have already demonstrated a reflectarray, and the rest of our discussion is anything but novel.

Though the original research used Inf. A, our testing was only performed with a surrogate bacteriophage. All experiments must be repeated with SARS-NCoV-2. \\


Accurate microwave measurements are rife with parasitics and sensitivities. It has occurred before that otherwise striking resonances were detected, and further replication found them to be artifacts of the measurement equipment. [Foster 1987] even offers this disclaimer:\\

\begin{quote}

"To detect the DNA resonances with the probe
technique requires correction for system errors that are
potentially much larger than the effect to be studied and
lead to resonance-like artifacts. This is true even with the
more precise instrumentation used in this study. Such data
are easily misinterpreted".

\end{quote}

The direct inactivation plaque and PCR data in the referenced papers are a somewhat convincing corroboration that this effect is not a thermal artifact; but hundreds of convincing and wrong papers exist in the literature.\\

We discuss some possible artifacts in the supplemental, but nothing seems plausible. It should be noted that all three papers that reference this technique use similar methods to determine the resonance mode.\\

Then again, our incredulity is itself probably unfounded. The data these papers provide are excellent; they passed peer-review in Nature Srep, which is a far closer scrutiny than we could possibly provide.

Our data is crude and imprecise, our equipment hastily constructed, and should not be considered validation of this effect even existing. \\

Even if this effect exists and is as effective as it appears, it could still be impractical to apply for any number of reasons.\\


In essence, please consider all claims with appropriate skepticism.

\end{autem}

\end{document}