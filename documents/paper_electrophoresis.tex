%!TeX root = paper_electrophoresis
\documentclass[paper.tex]{subfiles}
\begin{document}


\subsection{Paper electrophoresis}

There's a very interesting which has almost completely fallen into antiquity, and yet not obviously for any reason. Filter paper electrophoresis. It requires less work to set up than gel electrophoresis. because I believe it partially relies on a different mechanism. When power is applied to the filter paper baths, the buffer visibly pumps through. \cite{ELECTROPHORESIS1951} Electroosmotic flow 40 to 100 volts works fine, no gel needs to be poured beforehand.

I've heard that this is not effective for DNA et al, only for proteins or amino acids.


\cite{gelelectrophoresis}
\begin{fquote}
	While paper and other solid support materials proved to be an advantage over free solutions for the electrophoretic analysis of biomolecules, gels were adopted later because gels not only minimized diffusion better than paper supports they actually participated in the separation process by interacting with the migrating particles. Gels can be thought of as semi-solid matrices whose pore sizes aid in separation. The semisolid nature of the gel participates through a process known as molecular sieving. The three common media for gel electrophoresis are starch, polyacrylamide, and agarose. Of these, the starch gel medium is the least versatile whereas a wide range of separation effects can be achieved using the other two media. There are limitations to the use of both polyacrylamide and agarose but these are effectively minimized when the material to be analyzed is a nucleic acid. Even very small nucleic acids (i.e., oligonucleotides) are easily separated in an electrical field by one or the other medium. The primary criteria for choosing polyacrylamide or agarose gel electrophoresis are length and whether or not the nucleic acid is single stranded or double stranded. Short, single-stranded DNAs like oligonucleotides require polyacrylamide whereas long (>100bp), double stranded DNAs are best resolved on agarose.
\end{fquote}

The macgyver running buffer\cite{MACGYVER2005}.

\cite{Inexpensive1955}


\end{document}