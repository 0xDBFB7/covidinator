%!TeX root = background
\documentclass[paper.tex]{subfiles}
\begin{document}



\begin{multicols}{1}

\section{Background}



We were inspired to undertake this project by a set of publications by a certain Sun group, (primarily Liu et al \cite{Microwave2009} and Yang et al \cite{Efficient2015}, simultaneously demonstrated by Hung et al \cite{focusing2014}).

These reports attempt to establish that, ostensibly unlike human cellular structures, the studied species of viruses have about the right size, shape, stiffness, and net charge distribution such that the lowest normal mode is slightly underdamped and can be driven by intense electric fields. 

The group theoretically model and then experimentally demonstrate in various strains of Influenza A that - supposedly due to this acoustic, normal-mode effect - the field intensity required to crack the lipid envelope is near the safety limits for continuous exposure to humans.

The experiments are competently designed and many orthogonal measurements are made to suggest that such an effect exists.

There are a few minor issues with the experimental technique (see discussion below).

These results have attracted some recent attention. \cite{Optical2020} have observed a normal mode of the same nature optically, and work is proceeding by their team.

However, such a non-thermal mechanism is quite unprecedented and extremely dubious, oft-posited but never substantiated; so it has taken some convincing that this is not a particularly abstruse artifact, especially when considered in the context of the broader non-thermal effects literature.

For even the RF power limits set by standards organizations like \cite{ICNIRP2020} \cite{IEEE2006} appear to be based on the observation that any resonance modes which might exist in biological tissues are categorically overdamped due to viscous from the surrounding solvent to be of any importance \cite{Vibrational2002}, and such charge distributions do not lend themselves to sufficiently strong coupling to external fields. This is grounded in very solid {\it in vitro}, {\it in vivo} evidence and theoretical modelling. 

On the other hand, under clinical and laboratory circumstances, a few well-established non-resonant non-thermal effects have already been put to practical use. 

As an example, electroporation is now a common laboratory procedure in which an enormous electric field causes ions in solution to diffuse across a host cell's membrane, reversibly rendering it permeable (among other, more subtle mechanisms; the minutia of poration are extraordinarily complex and beyond the author's understanding\cite{Theoretical2007}). 

A variant using irreversible electroporation\cite{Irreversible2013} has recently been used in the clinic\cite{Nonthermal2013} for tumor ablation.% In vitro there is evidence to suggest that sub-cellular effects take place under truly tremendous fields. 

\lettrine{U}nfortunately, it appears that, in order to fit Yang et al's observed data on absorption cross section, and to explain the suggested inactivation mechanism, requires that the virus, a structure with not more than $6 \times 10^8$ atoms and not more than $1 \times 10^8$ solvent molecules, to have an effective charge of $q=10^7$. It is difficult to account for such a charge.

On the other hand, Uzunoglu have produced a formulation of the theoretical analysis which requires a more plausible charge of only $q=10^4$. However, whether even this charge is reasonable given solvent screening is not yet known to us.\footnotemark

% 300 MDa of hydrogen has 3e8 H atoms
% ((100 nm)^3 * (1000 kg/m^3))/((18.15 g/mol) / avogadroConstant)
% 600 MDa

\footnotetext{While induced polarization is not expected to occur on GHz time scales, especially where solvent ions are the charge carriers, (see ICNIRP arguments), it is possible that some subtle, extremely fast polarization effect regarding electrons could be used to explain this \cite{nature1986}. We have not investigated this in any depth.}

It is possible that this result could be due to a specific sort of lipid membrane leakage, attributed to oxidation, which has been observed in strikingly similar regime in liposomes (\hyperref[sec:liburdy]{c.f.}); but this would not easily explain the frequency-dependent inactivation seen in Figure 4b of \cite{Efficient2015}.

We are thoroughly mystified by the data presented. For the purposes of this paper, we adopt a working hypothesis that this does not represent a clinically useful effect - especially as the exposures required are close to - or greatly exceed - the thermal limit for safety.

In any case, this is a very interesting result, and should be replicated with better controls. Reuben's team appear to be on the case, so we will not address this regime, a 15 minute, 100 V/m, 8 GHz, exposure, in much detail.\\

Were such a mechanism to exist, it would seem to provide significant advantages over existing Far UV or cold-plasma sterilization.


%We leave this section here because it will be useful to inform us in the next section.



%
%
%There are some 20,000 papers on the topic of RF safety, with solid mechanistic, in vitro, in vivo and epidemiological evidence of safety in the 2.4 GHz and 5.8 GHz bands.
%
%However, "There are limited experimental human data upon which to set limits on exposures above 6 GHz" [Chan 2019]. Only 2\% of the above papers are on frequencies $>6$ GHz [Vijayalaxmi 2018]. 
%
%In addition, what data remains in this category is of varying quality. 
%
%
%
%As has been starkly demonstrated in the recent hydroxychloroquine contradictions [Gautret 2020] [Geleris 2020], biological research must be essentially perfectly conducted to offer any meaningful results.
%
%So, turning to, we find an entirely new set of problems. This effect is presumed to depend on mechanical forces of the entire virus. The large scales of physical virology all but prevent all-atom simulations that include the genome, but the genome is essential to the effectiveness of this technique. The same reasons currently rule out simulations including chemical reactions, important for lipid oxidation effects. Attempts at coarse-graining require careful tuning \footnote{or "parameterizing" in MD parlance}, where slight mis-steps in problem setup can lead to unphysical effects that completely alter the conclusion, particularly where solvent polarization. The kind which can only be obtained through extensive hard-wrought expertise (which the author does not possess) and careful verification, 
%
%Multiscale or hybrid techniques would seem to offer one route out of this pickle. Simulating a small patch of lipid in explicit solvent, with the pressure provided by
%
%The stymied writer turns to writing inconclusive high-level summaries which lack sufficient detail to be of true use.
%
%
%
%
%
%First, free-space techniques are easily implemented to prevent respiratory transmission. A \$10 emitter appears to suffice on a personal level, and \$100 systems for room-scale.\\
%
%
%Second, via cursory FDTD simulations, using off-the-shelf equipment and modified non-invasive microwave diathermy techniques, it appears to be possible to inactivate all virions in the outer 2 cm of the lungs and inner 1 cm of the bronchi without any effect on the surrounding tissue. 
%
%
%
%
%One might be tempted to cite long wavelength of microwaves as an advantage. A UV photon is approximately the same wavelength as the virus itself;  In the case of, the wavelength is >> diameter, meaning that 
%
%
%
%
%Like pumping a swing, this effect allows an otherwise inconsequential field magnitude to store energy over a small number of cycles until the virus is destroyed. \\
%\\\\
%
%
%It is common for.  However, in the classical case, a non-zero resonant frequency requires the ratio of the loss in energy to the surroundings to the oscillating mass to obey $b/(2m) < \omega_{res} \sqrt{2}$ \cite{Driven}; the solution to the harmonic oscillator equation is piecewise around this value. 
%
%(unless the various small-scale quantum mechanical oscillators are considered, which in our case appear to require fields in excess of 1 TW to access).
%
%
%Most things with a stiffness matrix has some resonant frequency\citationneeded. As far as we can tell, two justifications underpin the categorical implausibility of resonance-like behavior in 
%



Indeed, relatively trustworthy time-domain spectroscopy of proteins \cite{Microwave1994} has found that the relaxation time is proportional to the molecular weight of the structure (of which viruses are perhaps the ultimate example), and can be as slow as 100 MHz in some cases. 

%
%\begin{itemize}
%	\item Viscous damping from biological solvents is so strong that the relaxation time is 
%\end{itemize}
%
%To a first order, this simply doesn't seem to be true in the case of the virion.
%
%
%
%
%
%\begin{itemize}
%	\item The charge 
%\end{itemize}
%
%\footnote{It should be noted that this {\it confined} acoustic resonance is subtly distinct from common-and-garden pipe-organ acoustic resonance; this is apparently not a strictly classical phenomenon. Besides standard Coulomb-like and Lennard-Jones-like interactions between constituent particles, if Fr\"{o}hlich is to be believed at these nanoscopic scales there are also wave-function interactions among the particles of the virus which can shift storage to modes not otherwise expected.
%	
%	We confess to not yet understanding this phenomenon; fortunately, while helpful, the details of how this mode appears are not critical to implementing this technique.}


\footnote{All values have been converted to $\text{W/m}^2$ to avoid confusion. 100 $\text{W/m}^2 = 10 \text{ mW/cm}^2 = 10 \text{ dBm/cm}^2$.}


%
%So far as we are aware, there are no fundamental physical obstacles to non-thermal effects in biological systems. 
%
%
%This may account for why this paper has been largely ignored.
%
%The precise structure and charge of the virion appears to be a singular anomaly in this otherwise categorical non-existence.
%
%
%
%
%
%Any technique that requires background subtraction (see).
%
%
%We distinct regimes, which overlap in physical mechanisms (both involve the charge distribution and relaxation time) but which cite different bodies of literature. Vestiges 




\footnote{A note on units, from ICNIRP: ``{[B]elow about 6 GHz, where EMFs penetrate deep into tissue (and thus require depth to be considered), it is useful to describe this in terms of “specific energy absorption rate” (SAR), which is the power absorbed per unit mass $(W/kg)$. Conversely, above 6 GHz, where EMFs are absorbed more superficially (making depth less relevant), it is useful to describe exposure in terms of the density of absorbed power over area $W/m^2$, which we refer to as “absorbed power density”}''. }



 
From an energetics perspective, we can obtain the spring constant $k = \omega_{res}^2 m^*$ from the resonant frequency and therefore from the speed of sound and $m^*$, the reduced mass. With relatively sane choices of screened charge and resonant frequency, the energy per monocycle at 5 MV/m varies from 0.01 to 20 times the thermal energy $k_b T$. 

Compare to ~5 $k_b T$ per protein for 1 capsid segment binding energy \cite{Energies2012} \cite{Weak2002} and 6 to 30 $k_b T$ to form a single pore in a lipid bilayer \cite{Atomistic2014a}.
%17 or 78 kJ/mol / (boltzmannConstant * 310 K) / avogadroConstant = 6.5 to 30.3 kT

On the other hand, a 0.5 angstrom movement is only a 0.04\% deflection of the capsid, whereas a general value for host cells is about 3\% deflection. In addition, with a relaxation time on the order of 500 ps, it is difficult to imagine that the energy will be sufficiently concentrated to be destructive, as after the relaxation time the energy is a distant memory; only the mechanical effects that were induced during that time can remain.

At least we do not have to justify observing an effect at an exposure of $10^{-17} W/m^2$ \cite{Resonance1996}.

Certainly the virus will be destroyed at some finite electric field.

% google
% ((5 megavolts/meter * 100 electron charge)^2 / ((2*pi*(30 GHz))^2 * 60 MDa)) / (boltzmann constant * 310 Kelvin) = 0.00042 kb
% ((5 megavolts/meter * 10000 electron charge)^2 / ((2*pi*(5 GHz))^2 * 60 MDa)) / (boltzmann constant * 310 Kelvin) = 152.48 kb
% ((40 megavolts/meter * 100 electron charge)^2 / ((2*pi*(15 GHz))^2 * 100 MDa)) / (boltzmann constant * 310 Kelvin)

\clearpage 

An abundance of mechanisms exist to. Irritatingly, such mechanisms have almost universally been ruled out by careful experiment. Many entirely plausible mechanisms exist for which the only valid answer is that they simply do not occur in practice.

%
%On the other hand, the maximum amplitude of oscillation is only expected to be on the order of 50 picometers. 
%

Regurgitating Yang et al's line of reasoning, with a few; they study a 1 kV/m, 15 minute exposure, whereas we are concerned with a 3 MV/m, 500 picosecond regime. 


%for which our working hypothesis is that there are no clinically useful mechanisms, 

It stands to reason, then, that any pulse longer than the relaxation time is not efficiently using the non-thermal nature of the effect.

Less than 1 nanosecond (where poration and thermal effects start to become significant at these fields, and any previous absorbed power has been dispersed in the solvent) but greater than about 10 picoseconds (covered by existing laser research) and electric fields between 0.1 MV/m (for which biological effects at these timescales are only barely plausible) and 10 MV/m (the production of pulses which strain credulity).

%as dt decreases, there's lots of room to add more zeros. a pulse 4 orders of magnitude higher than the maximum field used by practically any existing system.





T4 capsid withstands a pressure differential of more than 30 atmospheres, crystallizing the genome packed within. \cite{Osmotic2003}


formal systematic review; in particular no search criteria were defined.

%
\footnote{It may be useful to clarify 'non-thermal'; it caused us some confusion, as the proteins of the virion locally absorb energy and undoubtedly increase in temperature; indeed the same denaturation processes may occur. The key is that, when excited in this manner, the energies in the envelope are poorly modeled by a Maxwell-Boltzmann distribution; they are not given sufficient time to 'thermalize'.} 
%


%The effects that we are concerned with occur 4 orders of magnitude above 

%Windows between caused by structural or 

%Additionally, other sterilization techniques may produce superior results with less faffing about and should be evaluated in the same context. For instance, data on far UV\cite{Germicidal2017} indicates safety.\\
%
%"Windows" of the same sort have been found, both in thermal and pulsed ultra-high-power UV Prodouz \cite{Use1987a}.
%
%(more data exists for influenza A; it is sufficiently structurally similar that ).
%
%Excellent theoretical arguments against the possibility of non-thermal effects can be found Adair \cite{Vibrational2002}, Adair \cite{Biological2002}, and Foster \cite{Viscous2000} on microtubules. 
%
%
%
%Adair \cite{Biophysics2000} on Brillouin.
%
%\nameref{sec:viscousdamping}
%
%Owrutsky
%
%We don't want a purely membrane-electroporative effect since the lipid membrane of the virus is similar to (derived from) the host cell and little selectivity can be expected. Unless the virus has a lower capacitance and charges faster than the cell.
%
%Also, the pore must stay open for long enough that damage is dealt to the virus. 
%


%Scope for clinical utility for this effect can be seen in the following contrast. 

Pakhomov \cite{Comparative} find no non-thermal effects on functioning frog heart pacemaker cells at 0.9 MV/m (although the field inside the tissue was not measured and could have been lower). 

Recent data by de Seze following continuous exposure to nanosecond fields at an external field of 3 MV/m \cite{Repeated2020} notes no acute effects, but 

The following statement echos the concensus of a number of literature reviews by standards organizations:

\begin{fquote}[\textit{NATO Research Task Group 189}, as cited by \textit{C95.1-2019, B.4.3 Rationale for pulsed RF field limits}][ \cite{treatyelectromagnetic}]
	HFM-189 found no published and replicated adverse health effects or biological mechanisms, beyond
	thermal interaction, for pulses shorter than 100 ms which suggested that neither the peak E-field limit in the
	IEEE C95.1-2005 safety standard {\normalfont[NB chapter 4.3, Table 9, subsection e. peak E field 100 kV/m]} nor the proposed limit in the Directive 2004/40/EC [33] (subsequently promulgated as 2013/35/EU [34]) have scientific basis. 
	
	Physical laws governing the propagation of E-fields in air already limit the maximum allowable peak E-field at ~3 MV/m (air breakdown). Current research efforts by members to expose biological organism(s), tissues, and cells to environmental
	fields up to this magnitude have been unable to elicit an acute biological response.
\end{fquote}

\subsection{Precedent}

Electropermeabilization of viral membranes and capsids has been observed.

Mizuno \cite{Inactivation1990} inactivate SVD Enterovirus (a 27 nm non-enveloped protein encapsidated ssRNA) and EHV (a 100 nm enveloped dsDNA) with 100\% effectiveness with 120 pulses, about 500 nanoseconds each, at 3 MV/m. Electron microscopy demonstrates that the mechanism is irreversible damage to the capsid or envelope respectively and subsequent release of the core, and is unlike thermal treatment.

While this result would seem to perfectly align with our hypothesis, extreme care must be taken in interpreting the results of any one study in isolation, especially in this field. 

See the section on pulse artifacts. Electrode insulation is not mentioned, so electrochemical effects are possible. The observed RNA damage is unexpected from a mechanical standpoint but is somewhat like the damage due to copper ions found by \cite{Microwave1987}; therefore electrochemical interaction may have been a confounding factor.

Also, although the temperature was monitored, the thermal artifact cannot necessarily be completely discounted in this experiment. The energy dissipated was sufficient to cause a significant temperature rise if localized hot-spots were to appear \footnote{(43 cal/cc * (3 microliters)) / (3 milligrams * 4.2 J/(g K)) = ~50 $\Delta$C, whereas melting temperatures are typically about 90 C for capsids \cite{Thermal1999} and 50 C for envelopes\cite{Stability1985}}.


Madiyar \cite{Manipulation2013} observe rapid permeabilization (perhaps through the slow, conventional charge electroporation mechanism) at a field of poorly-defined intensity up to about 10 MV/m, in the Vaccinia poxvirus (a 300 nm enveloped dsDNA virus). \cite{AC2017}. Interestingly, the same effect is not observed in T4 bacteriophage. 

Moreover, these results could easily be explained by a first-order theoretical. Both AFM data and surmised from osmotic pressure, the force scale required to crack the viral envelopes and capsids is about 1 nanonewton. 



\subsection{Dielectrophoresis and electrorotation}

Schnelle\cite{Trapping1996} and Fuhr\cite{Radiofrequency1994} and Grom \cite{Accumulation2006} and Gimsa \cite{New1999} report on using a modification of so-called dielectrophoresis techniques called electrorotation, in a so-called field cage.
 
 
Such electrorotation is always sub-synchronous (unlike optical centrifuge techniques) at a few dozen Hz, whereas the field rotates at between 0.1 MHz to over 50 MHz; in most cases the frequency is made high enough that ion does not occur.

In the virus tests by Schnelle, a field of 0.3 MV/m was used, without apparent damage to the virus.

Gimsa \cite{New1999} use a four-electrode cage at 25 kV/m to trap influenza.

Electrorotation large field but not seen.

de Seze





\pagebreak

(we use the term electropermeabilization)


If we compare in-vitro with in-vitro host cells, the issue becomes less clear.


NATO Research Task Group 189 \cite{treatyelectromagnetic}

zotero citation graph?


The field strengths are high enough to lead to Townsend breakdowns in air (if the duration is longer than the formative lag). In some sense, we are concerned with field magn

This idea, that resonances, is backed up by measurements on aggregate tissues by Gabriel; the slowest relaxation time in the microwave regime appear to be in the mammary glands, and. Small resonances will be disguised by the aggregation, however.


Dielectric measurements on different proteins demonstrate that the slowest relaxation 








Electric field has many more free variables than other treatments; 



best practices were not followed.





A provisional limit of 100 kV/m; the standard now that the fourier transform instantaneous pulse energy must not exceed 1/5 the energy over some specified averaging time.

C95.1-2005


When the waveform of the external electric field is non-sinusoidal, such as with pulsed or mixed frequency
waveforms, the rms value of the spatially averaged external field shall conform to the MPEs of Table 4, and
also to either of the criteria stated in 4.1.2.4.1 and 4.1.2.4.2.

 
Ultrasonic excitation or acoustic pulses have been considered in the literature; we have not analyzed these claims in any depth.







\footnote{A wave propagating into tissue with $\epsilon_r=70$ has a refractive index $n=\sqrt{\mu_r/\epsilon_r}$, and is transmitted as $\frac{2 n_1}{n_1 + n_2}$. This is of no concern for waveguide or bolus applicators, since the reflected energy is captured and returned.}






Cancer treatment involving irreversible electroporation is in a different regime. Long pulses involving high conduction currents through needles. Cardiac arrest.




Is this resonance of Q=2 really of any practical use?


In fact, this specific question is briefly noted in a comment by Saviot \cite{Comment2004} on 


\begin{fquote}[Saviot][ \cite{Comment2004}]
	We also would like to point out that these normal modes are damped when the virus particle is embedded inside a liquid...when the virus is inside water, for example, there is not much acoustic impedance mismatch at the surface. For this reason the normal modes will be broad and have short lifetimes. Therefore, the objective of killing viruses by sending out sound waves that resonate and destroy them is probably unworkable in such a configuration.	
\end{fquote}




Perilla \cite{}, in a trustworthy explicitly solvated all-atom MD simulation, describe a few surface-wave modes around 10 MHz, but don't report (or don't consider) modes in the GHz.




electropermeabilization




Unfortunately, this cannot be an exhaustive list. 20,000 papers.

significant work on elecroporation for cancer, water treatment, and food sterilization.


can we bind a highly-charged something-or-other to the virus?
or a little magnetite crystal - like that magnetoacoustic paper?


We have cycled between condemning as pathological science, and lauding it as the cure. Barring 


add "timeline"



\subsection{Liburdy's oxygen-dependent lipid leakage}
\label{sec:liburdy}


\subsection{Coherent control and laser chemistry}


Man \cite{Picosecond2016b} perform a trustworthy all-atom molecular dynamics simulation of a laser pulse tuned to provoke dissociation of a viral capsid. In this case the large-scale structual modes are not involved. As with many molecular dynamics simulations using applied fields, the field was enhanced by several orders of magnitude to an unrealistic 2 gigavolts per meter to make the reaction proceed fast enough to be captured without excessive computation time.


\subsection{Picosecond and femtosecond laser studies}

A group of literature exists concerning the excitation of these structural modes with picosecond or femtosecond lasers. A pulse some 300 fs long with a carrier of around 600 nm. Impulsive stimulated Raman scattering. We confess to not having studied this mechanism in great detail. The pulse does not directly act on the vibrational mode; the optimal pulse duration is some small fraction of the period of the mode in question.

M13 is a 880 nm by 6.6 nm rod \cite{M132015}. 

Dykeman \cite{Vibrational2009} perform an all-atom molecular dynamics simulation of ISRU of phage M13, albeit without a heat bath to accurately represent damping. An interesting result is that despite the broadband nature of the pulse, \ntilde80\% of the pulse energy couples into only 5 modes, starting at approx. 20 GHz. The binding energy is implied to be too high to explain the experimental results. 

Zhou \cite{Maximum2010} consider the optimal control theory for this problem. 

Unfortunately, the experimental results on M13 appear to have completely failed replication by Wigle \cite{No2011}, and those on lymphocytes do not agree with existing literature by two orders of magnitude \cite{Targeted2002}, observing that over 1 T$\text{W/m}^2$ of incident power is required to reproduce the effects by . 

$1\ \text{TW/m}^2 / (100\ \text{ps} / 0.3\ \text{ps}) = 3\ \text{GW/m}^2$

In any case, inactivation through this mechanism appears to require a very long duration of exposure.




\subsection{Free-electron-laser }







involving Schnelle 



A poignant summary of the consensus is found in \cite{IEEE2006}, Annex B, "Identification of levels of RF exposure responsible for adverse effects: summary of the literature".

\begin{quote}
	Further examination of the RF literature reveals no reproducible low level (non-thermal) effect that would
	occur even under extreme environmental exposures. The scientific consensus is that there are no accepted
	theoretical mechanisms that would suggest the existence of such effects. This consensus further supports the
	analysis presented in this section, i.e., that harmful effects are and will be due to excessive absorption of
	energy, resulting in heating that can result in a detrimentally elevated temperature. 
	
	The accepted mechanism is RF energy absorbed by the biological system through interaction with polar molecules (dielectric relaxation) or interactions with ions (ohmic loss) is rapidly dispersed to all modes of the system leading to an average energy rise or temperature elevation. 
	
	Since publication of ANSI C95.1-1982 [B6], significant advances have been made in our knowledge of the biological effects of exposure to RF energy. This increased knowledge strengthens the basis for and confidence in the statement that the MPEs and BRs in this	standard are protective against established adverse health effects with a large margin of safety.
\end{quote}

and, 

\begin{quote}
	...the two major groups that develop RF safety standards and
	guidelines (ICES and ICNIRP) agree that thermal effects continue to be the appropriate basis for protection
	against RF exposure at frequencies above 100 kHz.
\end{quote}



One of the assumptions that underpin 



Electrolysi


\cite{comparative2003} discuss the relative inactivation of bacterial hosts and MS2 and PRD1 phages. 



\clearpage
\end{multicols}
\section{Commentary on the Sun group results; matters to consider in the design of experiments and the evaluation of literature}

When confronted by an extremely surprising and novel result, it seems useful to consider precisely where deficiencies may lie, so that our follow up work might address these concerns; to consider all the ways in which the ultimate conclusion could be falsified; in this case, by exploring what ways the observed facts could be explained without invoking non-thermal mechanisms, so that scope for such explanations may later be reduced.



\begin{toolchain}
We must now stress decisively that, as with all scientific critique, this commentary must not be construed as a reflection on the quality of the original study, and especially not towards its authors.\\ 

We are adamant that, in all cases cited, we believe their studies were well designed and conducted: the authors went to considerable and sufficient lengths to demonstrate its validity. It is easy to say what could have happened, but much harder to say what did happen.
\end{toolchain}


\begin{fquote}[Professor David Baltimore][\cite{Issues2008}]
	The scientific literature is a conversation among scientists; we describe to each other what we have done, and we implicitly ask our colleagues to try to build on our results, testing their value and their ability to support a growing edifice.
	
	In fact, a scientist's first response upon reading a new paper may be to see a new interpretation, to conceive a different experiment that could provide a wholly new thrust for the paper. This is a thought process that is encouraged in all young scientists, and it is one of the driving forces in science.
\end{fquote}



\begin{multicols}{1}

Vijayalaxmi \cite{Biological2016} \cite{Comprehensive2018} \cite{Funding2019} already offer a number of excellent guidelines on best practices in this field; as do Chou \cite{chou1996radio} and \cite{Effects2016} . We only offer a few issues which we were particularly concerned about when designing our experiments and searching literature. 

; sham, blinding, and dosimetry demanded by Vjl. are not mentioned.


\subsection{Artifacts in past absorption spectroscopy experiments, and the subsequent development and invalidation of non-classical damping hypotheses}

Edwards et al in a series papers in 1984 and 1985 measured the reflection from an open-ended coaxial line dipped in DNA solution and found that DNA appeared to produce a resonance-like peak in the absorption spectrum.

Such an effect is so unexpected - Bigio et al describe it as “astonishing”. \\

Edwards hypothesized that the outer hydration layers could have a sort of elastic, non-viscous isolating effect.

This result seems to have triggered a large amount of work, both theoretical and experimental\cite{Structure1989}, in an attempt to model this interaction, and various other similar interactions, in an attempt to explain the observed resonance.

At first blush, such an explanation does not seem unreasonable. In the case of the virus, for instance, the expected amplitude of oscillation for which this mechanism applies is on the order of one water molecule (~0.1 nm). Indeed, it is now well-known that the layers formed by explicit solvents have significant effect on the conformations of proteins.

In the case of a uniform rotating sphere\cite{Molecular1963}, for instance, continuum Einsten-Stokes drag and discrete molecular drag diverge completely!

This really shouldn’t be treated classically, either - Frolich 

\lettrine{U}nfortunately for these hypotheses, Foster 1986, replicating Edwards’ coaxial line method with a more rigorous error analysis on plus an equally rigorous transmission-line absorption, Gabriel, and Bigio in 1993, an ingenious thermo-optical technique, quite definitively demonstrate - to absurd precision -  that the previous results were a characteristic artifact of coaxial-line spectrometry, and that DNA does not have any resonance mode at any microwave frequency.

The same issue, where a resonance is reported using microwave spectroscopy and subsequently irreplicable has occurred many times in the past - with Hagmann and Gandhi \cite{Substitution1982}. 

In Bigio’s words “Extreme care”. Because it necessarily involves performing differential measurement of two samples, subtracting an enormous background from a minuscule sample.

Of course, such subtraction and de-embedding are very common techniques in the 

In terms of volume, the 

%((120 nm)^3 * 10^13) / 50 ml = 0.035%

Whereas a 50 microliter pipette will have a volume tolerance of 2\%! 




Indeed, a sample of on a diean E5000 dielectric resonator 

The difficulty is that the loss tangent of water varies fro about 0.3 at 6 GHz to 0.45 at 10 GHz \cite{Metamaterial2015}. A microstrip might optimistically have a loss of 0.315 dB/cm - perhaps 2.5 or to 5\% power loss. 

$$f = \frac{233}{\sqrt{e_r}\cdot{V^{1/3}}}$$



%((1e7 * (electron charge) * 100 V/m)^2 / ((2*pi*(5 GHz))^2 * 50 MDa) * (10^13 / ml) * (50 microliter)) / (100 ps) = 1.56628917 watts


% 1 nanonewton * 1 nanometer  * 10^13

For a 0.5% tolerance in the frequency of a dielectric resonator (which), it must be machined to an accuracy of less than 0.05 mm.

https://exxelia.com/uploads/PDF/e6000-v1.pdf


In the case of our (admittedly crude, amateurish equipment), by making completely justifiable changes in calibration and post-processing correction for thermal drift, normalization, etc, we could produce a resonance mode at almost any frequency we liked. We were able to produce a candidate resonance even in pure water.


Liu and Yang use a coplanar waveguide, and offer hydration layers along the lines of Edwards as a theoretical basis.




However, as Foster et al demonstrated, there is nothing necessarily wrong with VNA-based dielectric spectroscopy if it is well conducted; but history would seem to suggest that we take some degree of caution with such results.


\subsection{Raman spectroscopy}

Very recently, Burkhartsmeyer \cite{Optical2020} have published data replicating the findings on bacteriophage PhiX174 using a single-particle Raman spectroscopy technique. We have not examined these claims in any detail, but believe that they are excellent evidence.

Perhaps these results cannot be trusted implicitly, however, because non-photothermal optical Raman methods can also apparently synthesize artifacts - see unpublished, personal communication with Taylor \cite{mechanisms1981}. As this is an unpublished negative result, we know nothing about the experimental details that led to this error.

\subsection{Thermal controls}

As any researchers who have scalded their mouths on a cheap dinner may know, microwave irradiation can produce unintuitive temperature gradients, with local hot-spots that can be difficult to measure directly.

An electric field necessarily dissipates energy in the surrounding solvent, and 

In inevitable temperature rise as a confounding factor. 

mouse was heating

A few reports in the 1940s suggested that bacteria might be non-thermally inactivated. However, Burton\cite{Effects1950} described an experiment using thermochromic pigments to examine heating profiles throughout the sample. With a half-second pulse of an exposure frequency of only 30 MHz, they observe highly localized regions that exceed 50 C even while the maximum temperature measured by thermocouple never exceeds 36-40 C. This is despite the use of a tube only 9 mm across, where it might be thought that thermal conduction would dissipate any . 

With the shorter wavelengths of microwave frequencies, it seems likely that such effects will be far more pronounced.

This is highlighted by the experiments of Kozempel \cite{Preliminary1997} and \cite{Inactivationa}, discussing the performance of an RF fluid sterilization unit. They take great pains to map the temperature profile using a fiber-optic thermometer; use thermographic indicators; and yet find substantial inactivation of {\it Pediococcus sp.} with a maximum measured temperature of only 40 C.

This is a fascinating result, and one which should be compared to the existing literature and very carefully replicated to isolate the mechanism. However, unless it is suggested that {\it Pediococcus sp.} also exhibits a confined microwave resonance of the same nature as the virus, it appears that temperature measurements within the physiological range are not necessarily a sufficiently stringent control to definitively {\it isolate} resonant effects from other mechanisms.


The temperature rise after 15 minutes radiation was 27.5°C up to 34.5°C.


To illustrate how sensitive this research is, [Chemeris 2004] mention:

\begin{quote}
	
	The increase in DNA damage after exposure of cells to HPPP EMF shown in Table 2 was due to the temperature rise in the cell suspension by $3.5\pm0.1^{\circ}  $C. This was confirmed in sham-exposure experiments and experiments with incubation of cells for 40 min under the corresponding temperature conditions."
	
\end{quote}

Of course, this is using an extremely sensitive assay, not an aggregate inactivation 

A previous report of microwave resonance, Grundler \cite{Sharp1983} use a calorimeter, a stream of cooling water controlled to within 0.1 C, an interferometric thermometer, and can state definitively that at no point does the temperature inhomogeneity exceed 0.02 C.

The experiments of the Sun group 

Conversely, if temperature control is not practical, the temperature can be allowed to rise - so long as a positive control is present which can be  

In Gandhi \cite{Basic1983}, nichrome heaters and copper-constantan thermocouples are used to generate the same temperature-time profile in both the treatment and this positive control to within 0.02 C, finding no effect. 


[Vj] mention the importance of precise dosimetry. Even simple structures can produce hot-spots 
Yang et al use both a plastic cuvette, and a single drop of solution on a glass slide; in either case, a sharp change dielectric constant is present.

\begin{quote}
	To confirm that our observation is not due to the microwave thermal heating effect, we had monitored the sample temperature change during the microwave illumination experiments with a radiated power density of 486W/m2 at a frequency of 6GHz by using an infrared thermal imaging camera with a temperature accuracy of 0.05°C (CHCT, P384-20).  We thus exclude the possible contribution of microwave thermal heating effect to inactivation H3N2 viruses under our experimental condition.
\end{quote}

Glass and water are not infrared transmissible; only the surface temperature of the sample can be measured. 

The frequency-dependent inactivation seen in Figure 4b of \cite{Efficient2015} is difficult to explain without invoking non-thermal mechanisms. Only one  It is not mentioned how this applied power level was measured. 

The QPJ-06183640 amplifier in question is not available on the supplier's website \cite{Microwaved} has gain flatness of 3.5 dB over 6-12 GHz - just over 1 octave. The amplifier used in [Hung] is not mentioned; because the frequency range mentioned (6 GHz to 18 GHz) is the same as that of the amplifier, we assume the same amplifier was used. 

If the amplifier happened to be matched better at 8 GHz than 12 GHz.

If the output power was calculated based on the input power (which appears to have been done) rather than being measured in free-space, a shape very similar to that seen will be formed.

Similarly, the gain of the EM-6969 antenna\cite{EM6969} varies from 17 dBi to 21 dBi over the frequency range in Figure 3.

However, that the spectra of all these effects would combine to produce a peak at precisely the resonant frequency determined by the entirely different microwave cuvette is highly implausible; and so Figure 4b really does seem to be good evidence of such an effect.


\subsection{Even weirder artifacts}

Bergqvist \cite{Effect1994a} find that the permeabilizing of their liposomes was not due to microwave exposure, but apparently a very peculiar interaction with the Teflon dish that was used only for positive microwave trials. Therefore, every part of the experimental protocol should be maintained precisely identical for non-sham and thermal sham trials.

Electrochemical the presence of various ions 



$20 \text{W/m}^2 * 6 minutes to \text{J/m}^2 = 7200 \text{J/m}^2$


Some research has been discounted due to attenuators that leak high 

Matzumoto \cite{Inactivation1991} find that in one case, an arc was struck within the medium, and that the UV light produced was sufficient to inactivate the organism.





\subsection{Meta issues with the field}

\footnotetext{
It is apparently easy to draw the wrong conculsion from a funnel plot, however. Here's one example of a flawed analysis: http://ies.fsv.cuni.cz/sci/publication/show/id/5277/lang/cs

Thanks to James Picone https://skeptics.stackexchange.com/a/49751/ for the analysis.
}

Publication bias in the positive direction. A p=0.001. Something along an anthropic principle. In effect, the literature selects for causes of a false positive that would be expected to be an extremely improbable set of circumstances; 

anthropic principle



 $<< 20 \text{ ps}$ \footnotemark measured in the structures of human tissue.\\
 
\footnotetext{This value for human structures is the aggregate Cole-Cole relaxation time from Gabriel, in the case of breast tissue (which has the highest value). This is somewhat an unfair comparison; the relaxation time of a heterogeneous mixture such as tissue often appears smaller than that of a pure sample. However, this value agrees well with the individual findings of cellular modes (for instance, the damping from Adair's estimates is approx. 10 ps)}

\end{multicols}



\end{document}









