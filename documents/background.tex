%!TeX root = background
\documentclass[paper.tex]{subfiles}
\begin{document}


We have cycled between condemning as pathological science, and lauding it as the cure. Barring 


we appear to have lost confidence in the concept of measurement itself.



the only analagous non-thermal mechansism we are aware of is electroporation; and that requires the diffusion of ionized sodium and potassium charge carriers through the membrane, which would seem to necessarily require a non-

Laser









A poignant summary of the consensus is found in \cite{IEEE2006}, Annex B, "Identification of levels of RF exposure responsible for adverse effects: summary of the literature".

\begin{quote}
	Further examination of the RF literature reveals no reproducible low level (non-thermal) effect that would
	occur even under extreme environmental exposures. The scientific consensus is that there are no accepted
	theoretical mechanisms that would suggest the existence of such effects. This consensus further supports the
	analysis presented in this section, i.e., that harmful effects are and will be due to excessive absorption of
	energy, resulting in heating that can result in a detrimentally elevated temperature. 
	
	The accepted mechanism is RF energy absorbed by the biological system through interaction with polar molecules (dielectric relaxation) or interactions with ions (ohmic loss) is rapidly dispersed to all modes of the system leading to an average energy rise or temperature elevation. 
	
	Since publication of ANSI C95.1-1982 [B6], significant advances have been made in our knowledge of the biological effects of exposure to RF energy. This increased knowledge strengthens the basis for and confidence in the statement that the MPEs and BRs in this	standard are protective against established adverse health effects with a large margin of safety.
\end{quote}

and, 

\begin{quote}
	...the two major groups that develop RF safety standards and
	guidelines (ICES and ICNIRP) agree that thermal effects continue to be the appropriate basis for protection
	against RF exposure at frequencies above 100 kHz.
\end{quote}



Previous works (Liu et al, ) appear to demonstrate that several species of viruses exhibit an anomalous dielectric relaxation time of $> 75 \text{ ps}$, as opposed to $< 20 \text{ ps}$ \footnotemark measured in the structures of human tissue.\\

Other studies (Yang et al, Hung et al, Sun et al) then appear to demonstrate experimentally that - due to this effect and the favorable combination of envelope and genome net charge, envelope and capsid yield stress, mass distribution, and overall stiffness - certain viruses can  apparently be made to ring up, leading their envelope to shatter, when exposed to intense but ostensibly safe CW tones in the X-band - very much like a singer might shatter a wine glass.\\

Such a non-thermal mechanism is quite unprecedented and extremely dubious, oft-posited but never substantiated; so it has taken some convincing that this is not a particularly abstruse artifact, especially when considered in the context of the broader non-thermal effects literature. For instance, the very paper (Edwards) which spurred development of the hydration-layer damping hypothesis used to explain these results, is has been quite\cite{Resonances1987} definitively\cite{Microwave1993a} shown to be an artifact, one which we unfortunately have not excluded in our study.\footnotemark \ In fact, even in light of the fine evidence that the other groups have collected, and the poor-quality data we have collected, we are still not entirely convinced that this effect exists - but perhaps this skepticism is unjustified.\\

\footnotetext{we should verify, based on simple solvated sphere damping, whether the damping is of the right order of magnitude}

We contribute almost nothing to the record (much of our discussion has already been covered by []), except by apparently replicating the effect in a T4 bacteriophage surrogate, using both microwave feedback and a luminometric beta-galactosidase infectivity assay. We appear to demonstrate that inactivation does not require 15 minutes, but is complete in less than 500 nanoseconds. This provides a headroom of some $10^6$ to all safety limits imposed by all known harm mechanisms, which can be traded for depth in tissue or distance in space. \\

This technique appears to be unique in that it is selective, save for Far UV or cold-plasma sterilization. Its advantages over those is that it operates over large volumes with inexpensive emitters, but, most critically, can be made to penetrate tissue without harm.

We briefly review the biological basis for the safety. There is solid in vitro and limited in vivo evidence of safety in this regime.

\footnotetext{This value for human structures is the aggregate Cole-Cole relaxation time from Gabriel, in the case of breast tissue (which has the highest value). This is somewhat an unfair comparison; the relaxation time of a heterogeneous mixture such as tissue often appears smaller than that of a pure sample. However, this value agrees well with the individual findings of cellular modes (for instance, the damping from Adair's estimates is approx. 10 ps)}

We briefly examine three applications. All depend significantly on what the microwave spectrum of SARS-NCoV-2 ends up being, which must be measured - or possibly reconstructed from existing whole-virion data or determined from coarse-grained MD simulation. 

With refinements, our inexpensive pulsed microwave spectrometer can obtain the required absorption and threshold data with sufficiently concentrated specimens; but proper dielectric spectroscopy labs (or, in a pinch, swept-EPR/ESR equipment \st{when used improperly}) will provide a much more detailed spectrum. A slightly safer low-titer primary specimen may be usable to avoid BSL-3 requirements. \\



First, free-space techniques are easily implemented to prevent respiratory transmission. A \$10 emitter appears to suffice on a personal level, and \$100 systems for room-scale.\\
 
%
Second, via cursory FDTD simulations, using off-the-shelf equipment and modified non-invasive microwave diathermy techniques, it appears to be possible to inactivate all virions in the outer 2 cm of the lungs and inner 1 cm of the bronchi without any effect on the surrounding tissue. 
\\

[] notes that with a bit more effort, by harnessing the dispersive nature of tissue using modulated Brillouin precursors, it may be possible to remove the virus from the body altogether.\\

The mechanism also provides a few distinctive microwave signatures that may be useful for detection; it remains to be seen whether these can be discerned in practice.

The claims above are extraordinary. We do not provide the extraordinary quality of evidence required to support them. Consider: is it more likely that an undergraduate in his parent's basement, with no experience in biology or microwave design, has misinterpreted a blip from a cobbled-together machine, or that 

expelling its viral contents into solution and 

Not to be confused with pseudoscientific 'resonance therapies', nor reports of a connection with 5G communications.

This means that the technique is {\it selective}.








is is apparently easy to draw the wrong conculsion from a funnel plot, however. Here's one example of a flawed analysis: http://ies.fsv.cuni.cz/sci/publication/show/id/5277/lang/cs

Thanks to James Picone https://skeptics.stackexchange.com/a/49751/ for the analysis.

"The funnels are heavily asymmetrical: the left-hand side of the funnels is almost completely missing, hence we have good reason to believe that publication selection bias is strong in this literature." - 










\subsection{Artifacts in past absorption spectroscopy experiments}

Edwards et al in 1984 and 1985 measured the reflection from an open-ended coaxial line dipped in DNA solution and found that DNA appeared to have a resonance-like peak.

Such an effect is so unexpected - Bigio et  aldescribe it as “astonishing”. Edwards hypothesized that the outer hydration layers could have a sort of elastic, non-viscous isolating effect. This result seems to have triggered a large amount of work to model this interaction, and various other interactions, in an attempt to explain the observed resonance.

Footnote [At first blush, such an explanation does not seem unreasonable. In the case of the virus, for instance, the expected amplitude of oscillation for which this mechanism applies is on the order of one water molecule (~0.1 nm).

Treating the interaction between the first few solvent layers surrounding some biological structure is indeed quite difficult; we encountered this in trying to setup MD This can be seen in [MD paper], so it is reasonable that continuum-like damping assumptions might break down. 

In the case of a uniform rotating sphere, for instance, the [Einstein - ?] viscous drag force diverges from the continuum hypothesis toward zero.]

This really shouldn’t be treated classically, either - Frolich 

Unfortunately for these hypotheses, Foster 1986, replicating Edwards’ coaxial line method with a more rigorous error analysis on plus an equally rigorous transmission-line absorption, and Bigio in 1993, an ingenious thermo-optical technique, quite definitively demonstrate - to absurd precision -  that the previous results were a characteristic artifact of coaxial-line spectrometry, and that DNA does not have any resonance mode at any microwave frequency.

In Bigio’s words “Extreme care”. Because it involves a differential measurement of two samples.

Hagmann and Gandhi \cite{Substitution1982}



For a 0.5% tolerance in the frequency of a dielectric resonator (which), it must be machined to an accuracy of less than 0.05 mm.

https://exxelia.com/uploads/PDF/e6000-v1.pdf


crude, amateurish equipment, by making completely justifiable changes in calibration and post-processing correction for thermal drift, normalization, etc, we could produce a resonance mode at many frequencies. almost any frequency we liked. We were able to produce many candidate peaks even in pure water.


autem We have not yet examined the details of the hydration-layer theory.


Liu and Yang use a coplanar waveguide, and offer hydration layers along the lines of Edwards as a theoretical basis.




However, as Foster et al demonstrated, there is nothing necessarily wrong with VNA-based dielectric spectrometry, if it is well conducted. The Liu and Yang group seem very competent experimenters, and the experiments are very well conducted; but history would seem to suggest that we take some degree of caution.




\subsection{Thermal controls}

In inevitable temperature rise as a confounding factor. Power dissipation necessarily produces a  As anyone who has scalded their mouths on microwaved dinner may know, microwave irradiation can produce unintuitive temperature gradients, with local hot-spots that can be difficult to measure directly.


Microwave engineering is, itself  difficult. What type of attenuator you use?

Publication bias in the positive direction. A p=0.001. Something along an anthropic principle. In effect, the literature selects for causes of a false positive that would be expected to be an extremely improbable set of circumstances; 




T4 capsid withstands a pressure differential of more than 30 atmospheres, crystallizing the genome packed within.





\cite{Osmotic2003}


\end{document}
