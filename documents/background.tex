%!TeX root = background
\documentclass[paper.tex]{subfiles}
\begin{document}

I was hoping I would have a definitive demonstration, so as to not redirect resources from more promising avenues.


TRL 0

suggestive, especially because a model organism


Regurgitating Yang et al's line of reasoning, with a few; they study a 1 kV/m, 15 minute exposure, whereas we are concerned with a 3 MV/m, 500 picosecond regime. 

\begin{multicols}{1}

T4 capsid withstands a pressure differential of more than 30 atmospheres, crystallizing the genome packed within. \cite{Osmotic2003}





(more data exists for influenza A; it is sufficiently structurally similar that ).


We don't want a purely membrane-electroporative effect since the lipid membrane of the virus is similar to (derived from) the host cell and little selectivity can be expected. Unless the virus has a lower capacitance and charges faster than the cell.

Also, the pore must stay open for long enough that damage is dealt to the virus. 



Electroporation of viral membranes and capsids has been observed.

Mizuno \cite{Inactivation1990} inactivate SVD Enterovirus (a 27 nm non-enveloped encapsidated ssRNA) and EHV (a 100 nm enveloped dsDNA) with 100\% effectiveness at 120 pulses, about 500 nanoseconds each, at 3 MV/m. 

The thermal artifact cannot necessarily be discounted. The energy \footnote{(43 cal/cc * (3 microliters)) / (3 milligrams * 4.2 J/(g K))}. The electrodes were un-insulated, so electrochemical effects are possible.

Interestingly


Madiyar observe rapid permeabilization at a field of approximately $10^7$ V/m, in the Vaccinia poxvirus (a 300 nm enveloped dsDNA virus). \cite{AC2017}. Interestingly, the same effect is not observed in T4 \cite{Manipulation2013}. 


The field strengths are high enough to lead to Townsend breakdowns in air (if the duration is longer than the formative lag). In some sense, we are concerned with field magn




Dielectric measurements on different proteins demonstrate that the slowest relaxation 







Electric field has many more free variables than other treatments; 



best practices were not followed.

\footnote{A wave propagating into tissue with $\epsilon=70$ has a refractive index $n=\sqrt{\mu/\epsilon}$, and is transmitted as $\frac{2 n_1}{n_1 + n_2}$. This is of no concern for waveguide or bolus applicators, since the reflected energy is captured and returned.}

NATO Research Task Group 189 \cite{treatyelectromagnetic}

zotero citation graph?

HFM-189 found no published and replicated adverse health effects or biological mechanisms, beyond
thermal interaction, for pulses shorter than 100 ms which suggested that neither the peak E-field limit in the
IEEE C95.1-2005 safety standard [n.b. 4.3, Table 9, subsection e. peak E field 100 kV/m] nor the proposed limit in the Directive 2004/40/EC [33] (subsequently promulgated as 2013/35/EU [34]) have scientific basis. 



" physical laws governing the propagation of E-fields in air already limit the maximum allowable peak E-field at ~3 MV/m (air breakdown). Current research efforts by members to expose biological organism(s), tissues, and cells to environmental
fields up to this magnitude have been unable to elicit an acute biological response."


\footnote{A note on units, from ICNIRP: ``{[B]elow about 6 GHz, where EMFs penetrate deep into tissue (and thus require depth to be considered), it is useful to describe this in terms of “specific energy absorption rate” (SAR), which is the power absorbed per unit mass $(W/kg)$. Conversely, above 6 GHz, where EMFs are absorbed more superficially (making depth less relevant), it is useful to describe exposure in terms of the density of absorbed power over area $W/m^2$, which we refer to as “absorbed power density”}''. }

A provisional limit of 100 kV/m; the standard now that the fourier transform instantaneous pulse energy must not exceed 1/5 the energy over some specified averaging time.

C95.1-2005


When the waveform of the external electric field is non-sinusoidal, such as with pulsed or mixed frequency
waveforms, the rms value of the spatially averaged external field shall conform to the MPEs of Table 4, and
also to either of the criteria stated in 4.1.2.4.1 and 4.1.2.4.2.




C95.1-2019, B.4.3 Rationale for pulsed RF field limits










Cancer treatment involving irreversible electroporation is in a different regime. Long pulses involving high conduction currents through needles. Cardiac arrest.



Pakhomov \cite{Comparative} expose frog heart cells. No effects on pacemaker at 0.9 kV/m (although the field inside the ). 


Note: Not being biologists or microwave engineers, the trained reader may find that we are too verbose in simple technique and not in another. 


Is this resonance of Q=2 really of any practical use?


In fact, this specific question is briefly noted in a comment by Saviot \cite{Comment2004} on 


\begin{fquote}[Saviot][ \cite{Comment2004}]
	We also would like to point out that these normal modes are damped when the virus particle is embedded inside a liquid...when the virus is inside water, for example, there is not much acoustic impedance mismatch at the surface. For this reason the normal modes will be broad and have short lifetimes. Therefore, the objective of killing viruses by sending out sound waves that resonate and destroy them is probably unworkable in such a configuration.	
\end{fquote}




Perilla \cite{}, in a trustworthy explicitly solvated all-atom MD simulation, describe a few surface-wave modes around 10 MHz, but don't report (or don't consider) modes in the GHz.




electropermeabilization




Unfortunately, this cannot be an exhaustive list. 20,000 papers.

significant work on elecroporation for cancer, water treatment, and food sterilization.


can we bind a highly-charged something-or-other to the virus?
or a little magnetite crystal - like that magnetoacoustic paper?


We have cycled between condemning as pathological science, and lauding it as the cure. Barring 


add "timeline"



\subsection{Liburdy's oxygen-dependent lipid leakage}



\subsection{Tsen's femtosecond laser ISRU}

Impulsive stimulated Raman scattering. We confess to not having studied these mechanisms in great detail. Does not directly act on 
the vibrational mode; the optimal pulse duration is some small fraction of the period of the mode in question.

Unfortunately, these results appear to have failed replication \cite{No2011}. In any case, inactivation requires a very long duration of exposure; unlikely to be of great clinical use.

\subsection{Free-electron-laser }


If this mechanism exists, it would seem to provide significant advantages over existing UV or cold-plasma sterilization. 

If the extensions made in our paper are valid, this is a non-ionizing, non-thermal
%
\footnote{It may be useful to define 'non-thermal'; it caused us some confusion. Certainly the proteins of the virion locally absorb energy and increase in temperature. The key is that, when excited in this manner, the energies in the envelope are poorly modeled by a Maxwell-Boltzmann distribution; they are not given sufficient time to 'thermalize'. In contrast, with typical 2.4 GHz microwave exposure, sterilization can only occur by aggregate heating of the fluid and tissue.} 
%
, non-chemical technique, harmless for continuous exposure to tissue, which can sterilize air and surfaces alike, including skin, eyes, and within hair; it evolves no ozone, can readily be generated with \$1 USD - scale devices, acts instantaneously, and - perhaps most critically - could be made to act {\bf within infected tissues}.




\footnote{It should be noted that this {\it confined} acoustic resonance is subtly distinct from common-and-garden pipe-organ acoustic resonance; this is apparently not a strictly classical phenomenon. Besides standard Coulomb-like and Lennard-Jones-like interactions between constituent particles, if Fr\"{o}hlich is to be believed at these nanoscopic scales there are also wave-function interactions among the particles of the virus which can shift storage to modes not otherwise expected.
	
	We confess to not yet understanding this phenomenon; fortunately, while helpful, the details of how this mode appears are not critical to implementing this technique.}


\footnote{All values have been converted to $\text{W/m}^2$ to avoid confusion. 100 $\text{W/m}^2 = 10 \text{ mW/cm}^2 = 10 \text{ dBm/cm}^2$.}




the only analagous non-thermal mechansism we are aware of is electroporation; and that requires the diffusion of ionized sodium and potassium charge carriers through the membrane, which would seem to necessarily require a non-

Laser


Additionally, other sterilization techniques may produce superior results with less faffing about and should be evaluated in the same context. For instance, data on far UV [Buonanno 2017] indicates safety.\\


So far as we are aware, there are no fundamental physical obstacles to non-thermal effects in biological systems. As an example, electroporation is a well-established - now a common laboratory procedure, in which an enormous electric field causes ions in solution to migrate across the membrane, 

However, there equally appear to be no fundamental reasons why some small-molecule drug could not bind to the wrong receptor, why 



Most things with a stiffness matrix has some resonant frequency\citationneeded. As far as we can tell, two justifications underpin the categorical implausibility of resonance-like behavior in 

Notice that a non-zero resonant frequency
requires $\beta = b/(2m) \i-t \sqrt{2}$ \cite{Driven}


\begin{itemize}
		\item Viscous damping from biological solvents is so strong that the relaxation time is 
\end{itemize}

To a first order, this simply doesn't seem to be true in the case of the virion.

\begin{itemize}
		\item The charge 
\end{itemize}








A poignant summary of the consensus is found in \cite{IEEE2006}, Annex B, "Identification of levels of RF exposure responsible for adverse effects: summary of the literature".

\begin{quote}
	Further examination of the RF literature reveals no reproducible low level (non-thermal) effect that would
	occur even under extreme environmental exposures. The scientific consensus is that there are no accepted
	theoretical mechanisms that would suggest the existence of such effects. This consensus further supports the
	analysis presented in this section, i.e., that harmful effects are and will be due to excessive absorption of
	energy, resulting in heating that can result in a detrimentally elevated temperature. 
	
	The accepted mechanism is RF energy absorbed by the biological system through interaction with polar molecules (dielectric relaxation) or interactions with ions (ohmic loss) is rapidly dispersed to all modes of the system leading to an average energy rise or temperature elevation. 
	
	Since publication of ANSI C95.1-1982 [B6], significant advances have been made in our knowledge of the biological effects of exposure to RF energy. This increased knowledge strengthens the basis for and confidence in the statement that the MPEs and BRs in this	standard are protective against established adverse health effects with a large margin of safety.
\end{quote}

and, 

\begin{quote}
	...the two major groups that develop RF safety standards and
	guidelines (ICES and ICNIRP) agree that thermal effects continue to be the appropriate basis for protection
	against RF exposure at frequencies above 100 kHz.
\end{quote}



One of the assumptions that underpin 



Electrolysi

\cite{comparative2003} discuss the relative inactivation of bacterial hosts and MS2 and PRD1 phages. 

\subsection{Artifacts in past absorption spectroscopy experiments, and the subsequent development and debunking of non-classical damping hypotheses}

Edwards et al in 1984 and 1985 measured the reflection from an open-ended coaxial line dipped in DNA solution and found that DNA appeared to have a resonance-like peak.

Such an effect is so unexpected - Bigio et al describe it as “astonishing”. Edwards hypothesized that the outer hydration layers could have a sort of elastic, non-viscous isolating effect. This result seems to have triggered a large amount of work to model this interaction, and various other interactions, in an attempt to explain the observed resonance.

Footnote [At first blush, such an explanation does not seem unreasonable. In the case of the virus, for instance, the expected amplitude of oscillation for which this mechanism applies is on the order of one water molecule (~0.1 nm).

Treating the interaction between the first few solvent layers surrounding some biological structure is indeed quite difficult; we encountered this in trying to setup MD This can be seen in [MD paper], so it is reasonable that continuum-like damping assumptions might break down. 

In the case of a uniform rotating sphere, for instance, the [Einstein - ?] viscous drag force diverges from the continuum hypothesis toward zero.]

This really shouldn’t be treated classically, either - Frolich 

\lettrine{U}nfortunately for these hypotheses, Foster 1986, replicating Edwards’ coaxial line method with a more rigorous error analysis on plus an equally rigorous transmission-line absorption, and Bigio in 1993, an ingenious thermo-optical technique, quite definitively demonstrate - to absurd precision -  that the previous results were a characteristic artifact of coaxial-line spectrometry, and that DNA does not have any resonance mode at any microwave frequency.

In Bigio’s words “Extreme care”. Because it involves a differential measurement of two samples.

Hagmann and Gandhi \cite{Substitution1982}



For a 0.5% tolerance in the frequency of a dielectric resonator (which), it must be machined to an accuracy of less than 0.05 mm.

https://exxelia.com/uploads/PDF/e6000-v1.pdf


In the case of our crude, amateurish equipment, by making completely justifiable changes in calibration and post-processing correction for thermal drift, normalization, etc, we could produce a resonance mode at almost any frequency we liked. We were able to produce many candidate peaks even in pure water.


Liu and Yang use a coplanar waveguide, and offer hydration layers along the lines of Edwards as a theoretical basis.




However, as Foster et al demonstrated, there is nothing necessarily wrong with VNA-based dielectric spectroscopy if it is well conducted. The Sun group seem very competent experimenters, and the experiments are very well conducted; but history would seem to suggest that we take some degree of caution.




\subsection{Thermal controls}

To illustrate how sensitive this research is, [Chemeris 2004] mention:

\begin{quote}
	
	The increase in DNA damage after exposure of cells to HPPP EMF shown in Table 2 was due to the temperature rise in the cell suspension by $3.5\pm0.1^{\circ}  $C. This was confirmed in sham-exposure experiments and experiments with incubation of cells for 40 min under the corresponding temperature conditions."
	
\end{quote}

Of course, this is using an extremely sensitive assay, not an aggregate inactivation 

A previous report of microwave resonance, Grundler \cite{Sharp1983}, use a calorimeter, a stream of cooling water controlled to within 0.1 C, an interferometric thermometer, and can state definitively that at no point does the temperature inhomogeneity exceed 0.02 C.

The experiments of the Sun group 

An electric field necessarily dissipates energy in the surrounding solvent, and 

In inevitable temperature rise as a confounding factor. 

As any researchers who have scalded their mouths on a cheap dinner may know, microwave irradiation can produce unintuitive temperature gradients, with local hot-spots that can be difficult to measure directly.

Kozempel \cite{Preliminary1997} and \cite{Inactivationa}. They take great pains to map the temperature profile using a fiber-optic thermometer; use thermographic indicators; and find substantial, almost complete inactivation of {\it Pediococcus sp.} with a measured discharge temperature of only 40 C.

This is a fascinating result, and one which should be compared to the existing literature and very carefully replicated to isolate the mechanism. However, unless it is suggested that {\it Pediococcus sp.} also exhibits a confined microwave resonance of the same nature, it appears that periodic temperature measurements within the physiological range are not necessarily a sufficiently stringent control to definitively {\it isolate} resonant effects from other mechanisms.

Conversely, if temperature control is not possible, the temperature can be allowed to rise so long as a positive control is present in which 

In Gandhi \cite{Basic1983}, nichrome heaters and copper-constantan thermocouples are used to generate the same temperature-time profile in both the treatment and this positive control to within 0.02 C, finding no effect. 

Burton\cite{Effects1950} describe the use of thermochromic pigments to examine heating profiles. Using low-frequency exposure, they are able to produce localized regions in excess of 50 C.






[Vj] mention the importance of precise dosimetry. Even simple structures can produce hot-spots 
Yang et al use both a plastic cuvette, and a single drop of solution on a glass slide; in either case, a sharp change dielectric constant is present.


We do not intend to denigrate the work or otherwise cast aspersions on the authors of the mentioned papers; their experiment extremely well-designed and performed. 

However, since the result is unexpected, we would like to mention a few means by which the effect described could be achieved without strong non-thermal coupling.

\begin{quote}
	
	To confirm that our observation is not due to the microwave thermal heating effect, we had monitored the sample temperature change during the microwave illumination experiments with a radiated power density of 486W/m2 at a frequency of 6GHz by using an infrared thermal imaging camera with a temperature accuracy of 0.05°C (CHCT, P384-20). The temperature rise after 15 minutes radiation was 7°C, from 27.5°C up to 34.5°C. We thus exclude the possible contribution of microwave thermal heating effect to inactivation H3N2 viruses under our experimental condition.
	
\end{quote}

The frequency-dependent inactivation seen in Figure 4b of \cite{Efficient2015} is difficult to explain without invoking non-thermal mechanisms. Only one  It is not mentioned how this applied power level was measured. 

The QPJ-06183640 amplifier in question is not available on the supplier's website \cite{Microwaved} has gain flatness of 3.5 dB over 6-12 GHz - just over 1 octave. The amplifier used in [Hung] is not mentioned; because the frequency range mentioned (6 GHz to 18 GHz) is the same as that of the amplifier, we assume the same amplifier was used. 

If the amplifier happened to be matched better at 8 GHz than 12 GHz.

If the output power was calculated based on the input power (which appears to have been done) rather than being measured in free-space, a shape very similar to that seen will be formed.

Similarly, the gain of the EM-6969 antenna\cite{EM6969} varies from 17 dBi to 21 dBi over the frequency range in Figure 3.

However, that the spectra of all these effects would combine to produce a peak at precisely the resonant frequency determined by the entirely different microwave cuvette is highly implausible; and so Figure 4b really does seem to be good evidence of such an effect.


\subsection{Other artifacts}

Bergqvist \cite{Effect1994a} find that the permeabilizing of their liposomes was not due to microwave exposure, but apparently a very peculiar interaction with the Teflon dish that was used only for positive microwave trials. Therefore, every part of the experimental protocol should be maintained precisely identical for non-sham and thermal sham trials.

Some research has been discounted due to attenuators that leak high 

Matzumoto \cite{Inactivation1991} find that in one case, an arc was struck within the medium, and that the UV light produced was sufficient to inactivate the organism.


\subsection{Meta issues with the field}

\footnotetext{
It is apparently easy to draw the wrong conculsion from a funnel plot, however. Here's one example of a flawed analysis: http://ies.fsv.cuni.cz/sci/publication/show/id/5277/lang/cs

Thanks to James Picone https://skeptics.stackexchange.com/a/49751/ for the analysis.
}

Publication bias in the positive direction. A p=0.001. Something along an anthropic principle. In effect, the literature selects for causes of a false positive that would be expected to be an extremely improbable set of circumstances; 

anthropic principle


\subsection{Raman spectroscopy}

Very recently, Burkhartsmeyer \cite{Optical2020} have published data replicating the findings using a single-particle Raman spectroscopy technique. We have not examined these claims in detail, but they are good evidence.

Non-photothermal optical Raman methods can also apparently synthesize artifacts - see unpublished, personal communication with Taylor \cite{mechanisms1981}. 



























Previous works (Liu et al, ) appear to demonstrate that several species of viruses exhibit an anomalous dielectric relaxation time of $> 75 \text{ ps}$, as opposed to $<< 20 \text{ ps}$ \footnotemark measured in the structures of human tissue.\\

Other studies (Yang et al, Hung et al, Sun et al) then appear to demonstrate experimentally that - due to this effect and the favorable combination of envelope and genome net charge, envelope and capsid yield stress, mass distribution, and overall stiffness - certain viruses can  apparently be made to ring up, leading their envelope to shatter, when exposed to intense but ostensibly safe CW tones in the X-band - very much like a singer might shatter a wine glass.\\

Such a non-thermal mechanism is quite unprecedented and extremely dubious, oft-posited but never substantiated; so it has taken some convincing that this is not a particularly abstruse artifact, especially when considered in the context of the broader non-thermal effects literature. For instance, the very paper (Edwards) which spurred development of the hydration-layer damping hypothesis used to explain these results, is has been quite\cite{Resonances1987} definitively\cite{Microwave1993a} shown to be an artifact, one which we unfortunately have not excluded in our study.\footnotemark \ In fact, even in light of the fine evidence that the other groups have collected, and the poor-quality data we have collected, we are still not entirely convinced that this effect exists - but perhaps this skepticism is unjustified.\\

\footnotetext{we should verify, based on simple solvated sphere damping, whether the damping is of the right order of magnitude}

We contribute almost nothing to the record (much of our discussion has already been covered by []), except by apparently replicating the effect in a T4 bacteriophage surrogate, using both microwave feedback and a luminometric beta-galactosidase infectivity assay. We appear to demonstrate that inactivation does not require 15 minutes, but is complete in less than 500 nanoseconds. This provides a headroom of some $10^6$ to all safety limits imposed by all known harm mechanisms, which can be traded for depth in tissue or distance in space. \\

This technique appears to be unique in that it is selective, save for Far UV or cold-plasma sterilization. Its advantages over those is that it operates over large volumes with inexpensive emitters, but, most critically, can be made to penetrate tissue without harm.

We briefly review the biological basis for the safety. There is solid in vitro and limited in vivo evidence of safety in this regime.

\footnotetext{This value for human structures is the aggregate Cole-Cole relaxation time from Gabriel, in the case of breast tissue (which has the highest value). This is somewhat an unfair comparison; the relaxation time of a heterogeneous mixture such as tissue often appears smaller than that of a pure sample. However, this value agrees well with the individual findings of cellular modes (for instance, the damping from Adair's estimates is approx. 10 ps)}

We briefly examine three applications. All depend significantly on what the microwave spectrum of SARS-NCoV-2 ends up being, which must be measured - or possibly reconstructed from existing whole-virion data or determined from coarse-grained MD simulation. 

With refinements, our inexpensive pulsed microwave spectrometer can obtain the required absorption and threshold data with sufficiently concentrated specimens; but proper dielectric spectroscopy labs (or, in a pinch, swept-EPR/ESR equipment \st{when used improperly}) will provide a much more detailed spectrum. A slightly safer low-titer primary specimen may be usable to avoid BSL-3 requirements. \\



First, free-space techniques are easily implemented to prevent respiratory transmission. A \$10 emitter appears to suffice on a personal level, and \$100 systems for room-scale.\\

%
Second, via cursory FDTD simulations, using off-the-shelf equipment and modified non-invasive microwave diathermy techniques, it appears to be possible to inactivate all virions in the outer 2 cm of the lungs and inner 1 cm of the bronchi without any effect on the surrounding tissue. 
\\

[] notes that with a bit more effort, by harnessing the dispersive nature of tissue using modulated Brillouin precursors, it may be possible to remove the virus from the body altogether.\\

The mechanism also provides a few distinctive microwave signatures that may be useful for detection; it remains to be seen whether these can be discerned in practice.

The claims above are extraordinary. We do not provide the extraordinary quality of evidence required to support them. Consider: is it more likely that an undergraduate in his parent's basement, with no experience in biology or microwave design, has misinterpreted a blip from a cobbled-together machine, or that 

expelling its viral contents into solution and 

Not to be confused with pseudoscientific 'resonance therapies', nor reports of a connection with 5G communications.

This means that the technique is {\it selective}.




\end{multicols}



\end{document}









