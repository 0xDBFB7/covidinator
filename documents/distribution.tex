%!TeX root = distribution
\documentclass[paper.tex]{subfiles}
\begin{document}
	
\begin{center}
	{\bf \cite{Microwave2009} already mention this inhomogeneity; \cite{Elastic} perform a much better analysis than we do.}
\end{center}
	

\cite{Efficient2015} theoretically model the virus to determine the minimum electric field required to destroy it. 

They assume that the virus is a simple damped harmonic oscillator, where the 'core' and 'shell' oscillate in opposition. 

They determine the net charge experimentally from microwave absorption measurements.



Since \cite{Efficient2015} try to compute the {\it threshold} field to obtain some amount of inactivation, they use a value of 400 piconewtons for the breaking stress of the envelope, obtained from \cite{Bending2011}. This threshold value agrees very well with their experimental data. However, \cite{Bending2011} also mentions that 

\begin{quote}
	More than 95\% [of] puncture events occurred above 0.4 nN.
\end{quote}

Inputting this yield stress distribution produces an inactivation field distribution that is far higher than that of \cite{Efficient2015}'s experimental data by some 200 V/m.

Let us therefore re-examine the reason why.

The liposome breaking force is distributed in an obstinately non-gaussian bimodal manner, so we override the covariance matrix and introduce a gaussian KDE sampled from the force histogram in \cite{Bending2011}.




"95%.".

The distribution of breaking strengths in [Li 2011] (Figure 5b) is bimodal, and a naive gaussian fit does not produce the above 95\% figure.

Smudged 

The simple-harmonic

This problem has a long history, stretching back to Cole-Cole's landmark 1942\cite{Dispersion1941} paper and \cite{Electrical1941}. Such concerning statements as 'mathematically ill-posed' are bandied about; however, standard optimization techniques work fine.

\begin{quote}
	As Fuoss and Kirkwood have shown, it is possible to calculate the distribution necessary to account for any given experimental results from the observed $\epsilon''$ - frequency relation.
\end{quote}

Within a sample of a single species, there is $\approx$ 12.7\%  ($109 \pm 16.65 \text{ nm}$\cite{lauffer1944biophysical}, $115 \pm 12 \text{ nm}$ \cite{Characterization1984}) variation in diameter and  \cite{Characterization1984} variation in mass.



Note that, per \cite{lauffer1944biophysical}, of Inf. A and B, "the infectious particles have diameters within the range 80 to 135[nm]". \cite{Efficient2015}'s MDCK plaque assay is already sensitive to infectivity; the extreme-sized non-infective mutants would already be filtered out. Their PCR assay might not be; the DNA of all non-infectious mutants would be considered.


Worse still, influenza exhibits pleiomorphy, stretching into a variety of \cite{Influenza2006}; 

Charge \footnotemark

\footnotetext{
	It has been shown that, over decades, single species of viruses can effect large-scale changes in the net charges of their proteins without altering their function\cite{Dynamics2010}. Use of this technique will provide a selection bias towards immunity to electromagnetic fields. We do not have the biological knowledge to determine whether this is plausible; it simply seemed worth mentioning. 
	
}

NCoV is closer to \cite{Viral2020}, but does not appear to exhibit pleiomorphy.

A cursory fit found parameters $\bar{\text{Q}}$=3.736,  $\sigma \ \text{Q} = $0.738, $\bar{F}_{res}=$8.606 GHz and $\sigma {F_{res}}=$1.073 \st{and covariance of +1.001}\footnote{warning}. This describes the observed spectrum equally well as Q=2.0:

We have not yet determined whether these values also explain the other datasets in \cite{Efficient2015}, such as the inactivation vs frequency plot.


T4 bacteriophage has far greater restraint, with only a 2.5\% $\pm \sigma$ variation in the dimensions of the icosahedral head \cite{Head1988}.\footnotemark

\footnotetext{T4 also occasionally finds itself as an isometric mutant (See PDB 5VF3).} 


Plugging a (single, non-distributed, which must be fixed; FIXME) Q of 3.75 into the above breaking force distribution, and applying some guesses for the covariance of various parameters, we get:

								Yang get:
6th percentile: 59.234       68 V/m
38th percentile: 127.376		87 V/m
63th percentile: 186.803		171 V/m
90th percentile: 298.882     100\% 275 V/m
99.9th percentile: 565.126 

Of course, 
\begin{quote}
	With four parameters I can fit an elephant, and with five I can make him wiggle his trunk. \\- J.v.N.
\end{quote}

and here we have some dozen parameters. So these approximations should be regarded with skepticism.

Since this entire covariance process will have to be repeated for SARS-NCoV \footnotemark we have not bothered to fine-tune any of these fits; and so the idea that the Q-factor is higher than previously reported must still be regarded as only a speculative hypothesis.

\footnotetext{...the CoV of NCoV}

This introduces a self-evident optimization: rather than applying a pulse of a single frequency, the pulses should be swept across the distribution. From the (extremely crude) approximations on Inf. A above, this has little effect on $<$ 90-th percentile virions, but decreases the inactivation field of the last 10\% by perhaps 50 V/m.

In practice, not having a single frequency to design a microwave system around makes the engineering quite a bit more difficult whether this tradeoff is worth making depends. "Broadbanding" makes all the $\lambda / 4$ structures behave rather poorly; depending on the bandwidth required, biasing may need to be made inductive or resistive.

In the case of patch antennas for phased arrays, U-slots have to be added to dull the impedance bandwidth; all sorts of free-space injection-locking techniques become untenable; waveguides and cavities start to run into overlapping transmission modes. Structures like magnetron cavities are hard to 

For the pre-engineered, off-the-shelf clinical without cost constraints, this is probably not a serious issue.

It may be useful to choose a few 

Are such high pulse powers achievable? 




	

\end{document}