%!TeX root = distribution
\documentclass[paper.tex]{subfiles}
\begin{document}
	
	

\cite{Efficient2015} theoretically model the virus to determine the minimum electric field required to destroy it. 

They assume that the virus is a simple damped harmonic oscillator, where the 'core' and 'shell' oscillate in opposition. 

They determine the net charge experimentally from microwave absorption measurements.

Since \cite{Efficient2015} try to compute the {\it threshold} field to obtain some amount of inactivation, they use a value of 400 piconewtons for the breaking stress of the envelope, obtained from \cite{Bending2011}. This threshold value agrees very well with their experimental data. However, \cite{Bending2011} also mentions that 

\begin{quote}
	More than 95\% [of] puncture events occurred above 0.4 nN.
\end{quote}



The liposome breaking force is distributed in an obstinately non-gaussian bimodal manner, so we override the covariance matrix and introduce a gaussian KDE sampled from the force histogram in \cite{Bending2011}.



"95%.".

The distribution of breaking strengths in [Li 2011] (Figure 5b) is bimodal, and a naive gaussian fit does not produce the above 95\% figure.

Smudged 



The simple-harmonic

This problem has a long history, stretching back to Cole-Cole's landmark paper in 1942\cite{Dispersion1941} and \cite{Electrical1941}.

\begin{quote}
	As Fuoss and Kirkwood have shown, it is possible to calculate the distribution necessary to account for any given experimental results from the observed $\epsilon''$ - frequency relation.
\end{quote}

Within a sample of a single species, there is $\approx$ 12.7\%  ($109 \pm 16.65 \text{ nm}$\cite{lauffer1944biophysical}, $115 \pm 12 \text{ nm}$ \cite{Characterization1984}) variation in diameter and  \cite{Characterization1984} variation in mass.

Note that, per \cite{lauffer1944biophysical}, of Inf. A and B, "the infectious particles have diameters within the range 80 to 135[nm]". \cite{Efficient2015}'s MDCK plaque assay is already sensitive to infectivity; the extreme-sized non-infective mutants would already be filtered out. Their PCR assay might not be; the DNA of all non-infectious mutants would be considered.


Worse still, influenza exhibits pleiomorphy \cite{Influenza2006}; 



NCoV is closer to \cite{Viral2020}, but does not exhibit pleiomorphy.

A cursory fit found parameters $\bar{\text{Q}}$=3.736,  $\sigma \ \text{Q} = $0.738, $\bar{F}_{res}=$8.606 GHz and $\sigma {F_{res}}=$1.073 and covariance of +1.001. This describes the observed spectrum equally well:

Whether these values also explain the other datasets in \cite{Efficient2015}, such as the inactivation threshold vs frequency has not yet been determined.

Plugging a (single, non-distributed, which must be fixed) Q of 3.75 into the above breaking force distribution, and applying some guesses for the covariance of various parameters, we get:

								Yang get:
6th percentile: 59.234       68 V/m
38th percentile: 127.376		87 V/m
63th percentile: 186.803		171 V/m
90th percentile: 298.882     100\% 275 V/m
99.9th percentile: 565.126 

Of course, 
\begin{quote}
	With four parameters I can fit an elephant, and with five I can make him wiggle his trunk. \\- J.v.N
\end{quote}

and here we have some dozen parameters. So these approximations should be regarded with great skepticism.


T4 bacteriophage has far greater restraint, with only a 2.5\% $\pm \sigma$ variation in the icosahedral head dimensions \cite{Head1988}.\footnotemark

\footnotetext{T4 also occasionally finds itself as an isometric mutant (See PDB 5VF3).} 

This also 

Since this entire covariance process will have to be repeated for SARS-NCoV-2 - that is, the CoV for NCoV, we have not bothered to fine-tune any of these fits. This must still be regarded as only a hypothesis.

\begin{center}
	\begin{tabular}{||c c c c||} 
		\hline
		Col1 & Col2 & Col2 & Col3 \\ [0.5ex] 
		\hline\hline
		1 & 6 & 87837 & 787 \\ 
		\hline
		2 & 7 & 78 & 5415 \\
		\hline
		3 & 545 & 778 & 7507 \\
		\hline
		4 & 545 & 18744 & 7560 \\
		\hline
		5 & 88 & 788 & 6344 \\ [1ex] 
		\hline
	\end{tabular}
\end{center}



	

\end{document}