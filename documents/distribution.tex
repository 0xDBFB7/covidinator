%!TeX root = distribution
\documentclass[paper.tex]{subfiles}
\begin{document}
	
	

\cite{Efficient2015} theoretically model the virus to determine the minimum electric field required to destroy it. 

They assume that the virus is a simple damped harmonic oscillator, where the 'core' and 'shell' oscillate in opposition. 

They determine the net charge experimentally from microwave absorption measurements.

Since [Yang] try to compute the {\it threshold} field to obtain some amount of inactivation, they use a value of 400 piconewtons for the breaking stress of the envelope, obtained from \cite{Bending2011}. This threshold value agrees very well. However, \cite{Bending2011} also mentions that 

\begin{quote}
	More than 95\% [of] puncture events occurred above 0.4 nN.
\end{quote}



The liposome breaking force is distributed in an obstinately non-gaussian bimodal manner, so we override the covariance matrix and introduce a gaussian KDE sampled from the force histogram in \cite{Bending2011}.



"95%.".

The distribution of breaking strengths in [Li 2011] (Figure 5b) is bimodal, and a naive gaussian fit does not produce the above 95\% figure.

Smudged 

The simple-harmonic

This problem has a long history, stretching back to Cole-Cole's landmark paper in 1942\cite{Dispersion1941}.

T4 bacteriophage has far greater restraint, with only a 2.5\% $\pm \sigma$ variation in the icosahedral head dimensions \cite{Head1988}. \footnote{} 

This also 

Since this covariance process will have to be repeated for SARS-NCoV-2 - that is, the CoV for NCoV - 

\begin{center}
	\begin{tabular}{||c c c c||} 
		\hline
		Col1 & Col2 & Col2 & Col3 \\ [0.5ex] 
		\hline\hline
		1 & 6 & 87837 & 787 \\ 
		\hline
		2 & 7 & 78 & 5415 \\
		\hline
		3 & 545 & 778 & 7507 \\
		\hline
		4 & 545 & 18744 & 7560 \\
		\hline
		5 & 88 & 788 & 6344 \\ [1ex] 
		\hline
	\end{tabular}
\end{center}



	

\end{document}