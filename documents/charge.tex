%!TeX root = charge
\documentclass[paper.tex]{subfiles}
\begin{document}



Protein net charges extracted from primary protein sequences compiled by the proteome-pI database \cite{ProteomepI2017} and UniProt using localCIDER \cite{CIDER2017} \ghfile{biology/data/net_charge.py}. 

*detailed source information will be added later! There are several hundred.*

\cite{Vibrational2002}
\begin{quote}
Although the value q = e, chosen to describe the oscillating dipole moment already implies a very large permanent dipole moment, if the oscillating charge were greater
than $2\times 10^4 e$, the radiative absorption from the canonical
incident microwave power density would be increased to
detectable levels if the relaxation time is as large as 5 ns.
But so large a charge leads to a moment that seems far
outside of our understanding of DNA or any other molecule.


The conclusion that there can be no microwave resonances in DNA in water is in accord with the results of
measurements in three different laboratories (Gabriel et al.,
1987; Foster et al., 1987) using techniques designed to
detect resonances with amplitudes less than 1/20th of that
reported by Edwards et al. (1984).
\end{quote}

Adair's \cite{Vibrational2002}, in the section {\it \bf Coherent Processes}. 

$$\frac{4 \pi (100 \text{nm}^2)}{(10 \text{ micrometers} \times 10 \text{ micrometers})} 10^7 e^- = $$

The arguments regarding Resonances in microtubules are also 



\end{document}