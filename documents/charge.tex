%!TeX root = charge
\documentclass[paper.tex]{subfiles}
\begin{document}

Back-of-the-envelope.


Yang et al determine that, to fit their microwave absorption cross-section data and produce the decrease in infectivity via the route they suggest, the "core" and "shell" of the virus should have an effective charge of about $10^7 e^-$. 

Is such a charge plausible?

Protein net charges extracted from primary protein sequences compiled by the proteome-pI database \cite{ProteomepI2017} and UniProt using localCIDER \cite{CIDER2017} \ghfile{biology/data/net_charge.py}. 


look into the Gibbs–Donnan effect



blood plasma


If we assume that the charge is not as. All other properties being equal, to account for the same coupling and absorption, one of the following would seem to need to be true:

1. the effective path length of the fancy test waveguide used by Yang may not be as they expected.
2. The concentration of virus was higher than measured.
3. 




On the largest scale, at a pH of 7, the net charges of the 11 proteins coded for by the Influenza A genome only contribute 


The polarization potential example that Adair borrows from Frolich, with a membrane potential of 100 V (see []) still only contributes.  

$$(4.523 \times 10^-14 \text{m}^2 \cdot 0.01 \text{F/m}^2 \cdot 100 \text{V}) \approx 10^5 e^- $$







As might be expected, the electrostatic properties of the virus differ from that of host cells, both on the sub-protein level\footnotemark

\footnotetext{
\cite{Icosahedral2019}\begin{quote}
	Different from insect, plant, and human, viruses had more positively charged than negatively charged segments. Viral proteins were the only class enriched in extremely high positively charged segments (charge ≥ + 17) when compared to the overall proteome (1B-inset with p-values).
\end{quote}
}

and in the lipid bilayer \cite{Lipid2015} \footnotemark.

\footnotetext{The virus steals its envelope from the host cell; but it tie dyes its duds before it bounces.}

*detailed source information will be added later! There are several hundred.*



Other enzymes

\cite{Vibrational2002}
\begin{quote}
Although the value q = e, chosen to describe the oscillating dipole moment already implies a very large permanent dipole moment, if the oscillating charge were greater
than $2\times 10^4 e$, the radiative absorption from the canonical
incident microwave power density would be increased to
detectable levels if the relaxation time is as large as 5 ns.
But so large a charge leads to a moment that seems far
outside of our understanding of DNA or any other molecule.


The conclusion that there can be no microwave resonances in DNA in water is in accord with the results of
measurements in three different laboratories (Gabriel et al.,
1987; Foster et al., 1987) using techniques designed to
detect resonances with amplitudes less than 1/20th of that
reported by Edwards et al. (1984).
\end{quote}

Adair's \cite{Vibrational2002}, in the section {\it \bf Coherent Processes}. 

$$\frac{4 \pi (100 \text{nm}^2)}{(10 \text{ micrometers} \times 10 \text{ micrometers})} 10^7 e^- = $$

The arguments regarding Resonances in microtubules are also 




In my field, I am used to being able to make a definite mechanistic claim. Such and such a gamma ray will not break the bond in so and so; the energy is too low. However, we might not be able to expect such a clear answer from biology; how many treatments do we know of where the interaction with every structure in the body is known? In some cases, all we appear to be able to say is that we've tested it in a convincingly large number of ways and just don't see it.

That seems defeatist, but it perhaps should be something we can prepare for.


\cite{Identification2018}



\section{Hello World}
\subsection{Hello World}
\label{sec:hello}


\hyperref[sec:hello]{See section.}


Rather than, for instance, less than 1/100 for Arginine or 1/600 for DNA.



it would be interesting to compare the potential energy of assembling such a structure to the known host metabolic burden of assembling a virus.

%https://www.macmillanlearning.com/studentresources/college/physics/tiplermodernphysics6e/classial_concept_review/chapter_11_ccr_20_electrostatic_energy_of_a_sphere_of_charge.pdf

$U = 3/5 * 1/4 pi epsilon_0 Q^2 / R \approx 2.3e-7 $
%U = (3/5) * 1/(4*pi*electricConstant) *((1e7 electricCharge)^2 / 60 nanometers) = 2.30708e-7 J

The total is $6 \times 10^11$ ATP-equivalent.

%50 kJ per mole, 6.022*10^23 = 9.9634673e-13 * 50 kj = 5e-8 J

of course, much of the potential energy may have already existed in the host cell, so this is probably an invalid comparison. And if most of the charge is contributed by osmotic, ion , this potential energy may not be considered in the above. 

\begin{autem}
TODO: is it possible for the charged core to compensate for the potential energy of the charged shell?
\end{autem}










Using electroporation data, is osmotic transfer a common 

If the intracellular concentration of 500 mM of ions  from https://www.cell.com/action/showPdf?pii=S0006-3495%2899%2976974-2 is extrapolated to viruses, 


$500 millimolar * (4/3 pi * ( 60 nanometers)^3)  3e5 e-$


Indeed, the intracellular concentration would need to reach almost 25 molar for these results to be explained.

This is actually not too extreme. Bacteriophages can reach 3 M.

"Osmotic Shock and the Strength of Viral Capsids"


For an impermeable virus like the even-numbered T's, the inside salt concentration is set by the incubation solution's value of nb. Anderson et al. (1953) found in the case of T6, upon rapid dilution with distilled water, for example, that incubation nb values on the order of 1.5 M were required for osmotic shock. From Eq. 2, and n0 ≈ 3 M, it follows that nin ≈ 0.62 M and hence n++ n− ≈ 4.2 M, which because of the Donnan equilibrium is indeed significantly larger than 2nb ≈ 3 M.

https://www-sciencedirect-com.ezproxy.library.yorku.ca/science/article/pii/S0006349503744555















\end{document}