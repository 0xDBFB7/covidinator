%!TeX root = charge
\documentclass[paper.tex]{subfiles}
%\usepackage{parselines} 


\begin{document}

\section{Charge}


\begin{autem}
Deserves a more detailed treatment.
\end{autem}




The adsorption of metal ions onto the surface of the genome
\begin{itemize}
\item The nucleocapsid. 
\item Induced polarization
		Persson cite a polarization re-arrangement of electrons as a source of drag. Polarization-like effects are not expected. We have not followed up on this.
\item 
\end{itemize}		





Poisson-Boltzmann family of equations is an extension of standard Laplace and Poisson potentials to solutions containing ions. As with all models, there are many nuances to using these correctly; whether various approximations hold, etc. We will ignore all subtleties. 

Božič \cite{How2012} consider an empty capsid.

The generalized Poisson equation is 

$$\nabla \cdot \left(\epsilon\ \nabla \bar\phi \right) = - \rho$$

We shamelessly steal Brackley \cite{Electrostatic2020}'s formulation (thanks for such detail!)  with one exception. Most solutions consider an equilibrium. However, we are specifically designing the pulse to be too short for ion equilibriation to occur. Therefore we solve in two steps: First the Poisson-Boltzmann equation is run to determine the charge density due to screening; then this is used as a constant to solve when a field is applied.

Reproducing here for convenience:

$$\nabla \cdot \left(\epsilon\ \nabla \bar\phi \right) - \epsilon \kappa^2 \sinh(\bar\phi) = - \rho$$ 

is the potential in V, kappa is a Debye-Huckel parameter. They use an internal dielectric constant of 5, an external of 80 ($\epsilon =  80\epsilon_0$). 1/10 k1

$$\hat\phi = \frac{e_0 \phi}{}$$

where $e_0$ is the electron charge, $k_b$ the Boltzmann constant, and 


\footnote{\textit{Sparselizard} has built-in support for membranes and stokes drag, and might be a good next step - requires PB to be input in weak form, which we haven't the patience for right now.}







effective 

\subsection{Empirical methods}



Birefringence e


electroroattion 


It is easy to determine an upper limit.




Donnan equilibrium

\begin{autem}
	What is the charge on phage x174 from Burkhauzens Raman spectroscopy data?
\end{autem}

Yang et al determine that in order to fit their microwave absorption data, and produce the decrease in infectivity via precisely the mechanism they suggest, the "core" and "shell" of Influenza A should have an effective charge of about $q=1 \times 10^7 e^-$. 

The mechanism given is that ions from solution counteract the protein charges to form an electric double layer. In any case, the best representation is two concentric spheres, with a net charge each.

What specifically contributes to this charge? For the purposes of coarse MD simulation, it seemed useful to localize the charge concentration. 

As might be expected, the electrostatic properties of the virus differ from that of host cell, both on the sub-protein level\cite{Icosahedral2019} \footnotemark \ and in the lipid bilayer \cite{Lipid2015} \footnotemark. 

\footnotetext{
	\cite{Icosahedral2019}: \begin{quote}
		Different from insect, plant, and human, viruses had more positively charged than negatively charged segments. Viral proteins were the only class enriched in extremely high positively charged segments (charge $> +17$) when compared to the overall proteome (1B-inset with p-values).
	\end{quote}
}

\footnotetext{The virus steals its envelope from the host cell; but it tie dyes its duds before it bounces.}

\footnote{Influenza also has several other fascinating electrostatic properties. First, the net charge on the virion changes on a decade-by-decade basis \cite{Dynamics2010}. }

\begin{autem}
	autem: Are you sure you're understanding the nature of this "effective" charge properly?
\end{autem}

\subsection{Major components}

At the largest scale, at a pH of 7 the net charges of the 4 proteins that comprise ~88\% of the protein mass\cite{Quantitative1981} of the virus (of the 11 coded for by the Influenza A genome) only contribute at most  $\approx 1 \times 10^5\  e^-$.

(29.8 kb to 29.9 kbp for NCoV-2 \cite{Genomic2020}).



(these values were only taken at a pH of 7. A larger charge imbalance might be found at less neutral pHes. However, any effect would be of little practical use if it does not occur at physiological pH.)



\begin{sidenote}
Protein net charges extracted from protein primary structure sequences\footnotemark compiled by the proteome-pI database \cite{ProteomepI2017} and UniProt\footnotemark using localCIDER \cite{CIDER2017}. See \ghfile{biology/data/net_charge.py} and structure.xlsx. Detailed source information will be added later!
\end{sidenote}

NOTE SSRNA Assuming a charge of $1e^-$ per base pair due to phosphate groups, the 13.5 kbp of ssRNA can hardly contribute $> 1.5e^- \times10^4$



Note that this is considering the net charge of a whole isolated protein. Some proteins poke through the envelope; if a zwitterionic polarization existed from one end of a protein to another, for instance, this would not count here but would affect the coupling to an EM field. Given the small extent of most of the structural proteins, however, this doesn't seem like a likely effect.

\begin{sidenote}
	It should be noted that these are the simplest, first-order primary structure charges. In practice, the charge is determined by the secondary structure and the bonding to adjacent structures.
	
	 Penhoet \cite{Structurea}, "the viral glycoproteins and nonstructural protein are heterogeneous in isoelectric point, forming constellations of interrelated protein spots." whether this is intra- or inter-virion hetrogenity is unknown to us. 
	
	Empirically-determined charges appear to be somewhat inconsistent:
	
	Privalsky and Penhoet \cite{Influenza1978} and Penhoet \cite{Structurea} find that the isoelectric point of the M protein is so acidic that it runs right off the end of their 2D electrophoresis gel. They also note that 
	
	However, the predicted value by [] is about 3.
	
	This is not of much importance for this section, but may be important when setting up MD simulations.
	
	Because CryoEM uses charged particles, it is theoretically possible to extract the actual charge distribution of a macrostructure directly from an un-fit CryoEM map \cite{Identification2018}. 
\end{sidenote}

This also agrees roughly with Perilla \cite{Physical2017}


\subsection{Electrostatic energy}



Let's compare the potential energy that would be required to assemble a single sphere of such a charge to the known host metabolic burden of assembling a virus.

%https://www.macmillanlearning.com/studentresources/college/physics/tiplermodernphysics6e/classial_concept_review/chapter_11_ccr_20_electrostatic_energy_of_a_sphere_of_charge.pdf

$$ U = \frac{3}{5}  \frac{1}{4 \pi \epsilon_0} \frac{Q^2}{R} \approx 2.3\text{e}^{-7} \text{ J} \approx 6 \times 10^{11}\  \text{ATP-equivalent} $$
%U = (3/5) * 1/(4*pi*electricConstant) *((1e7 electricCharge)^2 / 60 nanometers) = 2.30708e-7 J


%50 kJ per mole, 6.022*10^23 = 9.9634673e-13 * 50 kj = 5e-8 J

Mahmoudabadi\cite{Energetic2017} provide a total per-virion energetic cost of $10^8 $ ATP-equivalent , so this appears to be 3 orders of magnitude too large.

Of course, much of the potential energy may have already existed in the host cell, so this is probably an invalid comparison. And if most of the charge is contributed by osmotic, ion, this potential energy may not be considered in the above. 

\begin{autem}
	autem: is it possible for the charged core to compensate for the potential energy of the charged shell?
\end{autem}

\subsection{First principles}

Influenza is approximately 174 MDa Ruigrok. Assuming a typical 50\% H concentration, such a charge would require about 1 in every 15 atoms to have a free electron - a very different definition of "blood plasma". Compare to less than 1/100 for Arginine or 1/600 for DNA. 

\begin{autem}
	autem: What does a "free" electron mean in this context? Over what spatial extent does the "free" charge act?
\end{autem}



\subsection{Membrane potential perspective}


The polarization potential example that Adair borrows from Frolich, with a membrane potential of 100 V (see []) still only contributes.  

$$(4.523 \times 10^-14 \text{m}^2 \cdot 0.01 \text{F/m}^2 \cdot 100 \text{V}) \approx 10^5 e^- $$



\cite{Vibrational2002}
\begin{quote}
	Although the value q = e, chosen to describe the oscillating dipole moment already implies a very large permanent dipole moment, if the oscillating charge were greater
	than $2\times 10^4 e$, the radiative absorption from the canonical
	incident microwave power density would be increased to
	detectable levels if the relaxation time is as large as 5 ns.
	But so large a charge leads to a moment that seems far
	outside of our understanding of DNA or any other molecule.
	
	

\end{quote}


\subsection{Osmosis, ion ingress}

As noted previously, electroporation involves the ingress of large amounts of charge. This requires a pseudo-DC pulse to drive this charge ingress.

%If the intracellular concentration of 500 mM of ions  from https://www.cell.com/action/showPdf?pii=S0006-3495%2899%2976974-2 is extrapolated to viruses, 

%\[ 500 millimolar * (4/3 pi * ( 60 nanometers)^3)  3e5 e- \]

Indeed, the intracellular concentration would need to reach almost 25 molar for these results to be explained.

This is actually not too extreme. Some small bacteriophages can reach 3 M.

"Osmotic Shock and the Strength of Viral Capsids"

\footnotetext{
For an impermeable virus like the even-numbered T's, the inside salt concentration is set by the incubation solution's value of nb. Anderson et al. (1953) found in the case of T6, upon rapid dilution with distilled water, for example, that incubation nb values on the order of 1.5 M were required for osmotic shock. From Eq. 2, and n0 \ntilde 3 M, it follows that nin $\approx 0.62 M$ and hence n++ n- \ntilde 4.2 M, which because of the Donnan equilibrium is indeed significantly larger than 2nb \ntilde 3 M.
}
%https://www-sciencedirect-com.ezproxy.library.yorku.ca/science/article/pii/S0006349503744555



\begin{autem}
	autem: research the Gibbs-Donnan equilibrium
\end{autem}


\subsection{Discussion}

If we assume that the charge is not as. All other properties being equal, to account for the same coupling and absorption, it seems one or many of the following would seem to need to be true:

\begin{itemize}
\item We have grossly misunderstood the nature of the charge imbalance. Most likely.
\item The effective path length of the waveguide used by Sun may not be as they expected, or otherwise the corrections made are not of the correct scale.
\item The concentration of virus was higher than measured by plaque assay - perhaps a dilution was missed.
\end{itemize}







Adair's \cite{Vibrational2002}, in the section {\it \bf Coherent Processes}. 

$$\frac{4 \pi (100 \text{nm}^2)}{(10 \text{ micrometers} \times 10 \text{ micrometers})} 10^7 e^- = $$

The arguments regarding Resonances in microtubules are also 

\cite{Vibrational2002}
The conclusion that there can be no microwave resonances in DNA in water is in accord with the results of
measurements in three different laboratories (Gabriel et al.,
1987; Foster et al., 1987) using techniques designed to
detect resonances with amplitudes less than 1/20th of that
reported by Edwards et al. (1984).





Bosch \cite{Studies1985}







\end{document}