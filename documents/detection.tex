%!TeX root = detection
\documentclass[paper.tex]{subfiles}
\begin{document}


\begin{autem}
	I hasten to note that culturing and quantifying viral titer is a completely solved problem, and is a procedure often done in kindergarten labs. The amount of time and effort we spent on this problem was completely unjustifiable. It was driven by an acute lack of competence, equipment, and the 0.8 uL volume constraints of our original poorly designed microwave setup.\\
	
	None of the techniques that we mention are even remotely novel, and we present only the barest of experiences with each.
	Indeed, the fluorescence technique that were eventually decided on is common practice for lysis detection; our use is not novel in the slightest.
\end{autem}




Most of our results are tainted by two factors:

First, the fridge froze. A small "three-way" 

T4r+,

Two 





\begin{figure}[H]
	\captionsetup{singlelinecheck = false, justification=justified}
	\centering
	\includegraphics[width=\textwidth]{chunk_4}
	\caption{
		\light{
			\\
			We've got a thing\\
			that's called\\
			{\it Radar Love}\\}
		We've got a wave\\
		in the air}
\end{figure}

Essentially the only method of determining the activity of phage is to have it infect a suitable host.

Typically, this is done either in a large tube or a flat-sided optical cuvette.

Because the variation in the optical density is quite small, this poses some rather. 

A lot of the basic facts of phage, and most of the useful techniques were determined in the 1930s. 



A small vial of liquid has a much larger surface area relative to its thermal mass. sis removed, the temperature drops very rapidly; if the temperature drops outside the "physiological range" of around 33-40 C, growth is stunted \cite{effect2003}. \cite{growth1946}. I'm not sure if this was the problem I encountered, 

As a faculative aerobe, E.coli requires a concentration of oxygen to be present in the mixture for speedy replication \cite{Effect1965}; and this supply must also be kept sterile. One method to ensure this is by using a bubbler aerator; Carolina recommends an aquarium pump with cotton-ball filters. Probably Syringe filters are a good alternative.

Another method, often used in microfluidics\cite{Microfluidic}, is that of \cite{method1951}; the Manganese dioxide catalyzed decomposition of hydrogen peroxide. $MnO_2$ is readily obtained from alkaline batteries; and in this case no sterility issues are presented. However, it is not obvious how antimicrobial $H_2O_2$ vapor can be prevented from contacting the culture.

Bottles of zero air may be more effective.

In commercial fluid transfer machines, this is often resolved by HEPA filtering the input to an enclosing cabinet.

Shaking incubators, where the flasks are 

Ensuring that a great deal of headspace was present in a sealed tube (that is, culturing 2.5 mL in a 15 mL falcon tube) was sufficient for brief, low-density.



As the size of the well decreases, the Reynolds number also decreases \footnote{In fact, in the case of a circular well, it's not obvious that this is the case - the cross section decreases faster than the characteristic length}, and turbulent flow is greatly hampered.

Even slight rotation of the tube, such that the graduations 

"Chromogenic" substances \cite{Fluorogenic1991}, or the degradation of various dyes like Methylene Blue; tetrazolium dye methods;

Using 2.5 mm diameter x 3 mm high wells cut in polycarbonate and sealed with clear packing tape. However, the presence of small bubbles in the headspace of the wells prevented a consistent reading.

We tried simply adding 1 uL of culture to the microscope slide;

The 8-bit resolution and dynamic range of most cameras did not seem sufficient to discern the turbidity; and masking and correcting for 

\cite{Vision2016} use an ingenious technique. A background with sharp light-to-dark edges is used. Slicing the image into chunks only slightly larger than the object,


The 0.8 uL sample in each was not sufficient to produce a measurable change in turbidity in a 0.2 uL - or we were not experienced enough to perform the phage lysis protocol properly.

However, we were not able to make smaller volumes of culture enter the log phase.




Another very interesting technique is that of \cite{Study2003}, 



























It should be noted that 

\subsection{Detect the growth of the host, and measure effects of the virus on that growth}

\subsubsection{Plaque assay}

The plaque assay for viruses has remained essentially unchanged since d'Herelle and Delbruck's era.

Advantages over broth methods are the apparently longer infective period without agitation or 

Contamination - in both directions: if you don’t wash your hands properly before working, you might end up with a thick soup of some truly horrific bacteria.

Despite poor aseptic technique, contamination was not a serious issue - perhaps 1 contaminant region per plate.

Incubating the bacterial lawn for sufficient contrast to read the plaques imposes a ~1 to 3 day delay between test and results. Each test requires several dilution tubes

\subsubsection{Protocol and observations}

\begin{toolchain}

To simplify and for higher throughput we use the drop-cast absorption protocol of \cite{Simple2018}, which uses only a single layer of 1\% agar.

150 mL water in a 250 mL beaker covered with alu foil, with stirring, and boiled in a pot of water. 1.5 grams of nutrient agar powder \footnote{Seaweed Inc, Amazon} was added to 150  until molten. A magnetic stir bar was used to ensure all was fully melted,
and then the beaker was autoclaved for 15 minutes. After cooling to 60 C for 25 minutes in a water bath on a hotplate, 20 mL of agar was poured onto [100 mm?] polystyrene plates. The plates were left to set, then dried inverted with a flamed bent copper wire spacer for air exchange at 70 C for [1 hours?].
The plates were then stored inverted in plastic bags at 4 C until use.
\footnote{How agar percentages are specified seems somewhat ambiguous to me - is the percent of the agar itself (a viscosity measure), or the nutrient-agar mix (a growth measure)?
	Weight/weight? It appears to be w/v of the whole dry powder in distilled water.}

After cooling, 1 mL of overnight culture was poured across the plate, excess drained onto the next plate, and allowed to dry for 35 minutes at room temperature.

A control plate was made with pure stock phage and autoclaved phage.

The sample was retrieved from the cuvette by carefully removing the tape backing and withdrawing 0.8 uL from the reverse side, to avoid a silicone oil bubble.

Unfortunately, this ended up being very inconsistent.

Not having a 10 uL pipette, the 0.8 uL sample of phage was diluted in 7.5 uL of sterile broth in a microplate, then all 7.5 uL was added in 2.5 uL steps to the agar.

\st{Each 0.8 uL sample was diluted in 0.2 mL of nutrient broth, a 250x dilution. If the $10^6$ pfu.}

60c to pour was I think too high - condensation was high, even after 1 hour at 37c no change.

". The dryness of the agar is very important; if too moist, the
drops will run and coalesce, if too dry, the bacteria will grow
poorly"

The plates were dried at ~70 C in an oven with the slightly open.

The plates were incubated overnight. The lawn was very nicely formed; the phage concentration was

It was difficult to tell where the drops were placed, and they ran together.


\begin{figure}[H]
	\captionsetup{singlelinecheck = false, justification=justified}
	\centering
	\includegraphics[width=\textwidth]{plaque_assay}
\end{figure}



\end{toolchain}










\subsubsection{Broth assays}

This, however, requires a microplate reader which can simultaneously agitate for sufficient oxygen, incubate, and 

Tetrazolium dyes are metabolized 

GFP


\subsubsection{Impedance assay}


Because it would be difficult to maintain temperature on and multiplexing was difficult, low-frequency E. coli culture monitoring impedance spectra were performed separately.

An ultra-simple KHz impedance spectrometer was built, following \cite{unconventional2015}.

Purely contactless, capacitively coupled impedance measurement was found to have an unworkably small signal level;
however, a single conduction path and a single capacitive coupling through the Kapton film yielded a 10 mV signal level with 5V Vpp drive. A 110x amplifier was used to bring levels up.

0.7 mm gold-plated spring pins ('pogo' pins) pierced the silicone oil layer and pressed against the bottom tape; 1.4 mm pads were the second electrode.\footnote{Platinum, gold wire, anodized aluminum, or graphite would likely have worked equally well.} The pins were wiped with 99\% alcohol before each test.

2.5 uL stationary e. coli culture was added to 5 uL broth and 2.5 uL phage or autoclaved phage.

0 - 3 were phage; 4-7 were autoclaved phage.

Failure. decrease in signal rather than the expected increase, and no difference between autoclaved and phage.

Looking at this later - the fluid has leaked out of the wells after some time. That might be why this didn't work.



\section{E. coli flavin autofluorescence}



Added 2.5 uL to 5 uL broth. There does seem to be an fluorescence signal from; 100 uS long pulse, -70 mV is a good threshold.
Hz w/e.c, 2.9-5 Hz broth

Put the original eppendorf culture full of e. coli in. Jumped to 18 Hz @ -81 mV! Fantastic!

Thought I was all clever-like, using the oscope trigger out as the comparator and plugging that into a counter,
but above 20 Hz there's no thingy.

No. No go.

Weird background noise.

" By moving into this hydrophobic environment and away from the solvent, the ethidium cation is forced to shed any water molecules that were associated with it. As water is a highly efficient fluorescence quencher, the removal of these water molecules allows the ethidium to fluoresce.[citation needed]"




Otherwise, GFP-flouresecent strains of E. coli, or phage which has been labeled with a 




\subsection{Detect the release of intracellular material due to the lysis of the host}

One common target is the enzyme $\beta$-galactosidase. E. coli and other. Both  isopropyl $\beta$-D-thiogalactoside. Lactose itself is a reasonable substitute (Borralho \cite{Lactose2002}).

Ijzerman

X-gal \cite{Improved}


X-gal is only soluble in DMF and DMSO. We choose DMSO due to DMF's toxicity. DMSO is miscible with water, so that works out nicely. Solutions of < 0.75\% DMSO seem to not be cytotoxic.


I was introduced to this by Ijzerman \cite{liquid1993}, who recommend chlorophenol red $\beta$-D-galactopyranoside (CPRG) as the chromogen for a high sensitivity. CPRG is very hard to obtain except through Sigma.



beta-galactosidase



\begin{figure}[H]
	\captionsetup{singlelinecheck = false, justification=justified}
	\centering
	\includegraphics[width=0.3\textwidth]{x_gal.jpg}
	\caption{The very pretty opalescent blue culture caused by E. coli B trying to metabolize lactose in an X-gal $\beta$-galactosidase enzyme indicator.}
\end{figure}




\includegraphics[width=\textwidth]{x_gal_and_flourescence/CIMG6810.JPG}

 It’s probably in principle possible to synthesize them yourself, but it’s totally impractical. You will probably need a corporate address HS codes. 

T4 has a comparatively huge genome, which makes this easier.


\subsection{Detect the lysis of the virus, and the release of intra-viral material}

A fingerprint, for instance, has only about 0.2 ng of DNA (Subhani), so contamination is not a huge issue.

Phages release several small compounds like putrescine and spermadine when broken (Minagawa\cite{characteristics1961}). These can be indicated colorimetrically at ppm scales with ninhydrin, or at nanogram scales via fluorescamine. 


Biotium Gelgreen has a very specific advantage: to increase the safety of the dye, the flurophore is tied to some huge proprietary molecule, preventing it from diffusing through membranes or capsids. This has the side effect of making the fluorescence intensity strictly related to the quantity of genomic material dispersed in the solvent, rather than 




\subsubsection{Nucleic acid UV extinction }


The 260 nm extinction due to 60 MDa DNA at concentration $10^8 = 10e-9$. With a 1 mm path length is 0.002x. that's going to be swamped by everything else.


\subsection{Directly detect the virus itself}

PEG apparently concentrates to $10^{12}$ or 13 no matter what the starting concentration is (huh?), very convenient.

Early workers on phage mention\cite{ADSORPTION1940}  that cultures with a titer above about $10^10$ have an opalescent blue tint due to Tyndall scattering. Schlesinger in 1933 \cite{Beobachtung1933} (paper is in German). This technique appears to have been forgotten?



Concentrated can be put into the microgram range of high-sensitivity protein assays like BCA or Bradford, but that requires propagation and PEG precipitation, plus many such assays pose disposal hazards and stain labware. 







\end{document}