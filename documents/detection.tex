%!TeX root = background
\documentclass[paper.tex]{subfiles}
\begin{document}


It should be noted that 

- Detect the growth of the host bacteria, and measure effects of the phage on that growth

- Plaque plating


Contamination - in both directions: if you don’t wash your hands properly before working, you might end up with a thick soup of some truly horrific bacteria.

Plates are super sensitive, but in general I didn’t have huge issues with contamination; even my bad aseptic technique was fine - perhaps 1 contaminant region per plate.


- Detect the lysis of bacteria, the release of intracellular material


beta-galactosidase


A fingerprint, for instance, has only about 0.2 ng of DNA (Subhani), so contamination is not a huge issue.

Detect the lysis of the virus, and the release of intra-viral material



 It’s probably in principle possible to synthesize them yourself, but it’s totally impractical. You will probably need a corporate address HS codes. 

T4 has a comparatively huge genome, which makes this easier.

Directly detect the virus itself




PEG apparently concentrates to 10^12 or 13 no matter what the starting concentration is (huh?), very convenient.

Concentrated can be put into the microgram range of many standard protein assays like BCA, but that requires propagation and PEG precipitation, plus many such assays pose disposal hazards and stain labware. 
